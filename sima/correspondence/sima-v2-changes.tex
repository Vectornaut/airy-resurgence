\documentclass{letter}

\usepackage{url}
\usepackage{graphicx}
\usepackage{tikz}

% include signature image. based on suggestion from TeX.SE user Werner
%   https://tex.stackexchange.com/a/148458/6934
%%\makeatletter
%%\newcommand*{\signatureimg}[1]{\def\fromimg{#1}}
%%\renewcommand{\closing}[1]{\par\nobreak\vspace{\parskip}%
%%  \stopbreaks
%%  \noindent
%%  \ifx\@empty\fromaddress\else
%%  \hspace*{\longindentation}\fi
%%  \parbox{\indentedwidth}{\raggedright
%%       \ignorespaces #1\\%
%%       \includegraphics[height=6\medskipamount]{\fromimg}\\%
%%       \ifx\@empty\fromsig
%%           \fromname
%%       \else \fromsig \fi\strut}%
%%   \par}
%%\makeatother

\address{Veronica Fantini \\
Institut des Hautes \'{E}tudes Scientifiques \\
Le Bois-Marie, 35 route de Chartres \\
F-91440 Bures-sur-Yvette, France}

\signature{\vspace{-12mm}Veronica Fantini \\
\url{fantini@ihes.fr} \\[6pt]
Aaron Fenyes \\
\url{aaron.fenyes@fareycircles.ooo}}

%%\signatureimg{placeholder-sig.pdf}

\date{March \textcolor{violet}{--}, 2023}

\begin{document}
\begin{letter}{}
\begin{tikzpicture}[remember picture,overlay]
\node[xshift=4.2cm, yshift=-3.5cm] at (current page.north west) {\pgftext{\includegraphics[height=4.02cm]{ihes-logo}}};
\end{tikzpicture}
\opening{Dear Prof. Lipton:}

\color{violet}
Reviewer~2:
\begin{itemize}
    \item We corrected the sign error that reviewer~2 noticed on line~275.
    \item We corrected the factorial $j!$ as reviewer~2 suggested on line~706.
     \item We corrected the factorial $j!$ as reviewer~2 suggested on line~730. We rewrote the decomposition of $Q$. 
     \item We corrected the sign error that reviewer~2 noticed on line~767.
      \item We corrected the typo that reviewer~2 noticed on line~781.
\end{itemize}
Reviewer~1:
\begin{itemize}
\item Reviewer~1 asked us to write more concisely and more formally. Where these goals conflict, we have prioritized conciseness. For example, we have continued to use ``small enough'' instead of ``sufficiently small.''
\item There is no universal convention for naming the coordinates on the spaces involved in the Laplace transform. Conventions we have seen include:
\begin{center}
\begin{tabular}{c|c|l}
Frequency varaible & Position variable & Source \\ \hline
$z$ & $\zeta$ & Mitschi \& Sauzin \\
$s$ & $t$ & Schiff \\
$\hbar$ & $s$ & Mari\~{n}o \\
\end{tabular}
\end{center}
From a physical perspective, the Laplace transform turns functions of position or time into functions of spatial or temporal frequency. The terms ``frequency'' and ``position'' thus unambiguously distinguish the two spaces involved in the Laplace transform. We use these terms to remind the reader which variable-naming convention we are using.
\item We have removed all contractions.
\item We have removed the phrase ``tells us.''
\item We use the phrase ``over all'' to emphasize that we are describing a uniform bound. This is now explained explicitly in the section ``Notation for uniform bounds.''
\item We have changed our notation for uniform bounds to require the absolute value.
\end{itemize}
\color{black}
\closing{Best regards,}
\end{letter}
\end{document}
