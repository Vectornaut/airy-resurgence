\documentclass{letter}

\usepackage{url}
\usepackage{graphicx}
\usepackage{tikz}

% include signature image. based on suggestion from TeX.SE user Werner
%   https://tex.stackexchange.com/a/148458/6934
%%\makeatletter
%%\newcommand*{\signatureimg}[1]{\def\fromimg{#1}}
%%\renewcommand{\closing}[1]{\par\nobreak\vspace{\parskip}%
%%  \stopbreaks
%%  \noindent
%%  \ifx\@empty\fromaddress\else
%%  \hspace*{\longindentation}\fi
%%  \parbox{\indentedwidth}{\raggedright
%%       \ignorespaces #1\\%
%%       \includegraphics[height=6\medskipamount]{\fromimg}\\%
%%       \ifx\@empty\fromsig
%%           \fromname
%%       \else \fromsig \fi\strut}%
%%   \par}
%%\makeatother

\address{Veronica Fantini \\
Institut des Hautes \'{E}tudes Scientifiques \\
Le Bois-Marie, 35 route de Chartres \\
F-91440 Bures-sur-Yvette, France}

\signature{\vspace{-12mm}Campbell Wheeler \\
\url{wheeler@ihes.fr} \\[6pt]
Veronica Fantini \\
\url{veronica.fantini@universite-paris-saclay.fr}}

%%\signatureimg{placeholder-sig.pdf}

\date{January 10, 2025}

\begin{document}
\begin{letter}{}
\begin{tikzpicture}[remember picture,overlay]
\node[xshift=4.2cm, yshift=-3.5cm] at (current page.north west) {\pgftext{\includegraphics[height=4.02cm]{ihes-logo}}};
\end{tikzpicture}
\opening{Dear Prof. Lipton:}

We would like to submit the enclosed manuscript, ``Regular singular Volterra equations on complex domains,'' for publication in the \textit{SIAM Journal on Mathematical Analysis}. We believe this work is suitable for \textit{SIMA} because it is motivated by problems from mathematical physics, but its results are primarily analytical, and \textit{SIMA} is known for including both areas in its scope.

We have been using a new Laplace transform method to study certain special solutions of irregular singular ODEs, which appear often in work on resurgence and Borel summability. To characterize this method, we need function spaces and analytic results that we have not seen in the literature; we wrote this manuscript to describe them.

Because Laplace transform methods are widely used in science and engineering, we hope these spaces and results may be of interest beyond the scope of our work. This hope is reinforced by the physical flavor of the example ODEs mentioned in the introduction, which include the Airy equation from optics and the equation describing
the vibration modes of a solid triangular cantilever.

Our method turns ODEs into Volterra equations, which have appeared often in \textit{SIMA}. Articles by people in our field---including Tamara Grava, our suggested editor---have also appeared here. We therefore believe that \textit{SIMA} brings together readers from usually-separated communities that might be interested in this work.
\closing{Best regards,}
\end{letter}
\end{document}
