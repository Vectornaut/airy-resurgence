\documentclass{article}[a4paper]

\usepackage[left=3.5cm, right=3.5cm, top=1.5cm, bottom=1.5cm]{geometry}
\usepackage{amsmath}
\usepackage{titlesec}

\titleformat*{\section}{\large\bfseries}
\titlespacing{\section}{0pt}{*2}{*1}
\titleformat{\subsection}[runin]{\normalsize\bfseries}{\thesubsection\hspace{2ex}}{0pt}{}{}
\titlespacing{\subsection}{0pt}{*1}{*2}

\begin{document}
\begin{center}
\Large Synthesizing perspectives on Borel summation
\end{center}

\section*{Motivation}
In winter 2021--2022, we (Veronica Fantini and Aaron Fenyes) and some of our colleagues at IH\'{E}S met regularly to learn about Borel summation and resurgence. While discussing sources that introduce various aspects of Borel summation from different perspectives, we saw a need for sources that would detail the common threads binding together different approaches, examples, and parts of the theory.
\section*{Past work and goals}
Since then, we've been drafting an article---suitable for readers new to the field---which describes some of the common threads that have emerged from our discussions. Our main goals are:

\begin{itemize}
\item To focus on examples of functions which are both exponential integrals and solutions of differential equations, and to show how the same Borel plane emerges in different ways from these two characterizations.
\item To present Borel summation as a regularization process, and to explain from this common perspective why Borel summation can reconstruct both exponential integrals and solutions of certain differential equations from their asymptotic series.
\item To show how Borel summation links analogous structures across the division between functions and series, and the division between the frequency and position domains (that is, the $z$ and Borel planes).
\item To give the frequency and position domains a common geometry by presenting them as the singular fiber space and the base space of a cotangent bundle.
\end{itemize}
\section*{Plans for this visit}
During the proposed two-week visit, we expect to increase our writing pace through the more frequent, efficient, and spontaneous meetings made possible by working in person.

We also plan to disseminate and get feedback on our draft article through a series of two to four talks. Being close to the Plateau de Saclay and to Paris
\end{document}