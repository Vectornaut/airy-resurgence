\documentclass{article}

\usepackage{url}
\usepackage[hmargin=1.5in]{geometry}
\usepackage{amsmath}
\usepackage{amssymb}
\usepackage{graphicx}
\usepackage[svgnames]{xcolor}
\usepackage{tikz-cd}
% convenience aliases
\newcommand{\maps}{\colon}

% symbology
\newcommand{\Z}{\mathbb{Z}}
\newcommand{\R}{\mathbb{R}}
\newcommand{\C}{\mathbb{C}}

\newcommand{\series}[1]{\tilde{#1}}
\newcommand{\fracderiv}[3]{\partial^{#1}_{#2, #3}}
\newcommand{\holoL}[1]{\mathcal{H}L^{#1}} %% may no longer be needed
\newcommand{\blankbox}{{\fboxsep 0pt \colorbox{lightgray}{\phantom{$h$}}}}
\newcommand{\laplacepde}{\mathcal{D}}
\newcommand{\van}{\mathfrak{m}}
\DeclareMathOperator{\Ai}{Ai}
\usetikzlibrary{matrix,shapes,arrows,decorations.pathmorphing}
\tikzset{commutative diagrams/arrow style=math font}
\tikzset{commutative diagrams/.cd,
mysymbol/.style={start anchor=center,end anchor=center,draw=none}}
\newcommand\MySymb[2][\square]{%
  \arrow[mysymbol]{#2}[description]{#1}}
\tikzset{
shift up/.style={
to path={([yshift=#1]\tikztostart.east) -- ([yshift=#1]\tikztotarget.west) \tikztonodes}
}
}
\DeclareMathAlphabet{\mathpzc}{OT1}{pzc}{m}{it}

\newcommand*{\defeq}{\mathrel{\vcenter{\baselineskip0.5ex \lineskiplimit0pt
                     \hbox{\scriptsize.}\hbox{\scriptsize.}}}%
                     =}
\newcommand*{\defeqin}{\mathrel{\vcenter{\lineskiplimit0pt\baselineskip0.5ex
                     \hbox{\scriptsize.}\hbox{\scriptsize.}}}%
                     =}                     

%%\let\Re\relax
%%\DeclareMathOperator{\Re}{Re}

\newcommand{\laplace}{\mathcal{L}}
\newcommand{\borel}{\mathcal{B}}
\newcommand{\aexp}{\text{\ae}}
\newcommand{\deriv}[3]{\partial^{#1}_{#2 \text{ from } #3}}


\newtheorem{definition}{Definition}[section]
\newtheorem{prop}[definition]{Proposition}
\newtheorem{remark}[definition]{Remark}
\newtheorem{theorem}{Theorem}[section]
\newtheorem{corollary}[theorem]{Corollary}
\newtheorem{lemma}[definition]{Lemma}
%\newtheorem{conjecture}[definition]{Conjecture}
\newtheorem{claim}[definition]{Claim}
%\newtheorem{exercise}[definition]{Exercise}
%\newtheorem*{notation*}{Notation}

% drafting environments
\newenvironment{verify}{\color{ForestGreen}}{\color{black}}
\newenvironment{brainstorm}{\color{violet}\begin{itemize}}{\end{itemize}\color{black}}

% colors
\definecolor{ietocean}{RGB}{0, 30, 140}
\definecolor{ietcoast}{RGB}{0, 150, 173}
\definecolor{ietlagoon}{RGB}{0, 216, 180}

% pretty prince hyperref must always be the last thing in the preamble, always
\usepackage{hyperref}
\hypersetup{
  colorlinks,
  linkcolor={ietcoast},
  citecolor={ietcoast},
  urlcolor={ietcoast}
}


\title{Plan of the paper---old material}
\author{Veronica Fantini}

\begin{document}
\maketitle

\section{Plan of the paper}
\subsection{Results}

\color{orange}
\begin{itemize}
\item what does it mean to be Borel regular?
\item when does it happen?
\begin{itemize}
\item State new Borel regularity results
\begin{itemize}
\item Linear, homogeneous ODE with a regular singularity at 0 and irregular singularity at infinity [big idea in {\tt airy-resurgence}]
\begin{itemize}
\item Contextualize with previous work of Braaksma (``Multisummability and Stokes multipliers of linear meromorphic differential equations'')
\item Also contextualize with Balser, Braaksma, Sibuya, and Ramis (``Multisummability of formal power series solutions of linear ordinary differential equations'')
\item Also contextualize with Loday-Richaud
\end{itemize}
\item \emph{Borel regularity} for \textbf{thimbles integrals} can be stated a the commutativity of the following diagram:
\begin{equation}
\begin{tikzcd}
I_{\alpha}(z)\defeq\int_{\mathcal{C}_\alpha}e^{-zf}\nu \arrow[r,"\sim"]\arrow[dd, swap, "\laplace^\theta "] & \tilde{I}_{\alpha}(z)\arrow[dd,"\borel"]\\
& \\
\hat{\iota}_\alpha(\zeta)\arrow[r,equal,swap, "\text{sum}"] & \tilde{\iota}_{\alpha}(\zeta) 
\end{tikzcd}
\end{equation}
\item A priori, the Laplace transform of $\hat{\iota}_\alpha(\zeta)$ and $I_{\alpha}(z)$ have the same asymptotic behavior in a given sector (indeed taking the asymptotic of $I_\alpha(z)$ we \textit{loose} information); however Borel regularity guarantees that $I_{\alpha}(z)=\laplace^{\theta}\hat{\iota}_{\alpha}$ in a given sector.
\item thimble projection formula
\item Conjecturally, we expect $\hat{\varphi}_\alpha(\zeta)$ to have simple singularities. 
\item in the examples, $\hat{\varphi}_\alpha(\zeta)$ turn out be an hypergeometric function of type ${}_pF_{p-1}$ where $p$ is the number of critical values. 
\item We expect that hypergeometric functions play a special role in resurgence theory as they may always appear when there are only finitely many singularities.
\item \textcolor{magenta}{maybe we can say more about algebraic hypergeometric functions}
\end{itemize}
%\item Recall Watson condition (old): Let $R_N$ be the difference between a function and the partial sum
%\[ \frac{\varphi_0}{z} + \frac{\varphi_1}{z^2} + \frac{\varphi_2}{z^3} + \ldots + \frac{\varphi_{N-2}}{z^{N-1}} \]
%of its asymptotic series. Watson showed a century ago that the function is Borel regular whenever there's a constant $c \in (0, \infty)$ with
%\[ |R_N| \le \frac{c^{N+1} N!}{|z|^N} \]
%over all orders $N$ and all $z$ in a wide enough wedge around infinity.
\end{itemize}
\end{itemize}

\color{orange}
\subsection*{Outline}

\textbf{Title: Borel regularity and Resurgence of Exponential Integrals}

\begin{enumerate}
\item introduction
\begin{itemize}
\item Exponential integrals
\begin{itemize}
\item they are function of $z$ and they are defined from the data of $(X,f)$ and $[\mathcal{C}], [\nu]$
\item the choice of the path $\mathcal{C}$: 
\begin{itemize}
\item $\mathcal{C}\in H^{B,z}_{n}(X,f)$
\item Witten's formalism, $\mathcal{C}$ is a Lefschetz thimbles (or steepest descendent path)
\end{itemize}
\item they define a paring between the relative homology (rapid decaying homology)$H^{B,z}_{\bullet}(X,f)$ and the twisted de Rham cohomology  $H_{dR,z}^{\bullet}(X,f)$
\begin{itemize}
\item there is a comparison isomorphism (Maxim)
\end{itemize}
\item varying $z$ we have the Stokes phenomena
\item as $z\to\infty$, the asymptotic expansion of $I$ is a divergent series $\tilde{I}$, usually of Gevrey-class
\begin{itemize}
\item {\it exact resurgence relation} (Berry--Howls): divergence encodes contributions from other critical values
\item it is an example of resurgent series (\'Ecalle)
\item $\tilde{I}$ is resurgent in $\C\setminus\lbrace \text{poles of } \nu\,, \text{cirtical values of } f\rbrace$
\item it is a toy example of resurgent series because there are only finitely many singularities in the Borel plane
\item we have to compute the \textbf{Stokes constants} relative to the singular points in $B$ to fully understand $B$. There are two methods to compute Stokes constants: $\bullet$ geometric: using intersection theory of thimbles (Picard--Lefschtez, Witten, Maxim), $\bullet$ analytic: using \'Ecalle formalism   
\end{itemize} 
\end{itemize}
\item what are exponential integrals? \textcolor{gray}{has to be done}
\begin{itemize}
\item motivation
\begin{itemize}
\item In the classical theory of special functions, exponential integrals are often used to express solutions of linear differential and difference equations.
\item In physics ??
\item Geometrically they represent a Poincar\'e pairing (as explained by Kontsevich in \textbf{IHES lectures}).
\end{itemize}
\end{itemize}
\item What is the class of ODEs that we study? \textcolor{gray}{has to be done}
\item State results about resurgence of exponential integrals and Stokes phenomena
\begin{itemize}
\item Thimbles integrals [Kontsevich]: geometric computation of Stokes constants \textcolor{gray}{has to be done}
\item ODE and fractional derivative formula [{\tt draft2}]
\item if hypergeometric functions appear in a large class of examples: integral formulas for hypergeometric functions \textcolor{gray}{has to be done}
\end{itemize}
\end{itemize}
\item Formalism for Laplace transform [{\tt draft2}, ``The geometry of the Laplace transform'']
\begin{enumerate}
\item Analytic
\begin{enumerate}
\item Introduction
\item Brief revew of translation surfaces (we can refer to this from the introduction if we need to)
\item The Laplace transform of a holomorphic function
\begin{enumerate}
\item Over an ordinary point
\item Over a branch point
\item Differential equation
\end{enumerate}
\item Relating differential equations in the frequency domain to integral equations in the position domain
\end{enumerate}
\item Formal
\begin{enumerate}
\item Laplace transform of a formal series
\item Borel transform
\item Relating differential equations in the frequency variable to integral equations in the position variable
\end{enumerate}
\end{enumerate}
\item Review of integral equations
\begin{itemize}
\item Existence of solutions
\item Fractional integrals and derivatives
\item Going between integral and differential equations (slight functions)
\end{itemize}
\item General cases
\begin{enumerate}
\item Borel regularity
\begin{itemize}
\item General ODE of the form
\[ \left[ P\big(\tfrac{\partial}{\partial z}\big) + z^{-1} Q\big(\tfrac{\partial}{\partial z}\big) + z^{-2} R(z^{-1}) \right] \Phi = 0, \]
where $P$ is a polynomial, $Q$ is a polynomial of one degree lower, and $R$ is an entire function~\text[see {\tt airy-resurgence} and written notes]
\begin{itemize}
\color{DarkCyan}
\item More generally, for $P$ of degree $n$, we should be able to handle
\[ \left[ P\big(\tfrac{\partial}{\partial z}\big) + z^{-1} Q\big(\tfrac{\partial}{\partial z}\big) + z^{-2} R_2\big(\tfrac{\partial}{\partial z}\big) + \ldots + z^{-(n-1)} R_{n-1}\big(\tfrac{\partial}{\partial z}\big) + z^{-n} R(z^{-1}) \right] \Phi = 0, \]
where $R_k$ has degree $n-k$. \textcolor{gray}{has to be done}
\begin{itemize}
\item We want the most general ODE with a regular singularity at $z = 0$ and its only other singularity, typically irregular, at $z = \infty$. \textcolor{gray}{has to be done}
\item The singularity at $\infty$ should only be regular for an Euler equation. \textcolor{gray}{has to be done}
\end{itemize}
\color{orange}
\item Show that we can find a slight solution at each critical value.
\item Show that $\hat{\iota} = \tilde{\iota}$, where:
\begin{itemize}
\item $I = \laplace \iota$
\item $\hat{\iota}$ is the Taylor expansion of $\iota$
\item $\tilde{I}$ is the asymptotic series of $I$
\item $\tilde{\iota} = \borel \tilde{I}$
\item Idea: Show that $\hat{\iota}$ and $\tilde{\iota}$ have matching asymptotics at $\zeta = 0$. Since they both satisfy the position-domain integral equation, they must coincide.
\end{itemize}
\end{itemize}
\item General thimble integral (conditions?)
\begin{itemize}
\item Proof of Borel regularity
\item $3/2$-derivative formula
\item thimble projection reasoning
\end{itemize}
\end{itemize}
\item Resurgence
\begin{itemize}
\item Explain how Borel regularity relates resurgence of formal series to resurgence of holomorphic functions in the position domain. \textcolor{gray}{think more about what we're trying to say here}
\item Relate to \'Ecalle's formalism and the alien derivative
\item Stokes factors
\begin{itemize}
\item For ODEs
\item For thimble integrals
\end{itemize}
\end{itemize}
\end{enumerate}
\item Examples \textcolor{gray}{make sure each example contains a computation of the Borel transform, so we can see it matches}
\begin{enumerate}
\item The Airy example
\begin{itemize}
\item $I(z)$ is a solution of a linear ODE. We explicitly find its Borel transform, knowing the nature of singularities and the asymptotic behaviour of a basis of solution for the ODE  [{\tt airy-resurgence}]
\item Compute Stokes constants
\begin{itemize}
\item Using thimble projection formula and Borel transform computation [{\tt draft2}]
\item Using Picard-Lefschetz theory (Pham, Kontsevich, etc.)
\end{itemize}
\item Comparison with the literature \textcolor{gray}{has to be done}
\begin{itemize}
\item Mari\~{n}o
\item Sauzin
\item Kontsevich slides
\item Kawai--Takei? [might take too long to understand well enough]
\end{itemize}
\end{itemize}
\item The Airy--Lucas examples
\begin{itemize}
\item Compute Borel transform [{\tt airy-resurgence}]
\item Compute Stokes constants \textcolor{gray}{has to be done}
\end{itemize}
\item Bessel 0 (it is different because we have infinite cover)
\begin{itemize}
\item Compute Stokes constants [{\tt draft2}]
\end{itemize}
\item Bessel $\mu$ (follows from Bessel 0)
\begin{itemize}
\item Compute Stokes constants [{\tt modified Bessel}]
\end{itemize}
\item The generalized Airy example
\item The vibrating beam example
\begin{itemize}
\item In addition to the simple example, maybe we can do an example where the equation on the spatial domain includes fractional integrals since Andy is interested in that sort of thing
\end{itemize}
\end{enumerate}
\end{enumerate}

\subsection{Why Borel regularity works for thimble integrals}
\color{DarkBlue}
\begin{itemize}
%\item This is the setup. Let $X$ be an algebraic variety of dimension $N$, $f\colon X\to \mathbb{C}$ be a holomorphic Morse function with only simple critical points and $\nu\in\Gamma(X,\Omega^N)$.
%\begin{itemize}
%\item recall theory of homology and cohomology to define the thimbles integrals. Pham has briefly discussed it, but apparently was introduced by Malgrange, Milnor and ?.  
%\end{itemize} 
\item background: Pham studied the asymptotic behaviour of thimble integrals and we may interpret his result as a prof of Borel regularity in the algebraic setting, namely assuming $f$ a polynomial with distinct critical points (not necessary non--degenerate), $X=\C^N$, $\nu=g(x)dx_1\wedge...\wedge dx_N$ with $g(x)$ a polynomial. His proof uses microlocal analysis and classical results due to Varchenko. In the same algebraic setting, Malgrange states the Borel regularity result \cite[Theorem 6]{Malgrange22} 
\item \textcolor{red}{We give an explicit proof of Borel regularity beyond the polynomial case. Indeed I think we can simply assume $f$ holomorphic and proper.}
\item I think we should separate what can we do in general for N-dimensional integral and what is special for the $1$-dimensional one. 
%\item {[Thm 4]} 1-dim thimbles integrals are Borel regular, namely let 
%\begin{equation}
%I_{\alpha}(z)\defeq\int_{\Lambda_\alpha}e^{-zf}\nu
%\end{equation}

%then the following diagram is commutative
%\begin{equation}
%\begin{tikzcd}
%I_{\alpha}(z) \arrow[r,"\aexp^\theta"] & \tilde{I}_{\alpha}(z)\arrow[dd,"\borel"]\\
%& \\
%\hat{\iota}_\alpha(\zeta)\arrow[uu, "\laplace^\theta_\alpha "] & \tilde{\iota}_{\alpha}(\zeta) \arrow[l, "\text{sum}"]
%\end{tikzcd}
%\end{equation}
\begin{itemize}
%\item on the one hand take asymptotic expansion of integral via saddle point approximation and then study the Borel Laplace sum of the divergent series [from left upper corners clockwise]
\item on the other hand \textcolor{red}{see that the thimble integral is a generalized Laplace transform (which in certain cases can be rewritten as a usual Laplace transform).}
\item A priori, the Laplace transform of $\hat{\iota}_\alpha(\zeta)$ and $I_{\alpha}(z)$ have the same asymptotic behaviour in a given sector (indeed taking the asymptotic of $I_\alpha(z)$ we \textit{loose} information); however Borel regularity guarantees that $I_{\alpha}(z)=\laplace^{\theta}\hat{\iota}_{\alpha}$ in a given sector.
\item A corollary of [Thm 4] is the thimble projection formula.
\item \textcolor{red}{As a consequence of Thm 2.4 Pham}, if $f$ and $\nu$ are algebraic, we get that $\hat{\varphi}_\alpha(\zeta)$ to have simple singularities 
\begin{equation}
mmm
\end{equation} 
\end{itemize}
\item \text{[Cor]:} combining our result and the result of Maxim [IHES lectures] (generalizing Pham) we can deduce that Stokes constants for the Borel--Laplace sum of $\tilde{I}_j(z)$ are always integers as they are intersection numbers (use same argument of Maxim). See also the argument of KS21 sec 6.2 in the framework of analytic WCFs. 
\begin{itemize}
\item Prove integrality of Stokes constants using the fractional integral formula vs differential equation (see example in Appendix C) 
\end{itemize}
\end{itemize}
\color{black}



  

\bibliographystyle{utphys}
\bibliography{airy-resurgence}
\end{document}