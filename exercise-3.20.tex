\documentclass{article}

\usepackage{ajf}
\usepackage{amsthm}
\usepackage{stmaryrd}

\theoremstyle{definition}
\newtheorem{defn}{Definition}
\newtheorem*{exr}{Exercise}
\theoremstyle{plain}
\newtheorem{prop}{Proposition}
\newtheorem{thm}{Theorem}

\title{Exercise~3.20}
\author{Aaron Fenyes}

\newcommand{\phborel}{\mathcal{B}_\text{ph}}
\newcommand{\dict}{\mathcal{F}}

\begin{document}
\maketitle
Let $\mathcal{D}$ be the noncommutative $\C$-algebra generated by the number operator $N$ and the lowering operator $A$, with the relation $NA = A(N-1)$. Let $N$ act as $t \tfrac{\partial}{\partial t}$ on $\C\llparenthesis t \rrparenthesis$ and $\zeta \tfrac{\partial}{\partial \zeta}$ on $\C\llparenthesis \zeta \rrparenthesis$. Let $A$ act as $t^{-1}$ on $\C\llparenthesis t \rrparenthesis$ and
\begin{align*}
\frac{\zeta^n}{n!} & \mapsto \frac{\zeta^{n-1}}{(n-1)!} & n & > 0 \\
\zeta^n & \mapsto \zeta^{n-1} & n & \le 0
\end{align*}
on $\C\llparenthesis \zeta \rrparenthesis$. We use the physicist's Borel transform,
\begin{align*}
\phborel \maps \C\llbracket t \rrbracket & \to \C\llbracket \zeta \rrbracket \\
t^n & \mapsto \frac{\zeta^n}{n!}.
\end{align*}
\begin{exr}
Suppose $f \in \C\llbracket t \rrbracket$ is annihilated by $D \in \mathcal{D}$. Prove that $\phborel f \in \C\llbracket \zeta \rrbracket$ is annihilated by the operator $\dict(D)$ given by the alternate representation
\begin{align*}
\dict \maps \mathcal{D} & \to \big\{\C\llbracket \zeta \rrbracket \to \C\llparenthesis \zeta \rrparenthesis\big\} \\
A & \mapsto \tfrac{\partial}{\partial \zeta} \\
N & \mapsto t \tfrac{\partial}{\partial t}.
\end{align*}
\end{exr}
\begin{proof}[Solution]
Moving $t^{-1}$ to the right, we can write
\[ D = P_0(N)\,t^{-n} + P_1\,t^{1-n} + \ldots + P_{n-1}(N)\,t^{-1} + P_n(N) \]
in terms of polynomials $P_0, \ldots, P_n$. Recalling that the number operator $N$ is diagonal in the monomial basis, we can write $D$ as the matrix
\[ \begin{array}{l|lllll}
& 1 & t & t^2 & t^3 & \ldots \\ \hline
t^{0-n} & P_0(0-n) & \cdot & \cdot & \cdot & \ldots \\
t^{1-n} & P_1(1-n) & P_0(1-n) & \cdot & \cdot & \ldots \\
t^{2-n} & P_2(2-n) & P_1(2-n) & P_0(2-n) & \cdot & \ldots \\
t^{3-n} & P_3(3-n) & P_2(3-n) & P_1(3-n) & P_0(3-n) & \ldots \\
\vdots & \vdots & \vdots & \vdots & \vdots \\
t^{-1} & P_{n-1}(-1) & P_{n-2}(-1) & P_{n-3}(-1) & P_{n-4}(-1) & \cdots \\
1 & P_n(0) & P_{n-1}(0) & P_{n-2}(0) & P_{n-3}(0) & \cdots \\
t & \cdot & P_n(1) & P_{n-1}(1) & P_{n-2}(1) & \cdots \\
t^2 & \cdot & \cdot & P_n(2) & P_{n-1}(2) & \cdots \\
t^3 & \cdot & \cdot & \cdot & P_n(3) & \cdots \\
\vdots & \vdots & \vdots & \vdots & \vdots & \ddots \\
\end{array} \]

Now, observe that
\[ \dict(D) = P_0(N)\,\big(\tfrac{\partial}{\partial \zeta}\big)^n + P_1(N)\,\big(\tfrac{\partial}{\partial \zeta}\big)^{n-1} + \ldots + P_{n-1}(N)\,\tfrac{\partial}{\partial \zeta} + P_n(N). \]
Since $\big(\tfrac{\partial}{\partial \zeta}\big)^k = \Pi A^k$, where $\Pi$ is the projection $\C\llparenthesis \zeta \rrparenthesis \to \C\llbracket \zeta \rrbracket$ in the monomial basis, we can instead write
\[ \dict(D) = \Pi \left[ P_0(N)\,A^n + P_1(N)\,A^{n-1} + \ldots + P_{n-1}(N)\,A^1 + P_n(N) \right]. \]
\end{proof}
\end{document}