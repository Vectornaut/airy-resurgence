\documentclass{article}

\usepackage{ajf}
\usepackage{amsthm}
\usepackage{stmaryrd}

\theoremstyle{definition}
\newtheorem{defn}{Definition}
\newtheorem*{exr}{Exercise}
\theoremstyle{plain}
\newtheorem{prop}{Proposition}
\newtheorem{thm}{Theorem}

\title{Exercise~3.20}
\author{Aaron Fenyes}

\newcommand{\phborel}{\mathcal{B}_\text{ph}}
\newcommand{\dict}{\mathcal{F}}

\begin{document}
\maketitle
Let $\mathcal{D}$ be the noncommutative $\C$-algebra generated by the number operator $N$ and the lowering operator $A$, with the relation $NA = A(N-1)$. Let $N$ act as $t \tfrac{\partial}{\partial t}$ on $\C\llparenthesis t \rrparenthesis$ and $\zeta \tfrac{\partial}{\partial \zeta}$ on $\C\llparenthesis \zeta \rrparenthesis$. Let $A$ act as $t^{-1}$ on $\C\llparenthesis t \rrparenthesis$ and
\begin{align*}
\frac{\zeta^n}{n!} & \mapsto \frac{\zeta^{n-1}}{(n-1)!} & n & > 0 \\
\zeta^n & \mapsto \zeta^{n-1} & n & \le 0
\end{align*}
on $\C\llparenthesis \zeta \rrparenthesis$. These actions commute with the physicist's Borel transform
\begin{align*}
\phborel \maps \C\llbracket t \rrbracket & \to \C\llbracket \zeta \rrbracket \\
t^n & \mapsto \frac{\zeta^n}{n!}
\end{align*}
whenever both paths through the commutative square are well-defined.
\begin{exr}
Suppose $f \in \C\llbracket t \rrbracket$ is annihilated by $D \in \mathcal{D}$. Prove that $\phborel f \in \C\llbracket \zeta \rrbracket$ is annihilated by the operator $\dict(D)$ given by the alternate representation
\begin{align*}
\dict \maps \mathcal{D} & \to \big\{\C\llbracket \zeta \rrbracket \to \C\llparenthesis \zeta \rrparenthesis\big\} \\
A & \mapsto \tfrac{\partial}{\partial \zeta} \\
N & \mapsto \zeta \tfrac{\partial}{\partial \zeta}.
\end{align*}
\end{exr}
\begin{proof}[Solution]
Moving $A$ to the right, we can write
\begin{align*}
D & = P_0(N)\,A^n + P_1(N)\,A^{n-1} + \ldots + P_{n-1}(N)\,A + P_n(N) \\
\dict(D) & = P_0(N)\,\big(\tfrac{\partial}{\partial \zeta}\big)^n + P_1(N)\,\big(\tfrac{\partial}{\partial \zeta}\big)^{n-1} + \ldots + P_{n-1}(N)\,\tfrac{\partial}{\partial \zeta} + P_n(N)
\end{align*}
in terms of polynomials $P_0, \ldots, P_n$. Let $\Pi$ be the projection $\C\llparenthesis \zeta \rrparenthesis \to \C\llbracket \zeta \rrbracket$ in the monomial basis. Notice that $\big(\tfrac{\partial}{\partial \zeta}\big)^k$ and $\Pi A^k$ act the same on elements of $\C\llbracket \zeta \rrbracket$. Since $\Pi$ commutes with $N$, it follows that $\dict(D) = \Pi D$. Since the action of $\mathcal{D}$ commutes with $\phborel$, the assumption that $Df = 0$ implies that
\begin{align*}
\dict(D)\,\phborel f & = \Pi D \phborel f \\
& = \Pi \phborel Df \\
& = 0.
\end{align*}
\end{proof}
{\it Note from Alex: For holomorphic functions, the correct statement might be that $Df = 0$ implies $\tfrac{\partial}{\partial \zeta} \mathcal{F}(D)\,\phborel f = 0$. The idea is that $\mathcal{F}(A)$ only differs from $A$ because it annihilates constants, so a ``translation error'' in the differential equation can only leave constants left over.}
\end{document}