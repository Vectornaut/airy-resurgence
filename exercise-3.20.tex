\documentclass{article}

\usepackage{ajf}
\usepackage{amsthm}
\usepackage{stmaryrd}

\theoremstyle{definition}
\newtheorem{defn}{Definition}
\newtheorem*{exr}{Exercise}
\theoremstyle{plain}
\newtheorem{prop}{Proposition}
\newtheorem{thm}{Theorem}

\title{Exercise~3.20}
\author{Aaron Fenyes}

\newcommand{\phborel}{\mathcal{B}_\text{ph}}
\newcommand{\dict}{\mathcal{F}}

\begin{document}
\maketitle
Let $N_t = t \tfrac{\partial}{\partial t}$ and $N_\zeta = \zeta \tfrac{\partial}{\partial \zeta}$. We use the physicist's Borel transform,
\begin{align*}
\phborel \maps \C\llbracket t \rrbracket & \to \C\llbracket \zeta \rrbracket \\
t^n & \mapsto \zeta^n/n!.
\end{align*}
\begin{exr}
Suppose the formal series $f \in \C\llbracket t \rrbracket$ is annihilated by a differential operator $D \maps \C\llbracket t \rrbracket \to \C\llparenthesis t \rrparenthesis$ in the $\C$-algebra generated by $t^{-1}$ and $N_t$. Prove that $\phborel f$ is annihilated by the operator $\dict(D)$, where
\begin{align*}
\dict \maps t^{-1} & \mapsto \frac{\partial}{\partial \zeta} \\
N_t & \mapsto N_\zeta.
\end{align*}
\end{exr}
\begin{proof}[Solution]
Moving $t^{-1}$ to the right, we can write
\[ D = P_n(N_t)\,t^{-n} + P_{n-1}\,t^{-(n-1)} + \ldots + P_1(N_t)\,t^{-1} + P_0(N_t) \]
in terms of polynomials $P_n, \ldots, P_0$. After rescaling $f$ so its lowest-order coefficient is one, we can write $f = t^a + c_1 t^{a+1} + c_2 t^{a+2} + \ldots\,$. Monomials are eigenvectors of the number operator $N_t$, so $P_n(N_t)\,t^{-n} \cdot t^a$ is the lowest-order term of $Df$, unless it vanishes. The assumption that $Df = 0$ implies that this term does vanish. Simplifying the term to $P_n(a-n)\,t^{a-n}$, we learn that $P_n(a-n) = 0$.

Now we know that
\[ P_n(N_t)\,t^{-n} \cdot c_1 t^{a+1} + P_{n-1}(N_t)\,t^{-(n-1)} \cdot t^a \]
is the lowest-order term of $Df$, unless it vanishes. Again, by assumption, this term must vanish, so we learn that $c_1 P_n(a+1-n) + P_{n-1}(a+1-n) = 0$. \textbf{[...]}
\end{proof}
\end{document}