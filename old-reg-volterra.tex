\documentclass{article}

\usepackage{url}
\usepackage[hmargin=1.5in]{geometry}
\usepackage{amsmath}
\usepackage{amssymb}
\usepackage{amsthm}
%\usepackage{eqnarray}
\usepackage{stmaryrd} %% needed for mapsto arrows in commutative diagrams
\usepackage{bm} %% for putting series names in bold
%%\usepackage{amsthm}
\usepackage{graphicx}
\usepackage{fancyhdr}
\usepackage{xparse}
\usepackage{enumitem}
%\usepackage{eqnarray}
%% colors
\usepackage[svgnames]{xcolor}
\let\Re\relax
\DeclareMathOperator{\Re}{Re}
\let\Im\relax
\DeclareMathOperator{\Im}{Im}

%% editing
\newcommand{\done}[1]{\textcolor{gray}{#1}}

\theoremstyle{definition}
\newtheorem{defn}{Definition}
%%\theoremstyle{plain}
%%\newtheorem{prop}{Proposition}

% convenience aliases
\newcommand{\maps}{\colon}
\newcommand{\acts}{\mathbin{\raisebox{\depth}{\rotatebox{-90}{$\circlearrowright$}}}}


% symbology
\newcommand{\Z}{\mathbb{Z}}
\newcommand{\R}{\mathbb{R}}
\newcommand{\C}{\mathbb{C}}
\newcommand{\Q}{\mathbb{Q}}
\usepackage{tikz}
\usepackage{tikz-cd}
\usepackage{rotating}
\newcommand*{\isoarrow}[1]{\arrow[#1,"\rotatebox{90}{\(\sim\)}"
]}

\newcommand{\series}[1]{\tilde{#1}}
\newcommand{\fracderiv}[3]{\partial^{#1}_{#2, #3}}
\newcommand{\holoL}[1]{\mathcal{H}L^{#1}} %% may no longer be needed
\newcommand{\blankbox}{{\fboxsep 0pt \colorbox{lightgray}{\phantom{$h$}}}}
\newcommand{\laplacepde}{\mathcal{D}}
\newcommand{\van}{\mathfrak{m}}
\DeclareMathOperator{\Ai}{Ai}
\usetikzlibrary{matrix,shapes,arrows,decorations.pathmorphing}
\tikzset{commutative diagrams/arrow style=math font}
\tikzset{commutative diagrams/.cd,
mysymbol/.style={start anchor=center,end anchor=center,draw=none}}
\newcommand\MySymb[2][\square]{%
  \arrow[mysymbol]{#2}[description]{#1}}
\tikzset{
shift up/.style={
to path={([yshift=#1]\tikztostart.east) -- ([yshift=#1]\tikztotarget.west) \tikztonodes}
}
}
\DeclareMathAlphabet{\mathpzc}{OT1}{pzc}{m}{it}

\newcommand*{\defeq}{\mathrel{\vcenter{\baselineskip0.5ex \lineskiplimit0pt
                     \hbox{\scriptsize.}\hbox{\scriptsize.}}}%
                     =}
\newcommand*{\defeqin}{\mathrel{\vcenter{\lineskiplimit0pt\baselineskip0.5ex
                     \hbox{\scriptsize.}\hbox{\scriptsize.}}}%
                     =}                     

%%\let\Re\relax
%%\DeclareMathOperator{\Re}{Re}

\newcommand{\laplace}{\mathcal{L}}
\newcommand{\borel}{\mathcal{B}}
\newcommand{\aexp}{\text{\ae}}
\newcommand{\deriv}[3]{\partial^{#1}_{#2 \text{ from } #3}}


\newtheorem{definition}{Definition}[section]
\newtheorem{prop}[definition]{Proposition}
\newtheorem{remark}[definition]{Remark}
\newtheorem{theorem}{Theorem}[section]
\newtheorem{corollary}[theorem]{Corollary}
\newtheorem{lemma}[definition]{Lemma}
%\newtheorem{conjecture}[definition]{Conjecture}
\newtheorem{claim}[definition]{Claim}
%\newtheorem{exercise}[definition]{Exercise}
\newtheorem*{notation*}{Notation}

\title{Old regular Volterra equations}
\author{Veronica Fantini and Aaron Fenyes}

\begin{document}
\maketitle

\section{Integro-differential equations}
\subsection{Existence of solutions}
\subsubsection{Algebraic integral operators}
Take a simply connected open set $\Omega \subset \C$ that touches but doesn't contain $\zeta = 0$. Let $\holoL{\infty}(\Omega)$ be the space of bounded holomorphic functions on $\Omega$ with the supremum norm $\|\cdot\|_\infty$. For any $\sigma \in \R$, multiplying by $\zeta^{-\sigma}$ maps $\holoL{\infty}(\Omega)$ isomorphically onto another space of holomorphic functions on $\Omega$. We'll call this space $\holoL{\infty, \sigma}(\Omega)$ and give it the norm $\|f\|_{\infty, \sigma} = \|\zeta^\sigma f\|_\infty$, so that
\begin{align*}
\holoL{\infty}(\Omega) & \to \holoL{\infty, \sigma}(\Omega) \\
\phi & \mapsto \zeta^{-\sigma} \phi
\end{align*}
is an isometry. More generally,
\begin{align*}
\holoL{\infty, \rho}(\Omega) & \to \holoL{\infty, \rho+\delta}(\Omega) \\
f & \mapsto \zeta^{-\delta} f
\end{align*}
is an isometry for all $\rho \in \R$ and $\delta \in [0, \infty)$. This reduces to the previous statement when $\rho = 0$. For each $\delta \in [0, \infty)$, the functions in $\holoL{\infty, \rho}(\Omega)$ belong to $\holoL{\infty, \rho+\delta}(\Omega)$ too, and the inclusion map $\holoL{\infty, \rho}(\Omega) \hookrightarrow \holoL{\infty, \rho+\delta}(\Omega)$ has norm $\|\zeta^\delta\|_\infty$. Conceptually, $\|\zeta^\delta\|_\infty$ measures of the size of $\Omega$, so let's write it as $M^\delta$ with $M = \|\zeta\|_\infty$.

Since $\holoL{\infty}(\Omega)$ is a Banach algebra, the function space $\holoL{\infty, \infty}(\Omega):= \bigcup_{\sigma \in \R} \holoL{\infty, \sigma}(\Omega)$ is a graded algebra, with a different norm on each grade. For each $\rho, \delta \in \R$, multiplication by a function $m \in \holoL{\infty, \delta}(\Omega)$ gives a map $\holoL{\infty, \rho}(\Omega) \to \holoL{\infty, \rho+\delta}(\Omega)$ with norm $\|m\|_{\infty, \delta}$.

We'll study integral operators $\mathcal{G} \maps \holoL{\infty, \rho}(\Omega) \to \holoL{\infty, \sigma}(\Omega)$ of the form
\[ [\mathcal{G}f](a) = \int_{\zeta = 0}^{a} g(a, \cdot)\,f\,d\zeta, \]
where the kernel $g$ is an algebraic function over $\C^2$ which can be singular on $\Delta$, the diagonal.\footnote{Thanks to Alex Takeda for suggesting this.} To avoid ambiguity, we fix a branch of $g$ to use at the start of the integration path. The domain of $g$ is a covering of $\C^2$ which can be branched over $\Delta$. Continuing $g$ around $\Delta$ changes its phase by a root of unity, leaving its absolute value the same \textbf{[check]}. That makes $|g|$ a well-defined function on $\C^2 \smallsetminus \Delta$, which we can use to bound $\|\mathcal{G}\|$.

For each $a \in \C$, the expression $|g(a, \cdot)\,d\zeta|$ defines a {\em density} on $\Omega \smallsetminus \{a\}$---a norm on the tangent bundle which is compatible with the conformal structure. The square of a density is a Riemannian metric. Let $\ell^{\sigma, \rho}_{g, \Omega}(a)$ be the distance from $\zeta = 0$ to $a$ with respect to the density $|\zeta(a)^\sigma\,g(a, \cdot)\,\zeta^{-\rho}\,d\zeta|$ on $\Omega \smallsetminus \{a\}$. The bound
\begin{align*}
\big|[\zeta^\sigma \mathcal{G}f](a)\big| & \le \left| \zeta(a)^\sigma \int_{\zeta = 0}^{a} g(a, \cdot)\,f\,d\zeta \right| \\
& \le \int_{\zeta = 0}^{a} |\zeta^\rho f|\,|\zeta(a)^\sigma\,g(a, \cdot)\,\zeta^{-\rho}\,d\zeta| \\
& \le \|f\|_{\infty, \rho} \int_{\zeta = 0}^{a} |\zeta(a)^\sigma\,g(a, \cdot)\,\zeta^{-\rho}\,d\zeta|
\end{align*}
holds for any integration path. Taking the infimum over all paths, we see that
\[ \big|[\mathcal{G}f](a)\big| \le \ell^{\sigma, \rho}_{g, \Omega}(a)\,\|f\|_{\infty, \rho}. \]
so $\|\mathcal{G}\| \le \sup_{a \in \Omega} \ell^{\sigma, \rho}_{g, \Omega}(a)$. Crucially, we can always make $\|\mathcal{G}\|$ a contraction by restricting $\Omega$.
\subsubsection{The example of fractional integrals}
Setting $g(a, a') = (\zeta(a) - \zeta(a'))^{-\lambda-1}$ with $\lambda \in (-\infty, 0)$, we get the fractional integral $\partial^\lambda_{\zeta \text{ from } 0}$. The shortest path from $\zeta = 0$ to $a$ with respect to $|\zeta(a)^{\rho+\lambda}\,g(a, \cdot)\,\zeta^{-\rho}\,d\zeta|$ is the same as the shortest path with respect to $|d\zeta|$ \textcolor{magenta}{[check]}. It follows that
\begin{align*}
\ell^{\sigma, \rho}_{g, \Omega}(a) & = \int_0^{|\zeta(a)|} |\zeta(a)|^{\rho+\lambda}\,(|\zeta(a)| - r)^{-\lambda-1}\,r^{-\rho}\,dr \\
& = |\zeta(a)|^{\rho+\lambda} \int_0^1 \,(|\zeta(a)| - |\zeta(a)| t)^{-\lambda-1}\,(|\zeta(a)| t)^{-\rho}\,|\zeta(a)|\,dt \\
& = |\zeta(a)|^{\rho+\lambda-\lambda-1-\rho+1} \int_0^1 \,(1-t)^{-\lambda-1}\,t^{-\rho}\,dt \\
& = \int_0^1 \,(1-t)^{-\lambda-1}\,t^{-\rho}\,dt \\
& = B(-\lambda,\,1-\rho).
\end{align*}
The beta function $B$ can be written more explicitly as
\[ B(-\lambda,\,1-\rho) = \frac{\Gamma(-\lambda)\,\Gamma(1-\rho)}{\Gamma(1-\lambda-\rho)}. \]
Now we can see that for each $\lambda \in (-\infty, 0)$ and $\rho \in \R$, the fractional integral $\partial^\lambda_{\zeta \text{ from } 0}$ maps $\holoL{\infty, \rho}(\Omega)$ into $\holoL{\infty, \rho+\lambda}(\Omega)$, with norm $\|\partial^\lambda_{\zeta \text{ from } 0}\| \le B(-\lambda,\,1-\rho)$.
\subsubsection{Fractional integral equations near a regular singular point}\label{frac_int_exist}
\textcolor{SeaGreen}{\textbf{Angeliki:} Maybe Kato--Rellich perturbation theory can give existence immediately. It might not give uniqueness, though.}

Consider an integral operator $\mathcal{J}$ of the form
\[ p + \partial^{-1}_{\zeta \text{ from } 0} \circ q + \sum_{\lambda \in \Lambda} \partial^\lambda_{\zeta \text{ from } 0} \circ r_\lambda, \]
where:
\begin{itemize}
\item $p$ is a function in $\holoL{\infty, -1}(\Omega)$ that extends holomorphically over $\zeta = 0$, and its derivative at $\zeta = 0$ is non-zero.
\item $q$ is a function in $\holoL{\infty}(\Omega)$ that extends holomorphically over $\zeta = 0$.
\item $r_\lambda$ are functions in $\holoL{\infty}(\Omega)$.
\item $\Lambda$ is a countable subset of $(-\infty, -1)$ whose supremum is less than $-1$.
\end{itemize}
Our demand that $p$ and $q$ have convergent power series at $\zeta = 0$ can probably be relaxed; having convergent Novikov series, for example, should be enough. We could also probably replace $\partial^{-1}_{\zeta \text{ from } 0}$ with $\partial^{-1+\delta}_{\zeta \text{ from } 0} \circ \zeta^\delta$ for some $\delta \in [0, 1)$, or adjust the $\partial^\lambda_{\zeta \text{ from } 0}$ similarly.

We want to solve the equation $\mathcal{J}f = 0$. Let's look for a solution of the form $f = \zeta^{\tau-1} + \tilde{f}$ with $\tau \in (0, \infty)$ and $\tilde{f} \in \holoL{\infty, 1-\tau-\epsilon}(\Omega)$ for some $\epsilon \in (0, 1]$. When $\epsilon$ is small enough that $\Lambda \subset (-\infty, -1 - \epsilon]$, we'll see that we can always find such a solution, as long as we're willing to shrink $\Omega$. In fact, there's exactly one such solution. \textcolor{magenta}{[Add convergence conditions for $\partial^\lambda$ terms.]}

Let $p'_0$ and $q_0$ be the values of $\tfrac{\partial}{\partial \zeta} p$ and $q$, respectively, at $\zeta = 0$. We're assuming that $p$ and $q$ extend holomorphically over $\zeta = 0$, and the additional assumption that $p \in \holoL{\infty, -1}(\Omega)$ implies that $p$ has a first-order zero at $\zeta = 0$.

Since $p$ and $q$ extend holomorphically over $\zeta = 0$, and $p$ vanishes at $\zeta = 0$, we can write
\begin{alignat*}{2}
p & = p'_0 \zeta &\;+\;& \tilde{p} \\
q & = q_0 &\;+\;& \tilde{q}
\end{alignat*}
with $\tilde{p} \in \holoL{\infty, -2}(\Omega)$ and $\tilde{q} \in \holoL{\infty, -1}(\Omega)$. Then we have
\[ \mathcal{J} = p'_0\zeta + q_0\,\partial^{-1}_{\zeta \text{ from } 0} + \tilde{\mathcal{J}} \]
with
\[ \tilde{\mathcal{J}} = \tilde{p} + \partial^{-1}_{\zeta \text{ from } 0} \circ \tilde{q} + \sum_{\lambda \in \Lambda} \partial^\lambda_{\zeta \text{ from } 0} \circ r_\lambda \]
For any $\tau \in (0, \infty)$,
\[ \mathcal{J} \zeta^{\tau-1} = (p'_0 + q_0/\tau)\,\zeta^\tau + \tilde{\mathcal{J}} \zeta^{\tau-1}. \]
Setting $\tau = -q_0 / p'_0$ makes the first term vanish, leaving
\[ \mathcal{J} \zeta^{\tau-1} = \tilde{\mathcal{J}} \zeta^{\tau-1}. \]
Then the equation $\mathcal{J}f = 0$ becomes
\begin{align}
0 & = \tilde{\mathcal{J}}\zeta^{\tau-1} + \mathcal{J}\tilde{f} \nonumber \\
0 & = \tilde{\mathcal{J}}\zeta^{\tau-1} + \left[ p'_0\zeta + q_0\,\partial^{-1}_{\zeta \text{ from } 0} + \tilde{\mathcal{J}} \right]\tilde{f} \nonumber \\
-p'_0\zeta\tilde{f} & = \tilde{\mathcal{J}}\zeta^{\tau-1} + \left[ q_0\,\partial^{-1}_{\zeta \text{ from } 0} + \tilde{\mathcal{J}} \right]\tilde{f} \nonumber \\
\tilde{f} & = \left[ -\tfrac{1}{p'_0}\,\zeta^{-1} \circ \tilde{\mathcal{J}} \right]\,\zeta^{\tau-1} + \left[ \tau\,\zeta^{-1} \circ \partial^{-1}_{\zeta \text{ from } 0} - \tfrac{1}{p'_0}\,\zeta^{-1} \circ \tilde{\mathcal{J}} \right]\tilde{f}. \label{fixed-pt}
\end{align}

\color{Indigo}
\begin{center}
\begin{tikzcd}[column sep=25mm, row sep=2mm]
& & \holoL{\infty, \rho-1}(\Omega) \\
\holoL{\infty, \rho}(\Omega) \arrow[r, "\tilde{p}", "\|\tilde{p}\|_{\infty, -2}"'] & \holoL{\infty, \rho-2}(\Omega) \arrow[ru, hook, "\|\zeta\|_\infty"'] \arrow[rd, hook, "\|\zeta\|_\infty^{1-\epsilon}"'] \\
& & \holoL{\infty, \rho-1-\epsilon}(\Omega) \\
& & & \holoL{\infty, \rho-1}(\Omega) \\
\holoL{\infty, \rho}(\Omega) \arrow[r, "\tilde{q}", "\|\tilde{q}\|_{\infty, -1}"'] & \holoL{\infty, \rho-1}(\Omega) \arrow[r, "\partial^{-1}", "{B(1, 2-\rho) = \frac{1}{2-\rho}}"'] & \holoL{\infty, \rho-2}(\Omega) \arrow[ru, hook, "\|\zeta\|_\infty"'] \arrow[rd, hook, "\|\zeta\|_\infty^2"'] \\
& & & \holoL{\infty, \rho}(\Omega) \\
& & & \holoL{\infty, \rho-1}(\Omega) \\
\holoL{\infty, \rho}(\Omega) \arrow[r, "r_\lambda", "\|r_\lambda\|_\infty"'] & \holoL{\infty, \rho}(\Omega) \arrow[r, "\partial^\lambda", "{B(-\lambda, 1-\rho)}"'] & \holoL{\infty, \rho+\lambda}(\Omega) \arrow[ru, hook, "\|\zeta\|_\infty^{-1-\lambda}"'] \arrow[rd, hook, "\|\zeta\|_\infty^{-1-\epsilon-\lambda}"'] \\
& & & \holoL{\infty, \rho-1-\epsilon}(\Omega) \\
\holoL{\infty, \rho}(\Omega) \arrow[r, "\partial^{-1}", "{B(1, 1-\rho) = \frac{1}{1-\rho}}"'] & \holoL{\infty, \rho-1}(\Omega) \arrow[r, "\zeta^{-1}", "\|\zeta^{-1}\|_{\infty, 1} = 1"'] & \holoL{\infty, \rho}(\Omega) \\
\end{tikzcd}
\end{center}
\color{black}

From \textbf{[our previous discussion]}, we can work out that
\begin{align*}
\tilde{\mathcal{J}} & \maps \holoL{\infty, 1-\tau}(\Omega) \to \holoL{\infty, -\tau-\epsilon}(\Omega)\;\text{with} \\
\|\tilde{\mathcal{J}}\| & \le \left(\|\tilde{p}\|_{\infty, -2} + \tfrac{1}{1+\tau}\,\|\tilde{q}\|_{\infty, -1} \right) M^{1-\epsilon} + \sum_{\lambda \in \Lambda} B(-\lambda, \tau)\,\|r_\lambda\|_{\infty}\,M^{-1-\epsilon-\lambda}
\end{align*}
and
\begin{align*}
\tilde{\mathcal{J}} & \maps \holoL{\infty, 1-\tau-\epsilon}(\Omega) \to \holoL{\infty, -\tau-\epsilon}(\Omega)\;\text{with} \\
\|\tilde{\mathcal{J}}\| & \le \left(\|\tilde{p}\|_{\infty, -2} + \tfrac{1}{1+\tau+\epsilon}\,\|\tilde{q}\|_{\infty, -1} \right) M + \sum_{\lambda \in \Lambda} B(-\lambda, \tau+\epsilon)\,\|r_\lambda\|_{\infty}\,M^{-1-\lambda}.
\end{align*}
Since $\|\zeta^{-1}\|_{\infty, 1} = 1$, it follows that
\begin{align*}
\zeta^{-1} \circ \tilde{\mathcal{J}} & \maps \holoL{\infty, 1-\tau}(\Omega) \to \holoL{\infty, 1-\tau-\epsilon}(\Omega)\;\text{with} \\
\|\zeta^{-1} \circ \tilde{\mathcal{J}}\| & \le \left(\|\tilde{p}\|_{\infty, -2} + \tfrac{1}{1+\tau}\,\|\tilde{q}\|_{\infty, -1} \right) M^{1-\epsilon} + \sum_{\lambda \in \Lambda} B(-\lambda, \tau)\,\|r_\lambda\|_{\infty}\,M^{-1-\epsilon-\lambda} \label{bound:mollify}
\end{align*}
and
\begin{align}
\zeta^{-1} \circ \tilde{\mathcal{J}} & \acts \holoL{\infty, 1-\tau-\epsilon}(\Omega)\;\text{with} \\%\nonumber \\
\|\zeta^{-1} \circ \tilde{\mathcal{J}}\| & \le \left(\|\tilde{p}\|_{\infty, -2} + \tfrac{1}{1+\tau+\epsilon}\,\|\tilde{q}\|_{\infty, -1} \right) M + \sum_{\lambda \in \Lambda} B(-\lambda, \tau+\epsilon)\,\|r_\lambda\|_{\infty}\,M^{-1-\lambda}. \label{bound:perturb}
\end{align}
We can also see that
\begin{align}
\tau\,\zeta^{-1} \circ \partial^{-1}_{\zeta \text{ from } 0} & \acts \holoL{\infty, 1-\tau-\epsilon}(\Omega)\;\text{with}   \nonumber \\
\|\tau\,\zeta^{-1} \circ \partial^{-1}_{\zeta \text{ from } 0}\| & = \tfrac{\tau}{\tau+\epsilon} < 1 \label{bound:contract}.
\end{align}

Now, let's return to equation~\eqref{fixed-pt}, which tells us that $f = \zeta^{\tau-1} + \tilde{f}$ satisfies $\mathcal{J}f = 0$ when \textcolor{magenta}{[and only when?]} $\tilde{f}$ is a fixed point of the affine map $\mathcal{A}(\cdot) + b$, where
\begin{align*}
\mathcal{A} & = \tau\,\zeta^{-1} \circ \partial^{-1}_{\zeta \text{ from } 0} - \tfrac{1}{p'_0}\,\zeta^{-1} \circ \tilde{\mathcal{J}}  \\
b & = \left[ -\tfrac{1}{p'_0}\,\zeta^{-1} \circ \tilde{\mathcal{J}} \right]\,\zeta^{\tau-1}
\end{align*}
Choosing $\epsilon \in (0, 1]$ so that $\Lambda \subset (-\infty, -1 - \epsilon]$ has given us the domain and codomain statements in bounds \ref{bound:mollify} and \ref{bound:perturb}, which tell us that $\mathcal{A}(\cdot) + b$ sends $\holoL{\infty, 1-\tau-\epsilon}(\Omega)$ into itself. We'll show that when $\Omega$ is small enough, $\mathcal{A}(\cdot) + b$ contracts $\holoL{\infty, 1-\tau-\epsilon}(\Omega)$, and thus---by the contraction mapping theorem---has a unique fixed point.

An affine map is a contraction if and only if its linear part is a contraction. We know from bound~\ref{bound:contract} that $\tau\,\zeta^{-1} \circ \partial^{-1}_{\zeta \text{ from } 0}$ contracts $\holoL{\infty, 1-\tau-\epsilon}(\Omega)$. Since the supremum of $\Lambda$ is less than $-1$, all the powers of $M = \|\zeta\|_\infty$ in bound~\ref{bound:perturb} are positive. Thus, by shrinking $\Omega$, we can make the norm of $\zeta^{-1} \circ \tilde{\mathcal{J}}$ on $\holoL{\infty, 1-\tau-\epsilon}(\Omega)$ as small as we want---small enough to make $\mathcal{A}$ a contraction.
%\input{chapters/appendix}

\end{document}