\documentclass{article}

\usepackage{url}
\usepackage[hmargin=1.5in]{geometry}
\usepackage{amsmath}
\usepackage{amssymb}
\usepackage{graphicx}
\usepackage{xcolor}
% convenience aliases
\newcommand{\maps}{\colon}

% symbology
\newcommand{\Z}{\mathbb{Z}}
\newcommand{\R}{\mathbb{R}}
\newcommand{\C}{\mathbb{C}}
\newcommand{\laplace}{\mathcal{L}}
%\renewenvironment{proof}{{\scshape Proof.}}{\qed}

\makeatletter
\newenvironment{proofof}[1]{\par
  \pushQED{\qed}%
  \normalfont \topsep6\p@\@plus6\p@\relax
  \trivlist
  \item[\hskip3\labelsep
        \itshape
    Proof of #1\@addpunct{.}]\ignorespaces
}{%
  \popQED\endtrivlist\@endpefalse
}
\makeatother

% Def
%\def\be{\begin{equation}}    
%\def\ee{\end{equation}}
\def\into{\hookrightarrow}
\def\onto{\twoheadrightarrow}
\def\isom{\cong}  
\def\ra{\rightarrow}
\def\lra{\longrightarrow}
\def\surj{\twoheadrightarrow}
\def\Var{\mathrm{Var}}
\def\Sch{\mathrm{Sch}}
\def\Sets{\mathrm{Sets}}
\def\Def{\mathsf{Def}}
\def\KS{\mathsf{KS}}
\def\ad{\mathsf{ad}}
\def\St{\mathrm{St}}
\def\st{\mathrm{st}}

\def\L{\mathbb L}
\def\A{\mathcal A}
\def\B{\mathcal B}
\def\R{\mathbb R}
\def\C{\mathbb C}
\def\D{\mathbb D}
\def\P{\mathbb P}
\def\Q{\mathbb Q}
\def\G{\mathbb G}
\def\L{\mathbb{L}}
\def\SS{\mathcal S}
\def\RR{\mathbf R}
\def\X{\mathcal X}
\def\E{\mathcal E}
\def\Z{\mathbb Z}
\def\N{\mathbb N}
\def\ext{\mathrm{ext}}
\def\FF{\mathscr{F}}

\def\HS{\mathsf{HS}}
\def\O{\mathscr O}
\def\DDT{\mathsf{DT}}
\def\PPT{\mathsf{PT}}
\def\LL{\mathsf{L}}
\def\NN{\mathsf{N}}
\def\sc{\textrm{sc}}
\def\dcr{\textrm{d-crit}}
\def\loc{\textrm{loc}}
\def\Ad{\textrm{Ad}}
\def\reg{\textrm{reg}}
\def\red{\textrm{red}}
\def\relvir{\textrm{relvir}}
\def\pur{\textrm{pur}}
\def\vd{\mathrm{vd}}
\def\pure{\textrm{pure}}
\def\MF{\mathsf{MF}}
\def\WW{\mathsf{W}}
\def\HH{\mathsf{H}}
\def\h{\mathfrak{h}}
\def\at{\mathsf A}
\def\pt{\mathrm{pt}}

\def\CC{\mathrm{C}}
\def\KK{\mathrm{K}}
\DeclareMathOperator{\Mod}{Mod}
\DeclareMathOperator{\op}{op}
\DeclareMathOperator{\Tor}{Tor}
\DeclareMathOperator{\Mor}{Mor}
\DeclareMathOperator{\Fun}{Fun}
\DeclareMathOperator{\Vect}{Vect}
\DeclareMathOperator{\FDVect}{FDVect}
\DeclareMathOperator{\Rings}{Rings}
\DeclareMathOperator{\ev}{ev}
\DeclareMathOperator{\Quot}{Quot}
\DeclareMathOperator{\DD}{D}
\DeclareMathOperator{\Hilb}{Hilb}
\DeclareMathOperator{\Chow}{Chow}
\DeclareMathOperator{\Orb}{Orb}
\DeclareMathOperator{\Ob}{Ob}
\DeclareMathOperator{\ob}{ob}
\DeclareMathOperator{\Jac}{Jac}
\DeclareMathOperator{\ch}{ch}
\DeclareMathOperator{\Td}{Td}
\DeclareMathOperator{\tr}{tr}
\DeclareMathOperator{\id}{id}
\DeclareMathOperator{\Pic}{Pic}
\DeclareMathOperator{\codet}{codet}
\DeclareMathOperator{\Rep}{Rep}
\DeclareMathOperator{\Bl}{Bl}
\DeclareMathOperator{\ord}{ord}
\DeclareMathOperator{\aff}{aff}
\DeclareMathOperator{\vir}{vir}
\DeclareMathOperator{\QCoh}{QCoh}
\DeclareMathOperator{\Coh}{Coh}
\DeclareMathOperator{\Span}{Span}
\DeclareMathOperator{\mult}{mult}
\DeclareMathOperator{\Spec}{Spec\,}
\DeclareMathOperator{\Proj}{Proj\,}
\DeclareMathOperator{\Supp}{Supp\,}
\DeclareMathOperator{\coker}{coker}
\DeclareMathOperator{\Cone}{Cone}
\DeclareMathOperator{\Perf}{Perf}
\DeclareMathOperator{\im}{im}
\DeclareMathOperator{\DT}{DT}
\DeclareMathOperator{\PT}{PT}
\DeclareMathOperator{\RRR}{R}
\DeclareMathOperator{\GL}{GL}
\DeclareMathOperator{\SL}{SL}
\DeclareMathOperator{\dd}{d}
\DeclareMathOperator{\Tr}{Tr}
\DeclareMathOperator{\NCHilb}{NCHilb}
\DeclareMathOperator{\Sym}{Sym}
\DeclareMathOperator{\Aut}{Aut}
\DeclareMathOperator{\Ext}{Ext}
\DeclareMathOperator{\lExt}{{\mathscr Ext}}
\DeclareMathOperator{\Hom}{Hom}
\DeclareMathOperator{\lHom}{{\mathscr Hom}}
\DeclareMathOperator{\catA}{{\mathscr A}}
\DeclareMathOperator{\catB}{{\mathscr B}}
\DeclareMathOperator{\catC}{{\mathcal C}}
\DeclareMathOperator{\catD}{{\mathcal D}}
\DeclareMathOperator{\catT}{{\mathscr T}}
\DeclareMathOperator{\catF}{{\mathscr F}}
\DeclareMathOperator{\End}{End}
\DeclareMathOperator{\Eu}{Eu}
\DeclareMathOperator{\Exp}{Exp}
\DeclareMathOperator{\rk}{rk}
\DeclareMathOperator{\Nil}{Nil}
\DeclareMathOperator{\Tot}{Tot}
\DeclareMathOperator{\length}{length}
\DeclareMathOperator{\codim}{codim}
\DeclareMathOperator{\pr}{pr}
%\DeclareMathOperator{\at}{at}
\DeclareMathOperator{\Art}{Art}
\DeclareMathOperator{\uC}{\underline{\mathcal C}}
\DeclareMathOperator{\uA}{\underline{\mathscr A}}
\DeclareMathOperator{\F}{\mathcal F}
\DeclareMathOperator{\hh}{H}%Da togliere quando corregger� il capitolo 4
\DeclareMathOperator{\Der}{Der}
\DeclareMathOperator{\Ab}{Ab}


%%%%%%%%%%%%%%%%
%\theoremstyle{definition}
%
%\newtheorem*{lemma*}{Lemma}
%\newtheorem*{theorem*}{Theorem}
%\newtheorem*{example*}{Example}
%\newtheorem*{fact*}{Fact}
%\newtheorem*{notation*}{Notation}
%\newtheorem*{definition*}{Definition}
%\newtheorem*{prop*}{Proposition}
%\newtheorem*{remark*}{Remark}
%\newtheorem*{corollary*}{Corollary}
%\newtheorem*{conventions*}{Conventions}
%\newtheorem*{caution*}{Caution}

%\newtheorem{definition}{Definition}[section]
%\newtheorem{problem}[definition]{Problem}
%\newtheorem{example}[definition]{Example}
%\newtheorem{fact}[definition]{Fact}
%\newtheorem{aside}[definition]{Aside}
%%\newtheorem{prop}[definition]{Proposition}
%\newtheorem{question}[definition]{Question}
%\newtheorem{remark}[definition]{Remark}
%\newtheorem{theorem}[definition]{Theorem}
%%\newtheorem{corollary}[definition]{Corollary}
%\newtheorem{lemma}[definition]{Lemma}
%%\newtheorem{conjecture}[definition]{Conjecture}
%\newtheorem{claim}[definition]{Claim}
%\newtheorem{exercise}[definition]{Exercise}

%\newtheoremstyle{thm} % <name> % (ambienti con dimostrazione)
%        {3mm}% <Space above>
%        {3mm}% <Space below>
%        {\slshape}% <Body font> % 
%        {0mm}% <Indent amount>
%        {\bfseries}% <Theorem head font>
%        {.}% <Punctuation after theorem head>
%        {1mm}% <Space after theorem head>
%        {}% <Theorem head spec (can be left empty, meaning 'normal')> 
%\theoremstyle{thm}
%\newtheorem{theorem}[definition]{Theorem}
%\newtheorem{corollary}[definition]{Corollary}
%\newtheorem{lemma}[definition]{Lemma}
%\newtheorem{prop}[definition]{Proposition}
%\newtheorem{thm}{Theorem}
%\newtheorem{notation}{Notation}
%\renewcommand*{\thethm}{\Alph{thm}}



%\newtheoremstyle{sol} % <name> % (ambienti con dimostrazione)
%        {3mm}% <Space above>
%        {3mm}% <Space below>
%        {\normalfont}% <Body font> % 
%        {0mm}% <Indent amount>
%        {\scshape}% <Theorem head font>
%        {.}% <Punctuation after theorem head>
%        {1mm}% <Space after theorem head>
%        {}% <Theorem head spec (can be left empty, meaning 'normal')> 
%\theoremstyle{sol}
%\newtheorem{slogan}[definition]{Slogan}
%\newtheorem{assumption}[definition]{Assumption}
%%\newtheorem{claim}[definition]{Claim}
%\newtheorem{notation}[definition]{Notation}
%\newtheorem*{ssolution*}{Solution (sketch)}
%\newtheorem*{solution*}{Solution}


%%%%%%%%%%%%%%%%%%%%%%%%%

\usepackage{tikz}
\usepackage{tikz-cd}
\usepackage{rotating}
\newcommand*{\isoarrow}[1]{\arrow[#1,"\rotatebox{90}{\(\sim\)}"
]}
\usetikzlibrary{matrix,shapes,arrows,decorations.pathmorphing}
\tikzset{commutative diagrams/arrow style=math font}
\tikzset{commutative diagrams/.cd,
mysymbol/.style={start anchor=center,end anchor=center,draw=none}}
\newcommand\MySymb[2][\square]{%
  \arrow[mysymbol]{#2}[description]{#1}}
\tikzset{
shift up/.style={
to path={([yshift=#1]\tikztostart.east) -- ([yshift=#1]\tikztotarget.west) \tikztonodes}
}
}

\DeclareMathAlphabet{\mathpzc}{OT1}{pzc}{m}{it}

\newcommand*{\defeq}{\mathrel{\vcenter{\baselineskip0.5ex \lineskiplimit0pt
                     \hbox{\scriptsize.}\hbox{\scriptsize.}}}%
                     =}
\newcommand*{\defeqin}{\mathrel{\vcenter{\lineskiplimit0pt\baselineskip0.5ex
                     \hbox{\scriptsize.}\hbox{\scriptsize.}}}%
                     =}                     


\title{Resurgence of modified Bessel functions of second kind}
\author{Veronica Fantini}

\begin{document}
\maketitle

\section{Generalized Airy}

Let $p\geq 0$,  

\begin{equation}
I(z)\defeq\int_{e^{-i\tfrac{\pi}{3}}\infty}^{e^{i\tfrac{\pi}{3}}\infty}e^{-z(\frac{t^3}{3}-t)}\frac{dt}{t^p}
\end{equation}

Since $I(z)=z^{(p-1)/3}\mathrm{A}_1(z,p)$ which is a solution of 

\begin{equation}
\left[\partial_z^3-z\partial_z+(p-1)\right]\mathrm{A}_1(z,p)=0
\end{equation}
 it follows that $I(z)$ is a solution of 
\begin{equation}\label{eq:I}
\partial_z^3I-\frac{4}{9}\partial_zI+\frac{2-p}{z}\partial_z^2I-\frac{4(1-p)}{9}\frac{I}{z}-\frac{1+3p-3p^2}{9}\frac{\partial_zI}{z^2}+\frac{3+p-3p^2-p^3}{27}\frac{I}{z^3}=0
\end{equation}
 In particular, the formal integral solution of \eqref{eq:I} is a three parameter family 

\begin{equation}
\tilde{I}(z)=\sum_{\mathbf{k}\in\N^3}U^\mathbf{k}e^{-\tfrac{2}{3}(k_2-k_3)z}z^{-\tfrac{1}{2}(k_2+k_3)-(1-p)k_1}\tilde{w}_{\mathbf{k}}(z)
\end{equation} 

where $\tilde{w}_\mathbf{k}(z)\in\C[\![z^{-1]\!]}$ is a formal solution of 
\begin{equation}
\left[P_3(\partial,\mathbf{k})+\frac{1}{t}P_2(\partial,\mathbf{k})+\frac{1}{t^2}P_1(\partial,\mathbf{k})+\frac{1}{t^3}P_0(\partial,\mathbf{k})\right]\tilde{w}_{\mathbf{k}}(z)=0
\end{equation}

\begin{align*}
P_0(\partial,\mathbf{k})&=\frac{1}{9}+\frac{1}{9}k_1-k_1^2-k_1^3+\frac{1}{18}k_2 -k_1k_2-\frac{3}{2}k_1^2k_2-\frac{1}{4}k_2^2-\frac{3}{4}k_1k_2^2-\frac{1}{8}k_2^3+\frac{p^3}{3}k_1+\\
&\,\,+\frac{1}{18}k_3-k_1k_3-\frac{3}{2}k_1^2 k_3-\frac{1}{2}k_2k_3-\frac{3}{2}k_1k_2k_3-\frac{3}{8}k_2^2k_3-\frac{1}{4}k_3^2-k_1^2 p^3+k_1^3 p^3+\\
&\,\,-\frac{3}{4}k_1k_3^2-\frac{3}{8}k_2k_3^2-\frac{1}{8}k_3^3+\frac{p}{27}-\frac{7}{9}k_1 p+k_1^2 p+3k_1^3 p-\frac{p}{3}k_2+3 k_1^2k_2 p-\frac{p}{4}k_2^2+\\
&\,\,\frac{3}{4}k_1k_2^2 p-\frac{p}{3}k_3+3 k_1^2k_3 p-\frac{p}{2}k_2k_3+\frac{3}{2}k_1k_2k_3p-\frac{p}{4}k_3^2+\frac{3}{4}k_1k_3^2 p-\frac{p^2}{9}+\frac{p^2}{3}k_1+\\
&\,\,+k_1^2 p^2 -3k_1^3 p^2-\frac{p^2}{6}k_2+k_1k_2 p^2-\frac{3}{2}k_1^2k_2 p^2-\frac{p^2}{6}k_3+k_1k_3 p^2-\frac{3}{2}k_1^2k_3 p^2-\frac{p^3}{27}\\
P_1(\partial,\mathbf{k})&=(-\frac{1}{9}-k_1+3k_1^2-\frac{1}{2}k_2+3k_1 k_2+\frac{3}{4}k_2^2-\frac{1}{2}k_3+3k_1k_3+\frac{3}{2}k_2k_3+\frac{3}{4}k_3^2-\frac{p}{3}+\\
&\,\,+3 k_1p-6k_1^2 p+k_2 p -3 k_1k_2 p+k_3 p-3 k_1k_3 p+\frac{p^2}{3}-2k_1 p^2 +3k_1^2p^2 )\partial_z +\\
&\,\,+\frac{2}{27}k_2 +\frac{2}{3}k_1k_2-2k_1^2k_2+\frac{1}{3}k_2^2-2 k_1k_2^2-\frac{1}{2}k_2^3-\frac{2}{27}k_3-\frac{2}{3}k_1k_3+\\
&\,\,+2k_1^2k_3-\frac{1}{2}k_2^2k_3-\frac{1}{3}k_3^2+2k_1k_3^2+\frac{1}{2}k_2k_3^2+\frac{1}{2}k_3^3+\frac{2}{9}k_2 p-2k_1k_2 p\\
&\,\,+4 k_1^2k_2 p-\frac{2}{3}k_2^2 p+2k_1k_2^2 p-\frac{2}{9}k_3 p+2 k_1k_3 p-4k_1^2k_3 p+\frac{2}{3}k_3^2 p-2k_1k_3^2 p-\frac{2}{9}k_2 p^2+\\
&\,\,+\frac{4}{3}k_1k_2p^2-2k_1^2k_2p^2+\frac{2}{9}k_3p^2-\frac{4}{3}k_1k_3p^2+2k_1^2k_3p^2\\
P_2(\partial,\mathbf{k})&=(2-3k_1-\frac{3}{2}(k_2+k_3)-p+3k_1p)\partial^2_z +\\
&\,\, (-\frac{8}{3} k_2+ 4k_1k_2 +2k_2^2 +\frac{8}{3}k_3-4k_1k_3-2k_3^2+\frac{4}{3}k_2 p-4k_1k_2 p-\frac{4}{3}k_3 p+4k_1k_3 p) \partial_z +\\
&\,\,-\frac{4}{9}+\frac{4}{9} k_1+\frac{2}{9} k2 +\frac{8}{9}k_2^2 -\frac{4}{3}k_1 k_2^2-\frac{2}{3}k_2^3+\frac{2}{9} k_3-\frac{16}{9} k_2 k_3+\frac{8}{3}k_1k_2 k_3+\frac{2}{3}k_2^2+\frac{8}{9}k_3^2-\frac{4}{3}k_1k_3^2+\\
&\,\,+\frac{2}{3}k_2k_3^2-\frac{2}{3}k_3^3+\frac{4}{9}p-\frac{4}{9}k_1p-\frac{4}{9}k_2^2 p+\frac{4}{3} k_1 k_2^2 p +\frac{8}{9}k_2k_3 p-\frac{8}{3} k_1 k_2 k_3p-\frac{4}{9}k_3^2 p+\frac{4}{3}k_1 k_3^2 p \\
P_3(\partial,\mathbf{k})&=\partial_z^3-2(k_2-k_3)\partial_z^2+\frac{4}{3}(k_2-k_3)^2\partial_z-\frac{4}{9}\partial_z+\frac{8}{27}(k_2-k_3)(1-k_2+k_3)(1+k_2-k_3)
\end{align*}



Since $\tilde{w}_{\mathbf{k}}(z)=\sum_{j\geq 0}a_{\mathbf{k},j}z^{-j}$ there only  few values of $\mathbf{k}\in\N^3$ such that $\tilde{w}_{\mathbf{k}}\neq 0$, hence the formal integral solution of \eqref{eq:I} reduces to 

\begin{equation}
\tilde{I}(z)=U_1z^{p-1}\tilde{w}_1(z)+U_2e^{-\tfrac{2}{3}z}z^{-1/2}\tilde{w}_2(z)+U_3e^{\tfrac{2}{3}z}\tilde{w}_3(z)
\end{equation}

where $\tilde{w}_1, \tilde{w}_2$ and $\tilde{w}_3$ are the unique (we fix $a_{k,0}=1$ for $k=1,2,3$) solutions of 

\begin{multline}\label{w1}
\left[\partial_z^3-\frac{4}{9}\partial_z-\frac{1-2p}{z}\partial_z^2+\frac{17-30p+12p^2}{9z^2}\partial_z+\frac{8}{27}\frac{p^3-6p^2+11p-6}{z^3}\right]\tilde{w}_1(z)=0
\end{multline}


\begin{multline}\label{w2}
\left[\partial_z^3-2\partial_z^2+\frac{8}{9}\partial_z+\frac{1-2p}{2z}\partial_z^2-\frac{2-4p}{3z}\partial_z+\frac{5+24p+12p^2}{36z^2}\partial_z+\right.\\
\left.-\frac{5+24p+12p^2}{54z^2}-\frac{1}{z^3}\left(\frac{5}{24}+\frac{59}{108}p+\frac{5}{18}p^2+\frac{1}{27}p^3\right)\right]\tilde{w}_2(z)=0
\end{multline}

\begin{multline}\label{w3}
\left[\partial_z^3+2\partial_z^2+\frac{8}{9}\partial_z+\frac{1-2p}{2z}\partial_z^2+\frac{2-4p}{3z}\partial_z+\frac{5+24p+12p^2}{36z^2}\partial_z+\right.\\
\left.+\frac{5+24p+12p^2}{54z^2}-\frac{1}{z^3}\left(\frac{5}{24}+\frac{59}{108}p+\frac{5}{18}p^2+\frac{1}{27}p^3\right)\right]\tilde{w}_3(z)=0
\end{multline}

We first study the equation \eqref{w1}: its unique formal solution $\tilde{w}_1(z)$ has the following coefficients (we compute them with Mathematica)

\begin{equation}
a_{1,2j}=\frac{\Gamma\left(1+3j-p\right)}{3^j\Gamma(1+j)\Gamma(1-p)} \,\qquad j\geq 0
\end{equation}

Notice that $a_{1,j}$ are well defined only for $p\in\C\setminus\lbrace 1,2,3,...\rbrace$. However, for $p=1,2,3$ the solution is well defined and constant $\tilde{w}_1(z)=1$. Therefore the Borel transform of $\tilde{w}_1(z)$ is 
\begin{equation}
\hat{w}_1(\zeta)=\begin{cases}
\delta & \text{if } p=1,2,3 \\
\delta+\zeta\frac{\Gamma(4-p)}{3\Gamma(1-p)}{}_3F_2\left(\frac{4-p}{3},\frac{5-p}{3},2-\frac{p}{3};\frac{3}{2},2;\frac{9}{4}\zeta^2\right) & \text{if } p\in\C\setminus\lbrace 1,2,3,...\rbrace
\end{cases}
\end{equation}

In particular, the generalized hypergeometric series has a branch cut singularity at $\zeta=\pm\tfrac{3}{2}$. We expect that $\tilde{w}_1(z)$ is a simple resurgent fuction such that 
\begin{align*}
\hat{w}_1(\zeta+\tfrac{2}{3})&=\delta+\frac{C}{2\pi i\zeta}+\frac{1}{2\pi i}\log(\zeta)\hat{w}_2(\zeta)+\text{hol. fct.}\\
\hat{w}_1(\zeta-\tfrac{2}{3})&=\delta+\frac{C}{2\pi i\zeta}+\frac{1}{2\pi i}\log(\zeta)\hat{w}_3(\zeta)+\text{hol. fct.}
\end{align*}

Indeed, in [\textbf{D.B. Karp and E.G. Prilepkina} formula 3.1 (see also \textsf{https://arxiv.org/pdf/2110.12219.pdf} equation 27)] the authors compute the analytic continuation of generalized hypergeomtric functions across the branch cut
\begin{equation}
{}_qF_{q-1}\left(\mathbf{a};\mathbf{b};x+i0\right)-{}_qF_{q-1}\left(\mathbf{a};\mathbf{b};x-i0\right)=2\pi i\frac{\Gamma(\mathbf{b}-d+1)}{\Gamma(\mathbf{a}-d+1)}x^{d-1}G_{q,q}^{q,0}\left(d,\mathbf{b};\mathbf{a};\frac{1}{x}\right)
\end{equation}

which specifies to 

\begin{align*}
\frac{3}{2}\zeta&\left({}_3F_2\left(\frac{4-p}{3},\frac{5-p}{3},2-\frac{p}{3};\frac{3}{2},2;\frac{9}{4}\zeta^2+i0\right)-{}_3F_2\left(\frac{4-p}{3},\frac{5-p}{3},2-\frac{p}{3};\frac{3}{2},2;\frac{9}{4}\zeta^2-i0\right)\right)=\\
&=2\pi i\frac{\Gamma(\mathbf{b}+\tfrac{1}{2})}{\Gamma(\mathbf{a}+\tfrac{1}{2})}G_{3,3}^{3,0}\left(\frac{1}{2},\frac{3}{2},2;\frac{4-p}{3},\frac{5-p}{3},2-\frac{p}{3};\frac{4}{9}\zeta^{-2}\right)\\
&=2\pi i\frac{\Gamma(\mathbf{b}+\tfrac{1}{2})}{\Gamma(\mathbf{a}+\tfrac{1}{2})}G_{3,3}^{0,3}\left(\frac{p-1}{3},\frac{p-2}{3},\frac{p}{3}-1;\frac{1}{2},-\frac{1}{2},-1;\frac{9}{4}\zeta^{2}\right)
\end{align*}

When $p=1,2,3$ the solution is a trivial resurgent function, it is constant.


Let us now consider the formal solution $\tilde{w}_2(z)$ and $\tilde{w}_3(z)$. The recursive relation for $a_{2,j},a_{3,j}$ have not a simple expressions as it is for $a_{1,j}$. However, we compute numerically their first coefficients:

\begin{align*}
 &j=1 & 2 & 3& 4& 5& 6& 7 \\
a_{2,j}& & & & & & & \\
a_{3,j}& & & & & & & 
\end{align*}    

\bibliographystyle{utphys}
\bibliography{airy-resurgence}
\end{document}