\documentclass{article}

\usepackage{url}
\usepackage[hmargin=1.5in]{geometry}
\usepackage{amsmath}
\usepackage{amssymb}
%%\usepackage{amsthm}
\usepackage{graphicx}

%%\theoremstyle{definition}
%%\newtheorem{defn}{Definition}
%%\theoremstyle{plain}
%%\newtheorem{prop}{Proposition}

% convenience aliases
\newcommand{\maps}{\colon}

% symbology
\newcommand{\Z}{\mathbb{Z}}
\newcommand{\R}{\mathbb{R}}
\newcommand{\C}{\mathbb{C}}
\newcommand{\laplace}{\mathcal{L}}
\DeclareMathOperator{\Ai}{Ai}

\title{Resurgence of the Airy function \\ and other exponential integrals}
\author{Veronica Fantini and Aaron Fenyes}

\begin{document}
\maketitle
\section{Introduction}
\subsection{Why does Borel resummation work?}
\begin{itemize}
\item Each resummation method for asymptotic series makes some implicit assumption that allows us to reconstruct a holomorphic function from its asymptotic behavior.
\item The resummation method works correctly for functions which satisfy that assumption.
\item For the modified Bessel function $K_{1/3}$, Borel resummation works because the asymptotic series encodes a second-order differential equation.
\begin{itemize}
\item Different aspects of this example appear in various places (Mari\~{n}o, Kawai--Takei, Sauzin). We give a detailed, unified treatment.
\end{itemize}
\item We can generalize this argument to all $K_{1/n}$ and their limit $K_0$.
\item We can also generalize to all third-order exponential integrals.
\begin{itemize}
\item Most of them are equivalent to the $K_{1/3}$ integral, but there's also an interesting degeneration.
\end{itemize}
\end{itemize}
\subsection{Fractional derivative formula}
\begin{itemize}
\item Theorem~\ref{thm:three-halves} says that for a certain class of exponential integrals
\[ I(z) = \int_\Gamma e^{-zf}\;\nu, \]
the inverse Laplace \textbf{[better to say Borel?]} transform is the $\tfrac{3}{2}$ derivative of $d\zeta/df$, where $f^* d\zeta = \nu$ \textbf{[check]}.
\item the asymptotic expansion of $I(z)$ is a resurgent function.
\item Is it always a \emph{simple} resurgent function?
\begin{itemize}
\item \textbf{Maxim belives it is in general, and indeed in our examples we get simple resurgent functions. But how to prove it in general?}
\end{itemize} 
\end{itemize}
\subsection{Stokes phenomenon}
\begin{itemize}
\item For Bessel functions, we can see explicitly how solutions jump when the Laplace transform angle crosses a critical value.
\item The jump comes from the branch cut difference identity for hypergeometric functions.
\item Possible interpretation of the Stokes factors as intersections numbers in Morse--Novikov theory \textbf{[ask Maxim]}
\end{itemize}
\section{The Laplace and Borel transforms}
\subsection{The Laplace transform}
\begin{itemize}
\item Action on differential equations.
\begin{itemize}
\item Can we find a way to prove this when the differential operator spits out a function that's not integrable around zero?
\end{itemize}
\item Global picture?
\end{itemize}
\subsection{The Borel transform}
\begin{itemize}
\item Action on differential equations.
\begin{itemize}
\item No inhomogeneous terms! How is this consistent with the Laplace transform's action? Is there always an inhomogeneous solution with subexponential asymptotics?
\end{itemize}
\end{itemize}
\section{Third-order exponential integrals}
\begin{itemize}
\item Reduce to
\[ I(z) = \int \exp\left[-z\big(u^3 + pu + q)\right]\,du \]
using change of coordinate.
\item When $p \neq 0$, can reduce further to
\[ I(z) = p^{1/2} e^{-qz} K_{1/3}(p^{3/2} z). \]
\item As $p$ goes to zero, $I(z)$ degenerates to
\[ \left(\tfrac{1}{2}\right)^{2/3} e^{-qz} \Gamma(\tfrac{1}{3}) z^{-1/3} = \left(\tfrac{1}{2}\right)^{2/3} e^{-qz} \laplace_{\zeta,0}(\zeta^{-2/3}) = \left(\tfrac{1}{2}\right)^{2/3} \laplace_{\zeta_{-q},q}(\zeta^{-2/3}). \]
\end{itemize}
\end{document}