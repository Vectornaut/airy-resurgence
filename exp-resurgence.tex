\documentclass{article}

\usepackage{url}
\usepackage[hmargin=1.5in]{geometry}
\usepackage{amsmath}
\usepackage{amssymb}
\usepackage{bm} %% for putting series names in bold
%%\usepackage{amsthm}
%%\usepackage{graphicx}

%% colors
\usepackage[dvipsnames]{xcolor}

%%\theoremstyle{definition}
%%\newtheorem{defn}{Definition}
%%\theoremstyle{plain}
%%\newtheorem{prop}{Proposition}

% convenience aliases
\newcommand{\maps}{\colon}

% symbology
\newcommand{\Z}{\mathbb{Z}}
\newcommand{\R}{\mathbb{R}}
\newcommand{\C}{\mathbb{C}}
%%\let\Re\relax
%%\DeclareMathOperator{\Re}{Re}
\DeclareMathOperator{\Ai}{Ai}
\newcommand{\laplace}{\mathcal{L}}
\newcommand{\borel}{\mathcal{B}}
\newcommand{\deriv}[3]{\partial^{#1}_{#2 \text{ from } #3}}
\newcommand{\series}[1]{#1_\bullet}

\title{Resurgence of the Airy function \\ and other exponential integrals}
\author{Veronica Fantini and Aaron Fenyes}

\begin{document}
\maketitle
\section{Introduction}
\subsection{Why does Borel resummation work?}
Borel resummation is a way of turning a formal power series
\[ \series{\varphi} = z^\sigma \left( \frac{\varphi_0}{z} + \frac{\varphi_1}{z^2} + \frac{\varphi_2}{z^3} + \frac{\varphi_3}{z^4} + \ldots \right), \]
with $\sigma \in [0, 1)$, into a function which is asymptotic to $\series{\varphi}$ as $z \to \infty$. Different functions can be asymptotic to the same power series, and Borel resummation picks one of them, performing an implicit regularization~\textbf{[arXiv:1705.03071, or maybe arXiv:1412.6614]}. When a function matches the Borel sum of its asymptotic series, we'll say it's {\em Borel regular}. Several familiar kinds of regularity imply Borel regularity, and shed light on why it occurs.
%%Knowing that a function is Borel regular gives us extra information about it---enough to reconstruct it from its asymptotic series. What's the nature of this extra information?
%%Since different functions can be asymptotic to the same power series, Borel resummation must involve an {\em implicit regularization}, restricting its range to a class of functions which are uniquely determined by their formal power series.
\begin{itemize}
\item \textbf{Having a good asymptotic approximation}

Let $R_N$ be the difference between a function and the partial sum
\[ \frac{\varphi_0}{z} + \frac{\varphi_1}{z^2} + \frac{\varphi_2}{z^3} + \ldots + \frac{\varphi_{N-2}}{z^{N-1}} \]
of its asymptotic series. Watson showed a century ago that the function is Borel regular whenever there's a constant $c \in (0, \infty)$ with
\[ |R_N| \le \frac{c^{N+1} N!}{|z|^N} \]
over all orders $N$ and all $z$ in a wide enough wedge around infinity.
\item \textbf{Satisfying a singular differential equation}

\begin{itemize}
\item Think about conditions where this works.
\item Maybe the correct place is the setting of Ecalle's formal integral. See \S 5.2.2.1 of Delabaere's {\em Divergent Series, Summability and Resurgence III}.
\item Say there's a unique solution (up to scaling) that shrinks as you go right; everything else blows up exponentially. Then this is the only solution that can be expressed as a Laplace transform.
\item If the Borel-transformed equation has a subexponential solution $\hat{f}$ which is ``shifted holomorphic'' (we called this having a ``fractional power singularity'' in {\tt airy-resurgence}), then $\laplace \hat{f}$ satisfies the original equation, because there are no boundary terms.
\item Draw diagram showing formal vs. holomorphic solutions in time vs. frequency domains.
\end{itemize}
\item \textbf{Being a thimble integral}

Let $X$ be a translation surface---a Riemann surface carrying a holomorphic 1-form $\nu$. Suppose $X$ is of {\em meromorphic type}, meaning that we got it by puncturing a compact Riemann surface $\overline{X}$ at finitely many points, and $\nu$ has a pole at each puncture. A {\em translation coordinate} on $X$ is a local coordinate whose derivative is $\nu$.

Take another meromorphic-type translation surface $B$ and a holomorphic Morse\footnote{This condition means that the critical points of $f$ are isolated (the compactness of $\overline{X}$ guarantees this) and the 2-jet of $f$ is non-zero at every critical point.} map $f \maps \overline{X} \to \overline{B}$ that sends punctures to punctures. Suppose every singularity of $B$ is a critical value of $f$. \textbf{[Typical usage of ``Borel plane'' seems ambiguous, so maybe we can use ``Borel plane'' for $B$ and ``Borel cover'' for the Riemann surface of the Borel-transformed series. How to handle the Orr–Sommerfeld functions (DLMF~\S 9.13)? We know $f = 4u^3 - 3u$ is the pullback of a translation coordinate, but we also need a puncture at $f(0)$\ldots]} For each critical point $p$, let $\Gamma_p$ be the ray going rightward from $f(p)$, and let $\zeta_p$ be the translation coordinate around $\Gamma_p$ which vanishes at $f(p)$. These are well-defined as long as $\Gamma_p$ misses the critical values of $f$. The preimage $f^{-1}(\Gamma_p)$ is a bunch of disjoint curves, as long as $\Gamma_p$ misses the other critical values of $f$. The {\em Lefschetz thimble} $\Lambda_p$ is the component of $f^{-1}(\Gamma_p)$ that goes through $p$, oriented so that shifting it to its left would make its projection run clockwise around $\Gamma_p$. The {\em thimble integral}
\[ I_p = \int_{\Lambda_p} e^{-z f^*\zeta_p} \nu \]
is a holomorphic function on the right half-plane parametrized by $z$, and it turns out \textbf{[we hope]} to be Borel regular.

\textbf{[Talk about exponential integrals and their decomposition into thimble integrals.]}

In higher-dimensional complex manifolds, integrals over Lefschetz thimbles are still Borel regular~\textbf{[``Exponential integrals, Lefschetz thimbles and linear resurgence''][``Exponential Integral'' lectures?]}. This fact plays an important technical role in quantum mechanics, where infinite-dimensional exponential integrals are supposed to give the expectation values of observable quantities. Physicists often use Borel summation and related techniques to assign values to these integrals \textbf{[Costin \& Kruskal, ``On optimal truncation...'']}.

\color{Turquoise}
Choose a path $\gamma \maps \R \to X$ whose projection $f \circ \gamma$ starts out going leftward out of a puncture, ends up going rightward into a puncture, and never touches a critical value of $f$. Choose a translation coordinate $\zeta$ on $B$ and continue it along $f \circ \gamma$, noting that it may become multi-valued if $f \circ \gamma$ intersects itself. This data defines the {\em exponential integral}
\[ I = \int_\gamma e^{-z f^*\zeta} \nu, \]
a holomorphic function on the right half-plane parametrized by $z$. It turns out \textbf{[we hope]} that we can get $I$ by summing $e^{-\alpha_p z} I_p$ over various critical points---as long as none of the $\Gamma_p$ run into each other. \textbf{[We get jumps at phases where the $\Gamma_p$ do hit each other.]} The constants $\alpha_p$ are values of $\zeta$, continued to the critical points along certain paths.
\color{black}
\end{itemize}
\begin{itemize}
\item Each resummation method for asymptotic series makes some implicit assumption that allows us to reconstruct a holomorphic function from its asymptotic behaviour.
\item The resummation method works correctly for functions which satisfy that assumption.
\item For the modified Bessel function $K_{1/3}$, Borel resummation works because the asymptotic series encodes a second-order differential equation.
\begin{itemize}
\item Different aspects of this example appear in various places (Mari\~{n}o, Kawai--Takei, Sauzin). We give a detailed, unified treatment.
\end{itemize}
\item We can generalize this argument to all $K_{1/n}$ and their limit $K_0$.
\item We can also generalize to all third-order exponential integrals.
\begin{itemize}
\item Most of them are equivalent to the $K_{1/3}$ integral, but there's also an interesting degeneration.
\end{itemize}
\end{itemize}
\subsection{Fractional derivative formula}
\begin{itemize}
\item Theorem~\ref{thm:three-halves} says that for a certain class of exponential integrals
\[ I(z) = \int_\Gamma e^{-zf}\;\nu, \]
the inverse Laplace \textbf{[better to say Borel?]} transform is the $\tfrac{3}{2}$ derivative of $d\zeta/df$, where $f^* d\zeta = \nu$ \textbf{[check]}.
\item the asymptotic expansion of $I(z)$ is a resurgent function.
\item Is it always a \emph{simple} resurgent function?
\begin{itemize}
\item \textbf{Maxim belies it is in general, and indeed in our examples we get simple resurgent functions. But how to prove it in general?}
\end{itemize} 
\end{itemize}
\subsection{Stokes phenomenon}
\begin{itemize}
\item For Bessel functions, we can see explicitly how solutions jump when the Laplace transform angle crosses a critical value.
\item The jump comes from the branch cut difference identity for hypergeometric functions.
\item Possible interpretation of the Stokes factors as intersections numbers in Morse--Novikov theory \textbf{[ask Maxim]}
\end{itemize}
\section{The Laplace and Borel transforms}
\subsection{The Laplace transform}
\begin{itemize}
\item Action on differential equations.
\begin{itemize}
\item Can we find a way to prove this when the differential operator spits out a function that's not integrable around zero?
\end{itemize}
\item Global picture?
\end{itemize}
\subsection{The Borel transform}
\begin{itemize}
\item Action on differential equations.
\begin{itemize}
\item No inhomogeneous terms! How is this consistent with the Laplace transform's action? Is there always an inhomogeneous solution with subexponential asymptotics?
\end{itemize}
\end{itemize}
\section{Third-order exponential integrals}
\begin{itemize}
\item Reduce to
\[ I(z) = \int \exp\left[-z\big(u^3 + pu + q)\right]\,du \]
using change of coordinate.
\item When $p \neq 0$, can reduce further to
\[ I(z) = p^{1/2} e^{-qz} K_{1/3}(p^{3/2} z). \]
\item As $p$ goes to zero, $I(z)$ degenerates to
\[ \left(\tfrac{1}{2}\right)^{2/3} e^{-qz} \Gamma(\tfrac{1}{3}) z^{-1/3} = \left(\tfrac{1}{2}\right)^{2/3} e^{-qz} \laplace_{\zeta,0}(\zeta^{-2/3}) = \left(\tfrac{1}{2}\right)^{2/3} \laplace_{\zeta_{-q},q}(\zeta^{-2/3}). \]
\end{itemize}


\color{orange}
\textbf{Veronica's proposal}

\textbf{Title: Borel regularity and Resurgence of Exponential Integrals}

\begin{enumerate}
\item what are exponential integrals? Geometrically they represent a Poincar\'e pairing (as explained by Kontsevich in \textbf{IHES lectures}). In physics ?? \textcolor{gray}{has to be done}
\item what does it mean being Borel regular? and when does it happen? \textcolor{gray}{has to be done}
\begin{itemize}
\item Watson condition
\item linear, homogeneous ODE with regular singularity at 0 and irregular singularity at infinity [P.Ramis ?, Loday--Richaud ?, big idea in {\tt airy-resurgence}]
\item our theorem [{\tt draft2}]
\end{itemize}
\item Resurgence of exponential integrals and Stokes phenomena
\begin{itemize}
\item Thimbles integrals [Kontsevich]: geometric computation of Stokes constants \textcolor{gray}{has to be done}
\item ODE and fractional derivative formula [{\tt draft2}]: hypergeometric functions \textcolor{gray}{has to be done}
\end{itemize}
\item The Airy example
\begin{itemize}
\item $I(z)$ is a solution of a linear ODE. We explicitly find its Borel transform, knowing the nature of singularities and the asymptotic behaviour of a basis of solution for the ODE  [{\tt airy-resurgence}]
\item Stokes constants using fractional derivative formula and Borel transform computation [{\tt draft2}]
\item Comparison with the literature (Marino, Sauzin, Kontsevich slides) \textcolor{gray}{has to be done}
\end{itemize}
\item Bessel $1/n$ (follows from Airy) we compute Borel transform and Stokes constants \textcolor{gray}{has to be done}
\item Bessel 0 (it is different because we have infinite cover) we compute Stokes constants [{\tt draft2}]
\item Bessel $\mu$ (follows from Bessel 0) we compute Stokes constants [{\tt modified Bessel}]
\end{enumerate}


\end{document}