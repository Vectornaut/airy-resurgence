\documentclass{article}

\usepackage{url}
\usepackage[hmargin=1.5in]{geometry}
\usepackage{amsmath}
\usepackage{amssymb}
\usepackage{stmaryrd} %% needed for mapsto arrows in commutative diagrams
\usepackage{bm} %% for putting series names in bold
%%\usepackage{amsthm}
%%\usepackage{graphicx}

%% colors
\usepackage[svgnames]{xcolor}

%% editing
\newcommand{\done}[1]{\textcolor{gray}{#1}}

%%\theoremstyle{definition}
%%\newtheorem{defn}{Definition}
%%\theoremstyle{plain}
%%\newtheorem{prop}{Proposition}

% convenience aliases
\newcommand{\maps}{\colon}

% symbology
\newcommand{\Z}{\mathbb{Z}}
\newcommand{\R}{\mathbb{R}}
\newcommand{\C}{\mathbb{C}}
\usepackage{tikz}
\usepackage{tikz-cd}
\usepackage{rotating}
\newcommand*{\isoarrow}[1]{\arrow[#1,"\rotatebox{90}{\(\sim\)}"
]}
\usetikzlibrary{matrix,shapes,arrows,decorations.pathmorphing}
\tikzset{commutative diagrams/arrow style=math font}
\tikzset{commutative diagrams/.cd,
mysymbol/.style={start anchor=center,end anchor=center,draw=none}}
\newcommand\MySymb[2][\square]{%
  \arrow[mysymbol]{#2}[description]{#1}}
\tikzset{
shift up/.style={
to path={([yshift=#1]\tikztostart.east) -- ([yshift=#1]\tikztotarget.west) \tikztonodes}
}
}
\DeclareMathAlphabet{\mathpzc}{OT1}{pzc}{m}{it}

\newcommand*{\defeq}{\mathrel{\vcenter{\baselineskip0.5ex \lineskiplimit0pt
                     \hbox{\scriptsize.}\hbox{\scriptsize.}}}%
                     =}
\newcommand*{\defeqin}{\mathrel{\vcenter{\lineskiplimit0pt\baselineskip0.5ex
                     \hbox{\scriptsize.}\hbox{\scriptsize.}}}%
                     =}                     

%%\let\Re\relax
%%\DeclareMathOperator{\Re}{Re}
\DeclareMathOperator{\Ai}{Ai}
\newcommand{\laplace}{\mathcal{L}}
\newcommand{\borel}{\mathcal{B}}
\newcommand{\deriv}[3]{\partial^{#1}_{#2 \text{ from } #3}}
\newcommand{\series}{\tilde}

\title{Resurgence of the Airy function \\ and other exponential integrals}
\author{Veronica Fantini and Aaron Fenyes}

\begin{document}
\maketitle
\section{Introduction}
\subsection{The unreasonable effectiveness of Borel summation}\label{intro:summation}
%%\subsection{A familiar process: Borel summation}\label{intro:summation}
You can often find a formal power series \textcolor{magenta}{[double-check that this matches the $\tau$ from the position domain]}
\[ \series{\Phi} = \frac{c_0}{z^\tau} + \frac{c_1}{z^{\tau+1}} + \frac{c_2}{z^{\tau+2}} + \frac{c_3}{z^{\tau+3}} + \ldots, \]
with $\tau \in (0, 1]$, that looks or acts like a solution to a problem whose actual solutions are holomorphic functions of $z$. For example, if you want to understand how the solutions of the holomorphic ordinary differential equation (ODE)
\begin{equation}
\text{\bf [one-third Bessel equation, rescaled to match integral example]} \label{eqn:bessel-rescaled}
\end{equation}
behave near $z = \infty$, you might start by looking for formal {\em transmonomial} solutions $e^{-\alpha z}\,\series{\Phi}$, where $\series{\Phi}$ is a formal power series of the kind above. Setting $\alpha = -\tfrac{1}{12}$ and $\tau = \tfrac{1}{2}$ gives a well-behaved recurrence relation for $\series{\Phi}$, which produces the solution \textcolor{magenta}{[check]}
\begin{equation}
e^{z/12} \left[ \frac{(-1)!!}{z^{1/2}} + \frac{5}{6} \cdot \frac{1!!}{z^{3/2}} + \frac{385}{216} \cdot \frac{3!!}{z^{5/2}} + \frac{17017}{3888} \cdot \frac{5!!}{z^{7/2}} + \ldots \right] \label{series:bessel-ex}
\end{equation}
and its constant multiples. As another example, you might rewrite the integral
\color{DodgerBlue}
\[ \Phi(z) = \int_{\Lambda} \exp\left[-\tfrac{1}{12} z \left(4u^3 - 3u\right)\right]\,du \]
\color{black}
\[ \Phi(z) = \int_{\Lambda} \exp\left[-z \left(\tfrac{1}{3} u^3 - \tfrac{1}{4} u\right)\right]\,du \]
as
% see expansion.sage
\[ e^{z/12} \int_{-\infty}^\infty e^{-z\tau^2/2} \left[ 1 - \frac{2}{3} \tau + \frac{5}{6} \tau^2 - \frac{32}{27} \tau^3 + \frac{385}{216} \tau^4 - \frac{224}{81} \tau^5 + \frac{17017}{3888} \tau^6 - \ldots \right] d\tau \]
using the substitution $\tfrac{1}{2} \tau^2 = \tfrac{1}{3} u^3 - \tfrac{1}{4} u + \tfrac{1}{12}$. Na\"{i}vely integrating term by term, you again get the transmonomial~\eqref{series:bessel-ex}.
\color{DodgerBlue}
\begin{align*}
& e^{z/12} z^{-1/2} \left[ (-1)!! + \frac{5}{6}\,1!!\,z^{-1} + \frac{385}{216}\,3!!\,z^{-2} + \ldots \right] \\
& = e^{z/12} z^{1/2} \left[ \frac{(-1)!!}{z} + \frac{5}{6} \cdot \frac{1!!}{z^2} + \frac{385}{216} \cdot \frac{3!!}{z^3} + \frac{17017}{3888} \cdot \frac{5!!}{z^4} + \ldots \right] \\
& = e^{z/12} \left[ \frac{(-1)!!}{z^{1/2}} + \frac{5}{6} \cdot \frac{1!!}{z^{3/2}} + \frac{385}{216} \cdot \frac{3!!}{z^{5/2}} + \frac{17017}{3888} \cdot \frac{5!!}{z^{7/2}} + \ldots \right]
\end{align*}
\color{black}

Once you have the formal solution $\series{\Phi}$, you might try to get an actual solution by applying {\em Borel summation}, which turns a formal power series into a function asymptotic to it. Borel summation works in three steps.
\begin{enumerate}
\item Thinking of $z$ as a ``frequency variable,'' we take the formal inverse Laplace transform of $\series{\Phi}$, producing a formal power series $\series{\phi}$ in a new ``position variable'' $\zeta$.
\item With luck, $\series{\phi}$ has a positive radius of convergence. In this case, we say $\series{\Phi}$ is {\em $1$-Gevrey}. We sum $\series{\phi}$ to get a holomorphic function $\hat{\phi}$ on a neighborhood of $\zeta = 0$. Then, by analytic continuation, we expand the domain of $\hat{\phi}$ to a Riemann surface $B$ with a distinguished 1-form $\lambda$---the continuation of $d\zeta$. \textcolor{orange}{[Nikita has a complementary picture where the Borel plane is the cotangent fiber? Ask more about this.]}
\item With more luck, $\hat{\phi}$ grows slowly enough along an infinite ray $b + e^{i\theta}[0, \infty)$ \textcolor{magenta}{[change, explain, or link to notation]} for its Laplace transform $\laplace_b^\theta \hat{\phi}$ \textcolor{magenta}{[link to definition]} to be a holomorphic function of $z$, well-defined on some sector of the frequency plane. In this case, we say $\tilde{\Phi}$ is {\em Borel-summable}, and we call $\laplace_b^\theta \hat{\phi}$ its {\em Borel sum} at $b$.
\end{enumerate}
The Borel summation process is summarized in the following diagram.
\begin{center}
\begin{tikzcd}
& \text{problem} & \\
\hat{\Phi} \arrow[ru, dotted, no head, tail] & & \series{\Phi} \arrow[lu, no head, tail, "\parbox{15mm}{\centering\footnotesize formally solves}"'] \arrow[dd, mapsto, "\mathcal{B}"] \\
& & \\
\hat{\phi} \arrow[uu, mapsto, "\laplace_b^\theta"] & & \tilde{\phi}(\zeta) \arrow[ll, mapsto, "\text{sum}"] 
\end{tikzcd}
\end{center}

\begin{itemize}
\item \done{You can often find a formal power series $\tilde{\Phi} = \ldots$ that looks or acts like a solution to a problem whose actual solutions are holomorphic functions of $z$. For example\ldots}
\begin{itemize}
\item \done{Exponential integral: na\"{i}ve saddle point expansion.}
\item \done{ODE: formal solution.} Existence theorem in \cite{diverg-resurg-iii}?
\item Feynman diagram series?
\end{itemize}
\item Once you have $\tilde{\Phi}$, you might try applying {\em Borel summation}, which turns a formal power series into a function asymptotic to it.
\begin{itemize}
\item \done{Borel summation work in three steps. First we turn the formal power series $\tilde{\Phi}$ \textcolor{magenta}{[$\tilde{\Phi}(z)$]} in the ``frequency variable'' $z$ into a formal power series $\tilde{\phi}$ in a new ``position variable'' $\zeta$.}
\item \done{The series $\tilde{\phi}$ turns out to have a \textcolor{magenta}{[finite]} positive radius of convergence, so it defines a holomorphic function $\hat{\phi}$. By analytic continuation, we can expand the domain of $\hat{\phi}$ to a Riemann surface $B$ with a distinguished 1-form $\lambda$---the continuation of $d\zeta$.}
\item \done{If $\hat{\phi}$ grows slowly enough along an infinite ray $\Gamma_b^\theta := b + e^{i\theta}[0, \infty)$ \textcolor{magenta}{[decide notation]}, its Laplace transform $\laplace_b^\theta \hat{\phi} := \ldots$ is a holomorphic function of $z$, well-defined on some sector of the frequency plane. In this case, we say $\tilde{\Phi}$ is {\em Borel-summable}, and we call $\laplace_b^\theta \hat{\phi}$ its {\em Borel sum} at $b$.}
\item {[Draw the square!]}
\end{itemize}
\item Different functions can be asymptotic to the same power series, and Borel summation picks one of them.
\item In many cases, it picks correctly, producing an actual solution to your problem.
\begin{itemize}
\item The question of how that happens is the starting point for this paper.
\item (In both of the cases that we study, the Borel sum of $\tilde{\Phi}$ is always taken at a zero of $\lambda$, rather than an arbitrary point in $B$. For ODEs, our treatment explains why the zeroes of $\lambda$ play a special role.)
\end{itemize}
\end{itemize}
\subsection{A new perspective: Borel regularity}
\begin{itemize}
\item The central goal of this paper is to present a new perspective on Borel summation which helps explain why it works for at least two kinds of problems.
\begin{itemize}
\item The first problem is evaluating a certain kind of exponential integral: a one-dimensional {\em thimble integral}.
\item The second problem is solving a certain kind of ODE.
\item These two problems are closely linked. By playing with derivatives of an exponential integral, you can often find a linear ODE that the integral satisfies. Conversely, for many classical ODEs, there are useful bases of exponential integral solutions.
\end{itemize}
\item Suppose you have a holomorphic function $\Phi$ which is asymptotic to a formal power series $\series{\Phi}$ as $z$ goes to $\infty$ along a given ray.
\item If $\series{\Phi}$ is Borel-summable, as described in Section~\ref{intro:summation}, its Borel sum is a new holomorphic function.
%%\item {\em A priori}, we should expect $\hat{\Phi}$ to be different from $\Phi$, because we lose information in passing from $\Phi$ to its asymptotic series.
%%\item Taking the Borel sum of the asymptotic series of $\Phi$ is a regularization process, because it smooths away some perturbations---the ones that are asymptotic to zero along the given ray. We'll call it {\em Borel regularization}.
%%\item Taking the Borel sum of the asymptotic series of $\Phi$ can be seen as a regularization process, because taking asymptotics erases information. We'll call this process {\em Borel regularization}.
\item Since different functions can be asymptotic to the same power series, taking the Borel sum of the asymptotic series of $\Phi$ must smooth away some details. We'll therefore call this process {\em Borel regularization}.
\item We'll say $\Phi$ is {\em Borel-regularizable} if it's asymptotic to a Borel-summable power series.
\item We'll say $\Phi$ is {\em Borel regular} if it's Borel-regularizable and Borel regularization leaves it unchanged.
\item We have the following vector space inclusions:
\[ \text{Borel regular functions} \subset \text{Borel-regularizable functions} \subset \text{holomorphic functions}. \]
\item Nevanlinna's improvement of Watson's theorem (Sokal, ``An improvement of Watson's theorem on Borel summability'') tells us that for a function $\Phi$ with a well-defined inverse Laplace transform $\phi$, the following conditions are equivalent \textcolor{magenta}{[double check]}:
\begin{itemize}
\item $\Phi$ is Borel regular.
\item $\Phi$ is approximated well, in a certain sense (a weak version of being ``Gevrey-asymptotic''), by its asymptotic series.
\item $\phi$ grows at most exponentially along the Laplace transform ray.
\end{itemize}
\item When its domain is restricted to the space of Gevrey power series, Borel summation is invertible. \textcolor{magenta}{[Need to find a good reference for this, because a small change in the statement would mess up our conclusion. Does \texttt{arXiv:2112.08792}, \S A.2 have the statement we want?]} Thus, Borel regularization acts as a projection operator on the space of Borel-regularizable functions.
\item Borel regularity can help explain why Borel summation is so effective at solving certain kinds of problems, like the ones in Section~\ref{intro:summation}.
\item The formal solutions of a problem are typically supposed to generalize the actual solutions, in the sense that they include the asymptotic series of the actual solutions.
\item In this case, every Borel regular solution can be found by taking the Borel sum of a formal solution.
\item \ldots
\end{itemize}
\subsection{Goals}
\begin{itemize}
\item The central goal of this paper is to lay out two kinds of problems where we can prove that the Borel sum of a formal power series solution is always an actual solution.
\begin{itemize}
\item The first problem is evaluating a certain kind of exponential integral: a one-dimensional {\em thimble integral}.
\item The second problem is solving a certain kind of ODE.
\item These two problems are closely linked. By playing with derivatives of an exponential integral, you can often find a linear ODE that the integral satisfies. Conversely, for many classical ODEs, there are useful bases of exponential integral solutions.
\item \textcolor{magenta}{(Does this touch the Picard-Lefshetz perspective? Betti / de Rham relationship: ODE is a connection, and exponential integrals give flat sections?)}
\end{itemize}
\item Clearly separate the parts of the theory that deal with holomorphic functions and formal power series.
\item (Super-motivation: why do the zeroes of $\lambda$ play a special role?) As part of the treatment, we've made use of some new perspectives on the Laplace transform.
\begin{itemize}
\item \textbf{Geometric picture.} The spatial domain $B$ is a translation surface. If $b \in B$ is non-singular, the frequency domain for $\laplace_b^\theta$ is $T^* B_b$. If $b$ is a conical singularity, the frequency domain is more interesting, as we'll see in our main example.
\item \textbf{A new dictionary for ODEs.} The Laplace transform is often used to solve ODEs on the frequency domain by relating them to ODEs on the spatial domain. We find, however, that it's much easier and more natural to relate ODEs on the frequency domain to integral equations on the spatial domain. This clarifies why we take the Borel sums at zeroes of $\lambda$ when we're trying to solve an ODE. \textcolor{orange}{[Veronica---I dont't see why]}
\end{itemize}
\item Our picture helps explain why it's useful to work on the Borel plane (the position domain).
\begin{itemize}
\item Integral equations are more regular than differential equations.
\item A thimble integral in the frequency domain can be recast as the Laplace transform of a function in the position domain.
\end{itemize}
\item Illustrate with detailed treatments of several examples.
\begin{itemize}
\item Some have been discussed many times, using different approaches and conventions. We'll try to give an idea of how all these different treatments fit together. $\bullet$ The Airy function (Marino, Sauzin). $\bullet$ The anharmonic oscillator (Bender--Wu, Schiappa).
\item Others haven't been discussed much.
\end{itemize}
\item \textcolor{magenta}{To understand, for example, the Ecalle formalism, you can start with these toy examples, which illustrate what's going on, but can also be studied in a more elementary way. [How do we work this into the introduction?]} Recently, resurgence theory (first developed by \'Ecalle in the '80) has attracted interests in math and physics as a powerful alternative to Borel summability. Resurgence of lienar ODEs have been studied (see Costin slides for ReNewQuantum). Many results are also known for non linear ODEs (see Schiappa PI, Costin PI). For algebraic exponential integrals of the type we studied in this paper, resurgence of their asymptotic expansion can be understood geometrically (see Maxim's slides ReNewQuantum), however for more general exponential integrals (see examples in Maxim's talk) resurgence remains a conjecture. Despite their simplicity, our examples of linear ODEs and of exponential integrals show some features of resurgence and they are toy model to get a feeling on \'Ecalle formalism.       
\item \textcolor{magenta}{The examples give a place to compare more complicated formalisms like the Picard-Lefshetz (Morse theory) or Ecalle formalisms? [How do we work this into the introduction?]}
\end{itemize}
\subsection{Results}

\color{orange}
\begin{itemize}
\item what does it mean being Borel regular?
\item when does it happen?
\begin{itemize}
\item State new Borel regularity results
\begin{itemize}
\item Linear, homogeneous ODE with regular singularity at 0 and irregular singularity at infinity [big idea in {\tt airy-resurgence}]
\begin{itemize}
\item Contextualize with previous work of Braaksma (``Multisummability and Stokes multipliers of linear meromorphic differential equations'')
\item Also contextualize with Balser, Braaksma, Sibuya, and Ramis (``Multisummability of formal power series solutions of linear ordinary differential equations'')
\item Also contextualize with Loday-Richaud
\end{itemize}
\item \emph{Borel regularity} for \textbf{thimbles integrals} can be stated a the commutativity of the following diagram:
\begin{equation}
\begin{tikzcd}
I_{\alpha}(z)\defeq\int_{\mathcal{C}_\alpha}e^{-zf}\nu \arrow[r,"\sim"]\arrow[dd, swap, "\mathcal{L}^\theta "] & \tilde{I}_{\alpha}(z)\arrow[dd,"\mathcal{B}"]\\
& \\
\hat{\iota}_\alpha(\zeta)\arrow[r,equal,swap, "\text{sum}"] & \tilde{\iota}_{\alpha}(\zeta) 
\end{tikzcd}
\end{equation}
\item A priori, the Laplace transform of $\hat{\iota}_\alpha(\zeta)$ and $I_{\alpha}(z)$ have the same asymptotic behaviour in a given sector (indeed taking the asymptotic of $I_\alpha(z)$ we \textit{loose} information); however Borel regularity guarantees that $I_{\alpha}(z)=\mathcal{L}^{\theta}\hat{\iota}_{\alpha}$ in a given sector.
\item fractional derivative formula
\item Conjecturally, we expect $\hat{\varphi}_\alpha(\zeta)$ to have simple singularities. 
\item in the examples, $\hat{\varphi}_\alpha(\zeta)$ turn out be an hypergeometric function of type ${}_pF_{p-1}$ where $p$ is the number of critical values. 
\item We expect that hypergeometric functions play a special role in resurgence theory as they may always appear when there are only finitely many singularities.
\item \textcolor{magenta}{maybe we can say more about algebraic hypergeomtric functions}
\end{itemize}
\item Recall Watson condition (old): Let $R_N$ be the difference between a function and the partial sum
\[ \frac{\varphi_0}{z} + \frac{\varphi_1}{z^2} + \frac{\varphi_2}{z^3} + \ldots + \frac{\varphi_{N-2}}{z^{N-1}} \]
of its asymptotic series. Watson showed a century ago that the function is Borel regular whenever there's a constant $c \in (0, \infty)$ with
\[ |R_N| \le \frac{c^{N+1} N!}{|z|^N} \]
over all orders $N$ and all $z$ in a wide enough wedge around infinity.
\end{itemize}
\end{itemize}
\color{DarkBlue}
\subsection{Why does Borel resummation work?}
Borel resummation is a way of turning a formal power series
\[ \series{\varphi} = z^\sigma \left( \frac{\varphi_0}{z} + \frac{\varphi_1}{z^2} + \frac{\varphi_2}{z^3} + \frac{\varphi_3}{z^4} + \ldots \right), \]
with $\sigma \in [0, 1)$, into a function which is asymptotic to $\series{\varphi}$ as $z \to \infty$. Different functions can be asymptotic to the same power series, and Borel resummation picks one of them, performing an implicit regularization~\textbf{[arXiv:1705.03071, or maybe arXiv:1412.6614]}. When a function matches the Borel sum of its asymptotic series, we'll say it's {\em Borel regular}. Several familiar kinds of regularity imply Borel regularity, and shed light on why it occurs.
%%Knowing that a function is Borel regular gives us extra information about it---enough to reconstruct it from its asymptotic series. What's the nature of this extra information?
%%Since different functions can be asymptotic to the same power series, Borel resummation must involve an {\em implicit regularization}, restricting its range to a class of functions which are uniquely determined by their formal power series.
\begin{itemize}
\item \textbf{Having a good asymptotic approximation}

Let $R_N$ be the difference between a function and the partial sum
\[ \frac{\varphi_0}{z} + \frac{\varphi_1}{z^2} + \frac{\varphi_2}{z^3} + \ldots + \frac{\varphi_{N-2}}{z^{N-1}} \]
of its asymptotic series. Watson showed a century ago that the function is Borel regular whenever there's a constant $c \in (0, \infty)$ with
\[ |R_N| \le \frac{c^{N+1} N!}{|z|^N} \]
over all orders $N$ and all $z$ in a wide enough wedge around infinity (Sokal, ``An improvement of Watson's theorem on Borel summability''; Hardy, {\em Divergent Series}, Theorem~136; Watson, ``A theory of asymptotic series,'' \S 8?).
\item \textbf{Satisfying a singular differential equation}

\begin{itemize}
\item Think about conditions where this works.
\item Maybe the correct place is the setting of Ecalle's formal integral. See \S 5.2.2.1 of Delabaere's {\em Divergent Series, Summability and Resurgence III}.
\item Say there's a unique solution (up to scaling) that shrinks as you go right; everything else blows up exponentially. Then this is the only solution that can be expressed as a Laplace transform.
\item If the Borel-transformed equation has a subexponential solution $\hat{f}$ which is ``shifted holomorphic'' (we called this having a ``fractional power singularity'' in {\tt airy-resurgence}), then $\laplace \hat{f}$ satisfies the original equation, because there are no boundary terms.
\item Draw diagram showing formal vs. holomorphic solutions in time vs. frequency domains.
\end{itemize}
\item \textbf{Being a thimble integral}

Let $X$ be a translation surface---a Riemann surface carrying a holomorphic 1-form $\nu$. Suppose $X$ is of {\em meromorphic type}, meaning that we got it by puncturing a compact Riemann surface $\overline{X}$ at finitely many points, and $\nu$ has a pole at each puncture. A {\em translation coordinate} on $X$ is a local coordinate whose derivative is $\nu$.

Take another meromorphic-type translation surface $B$ and a holomorphic Morse\footnote{This condition means that the critical points of $f$ are isolated (the compactness of $\overline{X}$ guarantees this) and the 2-jet of $f$ is non-zero at every critical point.} map $f \maps \overline{X} \to \overline{B}$ \textcolor{magenta}{that sends punctures to punctures [actually, don't require this; the Orr--Sommerfeld integrals, for example, don't satisfy it]}. Suppose every singularity of $B$ is a critical value of $f$. \textbf{[Typical usage of ``Borel plane'' seems ambiguous, so maybe we can use ``Borel plane'' for $B$ and ``Borel cover'' for the Riemann surface of the Borel-transformed series. How to handle the Orr–Sommerfeld functions (DLMF~\S 9.13)? We know $f = 4u^3 - 3u$ is the pullback of a translation coordinate, but we also need a puncture at $f(0)$\ldots]} For each critical point $p$, let $\Gamma_p$ be the ray going rightward from $f(p)$, and let $\zeta_p$ be the translation coordinate around $\Gamma_p$ which vanishes at $f(p)$. These are well-defined as long as $\Gamma_p$ misses the critical values of $f$. The preimage $f^{-1}(\Gamma_p)$ is a bunch of disjoint curves, as long as $\Gamma_p$ misses the other critical values of $f$. The {\em Lefschetz thimble} $\Lambda_p$ is the component of $f^{-1}(\Gamma_p)$ that goes through $p$, oriented so that shifting it to its left would make its projection run clockwise around $\Gamma_p$. The {\em thimble integral}
\[ I_p = \int_{\Lambda_p} e^{-z f^*\zeta_p} \nu \]
is a holomorphic function on the right half-plane parametrized by $z$, and it turns out \textbf{[we hope]} to be Borel regular.

\textbf{[Talk about exponential integrals and their decomposition into thimble integrals.]}

In higher-dimensional complex manifolds, integrals over Lefschetz thimbles are still Borel regular~\textbf{[``Exponential integrals, Lefschetz thimbles and linear resurgence''][``Exponential Integral'' lectures?]}. This fact plays an important technical role in quantum mechanics, where infinite-dimensional exponential integrals are supposed to give the expectation values of observable quantities. Physicists often use Borel summation and related techniques to assign values to these integrals \textbf{[Costin \& Kruskal, ``On optimal truncation...'']}.

\color{Turquoise}
Choose a path $\gamma \maps \R \to X$ whose projection $f \circ \gamma$ starts out going leftward out of a puncture, ends up going rightward into a puncture, and never touches a critical value of $f$. Choose a translation coordinate $\zeta$ on $B$ and continue it along $f \circ \gamma$, noting that it may become multi-valued if $f \circ \gamma$ intersects itself. This data defines the {\em exponential integral}
\[ I = \int_\gamma e^{-z f^*\zeta} \nu, \]
a holomorphic function on the right half-plane parametrized by $z$. It turns out \textbf{[we hope]} that we can get $I$ by summing $e^{-\alpha_p z} I_p$ over various critical points---as long as none of the $\Gamma_p$ run into each other. \textbf{[We get jumps at phases where the $\Gamma_p$ do hit each other.]} The constants $\alpha_p$ are values of $\zeta$, continued to the critical points along certain paths.
\color{black}
\end{itemize}
\begin{itemize}
\item Each resummation method for asymptotic series makes some implicit assumption that allows us to reconstruct a holomorphic function from its asymptotic behaviour.
\item The resummation method works correctly for functions which satisfy that assumption.
\item For the modified Bessel function $K_{1/3}$, Borel resummation works because the asymptotic series encodes a second-order differential equation.
\begin{itemize}
\item Different aspects of this example appear in various places (Mari\~{n}o, Kawai--Takei, Sauzin). We give a detailed, unified treatment.
\end{itemize}
\item We can generalize this argument to all $K_\nu$ with $\nu \in \mathbb{Q}$.
\item We can also generalize to all third-order exponential integrals.
\begin{itemize}
\item Most of them are equivalent to the $K_{1/3}$ integral, but there's also an interesting degeneration.
\end{itemize}
\end{itemize}
\color{black}
\subsection{Fractional derivative formula}
\begin{itemize}
\item Theorem~\ref{thm:three-halves} says that for a certain class of exponential integrals
\[ I(z) = \int_\Gamma e^{-zf}\;\nu, \]
the inverse Laplace \textbf{[better to say Borel?]} transform is the $\tfrac{3}{2}$ derivative of $d\zeta/df$, where $f^* d\zeta = \nu$ \textbf{[check]}.
\item the asymptotic expansion of $I(z)$ is a resurgent function.
\item Is it always a \emph{simple} resurgent function?
\begin{itemize}
\item \textbf{Maxim belies it is in general, and indeed in our examples we get simple resurgent functions. But how to prove it in general?}
\end{itemize} 
\end{itemize}
\subsection{Stokes phenomenon}
\begin{itemize}
\item For Bessel functions, we can see explicitly how solutions jump when the Laplace transform angle crosses a critical value.
\item The jump comes from the branch cut difference identity for hypergeometric functions.
\item Possible interpretation of the Stokes factors as intersections numbers in Morse--Novikov theory \textbf{[ask Maxim]}
\end{itemize}
\section{The Laplace and Borel transforms}
\subsection{The Laplace transform}
\begin{itemize}
\item Action on differential equations.
\begin{itemize}
\item Can we find a way to prove this when the differential operator spits out a function that's not integrable around zero?
\end{itemize}
\item Global picture?
\end{itemize}
\subsection{The Borel transform}
\begin{itemize}
\item Action on differential equations.
\begin{itemize}
\item No inhomogeneous terms! How is this consistent with the Laplace transform's action? Is there always an inhomogeneous solution with subexponential asymptotics?
\end{itemize}
\end{itemize}
\section{Third-order exponential integrals}
\begin{itemize}
\item Reduce to
\[ I(z) = \int \exp\left[-z\big(u^3 + pu + q)\right]\,du \]
using change of coordinate.
\item When $p \neq 0$, can reduce further to
\[ I(z) = p^{1/2} e^{-qz} K_{1/3}(p^{3/2} z). \]
\item As $p$ goes to zero, $I(z)$ degenerates to
\[ \left(\tfrac{1}{2}\right)^{2/3} e^{-qz} \Gamma(\tfrac{1}{3}) z^{-1/3} = \left(\tfrac{1}{2}\right)^{2/3} e^{-qz} \laplace_{\zeta,0}(\zeta^{-2/3}) = \left(\tfrac{1}{2}\right)^{2/3} \laplace_{\zeta_{-q},q}(\zeta^{-2/3}). \]
\end{itemize}


\color{orange}
\section*{Outline}

\textbf{Title: Borel regularity and Resurgence of Exponential Integrals}

\begin{enumerate}
\item introduction
\begin{itemize}
\item Exponential integrals
\begin{itemize}
\item they are function of $z$ and they are defined from the data of $(X,f)$ and $[\mathcal{C}], [\nu]$
\item the choice of the path $\mathcal{C}$: 
\begin{itemize}
\item $\mathcal{C}\in H^{B,z}_{n}(X,f)$
\item Witten's formalism, $\mathcal{C}$ is a Lefschetz thimbles (or steepest descendent path)
\end{itemize}
\item they define a paring between the relative homology (rapid decaying homology)$H^{B,z}_{\bullet}(X,f)$ and the twisted de Rham cohomology  $H_{dR,z}^{\bullet}(X,f)$
\begin{itemize}
\item there is a comparison isomorphism (Maxim)
\end{itemize}
\item varying $z$ we have the Stokes phenomena
\item as $z\to\infty$, the asymptotic expansion of $I$ is a divergent series $\tilde{I}$, usually of Gevrey-class
\begin{itemize}
\item {\it exact resurgence relation} (Berry--Howls): divergence encodes contributions from other critical values
\item it is an example of resurgent series (\'Ecalle)
\item $\tilde{I}$ is resurgent in $\C\setminus\lbrace \text{poles of } \nu\,, \text{cirtical values of } f\rbrace$
\item it is a toy example of resurgent series because there are only finitely many singularities in the Borel plane
\item we have to compute the \textbf{Stokes constants} relative to the singular points in $B$ to fully understand $B$. There are two methods to compute Stokes constants: $\bullet$ geometric: using intersection theory of thimbles (Picard--Lefschtez, Witten, Maxim), $\bullet$ analytic: using \'Ecalle formalism   
\end{itemize} 
\end{itemize}
\item what are exponential integrals? \textcolor{gray}{has to be done}
\begin{itemize}
\item motivation
\begin{itemize}
\item In the classical theory of special functions, exponential integrals are often used to express solutions of linear differential and difference equations.
\item In physics ??
\item Geometrically they represent a Poincar\'e pairing (as explained by Kontsevich in \textbf{IHES lectures}).
\end{itemize}
\end{itemize}
\item What is the class of ODEs that we study? \textcolor{gray}{has to be done}
\item State results about resurgence of exponential integrals and Stokes phenomena
\begin{itemize}
\item Thimbles integrals [Kontsevich]: geometric computation of Stokes constants \textcolor{gray}{has to be done}
\item ODE and fractional derivative formula [{\tt draft2}]
\item if hypergeometric functions appear in a large class of examples: integral formulas for hypergeometric functions \textcolor{gray}{has to be done}
\end{itemize}
\end{itemize}
\item Formalism for Laplace transform [{\tt draft2}, ``The geometry of the Laplace transform'']
\begin{enumerate}
\item Analytic
\begin{enumerate}
\item Introduction
\item Brief revew of translation surfaces (we can refer to this from the introduction if we need to)
\item The Laplace transform of a holomorphic function
\begin{enumerate}
\item Over an ordinary point
\item Over a branch point
\item Differential equation
\end{enumerate}
\item Relating differential equations in the frequency domain to integral equations in the position domain
\end{enumerate}
\item Formal
\begin{enumerate}
\item Laplace transform of a formal series
\item Borel transform
\item Relating differential equations in the frequency variable to integral equations in the position variable
\end{enumerate}
\end{enumerate}
\item Review of integral equations
\begin{itemize}
\item Existence of solutions
\item Fractional integrals and derivatives
\item Going between integral and differential equations (slight functions)
\end{itemize}
\item General cases
\begin{enumerate}
\item Borel regularity
\begin{itemize}
\item General ODE of the form
\[ \left[ P\big(\tfrac{\partial}{\partial z}\big) + z^{-1} Q\big(\tfrac{\partial}{\partial z}\big) + z^{-2} R(z^{-1}) \right] \Phi = 0, \]
where $P$ is a polynomial, $Q$ is a polynomial of one degree lower, and $R$ is an entire function~\text[see {\tt airy-resurgence} and written notes]
\begin{itemize}
\color{DarkCyan}
\item More generally, for $P$ of degree $n$, we should be able to handle
\[ \left[ P\big(\tfrac{\partial}{\partial z}\big) + z^{-1} Q_1\big(\tfrac{\partial}{\partial z}\big) + z^{-2} Q_2\big(\tfrac{\partial}{\partial z}\big) + \ldots + z^{-(n-1)} Q_{n-1}\big(\tfrac{\partial}{\partial z}\big) + z^{-n} R(z^{-1}) \right] \Phi = 0, \]
where $Q_k$ has degree $n-k$. \textcolor{gray}{has to be done}
\begin{itemize}
\item We want the most general ODE with a regular singularity at $z = 0$ and its only other singularity, typically irregular, at $z = \infty$. \textcolor{gray}{has to be done}
\item The singularity at $\infty$ should only be regular for an Euler equation. \textcolor{gray}{has to be done}
\end{itemize}
\color{orange}
\item Show that we can find a slight solution at each critical value.
\item Show that $\hat{\iota} = \tilde{\iota}$, where:
\begin{itemize}
\item $I = \laplace \iota$
\item $\hat{\iota}$ is the Taylor expansion of $\iota$
\item $\tilde{I}$ is the asymptotic series of $I$
\item $\tilde{\iota} = \borel \tilde{I}$
\item Idea: Show that $\hat{\iota}$ and $\tilde{\iota}$ have matching asymptotics at $\zeta = 0$. Since they both satisfy the position-domain integral equation, they must coincide.
\end{itemize}
\end{itemize}
\item General thimble integral (conditions?)
\begin{itemize}
\item Proof of Borel regularity
\item $3/2$-derivative formula
\item Contour argument
\end{itemize}
\end{itemize}
\item Resurgence
\begin{itemize}
\item Explain how Borel regularity relates resurgence of formal series to resurgence of holomorphic functions in the position domain. \textcolor{gray}{think more about what we're trying to say here}
\item Relate to Ecalle's formalism and the alien derivative
\item Stokes factors
\begin{itemize}
\item For ODEs
\item For thimble integrals
\end{itemize}
\end{itemize}
\end{enumerate}
\item Examples \textcolor{gray}{make sure each example contains a computation of the Borel transform, so we can see it matches}
\begin{enumerate}
\item The Airy example
\begin{itemize}
\item $I(z)$ is a solution of a linear ODE. We explicitly find its Borel transform, knowing the nature of singularities and the asymptotic behaviour of a basis of solution for the ODE  [{\tt airy-resurgence}]
\item Compute Stokes constants
\begin{itemize}
\item Using fractional derivative formula and Borel transform computation [{\tt draft2}]
\item Using Picard-Lefschetz theory (Pham, Kontsevich, etc.)
\end{itemize}
\item Comparison with the literature \textcolor{gray}{has to be done}
\begin{itemize}
\item Mari\~{n}o
\item Sauzin
\item Kontsevich slides
\item Kawai--Takei? [might take too long to understand well enough]
\end{itemize}
\end{itemize}
\item The Airy--Lucas examples
\begin{itemize}
\item Compute Borel transform [{\tt airy-resurgence}]
\item Compute Stokes constants \textcolor{gray}{has to be done}
\end{itemize}
\item Bessel 0 (it is different because we have infinite cover)
\begin{itemize}
\item Compute Stokes constants [{\tt draft2}]
\end{itemize}
\item Bessel $\mu$ (follows from Bessel 0)
\begin{itemize}
\item Compute Stokes constants [{\tt modified Bessel}]
\end{itemize}
\item The generalized Airy example
\item The vibrating beam example
\begin{itemize}
\item In addition to the simple example, maybe we can do an example where the equation on the spatial domain includes fractional integrals, since Andy is interested in that sort of thing
\end{itemize}
\end{enumerate}
\end{enumerate}

\bibliographystyle{plain}
\bibliography{airy-resurgence}
\end{document}