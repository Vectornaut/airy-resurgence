\DeclareSymbolFont{AMSb}{U}{msb}{m}{n}
\documentclass[11pt,a4paper,twoside,leqno,noamsfonts]{amsart}
           \usepackage{setspace}
\linespread{1.34}           
           %\onehalfspacing
\usepackage[english]{babel}
\usepackage[dvipsnames]{xcolor}
\definecolor{britishracinggreen}{rgb}{0.0, 0.26, 0.15}
\definecolor{cobalt}{rgb}{0.0, 0.28, 0.67}
\usepackage[utopia]{mathdesign}
    \DeclareSymbolFont{usualmathcal}{OMS}{cmsy}{m}{n}
    \DeclareSymbolFontAlphabet{\mathcal}{usualmathcal}
\usepackage[a4paper,top=4cm,bottom=3cm,left=3.5cm,
           right=3.5cm,bindingoffset=5mm]{geometry}
\usepackage[utf8]{inputenc}
\usepackage{braket,caption,comment,mathtools,stmaryrd}
\usepackage{multirow,booktabs,microtype}
\usepackage{latexsym}
\usepackage{todonotes}
\usepackage{fancyhdr}
%\renewcommand{\sectionmark}[1]{\markboth{\thesection\ #1}{}}
\pagestyle{fancy}
% Clear the header and footer
\fancyhead{}
\fancyfoot{}
% Set the right side of the footer to be the page number
\fancyfoot[R]{\thepage}
\addtolength{\headheight}{\baselineskip}
%\fancyhead[RE]{\rightmark}
%\fancyhead[RE]{}
\usepackage{soul} % per testo barrato
\usepackage[colorlinks,bookmarks]{hyperref} %
\hypersetup{colorlinks,%
            citecolor=britishracinggreen,%
            filecolor=black,%
            linkcolor=cobalt,%
            urlcolor=black}
\setcounter{tocdepth}{2}
%\setcounter{section}{-1}
\numberwithin{equation}{section}

% Veronica's custom commands
%\renewenvironment{proof}{{\scshape Proof.}}{\qed}

\makeatletter
\newenvironment{proofof}[1]{\par
  \pushQED{\qed}%
  \normalfont \topsep6\p@\@plus6\p@\relax
  \trivlist
  \item[\hskip3\labelsep
        \itshape
    Proof of #1\@addpunct{.}]\ignorespaces
}{%
  \popQED\endtrivlist\@endpefalse
}
\makeatother

% Def
%\def\be{\begin{equation}}    
%\def\ee{\end{equation}}
\def\into{\hookrightarrow}
\def\onto{\twoheadrightarrow}
\def\isom{\cong}  
\def\ra{\rightarrow}
\def\lra{\longrightarrow}
\def\surj{\twoheadrightarrow}
\def\Var{\mathrm{Var}}
\def\Sch{\mathrm{Sch}}
\def\Sets{\mathrm{Sets}}
\def\Def{\mathsf{Def}}
\def\KS{\mathsf{KS}}
\def\ad{\mathsf{ad}}
\def\St{\mathrm{St}}
\def\st{\mathrm{st}}

\def\L{\mathbb L}
\def\A{\mathcal A}
\def\B{\mathcal B}
\def\R{\mathbb R}
\def\C{\mathbb C}
\def\D{\mathbb D}
\def\P{\mathbb P}
\def\Q{\mathbb Q}
\def\G{\mathbb G}
\def\L{\mathbb{L}}
\def\SS{\mathcal S}
\def\RR{\mathbf R}
\def\X{\mathcal X}
\def\E{\mathcal E}
\def\Z{\mathbb Z}
\def\N{\mathbb N}
\def\ext{\mathrm{ext}}
\def\FF{\mathscr{F}}

\def\HS{\mathsf{HS}}
\def\O{\mathscr O}
\def\DDT{\mathsf{DT}}
\def\PPT{\mathsf{PT}}
\def\LL{\mathsf{L}}
\def\NN{\mathsf{N}}
\def\sc{\textrm{sc}}
\def\dcr{\textrm{d-crit}}
\def\loc{\textrm{loc}}
\def\Ad{\textrm{Ad}}
\def\reg{\textrm{reg}}
\def\red{\textrm{red}}
\def\relvir{\textrm{relvir}}
\def\pur{\textrm{pur}}
\def\vd{\mathrm{vd}}
\def\pure{\textrm{pure}}
\def\MF{\mathsf{MF}}
\def\WW{\mathsf{W}}
\def\HH{\mathsf{H}}
\def\h{\mathfrak{h}}
\def\at{\mathsf A}
\def\pt{\mathrm{pt}}

\def\CC{\mathrm{C}}
\def\KK{\mathrm{K}}
\DeclareMathOperator{\Mod}{Mod}
\DeclareMathOperator{\op}{op}
\DeclareMathOperator{\Tor}{Tor}
\DeclareMathOperator{\Mor}{Mor}
\DeclareMathOperator{\Fun}{Fun}
\DeclareMathOperator{\Vect}{Vect}
\DeclareMathOperator{\FDVect}{FDVect}
\DeclareMathOperator{\Rings}{Rings}
\DeclareMathOperator{\ev}{ev}
\DeclareMathOperator{\Quot}{Quot}
\DeclareMathOperator{\DD}{D}
\DeclareMathOperator{\Hilb}{Hilb}
\DeclareMathOperator{\Chow}{Chow}
\DeclareMathOperator{\Orb}{Orb}
\DeclareMathOperator{\Ob}{Ob}
\DeclareMathOperator{\ob}{ob}
\DeclareMathOperator{\Jac}{Jac}
\DeclareMathOperator{\ch}{ch}
\DeclareMathOperator{\Td}{Td}
\DeclareMathOperator{\tr}{tr}
\DeclareMathOperator{\id}{id}
\DeclareMathOperator{\Pic}{Pic}
\DeclareMathOperator{\codet}{codet}
\DeclareMathOperator{\Rep}{Rep}
\DeclareMathOperator{\Bl}{Bl}
\DeclareMathOperator{\ord}{ord}
\DeclareMathOperator{\aff}{aff}
\DeclareMathOperator{\vir}{vir}
\DeclareMathOperator{\QCoh}{QCoh}
\DeclareMathOperator{\Coh}{Coh}
\DeclareMathOperator{\Span}{Span}
\DeclareMathOperator{\mult}{mult}
\DeclareMathOperator{\Spec}{Spec\,}
\DeclareMathOperator{\Proj}{Proj\,}
\DeclareMathOperator{\Supp}{Supp\,}
\DeclareMathOperator{\coker}{coker}
\DeclareMathOperator{\Cone}{Cone}
\DeclareMathOperator{\Perf}{Perf}
\DeclareMathOperator{\im}{im}
\DeclareMathOperator{\DT}{DT}
\DeclareMathOperator{\PT}{PT}
\DeclareMathOperator{\RRR}{R}
\DeclareMathOperator{\GL}{GL}
\DeclareMathOperator{\SL}{SL}
\DeclareMathOperator{\dd}{d}
\DeclareMathOperator{\Tr}{Tr}
\DeclareMathOperator{\NCHilb}{NCHilb}
\DeclareMathOperator{\Sym}{Sym}
\DeclareMathOperator{\Aut}{Aut}
\DeclareMathOperator{\Ext}{Ext}
\DeclareMathOperator{\lExt}{{\mathscr Ext}}
\DeclareMathOperator{\Hom}{Hom}
\DeclareMathOperator{\lHom}{{\mathscr Hom}}
\DeclareMathOperator{\catA}{{\mathscr A}}
\DeclareMathOperator{\catB}{{\mathscr B}}
\DeclareMathOperator{\catC}{{\mathcal C}}
\DeclareMathOperator{\catD}{{\mathcal D}}
\DeclareMathOperator{\catT}{{\mathscr T}}
\DeclareMathOperator{\catF}{{\mathscr F}}
\DeclareMathOperator{\End}{End}
\DeclareMathOperator{\Eu}{Eu}
\DeclareMathOperator{\Exp}{Exp}
\DeclareMathOperator{\rk}{rk}
\DeclareMathOperator{\Nil}{Nil}
\DeclareMathOperator{\Tot}{Tot}
\DeclareMathOperator{\length}{length}
\DeclareMathOperator{\codim}{codim}
\DeclareMathOperator{\pr}{pr}
%\DeclareMathOperator{\at}{at}
\DeclareMathOperator{\Art}{Art}
\DeclareMathOperator{\uC}{\underline{\mathcal C}}
\DeclareMathOperator{\uA}{\underline{\mathscr A}}
\DeclareMathOperator{\F}{\mathcal F}
\DeclareMathOperator{\hh}{H}%Da togliere quando corregger� il capitolo 4
\DeclareMathOperator{\Der}{Der}
\DeclareMathOperator{\Ab}{Ab}


%%%%%%%%%%%%%%%%
\theoremstyle{definition}

\newtheorem*{lemma*}{Lemma}
\newtheorem*{theorem*}{Theorem}
\newtheorem*{example*}{Example}
\newtheorem*{fact*}{Fact}
\newtheorem*{notation*}{Notation}
\newtheorem*{definition*}{Definition}
\newtheorem*{prop*}{Proposition}
\newtheorem*{remark*}{Remark}
\newtheorem*{corollary*}{Corollary}
\newtheorem*{conventions*}{Conventions}
\newtheorem*{caution*}{Caution}

\newtheorem{definition}{Definition}[section]
\newtheorem{problem}[definition]{Problem}
\newtheorem{example}[definition]{Example}
\newtheorem{fact}[definition]{Fact}
\newtheorem{aside}[definition]{Aside}
\newtheorem{prop}[definition]{Proposition}
\newtheorem{question}[definition]{Question}
\newtheorem{remark}[definition]{Remark}
\newtheorem{theorem}[definition]{Theorem}
\newtheorem{corollary}[definition]{Corollary}
\newtheorem{lemma}[definition]{Lemma}
%\newtheorem{conjecture}[definition]{Conjecture}
\newtheorem{claim}[definition]{Claim}
%\newtheorem{exercise}[definition]{Exercise}

%\newtheoremstyle{thm} % <name> % (ambienti con dimostrazione)
%        {3mm}% <Space above>
%        {3mm}% <Space below>
%        {\slshape}% <Body font> % 
%        {0mm}% <Indent amount>
%        {\bfseries}% <Theorem head font>
%        {.}% <Punctuation after theorem head>
%        {1mm}% <Space after theorem head>
%        {}% <Theorem head spec (can be left empty, meaning 'normal')> 
%\theoremstyle{thm}
%\newtheorem{theorem}[definition]{Theorem}
%\newtheorem{corollary}[definition]{Corollary}
%\newtheorem{lemma}[definition]{Lemma}
%\newtheorem{prop}[definition]{Proposition}
%\newtheorem{thm}{Theorem}
%\newtheorem{notation}{Notation}
%\renewcommand*{\thethm}{\Alph{thm}}



%\newtheoremstyle{sol} % <name> % (ambienti con dimostrazione)
%        {3mm}% <Space above>
%        {3mm}% <Space below>
%        {\normalfont}% <Body font> % 
%        {0mm}% <Indent amount>
%        {\scshape}% <Theorem head font>
%        {.}% <Punctuation after theorem head>
%        {1mm}% <Space after theorem head>
%        {}% <Theorem head spec (can be left empty, meaning 'normal')> 
\theoremstyle{sol}
%\newtheorem{slogan}[definition]{Slogan}
\newtheorem{assumption}[definition]{Assumption}
%%\newtheorem{claim}[definition]{Claim}
%\newtheorem{notation}[definition]{Notation}
%\newtheorem*{ssolution*}{Solution (sketch)}
%\newtheorem*{solution*}{Solution}


%%%%%%%%%%%%%%%%%%%%%%%%%

\usepackage{tikz}
\usepackage{tikz-cd}
\usepackage{rotating}
\newcommand*{\isoarrow}[1]{\arrow[#1,"\rotatebox{90}{\(\sim\)}"
]}
\usetikzlibrary{matrix,shapes,arrows,decorations.pathmorphing}
\tikzset{commutative diagrams/arrow style=math font}
\tikzset{commutative diagrams/.cd,
mysymbol/.style={start anchor=center,end anchor=center,draw=none}}
\newcommand\MySymb[2][\square]{%
  \arrow[mysymbol]{#2}[description]{#1}}
\tikzset{
shift up/.style={
to path={([yshift=#1]\tikztostart.east) -- ([yshift=#1]\tikztotarget.west) \tikztonodes}
}
}

\DeclareMathAlphabet{\mathpzc}{OT1}{pzc}{m}{it}

\newcommand*{\defeq}{\mathrel{\vcenter{\baselineskip0.5ex \lineskiplimit0pt
                     \hbox{\scriptsize.}\hbox{\scriptsize.}}}%
                     =}
\newcommand*{\defeqin}{\mathrel{\vcenter{\lineskiplimit0pt\baselineskip0.5ex
                     \hbox{\scriptsize.}\hbox{\scriptsize.}}}%
                     =}                     


% symbology
\newcommand{\blankbox}{{\fboxsep 0pt \colorbox{lightgray}{\phantom{$h$}}}}
\newcommand{\maps}{\colon}
\newcommand{\van}{\mathfrak{m}}
\newcommand{\laplace}{\mathcal{L}}
\newcommand{\borel}{\mathcal{B}}
\newcommand{\laplacepde}{\mathcal{D}}
\DeclareMathOperator{\Ai}{Ai}

\DeclareRobustCommand{\subtitle}[1]{\\#1}

\title{Resurgence and Borel regularity for ODEs \\ Seminar at Matamzee}
\author{Veronica Fantini}

\begin{document}
\maketitle
\section{What is resurgence}

Theory of resurgence was introduced by Ecalle in the '80 and it deals with divergent power series

\begin{equation}
\tilde{\Phi}(z)=\sum_{n\geq 0}a_nz^{-n-1}\in\C[\![z^{-1}]\!], \quad \text{ with } a_n\sim n!
\end{equation} 

they have zero radius of convergence. 


\subsection{Paradigma of Borel--Laplace sum}


\[
\begin{tikzcd}
\C[\![z^{-1}]\!]\ni\tilde{\Phi}\defeq\sum_{n\geq 0}a_nz^{-n-1} \arrow[rr, "\mathcal{B}"] & & \tilde{\phi}(\zeta)=\sum_{n\geq 0}\frac{a_n}{n!}\zeta^{n}\in\C\lbrace\zeta\rbrace
\end{tikzcd}
\]
\textcolor{magenta}{[make drawings of the Borel plane]}

Study the analytic continuation of $\tilde{\phi}(\zeta)$ and if $\hat{\phi}(\zeta)$ \textbf{behaves well} you can go back to the $z$-plane via Laplace transform 

\[
\begin{tikzcd}
\C[\![z^{-1}]\!]\ni\tilde{\Phi}\defeq\sum_{n\geq 0}a_nz^{-n-1} \arrow[rr, "\mathcal{B}"] & & \tilde{\phi}(\zeta)=\sum_{n\geq 0}\frac{a_n}{n!}\zeta^{n}\in\C\lbrace\zeta\rbrace\arrow[ld,"\mathcal{L}^\theta "]\\
& \Phi\in\mathcal{O}(H_\theta) &
\end{tikzcd}
\]

\textcolor{magenta}{[make drawings of the $z$-plane]}

The relation with $\tilde{F}$ is via asymptotic expansion as $\Re ze^{i\theta}\to \infty$

\[
\begin{tikzcd}
\C[\![z^{-1}]\!]\ni\tilde{\Phi}\defeq\sum_{n\geq 0}a_nz^{-n-1} \arrow[rr, "\mathcal{B}"] & & \tilde{\phi}(\zeta)=\sum_{n\geq 0}\frac{a_n}{n!}\zeta^{n}\in\C\lbrace\zeta\rbrace\arrow[ld,"\mathcal{L}^\theta "]\\
& \Phi\in\mathcal{O}(H_\theta)\arrow[lu,dashed, "\text{asymptotics}"] &
\end{tikzcd}
\]

The Laplace transform is defined 

\begin{equation}
F(z)\cong\mathcal{L}^{\theta}f(z)=\int_0^{+\infty e^{i\theta}}e^{-z\zeta}f(\zeta)d\zeta
\end{equation}
for $\theta\in [0,2\pi)$ and $F\in\mathcal{O}(H_\theta)$ if and only if $|f|\leq Ae^{-c|z|}$ for every $z$ in a tubular neighbourhood. In general one has to check 
\begin{itemize}
\item $\hat{f}$ can be defined through infinity;
\item $\hat{f}$ has the right decaying at $\infty$. \textcolor{red}{HARDER}
\end{itemize}

For the first one, we have to study the singularities of $\hat{f}$. 

\subsubsection{Ecalle's resurgent function}

$f\in\C\lbrace \zeta\rbrace$ is resurgent if it endlessly analytically continuable, i.e.

\textcolor{magenta}{[draw picture]}

He defined the \textbf{theory of singularities} and \textbf{relaxed the definition of Laplace transform} to deal with log- singularities, square root, poles, etc.

Examples of (\textit{minor} of) singularities 
\begin{align*}
\overset{\vee}{I}_c(\zeta)\defeq\frac{\zeta^{c-1}}{(1-e^{-2\pi ic})\Gamma(c)}\quad c\in\C\setminus\Z_{>0}\\
\overset{\vee}{I}_c(\zeta)\defeq\frac{\zeta^{c-1}}{2\pi i \Gamma(c)}\log\zeta \quad c\in\Z_{>0}
\end{align*}  

\textcolor{magenta}{draw singularity in a row (like Painlev\'e)}

\subsubsection{The role of $\theta$}

Varying $\theta$, as long as $|\theta-\theta'|<\pi$, $\mathcal{L}^\theta f=\mathcal{L}^{\theta'}$ on $H_\theta\cap H_{\theta'}$. More generally, they disagree

\begin{align*}
\mathcal{L}^\theta-\mathcal{L}^{\theta'}=S_{\theta\theta'}\mathcal{L}^{\theta'}\quad S_{\theta\theta'}\in\C
\end{align*}

These phenomena is called the Stokes phenomena. Computing the Stokes constant is crucial to understand the structure of $F$, and Ecalle developed the\textit{ Alien calculus} to study the  Stokes phenomena.   

Summarizing: the core of resurgence theory is 
\begin{enumerate}
\item study the Borel plane;
\item compute the Stokes constants. 
\end{enumerate} 

\textcolor{blue}{stress that resurgence is about Borel plane, but knows about the $z$-plane.}

\subsection{Motivation: why divergent series?}

\begin{itemize}
\item \textbf{thimbles integrals}:

\textcolor{Orchid}{COOL}: if $f$ is algebraic, $I(z)$ is a period, i.e. it is a geometric object studied (Deligne, Malgrange, Pham, Kontsevich--Soibelman $\sim$ generalization when $f$ is local coordinate for 1-form $f=\int \alpha$) and the resurgence of $\tilde{I}$ has a geometric nature 
\begin{itemize}
\item critical values of $f$ $\sim$ singularity in the Borel plane
\item Picard--Lefschetz theory $\sim$ Stokes phenomena
\item gradient lines $\sim$ Stokes indexes
\end{itemize}
\item \textbf{ODEs} with irregular singularity at $\infty$: they admits formal solutions, and if the ODE is regular enough M.A.E.T. assures the existence of a holomorphic solution asymptotic to the formal one. 
\begin{itemize}
\item non linear ODEs have interesting behaviours (maybe be resonant)
\end{itemize}
\item \textbf{q--difference equation}: like ODEs, 
\begin{equation}
n! \rightsquigarrow q^{n(n+1)/2}\,\, |q|>1
\end{equation}
there is a dictionary between ODEs and q-difference equation. 
\end{itemize}

\section{ODEs}
\subsection{Which class of ODEs we consider}
\begin{equation}\label{eq}
[P(\partial/\partial_z)+\frac{1}{z}Q(\partial/\partial_z)+\sum_{j=2}^dz^{-j}R_j(\partial/\partial_z)]\Phi(z)=0
\end{equation}
with
\begin{itemize}
\item $P(\lambda)$ a degree $d$ polynomial
\item $Q(\lambda)$ a degree $d-1$ polynomial 
\item $R_j(\lambda)$ are degree $d-j$ polynomials 
\end{itemize}
they are defined by Poincar\'e as \textit{series normal de Ier ordre}. As a general fact, if $P(-\lambda)$ has simple roots $\alpha_1,...,\alpha_d$ then \eqref{eq} admits $d$ formal solution of the form
\begin{equation}
\tilde{\Phi}_j(z)=e^{-\alpha_jz}z^{-\tau_j}\tilde{\phi}_j(z)\in e^{-\alpha_jz}z^{-\tau_j}\C[\![z^{-1}]\!][\log(z)]
\end{equation}  
where $\tau_j=-Q(\alpha_j)/P'(\alpha_j)$. 

The M.A.E.T theorem, guarantees under suitable assumptions on the ODE, the existence of an holomorphic solution $\Phi_j(z)$ asymptotic to $\tilde{\Phi}_j$ in a suitable sector. 

Our goal is to prove that $\Phi_j(z)\propto\mathcal{L}_{\zeta_j}^{\theta}\mathcal{B}\tilde{\Phi}_j$ for some angle $\theta$.

Equivalently said, Borel--Laplace summability picks an actual solution of the ODE.

\subsection{Borel regularity for ODEs}

\[
\begin{tikzcd}
& P\left(\frac{\partial}{\partial_z}\right)+\frac{1}{z}Q\left(\frac{\partial}{\partial_z}\right)+\sum_{j=2}^dz^{-j}R_j\left(\frac{\partial}{\partial_z}\right) & \\
\Phi_j\arrow[ru,red] & & \tilde{\Phi}_j\arrow[lu]\arrow[dd, "\mathcal{B}"] \\
& & \\
\hat{\phi}_j\arrow[uu,  "\mathcal{L}_{\zeta,\alpha_j}^\theta"]& & \tilde{\phi}_j(\zeta)\arrow[ll,"\text{sum}"] 
\end{tikzcd}
\]
a priori it is not guarantee that $\mathsf{L}_{\zeta,\alpha_j}^\theta\mathcal{B}\tilde{\Phi}_j$ is a solution of \eqref{eq}, because different functions may have the same asymptotic. 

Idea of the proof is based on the following diagram: 

\[
\begin{tikzcd}
& P\left(\frac{\partial}{\partial_z}\right)+\frac{1}{z}Q\left(\frac{\partial}{\partial_z}\right)+\sum_{j=2}^dz^{-j}R_j\left(\frac{\partial}{\partial_z}\right)\arrow[dd,"\mathcal{B}"] & \\
\Phi_j\arrow[ru,red] & & \tilde{\Phi}_j\arrow[lu]\arrow[dd, "\mathcal{B}"] \\
&P(-\zeta)+\partial_\zeta^{-1}\circ Q(-\zeta)+\sum_{j=2}^d \partial_\zeta^{-j}\circ R_j(-\zeta) & \\
\hat{\phi}_j\arrow[uu,  "\mathcal{L}_{\zeta,\alpha_j}^\theta"]& & \tilde{\phi}_j(\zeta)\arrow[ll,"\text{sum}"]\arrow[ul] 
\end{tikzcd}
\]

there exists a solution $\tilde{\phi}_j(\zeta)$ which is also slight, and in suitable coordinates it is a convergent series in $\overset{\vee}{I}_{n+\tau_j}(\zeta_j)$.  
\[
\begin{tikzcd}
& P\left(\frac{\partial}{\partial_z}\right)+\frac{1}{z}Q\left(\frac{\partial}{\partial_z}\right)+\sum_{j=2}^dz^{-j}R_j\left(\frac{\partial}{\partial_z}\right)\arrow[dd,"\mathcal{B}"] & \\
\Phi_j\arrow[ru,red] & & \tilde{\Phi}_j\arrow[lu]\arrow[dd, "\mathcal{B}"] \\
&P(-\zeta)+\partial_\zeta^{-1}\circ Q(-\zeta)+\sum_{j=2}^d \partial_\zeta^{-j}\circ R_j(-\zeta)\arrow[uu,"\mathcal{L}"] & \\
\hat{\phi}_j\arrow[uu,  "\mathcal{L}_{\zeta,\alpha_j}^\theta"]& & \tilde{\phi}_j(\zeta)\arrow[ll,"\text{sum}"] \arrow[ul] 
\end{tikzcd}
\]

the Laplace transform along a Hankel contour gives an inverse for $\mathcal{B}$. 

[repeat again the idea: a solution of \eqref{eq} is Borel regular, because its Borel--Laplace sum gives an actual solution.]

\begin{remark}
\textbf{[Extend the definition of $\mathcal{B}$ and $\mathcal{L}$]}
\begin{itemize}
\item As far as $\Gamma(\tau)$ is well defined, we can allow $\tau\in\R$ and extend the definition of $\mathcal{B}$. 
\item Ecalle's theory of singularity introduces a generalized Laplace transform which take as contour a Hankel contour rather than a straight line. This is not the only generalization of Laplace transform Ecalle introduced. 
\end{itemize}
\end{remark}

\subsection{Proof of Borel regularity}
\begin{itemize}
\item Borel transform the $\text{ODE}_z$ and we get $\text{IE}_\zeta$. 
\begin{itemize}
 \item IE are not easy to be solved so usually in the application we differentiate them to get an $\text{ODE}_\zeta$. If $\hat{\phi}$ is slight we don't loose informations by differentiating.
\end{itemize}
\item by Prop 1, there exists a solution $\hat{\phi}(\zeta_j)$ 
\begin{align*}
\hat{\phi}(\zeta_j)&=\zeta_j^{\tau_j-1}+\tilde{g}_j & \tilde{g}_j\in\mathcal{HL}^{\infty,1-\tau_j-\epsilon}\\
&=\sum_{k\geq 0}a_k\zeta_j^{\tau_j-1+k} +\text{h.f.} & \\
&=(1-e^{-2\pi i\tau_j})\sum_{n\geq 0}\tilde{a}_n\overset{\vee}{I}_{\tau_j+n}(\zeta_j) & \quad \tilde{a}_n=a_n\Gamma(\tau_j+n), \\
& & \overset{\vee}{I}_c(\xi)=\frac{\xi^{c-1}}{(1-e^{-2\pi ic})\Gamma(c)}\\
\end{align*}
\item $\hat{\phi}(\zeta_j) $ is a germ of meromorphic function
\begin{align*}
\limsup_{n\to\infty}\sqrt[n]{\frac{|\tilde{a}_n|}{\Gamma(\tau_j+n)(1-e^{-2\pi i(\tau_j+n)}}}&=\limsup_{n\to\infty}\sqrt[n]{\frac{{a}_n\Gamma(\tau_j+n)}{\Gamma(\tau_j+n)(1-e^{-2\pi i(\tau_j+n)}}}=\\
&=\limsup_{n\to\infty}\sqrt[n]{\frac{{a}_n}{(1-e^{-2\pi i(\tau_j+n)}}}<+\infty
\end{align*}
 where in the last step we use 
 \begin{multline}
 \infty+>||\tilde{g}_j \zeta_j^{\tau_j+\epsilon-1}||_\infty=||\sum_{n\geq 1}\zeta_j^{\tau_j-1+n}\zeta_j^{-(\tau_j+\epsilon)+1} ||_\infty=||\sum_{n\geq 1}a_n\zeta_j^{n-\varepsilon}||_\infty
 \end{multline}
 \item by Ecalle definition of Laplace transform [see Sauzin], $\hat{\phi}(\zeta_j)$ has a well defined Laplace transform which is asymptotic to $\sum_{n\geq 0}\tilde{a}_nz^{-\tau_j-n}(1-e^{-2\pi i \tau_j})=(1-e^{-2\pi i \tau_j})\sum_{n\geq 0}a_n\Gamma(\tau_j+n)z^{-\tau_j-n}$
 \begin{align*}
 (1-e^{-2\pi i \tau_j})\sum_{n\geq 0}a_n\Gamma(\tau_j+n)z^{-\tau_j-n}\sim \mathcal{L}_{\zeta_j}^\theta\hat{\phi}_j=e^{\alpha_jz}\mathcal{L}_{\zeta,\alpha_j}^\theta\hat{\phi}
 \end{align*}
 hence 
 \begin{equation}
 \mathcal{L}_{\zeta,\alpha_j}^\theta\hat{\phi} \sim e^{-\alpha_jz}z^{-\tau_j} \sum_{n\geq 0}a_n\Gamma(\tau_j+n)z^{-n}\propto \tilde{\Phi}_j(z)
 \end{equation}
\end{itemize}

\subsection{Corollary: construct an explicit holomorphic solution predicted by M.A.E.T.}

\begin{corollary}
The holomorphic solution which exists by M.A.E.T. can be characterized as the Borel--Laplace sum of $\tilde{\Phi}_j$.  
\end{corollary}


\section{q-difference equations}

\subsection{Formal solutions vs actual solutions: paradigma of $q$-Borel Laplace summability}

\subsubsection{From ODEs to $q$- difference equations}

\begin{center}
\begin{tabular}{c|c}
\hline
ODE     & $q$-difference, $|q|>1$ \\
\hline
$\sum_{n\geq 0}^d a_{n}(z)\frac{\partial^n}{\partial z^n}\Phi=a\in\mathbb{C}(z)$ and $a_n\in\mathbb{C} (z)$    & $\sum_{n\geq 0} a_{q,n}\sigma_q^n f=a_q\in\mathbb{C}(x)$ and $a_{q,n}\in\mathbb{C}(x)$ \\
& \\
$z^2\Phi'(z)+\Phi(z)=z$ & $x\sigma_q f(x)+f(x)=x$\\
$\tilde{\Phi}(z)=\sum_{n\geq 0}(-1)^{n+1} n!z^{-n-1}$ & $\hat{f}(x)=\sum_{n\geq 0}(-1)^{n}q^{n(n+1)/2}x^{n+1}$ \\
& \\
\hline
General notation & \\
\hline
$z\frac{d}{dz}$ & $\sigma_qf(x)=f(qx)$ \\
$\frac{d}{dz}$ & $\delta_q f(x)=\frac{f(qx)-f(x)}{(q-1)x}$ \\
$n!$ & $q^{n(n+1)/2}$ \\
$e^z$ & $e_q(x)=\log(1/q)\Theta_q(x)$
\end{tabular}

\end{center}

\begin{remark} Why $n!$ corresponds to $q^{-n(n+1)/2}$
\begin{equation}
\Gamma(n)=\int_0^{+\infty}e^{-t}t^n\frac{dt}{t}
\end{equation}

\begin{equation}
q^{-n(n-1)/2}=\int_0^{+\infty}\frac{t^n}{e_q(t)}\frac{dt}{t}
\end{equation}
\end{remark}

There is also another interesting relation between $q$- factorial and $n!$

\begin{equation}
[n]_q!= \frac{(q;q)_n}{(1-q)^n} \rightarrow_{q\to 1} n!
\end{equation}

\[
\begin{tikzcd}
\hat{f}(x)=\sum_{n\geq 0}a_n x^n\in\mathbb{C}[\![x]\!]_{(q;1)} \arrow[rr,"\mathcal{B}_{(q;1)}"] & & \varphi(\xi)=\sum_{n\geq 0}a_n q^{-n(n-1)/2}\xi^n\arrow[dl,"\mathsf{L}^{[\lambda]}_{q;1}"]\\
&f(x)\arrow[lu,dashed ,"\text{asymptotics}"] &
\end{tikzcd}
\]

\subsection{Borel regularity: Dreyfus's theorem}

Let $\mathcal{B}_{\mu}$ be the $q$- Borel transform for $q$-Gevrey $\mu$ series
\begin{equation}
    \mathcal{B}_{\mu}\colon\sum_{n\geq 0}a_nx^n\to \sum_{n\geq 0}a_n q^{-n(n-1)/(2\mu)}\xi^n
\end{equation}
and let $\mathsf{L}_{\mu,\kappa}^{[\lambda]}$ be the $q$- Laplace transform with parameters $\mu\in\Q_{>0}$, $\kappa\in\N^*$ ($\mathsf{L}_{q;1}^{[\lambda]}:=\mathsf{L}_{1,1}^{[\lambda]}$)
\begin{equation}
    \mathsf{L}_{\mu,\kappa}^{[\lambda]}\varphi(x)=\frac{\mu}{\kappa}\sum_{l\in \kappa^{-1}\mathbb{Z}}\frac{\varphi(q^l\lambda)}{\Theta_{q^{1/\mu}}(\frac{q^{1/\mu+l}\lambda}{x})}\quad \lambda\in\C^*/q^{\kappa^{-1}\Z}
\end{equation}
Let $\mathbb{H}_{\mu,\kappa}^{[\lambda]}$ be the space of functions $\varphi\in\mathcal{M}(\C^*)$, such that there exists $\varepsilon>0$, $\Omega\subset \C$ connected 
\begin{itemize}
\item $\bigcup_{l\in\kappa^{-1}\Z}\lbrace x\in\C^* | |x-\lambda q^{l}|<\varepsilon |\lambda q^l|\rbrace\subset\Omega$
\item $\varphi$ can be continued analytically in $\Omega$ with $q^{1/\mu}$ exponential growth
\[|\varphi(\xi)|<C|\Theta_{|q|^{1/\mu}}(A|\xi|)|\] 
\end{itemize} 

For every $\varphi(\xi)\in\mathbb{H}_{\mu,\kappa}^{[\lambda]}$, $\mathsf{L}_{\mu,\kappa}^[\lambda]\varphi(x)\in$. 

Under suitable assumptions on the $q$-difference equation, the following result holds true 
\begin{theorem}
Let $\hat{h}$ be a formal power series solution of a linear q-difference equation with coefficients in $\mathbb{C}(x)$. There exist $\kappa_1,...,\kappa_r\in\mathbb{Q}_{>0}$, $n,K\in\mathbb{N}^*$ and a finite set $\Sigma\subset\mathbb{C}^*/ q^{n^{-1}\mathbb{Z}}$, we may compute from the q-difference equation, such that for all $\lambda\in (\mathbb{C}^*/q^{n^{-1}\mathbb{Z}})\setminus\mathbb{Z} $,
\[
\mathsf{S}^{[\lambda]}(\hat{h}):=\mathsf{L}^{[\lambda]}_{\kappa_r,n}\circ\mathsf{L}^{[\lambda]}_{\kappa_{r-1},K}\circ...\circ\mathsf{L}^{[\lambda]}_{\kappa_1,K}\circ\mathcal{B}_{\kappa_1}\circ...\circ\mathcal{B}_{\kappa_r}\hat{h}\]
is meromorphic on $\mathbb{C}^*$, and is solution of the same equation as $\hat{h}$. Moreover, $\mathsf{S}^{[\lambda]}(\hat{h})$ is asymptotic to $\hat{h}$ and, for $|x|$ close to $0$ it has poles of order at most $1$ that are contained in $\lambda q^{n-1}, n\in\mathbb{Z}$. 
\end{theorem}

Idea of the proof: let $P(\hat{f})=\big[\sum_{n=0}^ma_n\sigma_q^n\big]\hat{f}$

\[
\begin{tikzcd}
& P\hat{f}=a\arrow[dd] & \\
f(x)\arrow[ru,red] & &\hat{f}(x)\arrow[lu]\arrow[dd, "\mathcal{B}_{\kappa_1}\circ...\circ\mathcal{B}_{\kappa_r}"] \\
& q-\text{difference equation} & \\
\varphi(\xi)\in\mathbb{H}^{[\lambda]}_{\kappa_1,K}, \text{ for some }\lambda \arrow[uu,  "\mathsf{L}^{[\lambda]}_{\kappa_r,n}\circ...\circ\mathsf{L}^{[\lambda]}_{\kappa_1,K}"]& & \varphi(\xi)\in\C\lbrace\xi\rbrace, \text{ of slope } 0 \arrow[ll,Leftrightarrow]\arrow[ul] 
\end{tikzcd}
\]

\subsection{Summary}

\begin{center}
\begin{tabular}{c|c}
\hline
\textbf{Formal series} & \\
\hline
$\tilde{\Phi}(z)=\sum_{n\geq 0} a_n z^{-n-1}\in\mathbb{C}[\![z^{-1}]\!]_1$     &  $\hat{f}(x)=\sum_{n\geq 0}a_n x^n\in\mathbb{C}[\![x]\!]_{(q;1)}$ \\
& \\
\hline
\textbf{Borel transform} & \\
\hline
  $\tilde{\phi}(\zeta)=\sum_{n\geq0}a_n\frac{\zeta^n}{n!}\in\mathbb{C}\lbrace \zeta\rbrace$   & $\varphi(\xi)=\sum_{n\geq 0}a_n q^{n(n-1)/2}\xi^n \in\mathbb{C}(\!(\xi)\!)$\\
  & \\
  \hline
  \textbf{Laplace transform} & \\
  \hline
 $\mathcal{L}_\zeta^{\theta}\hat{\phi}(z)=\int_0^{+e^{i\theta}\infty} e^{-z\zeta}\hat{\phi}(\zeta)d\zeta\in\mathcal{O}(H_\theta)$ & $\mathcal{L}^{\theta}_{q;1}\varphi(x)=\int_0^{e^{i\theta} \infty}\frac{\varphi(\xi)}{e_q(\frac{\xi}{x})}d\xi $ \\
 & $\Theta$ function: $\Theta_q(x)=\sum_{n\in\mathbb{Z}}q^{-n(n+1)/2}x^n$\\
 & $\mathsf{L}_{q;1}^{[\lambda]}\varphi(x)=\sum_{n\geq 0}\frac{\varphi(\lambda q^n)}{\Theta_q(\frac{\lambda q^n}{x})}\in\mathcal{O}(H^{\lambda}),\, \lambda\in\mathbb{C}^*$\\
 & $H^{\lambda}=D_r^*\setminus\lbrace -\lambda q^n, n\in\mathbb{Z}\rbrace$ \\
 & \\
 \hline
 \textbf{Domain of definition} & \\
 \hline 
 $\hat{\phi}(\zeta)$ s.t. $\exists C,a>0 $ & $\varphi(\xi)$ s.t. $\exists C,a>0 $\\
 $|\hat{\phi}(\zeta)|<Ce^{a|\zeta|}$, $\zeta\in S_{\delta}$ & $|\varphi(\xi)|<C|\xi|^aq^{\frac{1}{2}\left(\frac{\log|\xi|}{\log q}\right)^2}$, $\xi\in S_{\delta}\cap\C^*$ \\
 $S_\delta$ is an half-strip& \\
 & $\varphi(\xi)\in\mathbb{H}_{\mu,\kappa}^{[\lambda]}$\\
 & \\
 \hline
 \textbf{Gevrey asymptotics} & \\
 \hline
$\mathcal{L}_{\zeta}^{\theta}\hat{\phi}(z)\sim_1 \tilde{\Phi}(z)$ & $\mathsf{L}_{q;1}^{[\lambda]}\varphi(x)\sim_{q;1} \hat{f}(x)$ \\
& \\
\hline
\textbf{Stokes phenomena} & \\
\hline 
 varying $\theta$, $\mathcal{L}^{\theta}$ jumps & varying $\mu\not\equiv_{q}\lambda$, $\mathsf{L}_{q;1}^{[\mu]}$ jumps\\
 & \\
 \hline
 \textbf{Borel regularity} & \\
 \hline
 M.A.E.T: existence of hol. solutions & [Praagman, 86]: existence of merom. solutions \\
 F.,Feynes : BL sum gives an actual solution & [Dreyfus, 14]: q--BL gives an actual solution \\
 \textit{slight functions} & \textit{slope} $0$ operators \\
\end{tabular}
\end{center}



\bibliographystyle{plain}
\bibliography{airy-resurgence}
\end{document}