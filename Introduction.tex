\DeclareSymbolFont{AMSb}{U}{msb}{m}{n}
\documentclass[11pt,a4paper,twoside,leqno,noamsfonts]{amsart}
           \usepackage{setspace}
\linespread{1.34}           
           %\onehalfspacing
\usepackage[english]{babel}
\usepackage[dvipsnames]{xcolor}
\definecolor{britishracinggreen}{rgb}{0.0, 0.26, 0.15}
\definecolor{cobalt}{rgb}{0.0, 0.28, 0.67}
\usepackage[utopia]{mathdesign}
    \DeclareSymbolFont{usualmathcal}{OMS}{cmsy}{m}{n}
    \DeclareSymbolFontAlphabet{\mathcal}{usualmathcal}
\usepackage[a4paper,top=4cm,bottom=3cm,left=3.5cm,
           right=3.5cm,bindingoffset=5mm]{geometry}
\usepackage[utf8]{inputenc}
\usepackage{braket,caption,comment,mathtools,stmaryrd}
\usepackage{multirow,booktabs,microtype}
\usepackage{latexsym}
\usepackage{todonotes}
\usepackage{fancyhdr}
\usepackage{tikz-cd}
%\renewcommand{\sectionmark}[1]{\markboth{\thesection\ #1}{}}
\pagestyle{fancy}
% Clear the header and footer
\fancyhead{}
\fancyfoot{}
% Set the right side of the footer to be the page number
\fancyfoot[R]{\thepage}
\addtolength{\headheight}{\baselineskip}
%\fancyhead[RE]{\rightmark}
%\fancyhead[RE]{}
\usepackage{soul} % per testo barrato
\usepackage[colorlinks,bookmarks]{hyperref} %
\hypersetup{colorlinks,%
            citecolor=britishracinggreen,%
            filecolor=black,%
            linkcolor=cobalt,%
            urlcolor=black}
\setcounter{tocdepth}{2}
%\setcounter{section}{-1}
\numberwithin{equation}{section}

% Veronica's custom commands
%\renewenvironment{proof}{{\scshape Proof.}}{\qed}

\makeatletter
\newenvironment{proofof}[1]{\par
  \pushQED{\qed}%
  \normalfont \topsep6\p@\@plus6\p@\relax
  \trivlist
  \item[\hskip3\labelsep
        \itshape
    Proof of #1\@addpunct{.}]\ignorespaces
}{%
  \popQED\endtrivlist\@endpefalse
}
\makeatother

% Def
%\def\be{\begin{equation}}    
%\def\ee{\end{equation}}
\def\into{\hookrightarrow}
\def\onto{\twoheadrightarrow}
\def\isom{\cong}  
\def\ra{\rightarrow}
\def\lra{\longrightarrow}
\def\surj{\twoheadrightarrow}
\def\Var{\mathrm{Var}}
\def\Sch{\mathrm{Sch}}
\def\Sets{\mathrm{Sets}}
\def\Def{\mathsf{Def}}
\def\KS{\mathsf{KS}}
\def\ad{\mathsf{ad}}
\def\St{\mathrm{St}}
\def\st{\mathrm{st}}

\def\L{\mathbb L}
\def\A{\mathcal A}
\def\B{\mathcal B}
\def\R{\mathbb R}
\def\C{\mathbb C}
\def\D{\mathbb D}
\def\P{\mathbb P}
\def\Q{\mathbb Q}
\def\G{\mathbb G}
\def\L{\mathbb{L}}
\def\SS{\mathcal S}
\def\RR{\mathbf R}
\def\X{\mathcal X}
\def\E{\mathcal E}
\def\Z{\mathbb Z}
\def\N{\mathbb N}
\def\ext{\mathrm{ext}}
\def\FF{\mathscr{F}}

\def\HS{\mathsf{HS}}
\def\O{\mathscr O}
\def\DDT{\mathsf{DT}}
\def\PPT{\mathsf{PT}}
\def\LL{\mathsf{L}}
\def\NN{\mathsf{N}}
\def\sc{\textrm{sc}}
\def\dcr{\textrm{d-crit}}
\def\loc{\textrm{loc}}
\def\Ad{\textrm{Ad}}
\def\reg{\textrm{reg}}
\def\red{\textrm{red}}
\def\relvir{\textrm{relvir}}
\def\pur{\textrm{pur}}
\def\vd{\mathrm{vd}}
\def\pure{\textrm{pure}}
\def\MF{\mathsf{MF}}
\def\WW{\mathsf{W}}
\def\HH{\mathsf{H}}
\def\h{\mathfrak{h}}
\def\at{\mathsf A}
\def\pt{\mathrm{pt}}

\def\CC{\mathrm{C}}
\def\KK{\mathrm{K}}
\DeclareMathOperator{\Mod}{Mod}
\DeclareMathOperator{\op}{op}
\DeclareMathOperator{\Tor}{Tor}
\DeclareMathOperator{\Mor}{Mor}
\DeclareMathOperator{\Fun}{Fun}
\DeclareMathOperator{\Vect}{Vect}
\DeclareMathOperator{\FDVect}{FDVect}
\DeclareMathOperator{\Rings}{Rings}
\DeclareMathOperator{\ev}{ev}
\DeclareMathOperator{\Quot}{Quot}
\DeclareMathOperator{\DD}{D}
\DeclareMathOperator{\Hilb}{Hilb}
\DeclareMathOperator{\Chow}{Chow}
\DeclareMathOperator{\Orb}{Orb}
\DeclareMathOperator{\Ob}{Ob}
\DeclareMathOperator{\ob}{ob}
\DeclareMathOperator{\Jac}{Jac}
\DeclareMathOperator{\ch}{ch}
\DeclareMathOperator{\Td}{Td}
\DeclareMathOperator{\tr}{tr}
\DeclareMathOperator{\id}{id}
\DeclareMathOperator{\Pic}{Pic}
\DeclareMathOperator{\codet}{codet}
\DeclareMathOperator{\Rep}{Rep}
\DeclareMathOperator{\Bl}{Bl}
\DeclareMathOperator{\ord}{ord}
\DeclareMathOperator{\aff}{aff}
\DeclareMathOperator{\vir}{vir}
\DeclareMathOperator{\QCoh}{QCoh}
\DeclareMathOperator{\Coh}{Coh}
\DeclareMathOperator{\Span}{Span}
\DeclareMathOperator{\mult}{mult}
\DeclareMathOperator{\Spec}{Spec\,}
\DeclareMathOperator{\Proj}{Proj\,}
\DeclareMathOperator{\Supp}{Supp\,}
\DeclareMathOperator{\coker}{coker}
\DeclareMathOperator{\Cone}{Cone}
\DeclareMathOperator{\Perf}{Perf}
\DeclareMathOperator{\im}{im}
\DeclareMathOperator{\DT}{DT}
\DeclareMathOperator{\PT}{PT}
\DeclareMathOperator{\RRR}{R}
\DeclareMathOperator{\GL}{GL}
\DeclareMathOperator{\SL}{SL}
\DeclareMathOperator{\dd}{d}
\DeclareMathOperator{\Tr}{Tr}
\DeclareMathOperator{\NCHilb}{NCHilb}
\DeclareMathOperator{\Sym}{Sym}
\DeclareMathOperator{\Aut}{Aut}
\DeclareMathOperator{\Ext}{Ext}
\DeclareMathOperator{\lExt}{{\mathscr Ext}}
\DeclareMathOperator{\Hom}{Hom}
\DeclareMathOperator{\lHom}{{\mathscr Hom}}
\DeclareMathOperator{\catA}{{\mathscr A}}
\DeclareMathOperator{\catB}{{\mathscr B}}
\DeclareMathOperator{\catC}{{\mathcal C}}
\DeclareMathOperator{\catD}{{\mathcal D}}
\DeclareMathOperator{\catT}{{\mathscr T}}
\DeclareMathOperator{\catF}{{\mathscr F}}
\DeclareMathOperator{\End}{End}
\DeclareMathOperator{\Eu}{Eu}
\DeclareMathOperator{\Exp}{Exp}
\DeclareMathOperator{\rk}{rk}
\DeclareMathOperator{\Nil}{Nil}
\DeclareMathOperator{\Tot}{Tot}
\DeclareMathOperator{\length}{length}
\DeclareMathOperator{\codim}{codim}
\DeclareMathOperator{\pr}{pr}
%\DeclareMathOperator{\at}{at}
\DeclareMathOperator{\Art}{Art}
\DeclareMathOperator{\uC}{\underline{\mathcal C}}
\DeclareMathOperator{\uA}{\underline{\mathscr A}}
\DeclareMathOperator{\F}{\mathcal F}
\DeclareMathOperator{\hh}{H}%Da togliere quando corregger� il capitolo 4
\DeclareMathOperator{\Der}{Der}
\DeclareMathOperator{\Ab}{Ab}


%%%%%%%%%%%%%%%%
\theoremstyle{definition}

\newtheorem*{lemma*}{Lemma}
\newtheorem*{theorem*}{Theorem}
\newtheorem*{example*}{Example}
\newtheorem*{fact*}{Fact}
\newtheorem*{notation*}{Notation}
\newtheorem*{definition*}{Definition}
\newtheorem*{prop*}{Proposition}
\newtheorem*{remark*}{Remark}
\newtheorem*{corollary*}{Corollary}
\newtheorem*{conventions*}{Conventions}
\newtheorem*{caution*}{Caution}

\newtheorem{definition}{Definition}[section]
\newtheorem{problem}[definition]{Problem}
\newtheorem{example}[definition]{Example}
\newtheorem{fact}[definition]{Fact}
\newtheorem{aside}[definition]{Aside}
\newtheorem{prop}[definition]{Proposition}
\newtheorem{question}[definition]{Question}
\newtheorem{remark}[definition]{Remark}
\newtheorem{theorem}[definition]{Theorem}
\newtheorem{corollary}[definition]{Corollary}
\newtheorem{lemma}[definition]{Lemma}
%\newtheorem{conjecture}[definition]{Conjecture}
\newtheorem{claim}[definition]{Claim}
%\newtheorem{exercise}[definition]{Exercise}

%\newtheoremstyle{thm} % <name> % (ambienti con dimostrazione)
%        {3mm}% <Space above>
%        {3mm}% <Space below>
%        {\slshape}% <Body font> % 
%        {0mm}% <Indent amount>
%        {\bfseries}% <Theorem head font>
%        {.}% <Punctuation after theorem head>
%        {1mm}% <Space after theorem head>
%        {}% <Theorem head spec (can be left empty, meaning 'normal')> 
%\theoremstyle{thm}
%\newtheorem{theorem}[definition]{Theorem}
%\newtheorem{corollary}[definition]{Corollary}
%\newtheorem{lemma}[definition]{Lemma}
%\newtheorem{prop}[definition]{Proposition}
%\newtheorem{thm}{Theorem}
%\newtheorem{notation}{Notation}
%\renewcommand*{\thethm}{\Alph{thm}}



%\newtheoremstyle{sol} % <name> % (ambienti con dimostrazione)
%        {3mm}% <Space above>
%        {3mm}% <Space below>
%        {\normalfont}% <Body font> % 
%        {0mm}% <Indent amount>
%        {\scshape}% <Theorem head font>
%        {.}% <Punctuation after theorem head>
%        {1mm}% <Space after theorem head>
%        {}% <Theorem head spec (can be left empty, meaning 'normal')> 
\theoremstyle{sol}
%\newtheorem{slogan}[definition]{Slogan}
\newtheorem{assumption}[definition]{Assumption}
%%\newtheorem{claim}[definition]{Claim}
%\newtheorem{notation}[definition]{Notation}
%\newtheorem*{ssolution*}{Solution (sketch)}
%\newtheorem*{solution*}{Solution}


%%%%%%%%%%%%%%%%%%%%%%%%%

\usepackage{tikz}
\usepackage{tikz-cd}
\usepackage{rotating}
\newcommand*{\isoarrow}[1]{\arrow[#1,"\rotatebox{90}{\(\sim\)}"
]}
\usetikzlibrary{matrix,shapes,arrows,decorations.pathmorphing}
\tikzset{commutative diagrams/arrow style=math font}
\tikzset{commutative diagrams/.cd,
mysymbol/.style={start anchor=center,end anchor=center,draw=none}}
\newcommand\MySymb[2][\square]{%
  \arrow[mysymbol]{#2}[description]{#1}}
\tikzset{
shift up/.style={
to path={([yshift=#1]\tikztostart.east) -- ([yshift=#1]\tikztotarget.west) \tikztonodes}
}
}

\DeclareMathAlphabet{\mathpzc}{OT1}{pzc}{m}{it}

\newcommand*{\defeq}{\mathrel{\vcenter{\baselineskip0.5ex \lineskiplimit0pt
                     \hbox{\scriptsize.}\hbox{\scriptsize.}}}%
                     =}
\newcommand*{\defeqin}{\mathrel{\vcenter{\lineskiplimit0pt\baselineskip0.5ex
                     \hbox{\scriptsize.}\hbox{\scriptsize.}}}%
                     =}                     


% symbology
\newcommand{\blankbox}{{\fboxsep 0pt \colorbox{lightgray}{\phantom{$h$}}}}
\newcommand{\maps}{\colon}
\newcommand{\van}{\mathfrak{m}}
\newcommand{\laplace}{\mathcal{L}}
\newcommand{\borel}{\mathcal{B}}
\newcommand{\laplacepde}{\mathcal{D}}
\DeclareMathOperator{\Ai}{Ai}
\newcommand{\deriv}[3]{\partial^{#1}_{#2 \text{ from } #3}}
\newcommand{\series}[1]{#1_\bullet}

\DeclareRobustCommand{\subtitle}[1]{\\#1}
\title[Borel regularity for exponential integrals and ODEs]{Borel regularity for exponential integrals and ODEs\\ [1ex]
  }

\author{
Veronica Fantini and  Aaron Fenyes 
}
\begin{document}
\maketitle
\tableofcontents

\section{Introduction}

\subsection{Motivation}

\subsection{What are exponential integrals}

Let $X$ be an $n$-dimensional algebraic variety, we consider exponential integrals of the form 
\begin{equation}
\label{:exp_integral}
I(z)=\int_{\mathcal{C}}e^{-zf}\nu
\end{equation}
where $f\colon X\to \C$ is an algebraic function, $\nu\in\Gamma(X,\Omega^n)$, $z\in\C$ and $\mathcal{C}$ is an $n$-cycle of integration such that the integral is well defined. Assuming $\theta\defeq\arg(z)$ fixed, we define for every $c>0$ 
\begin{align*}
S^+_c\defeq\lbrace \zeta\in\C\vert \mathrm{Re}(\zeta e^{i\theta})\geq c\rbrace\\
S^-_c\defeq\lbrace \zeta\in\C\vert \mathrm{Re}(\zeta e^{i\theta})\leq c\rbrace.
\end{align*}
and we define $H_\bullet(X,zf)$ as the relative homology groups, namely \[H_{\bullet}(X,zf)=H_{\bullet}(X,f^{-1}(S_c^+))\] for $c$ large enough so that $f^{-1}(S_c^+)$ does not contain singular values of $f$ (i.e. points $\zeta_\alpha\defeq f(x_\alpha)$ where $df(x_\alpha)=0$ \footnote{In dimension $n>1$ there are other singular values due to the fact that $f$ is not proper. However under suitable assumptions it is possible to overcome this issue (see \cite{pham} part $1$, $\S 2$).}). Then we assume $[\mathcal{C}]\in H_n(X,zf)$. 

Notice that $I(z)$ only depends on the class of $\mathcal{C}$ and on the class of $[\nu]\in H_{dR}^n(X,zf)\defeq\mathbb{H}^n\left(X,(\Omega^\bullet_X,d_W\defeq e^{zf}\circ d\circ e^{-zf})\right)$: indeed let $\mathcal{C}'=\mathcal{C}+\partial\gamma\in [\mathcal{C}]$,
\begin{align*}
\int_{\mathcal{C}'}e^{-zf}\nu=\int_{\mathcal{C}}e^{-zf}\nu+\int_{\partial\gamma}e^{-zf}\nu=\int_{\mathcal{C}}e^{-zf}\nu
\end{align*} 
because $e^{-zf}$ is rapidly decaying at $\partial\gamma$. Similarly, if $\nu'=\nu+d_W\eta\in [\nu]$
\begin{multline*}
\int_{\mathcal{C}}e^{-zf}\nu'=\int_{\mathcal{C}}e^{-zf}\nu+\int_{\mathcal{C}}e^{-zf}d_{W}\eta=\int_{\mathcal{C}}e^{-zf}\nu+\int_{\mathcal{C}}d(e^{-zf}\eta)=\\
=\int_{\mathcal{C}}e^{-zf}\nu+\int_{\partial\mathcal{C}}e^{-zf}\eta=\int_{\mathcal{C}}e^{-zf}\nu.
\end{multline*}
Moreover, for $\theta$ generic, i.e. $\theta\neq\arg(\zeta_\alpha-\zeta_\beta)$ and $\alpha\neq\beta$, there is a decomposition of $H_\bullet(X,zf)$ as direct sum over the singular values of $f$ of the homology relative to these singular values (see \cite{pham}\cite{KKP}): 
\begin{align}\label{decomposition H_B}
H_\bullet(X,zf)=\bigoplus_{\alpha}H_\bullet\left(f^{-1}(B_\alpha(\varepsilon))\cap f^{-1}(S_\alpha^+(\theta,\varepsilon))\right)
\end{align}     
where $B_\alpha(\varepsilon)\defeq\lbrace \zeta \in\C\big\vert\,\, |\zeta-\zeta_\alpha|<\varepsilon\rbrace$ and $S_\alpha^+(\theta,\varepsilon)\defeq B_\alpha(\varepsilon)\cap\lbrace \mathrm{Re}(\zeta e^{i\theta})\geq \tfrac{\varepsilon}{2}\rbrace$ (see Figure \ref{fig:relative_homology}).
\begin{figure}
\caption{Relative homology}\label{fig:relative_homology}
\end{figure}
We furthermore assume the following:
\begin{assumption}\label{assumption_on_f}
The map $f\colon X\to\C$ has isolated, simple and non degenerate critical values $\zeta_\alpha$. 
\end{assumption}

which guarantees that 
\begin{itemize}
\item $H_k\left(f^{-1}(B_\alpha(\varepsilon))\cap f^{-1}(S_\alpha^+(\theta,\varepsilon))\right)$ is isomorphic to $H_{k-1}(B_\alpha^M(\varepsilon)\cap f^{-1}(\zeta-\zeta_\alpha))$, where $B_\alpha
^M(\varepsilon)$ is an open ball centred at $x_\alpha$ of radius $\varepsilon$ such that for every $\varepsilon'<\varepsilon$ $B_\alpha(\varepsilon')$ is transverse to $f^{-1}(\zeta_\alpha)$. 
\item $H_{k-1}(B_\alpha^M(\varepsilon)\cap f^{-1}(\zeta-\zeta_\alpha))=0$ if $k\neq n$ and $H_{n-1}(B_\alpha^M(\varepsilon)\,\cap f^{-1}(\zeta-\zeta_\alpha))=\Z^\mu$, where $\mu$ is called the Milnor number.   
\end{itemize}
\color{magenta}
Let $\theta=0$, locally in a neighbourhood of a critcal point $x_\alpha$ we can choose a coordinate system $x_1,...,x_n$ such that $f(x)-f(x_\alpha)=x_1^2+...+x_n^2$. The Lefschetz thimbles $\Gamma_{\alpha,z}$ are defined as follows (see \cite{pham}): denote $x'=\mathrm{Re}(x)$ and $x''=\mathrm{Im}(x)$, then 
\begin{equation}
\mathcal{C}_{\alpha,z}\defeq\lbrace x\in X\vert x_1''=x_2''=...=x_n''=0, x_1'^2+...+x_n'^2\leq \zeta-\zeta_\alpha\rbrace
\end{equation}
\begin{figure}
\caption{Lefschetz thimbles}\label{Lefschetz_thimbles}
\end{figure}
In addition, they generate the relative homology for every critical values \textbf{Why?}. If $\theta\neq 0$ we consider $xe^{i\theta/2}$. 

\textbf{[add duality]}  
\color{black}

Therefore for $\theta$ generic, the integral $I(z)$ can be decomposed as sum over Lefschetz thimbles: we denote 
\begin{equation}\label{int_alpha}
I_{\alpha}(z)\defeq\int_{\mathcal{C}_{\alpha,z}}e^{-zf}\nu
\end{equation}
and we refer to $I_{\alpha}(z)$ as a thimble integral. Then as a consequence of Pham's duality result ($\S 5$ \cite{pham}), for every $\mathcal{C}\in H_n(X,zf)$
\begin{align*}
I(z)=\sum_{\alpha}(-1)^{n(n-1)/2}\langle \mathcal{C}_{\alpha,\theta+\pi},\mathcal{C}\rangle I_{\alpha}(z). 
\end{align*}


\subsection{Borel regularity}

Borel resummation is a way of turning a formal power series
\[ \series{\varphi} = z^\sigma \left( \frac{\varphi_0}{z} + \frac{\varphi_1}{z^2} + \frac{\varphi_2}{z^3} + \frac{\varphi_3}{z^4} + \ldots \right), \]
with $\sigma \in [0, 1)$, into a function which is asymptotic to $\series{\varphi}$ as $z \to \infty$. Different functions can be asymptotic to the same power series, and Borel resummation picks one of them, performing an implicit regularization~\textbf{[arXiv:1705.03071, or maybe arXiv:1412.6614]}. When a function matches the Borel sum of its asymptotic series, we'll say it's {\em Borel regular}. Several familiar kinds of regularity imply Borel regularity, and shed light on why it occurs.
%%Knowing that a function is Borel regular gives us extra information about it---enough to reconstruct it from its asymptotic series. What's the nature of this extra information?
%%Since different functions can be asymptotic to the same power series, Borel resummation must involve an {\em implicit regularization}, restricting its range to a class of functions which are uniquely determined by their formal power series.
\begin{itemize}
\item \textbf{Having a good asymptotic approximation}

Let $R_N$ be the difference between a function and the partial sum
\[ \frac{\varphi_0}{z} + \frac{\varphi_1}{z^2} + \frac{\varphi_2}{z^3} + \ldots + \frac{\varphi_{N-2}}{z^{N-1}} \]
of its asymptotic series. Watson showed a century ago that the function is Borel regular whenever there's a constant $c \in (0, \infty)$ with
\[ |R_N| \le \frac{c^{N+1} N!}{|z|^N} \]
over all orders $N$ and all $z$ in a wide enough wedge around infinity.
\item \textbf{Satisfying a singular differential equation}

\begin{itemize}
\item Think about conditions where this works.
\item Maybe the correct place is the setting of Ecalle's formal integral. See \S 5.2.2.1 of Delabaere's {\em Divergent Series, Summability and Resurgence III}.
\item Say there's a unique solution (up to scaling) that shrinks as you go right; everything else blows up exponentially. Then this is the only solution that can be expressed as a Laplace transform.
\item If the Borel-transformed equation has a subexponential solution $\hat{f}$ which is ``shifted holomorphic'' (we called this having a ``fractional power singularity'' in {\tt airy-resurgence}), then $\laplace \hat{f}$ satisfies the original equation, because there are no boundary terms.
\item Draw diagram showing formal vs. holomorphic solutions in time vs. frequency domains.
\end{itemize}
\item \textbf{Being a thimble integral}

Under the previous assumptions, we can now state a Borel regularity result for thimble integrals: 
 
\begin{theorem}\label{thm:borel integrals}
Borel regularity for thimbles integrals can be stated a the commutativity of the following diagram:
\begin{equation}
\begin{tikzcd}
I_{\alpha}(z)\defeq\int_{\mathcal{C}_\alpha}e^{-zf}\nu \arrow[r,"\sim"]\arrow[dd, swap, "\mathcal{L}^\theta "] & \tilde{I}_{\alpha}(z)\arrow[dd,"\mathcal{B}"]\\
& \\
\hat{\iota}_\alpha(\zeta)\arrow[r,equal,swap, "\text{sum}"] & \tilde{\iota}_{\alpha}(\zeta) 
\end{tikzcd}
\end{equation}
\end{theorem}

A priori, the Laplace transform of $\hat{\iota}_\alpha(\zeta)$ and $I_{\alpha}(z)$ have the same asymptotic behaviour in a given sector (indeed taking the asymptotic of $I_\alpha(z)$ we \textit{loose} information); however Borel regularity guarantees that $I_{\alpha}(z)=\mathcal{L}^{\theta}\hat{\iota}_{\alpha}$ in a given sector. We give a proof of Theorem \ref{thm:borel integrals} in Section ?? \ref{thm:maxim} when $N=1$, and the proof is based on the fact that $I_\alpha(z)$ can be rewritten as the Laplace transform--type integral. In higher dimension the same result was stated in \cite{Dunne-Unsal 15} (but the proof was not written).

Generically, $\tilde{I}_{\alpha}(z)$ is a divergent series whose coefficients factorially grow and, as proved by Berry and Howls \cite{BH}\cite{H} the divergence of $\tilde{I}_{\alpha}(z)$ encodes contributions from the other critical points of $f$ in the form of \textit{exact resurgence relation} (see equation 19 \cite{BH}). Indeed, the Borel transform $\hat{\iota}_\alpha(\zeta)$ yet contains information about the Borel transform $\hat{\iota}_{\beta}(\zeta)$ at other critical values $\zeta_\beta$. This is the idea of resurgence of $\tilde{I}_{\alpha}(z)$ \cite{EcalleI}\cite{EcalleII}\cite{EcalleIII}. However, since $f$ has finitely many critical points and $\nu$ is meromorphic, Borel--Laplace summability properties and resurgence of exponential integrals are straightforward to be proven. However, in order to completely describe the Borel plane one needs to compute the Stokes data which provide the analytic continuation across the branch cut. In particular for a certain class of exponential integrals the analytic theory developed by J. Ecalle to compute the Stokes data (see \cite{Ecalle}\cite{diverg-resurg-i}\cite{Dorigoni}\cite{Schiappa}) has a geometric interpretation relying on intersection theory of dual relative homology classes in the sense of F. Pham \cite{pham} (see also \cite{Maxim_lectures}\cite{Maxim_slide_ERC}).   
\end{itemize}


\subsection{Stokes phenomena for exponential integrals}
So far we have considered $\theta=\arg(z)$ fixed and generic (i.e. $\theta\neq\arg(\zeta_\alpha-\zeta_\beta)$), but as we let $\theta$ varying $I_{\alpha}(z)$ becomes a multivalued function, indeed when $\theta$ crosses a Stokes ray $\ell_{\alpha\beta}\defeq\arg(\zeta_\alpha-\zeta_\beta)\R$, $I_\alpha(z)$ jumps. Computing the jumps of $I_{\alpha}(z)$ is equivalent to determine the analytic continuation of $I_\alpha(z)$ for $z\in\C$. There are different ways to compute the Stokes jumps across the rays $\ell_{\alpha\beta}$: one is purely geometric, based on the way the decomposition \eqref{decomposition H_B} varies as $\theta$ varies. Another one is based on the resurgence analysis of the asymptotic expansion of $I_{\alpha}(z)$ as $\mathrm{Re}(z e^{i\theta})\to +\infty$. In Section \ref{sec:Stokes} we prove that the geometric and the resurgence approach to compute the Stokes jumps are equivalent:
\begin{theorem}[Theorem \ref{Stokes}]

\end{theorem}  

    
\subsubsection{Thimbles integrals [Kontsevich]: geometric computation of Stokes constants}



\subsubsection{ ODE and fractional derivative formula [draft2]}

\subsubsection{If hypergeometric functions appear in a large class of examples: integral formulas for hypergeometric functions }

\subsection{Plan of the paper}

\section{Formalism of the Laplace transform}

\section{Borel regularity}

\subsection{ODEs}

\subsection{Thimble integrals}
We are going to prove Theorem \ref{thm:borel integrals}. 
Let $X$ be a $n$-dimensional algebraic variety, $f\colon X\to\C$ be an algebraic function with simple, isolated, non-degenerate critical points, and $\nu\in\Gamma(X,\Omega^n)$, and set
\begin{equation}
I(z)\defeq\int_{\mathcal{C}}e^{-zf}\nu
\end{equation}
where $\mathcal{C}$ is a suitable contour such that the integral is well defined. Indeed, $I(z)$ represents the pairing between a relative homology class $[\mathcal{C}]\in H_{n}^{B}(X,zf)$ and a cohomology class $[\nu]\in H_{dR}^n(X,zf)$ (see Section 1.3.1 Thimble integrals in the introduction). 
Let us restrict to one dimensional $X$. For any critical points $x_\alpha$ (satisfying the previous assumptions) of $f$, the saddle point approximation allows to compute the asymptotic expansion of $I_\alpha(z)$ 
\begin{equation}\label{exp-int}
I_{\alpha}(z)\defeq\int_{\mathcal{C}_\alpha}e^{-zf}\nu\sim \tilde{I}_{\alpha}\defeq e^{-zf(x_\alpha)}\sqrt{2\pi} z^{-1/2}\sum_{k\geq 0}a_{\alpha,k}z^{-k} \qquad \text{ as } \operatorname{Re} (ze^{i\theta})\to\infty
\end{equation}
where \textcolor{magenta}{ $\mathcal{C}_\alpha$ is a steepest descent path through the critical point $x_\alpha$ and $\theta$ is chosen such that $f(x_\beta)\notin f(x_\alpha)+[0,e^{i\theta}\infty)$ for $\beta\neq\alpha$}\footnote{Such a $\theta$ exists because $f$ has a finite number of critical points.}. Notice that $f \circ \mathcal{C}_\alpha$ lies in the ray $\zeta_\alpha +[0, e^{i\theta}\infty)$, where $\zeta_\alpha := f(x_\alpha)$.

\begin{theorem}[Theorem ??]\label{thm:maxim} Let $n=1$. Let ${I}_{\alpha}(z)$ defined as in \eqref{exp-int} for every critical point $x_\alpha$. Then $\tilde{I}_\alpha$ is Borel regular for $\operatorname{Re}(ze^{i\theta})>0$:
\begin{enumerate}
\item\label{int:series-gevrey} The series $\tilde{I}_\alpha(z)=e^{-zf(x_\alpha)}\sqrt{2\pi} z^{-1/2}\sum_{k\geq 0}a_{\alpha,k}z^{-k}$ is Gevrey-1.
\item\label{int:resum-converges} The series $\tilde{\iota}_\alpha(\zeta)\defeq\mathcal{B}(\tilde{I}_\alpha)$ converges near $\zeta=\zeta_{\alpha}$.
\item\label{int:resum-valid} If you continue the sum of $\tilde{\iota}_\alpha$ along the ray going rightward from $\zeta_\alpha$ in the direction $\theta$, and take its Laplace transform along that ray, you'll recover $I_\alpha$.
\end{enumerate}
\end{theorem}

\begin{remark}
\begin{enumerate}
\item We may drop the assumption of non degenerate critical points for $f$, however the asymptotic expansion of $I_\alpha(z)$ will depend on the order $m$ such that $f^{(m)}(x_\alpha)\neq 0$ and $f^{(j)}(x_\alpha)=0$ for every $j=1,...,m-1$ (see [Zorich] Theorem 1 Section 19.2.5).  
\item in [Malgrange74] (see also Chapter 5 of [Mistergard Phd thesis] for a general review), the author computes the asymptotic expansion of exponential integrals for $n>1$ which get logarithmic terms like 
\[\tilde{I}(z)=\sum_{j\in A} \sum_{k\geq 0}\sum_{q=0}^{n-1}a_{k,q,j}z^{-k-j}(\log z)^q,\] for $A\subset\Q_{\geq 0}$ finite. Due to the presence of logarithmic terms, the definition of Borel transform has to be further extended  (see [Mistergard phd] Definition pag 5) and the study of Borel regularity becomes more involved.
\item in the proof of Theorem \ref{thm:maxim} we will derive formula \eqref{exp-int} using Watson's lemma. However, the same result can be computed from geometric arguments as in Theorem 5.3.3 [Mistergard phd].     
\end{enumerate}
\end{remark}
\begin{proof}
Part~\eqref{int:series-gevrey}: Since $f$ is Morse, we can find a holomorphic chart $\tau$ around $x_\alpha$ with $\tfrac{1}{2} \tau^2 = f - \zeta_\alpha$. Let $\mathcal{C}^-_\alpha$ and $\mathcal{C}^+_\alpha$ be the parts of $\mathcal{C}_\alpha$ that go from the past to $x_\alpha$ and from $x_\alpha$ to the future, respectively. We can arrange for $\tau$ to be valued in $(-\infty e^{i\theta}, 0]$ and $[0, e^{i\theta}\infty)$ on $\mathcal{C}^-_\alpha$ and $\mathcal{C}^+_\alpha$, respectively. \textbf{[We should explicitly spell out and check the conditions that make this possible. I think we're implicitly orienting $\mathcal{C}_\alpha$ so that $\tau$ in the upper half-plane.]} Since $\nu$ is holomorphic, we can express it as a Taylor series
\[ \nu = \sum_{k \ge 0} b_k^\alpha \tau^k\,d\tau \]
that converges in some disk $|\tau| < \varepsilon$.


In coordinates $\tau $ the integral $I_\alpha(z)$ can be approximated as 
\[ I_\alpha(z) \sim  e^{-z\zeta_\alpha}\int_{\tau \in [-\varepsilon, \varepsilon]} e^{-z\tau^2/2} \nu \]
as $\operatorname{Re}(ze^{i\theta}) \to \infty$ (see Lemma 1 in Section 19.2.2  Zorich). \textbf{[I need to learn how this works! Do we get asymptoticity at all orders? ---Aaron]} Plugging in the Taylor series above, we get
\begin{align*}
 I_\alpha(z) & \sim e^{-z\zeta_\alpha}\int_{-\varepsilon}^\varepsilon e^{-z\tau^2/2} \sum_{k \ge 0} b_k^\alpha \tau^k\,d\tau \\
& = e^{-z\zeta_\alpha}\int_{-\varepsilon}^\varepsilon e^{-z\tau^2/2} \sum_{k \ge 0} b_{2k}^\alpha \tau^{2k}\,d\tau\\
& = 2e^{-z\zeta_\alpha}\int_{0}^\varepsilon e^{-z\tau^2/2} \sum_{k \ge 0} b_{2k}^\alpha \tau^{2k}\,d\tau.
\end{align*}


By Watson's Lemma (see Lemma 4 Section 19.2.2 Zorich)

\begin{align*}
I_\alpha(z) &\sim e^{-z\zeta_\alpha}\sum_{k \ge 0} b_{2k}^\alpha \Gamma\left(k+\tfrac{1}{2}\right)2^{k+1/2}z^{-k-1/2}\\
&= e^{-z\zeta_\alpha}\sqrt{2\pi}\sum_{k \ge 0} b_{2k}^\alpha (2k-1)!!z^{-k-1/2}
\end{align*}


%By the dominated convergence theorem,\footnote{Notice that the sum over $k$ is empty when $n = 0$.}
%\begin{align*}
%I_\alpha(z) & \approx e^{-z\zeta_\alpha} \sum_{n \ge 0} b_{2n}^\alpha \int_{-\varepsilon}^\varepsilon e^{-z\tau^2/2} \tau^{2n}\,d\tau \\
%& = e^{-z\zeta_\alpha} \sum_{n \ge 0} (2n-1)!!\,b_{2n}^\alpha \left[ \sqrt{2\pi}\,z^{-(n+1/2)} \operatorname{erf}\big(\varepsilon \sqrt{z/2}\big) - 2e^{-z\varepsilon^2/2} \sum_{k=1}^n \frac{\varepsilon^{2k-1}}{(2k-1)!!} z^{-n+k-1} \right].
%\end{align*}

%The annoying $e^{-z\varepsilon^2/2}$ correction terms are dwarfed by their $z^{-(n+1/2)}$ counterparts when $z$ is large. These terms are crucial, however, for the convergence of the sum. To see why, consider their absolute sum $C_\text{exp}$. When $z \in [0, \infty)$,
%\begin{align*}
%C_\text{exp} & = -2e^{-z\varepsilon^2/2} \sum_{n \ge 1} (2n-1)!!\,\left| b_{2n}^\alpha \sum_{k=1}^n \frac{\varepsilon^{2k-1}}{(2k-1)!!} z^{-(n-k+1)} \right| \\
%& =-2e^{-z\varepsilon^2/2} \sum_{n \ge 1} (2n-1)!!\,\left|b_{2n}^\alpha\right| \sum_{k=1}^n \frac{\varepsilon^{2k-1}}{(2k-1)!!} z^{-(n-k+1)} \\
%& \ge -2\varepsilon e^{-z\varepsilon^2/2} \sum_{n \ge 1} (2n-1)!!\,\left|b_{2n}^\alpha\right| nz^{-n},
%\end{align*}
%which diverges for typical $f$ and $\nu$. \textbf{[Does it? Veronica points out that we expect $b_{2n}$ to shrink at least as fast as $(n!)^{-1}$.]}

%This argument suggests that no matter how tiny the correction terms get, we can't expect to swat them all aside. We can, however, set aside any finite set of them. \color{violet}\textbf{[Use Miller's proof of Watson's lemma in place of the following argument, which has a few soft spots. See also Loday-Richaud, \S 5.1.5, Theorem~5.1.3]} For each cutoff $N$, the tail
%\[ \sum_{n \ge N} b_{2n}^\alpha \int_{-\varepsilon}^\varepsilon e^{-z\tau^2/2} \tau^{2n}\,d\tau \]
%
%\color{CarnationPink}
%For each cutoff $N$, the tail error \textbf{[check]}
%\begin{align*}
%\left| \sum_{n \ge N} b_{2n}^\alpha \int_{-\varepsilon}^\varepsilon e^{-z\tau^2/2} \tau^{2n}\,d\tau \right| & \le \sum_{n \ge N} \left| b_{2n}^\alpha \right| \int_{-\varepsilon}^\varepsilon e^{-|z|\tau^2/2} \tau^{2n}\,d\tau \\
%& \le \sum_{n \ge N} \left| b_{2n}^\alpha \right| \int_{-\infty}^\infty e^{-|z|\tau^2/2} \tau^{2n}\,d\tau \\
%& = \sqrt{2\pi} \sum_{n \ge N} (2n-1)!!\,\left| b_{2n}^\alpha \right| |z|^{-(n+1/2)} \\
%& \lesssim \sum_{n \ge N} (2n-1)!!\,\varepsilon^{-2n} |z|^{-(n+1/2)} \\
%& = \varepsilon \sum_{n \ge N} (2n-1)!!\,\big(\varepsilon^{-1}\big)^{2n+1} \big(|z|^{-1/2}\big)^{2n+1} \\
%& = \varepsilon \sum_{n \ge N} (2n-1)!!\,\big(\varepsilon^{-1} |z|^{-1/2}\big)^{2n+1} \\
%& = \textbf{uh-oh!}
%\end{align*}
%\color{violet}
%is in $o_{z \to \infty}(z^{-N})$ \textbf{[check]}, and the absolute sum
%\begin{align*}
%C_\text{exp}^N & = 2e^{-\operatorname{Re}(z)\varepsilon^2/2} \sum_{n = 1}^{N-1} (2n-1)!!\,\left| b_{2n}^\alpha \sum_{k=1}^n \frac{\varepsilon^{2k-1}}{(2k-1)!!} z^{-(n-k+1)} \right| \\
%& \le 2e^{-\operatorname{Re}(z)\varepsilon^2/2} \sum_{n = 1}^{N-1} (2n-1)!!\,\left|b_{2n}^\alpha\right| \sum_{k=1}^n \frac{\varepsilon^{2k-1}}{(2k-1)!!} |z|^{-(n-k+1)} \\
%& \leq 2\varepsilon e^{-\operatorname{Re}(z)\varepsilon^2/2}\sum_{n=1}^{N-1}(2n-1)!!|b_{2n}^\alpha|n z^{-n}\\
%%& \ge -2\varepsilon e^{-z\varepsilon^2/2} \sum_{n \ge 1} (2n-1)!!\,\left|b_{2n}^\alpha\right| z^{-n},
%\end{align*}
%is in $o_{z \to \infty}(z^{-m})$ for every $m$ \textbf{[check]}.\color{black} Hence,
%\[  I_\alpha(z) \sim e^{-z\zeta_\alpha}\sqrt{2\pi} \sum_{n \ge 0} (2n-1)!!\,b_{2n}^\alpha\,z^{-(n+1/2)} \operatorname{erf}\big(\varepsilon \sqrt{z/2}\big). \]
%The differences $1 - \operatorname{erf}\big(\varepsilon \sqrt{z/2}\big)$ shrink exponentially as $z$ grows, allowing the simpler estimate
%\[  I_\alpha(z) \sim e^{-z\zeta_\alpha}\sqrt{2\pi} \sum_{n \ge 0} (2n-1)!!\,b_{2n}^\alpha\,z^{-(n+1/2)}. \]
Call the right-hand side $\tilde{I}_\alpha$. We now see that $a_{\alpha,k} = (2k-1)!!\,b_{2k}^\alpha$ in the statement of the theorem. We know from the definition of $\varepsilon$ that $\left|b_k^\alpha\right| \varepsilon^k \lesssim 1$. Recalling that $(2k - 1)!! \sim (\pi k)^{-1/2}\,4^k\,k!$ as $k \to \infty$, we deduce that $|a_{\alpha,k}| \lesssim \left(\tfrac{4}{\varepsilon^2}\right)^k\,k!\,$, showing that $\tilde{I}_\alpha$ is Gevrey-1.%



Part~\eqref{int:resum-converges}: note that \textbf{[explain formally what it means to center at $\zeta_\alpha$]}
\begin{align*}
\tilde{\iota}_{\alpha}\defeq \mathcal{B}_{\zeta_\alpha} \tilde{I}_\alpha & = \sqrt{2\pi} \sum_{k \ge 0} (2k-1)!!\,b_{2k}^\alpha\,\frac{(\zeta - \zeta_\alpha)^{k-1/2}}{\Gamma\big(k+\tfrac{1}{2}\big)} 
\end{align*}

Since ${(2k-1)!!}=\pi^{-1/2} 2^k{\Gamma\left(k+\tfrac{1}{2}\right)}$ and $|b_k^\alpha|\epsilon^n\lesssim 1$, then $\tilde{\iota}_{\alpha}(\zeta)$ has a finite radius of convergence. 

%\begin{multline*}
%\tilde{\varphi}_\alpha(\zeta)=\mathcal{B}\left(e^{-zf(x_\alpha)}(2\pi)^{1/2} \sum_{n\geq 0}a_{\alpha,n}z^{-n}\right)(\zeta)=T_{f(x_\alpha)}(2\pi)^{1/2} \left(\delta a_{\alpha,0}+\sum_{n\geq 0}a_{\alpha,n+1}\frac{\zeta^n}{n!}\right)\\
%=(2\pi)^{1/2} \left(\delta(f_{x_\alpha}) a_{\alpha,0}+\sum_{n\geq 0}a_{\alpha,n+1}\frac{(\zeta-f(x_\alpha))^n}{n!}\right)
%\end{multline*}
%Since $|a_{\alpha,n}|\leq C_\alpha A_\alpha^nn!$, the series $\sum_{n\geq 0}a_{n+1}\frac{(\zeta-f(x_\alpha))^n}{n!}$ has a finite radius of convergence. 

Part~\eqref{int:resum-valid}: Let's recast the integral $I_\alpha$ into the $f$ plane. As $\zeta$ goes rightward from $\zeta_\alpha$, the start and end points of $\mathcal{C}_\alpha(\zeta)$ sweep backward along $\mathcal{C}^-_\alpha(\zeta)$ and forward along $\mathcal{C}^+_\alpha(\zeta)$, respectively. Hence, we have
\begin{align*}
I_\alpha(z) & = \int_{\mathcal{C}_{\alpha}} e^{-zf} \nu \\
&=\int_{\mathcal{H}_{\alpha}}e^{-z\zeta}\left(\int_{f^{-1}(\zeta)}\frac{\nu}{df}\right)d\zeta \\
& = \int_{\zeta_\alpha}^{e^{i\theta}\infty} e^{-z\zeta} \left[\frac{\nu}{df}\right]_{\operatorname{start} \mathcal{C}_\alpha(\zeta)}^{\operatorname{end} \mathcal{C}_\alpha(\zeta)}\,d\zeta.
\end{align*}
where $\mathcal{H}_{\alpha}$ is the Hanckel contour through the point $\zeta_{\alpha}$ (see Figure \cite{fig.paths}) with ends in the $\theta$ direction.
\begin{figure}
\caption{The contour $\mathcal{C}_\alpha$, its image under $f$ which is the Hankel contour $\mathcal{H}_{\alpha}=f(\mathcal{C}_{\alpha})$ and the ray $[\zeta_\alpha,+\infty]$. }
\end{figure}   
Noticing that the last integral is a Laplace transform for the initial choice of $\theta$, we learn that
\begin{equation}\label{thimble-difference}
\hat{\iota}_\alpha(\zeta) = \left[\frac{\nu}{df}\right]_{\operatorname{start} \mathcal{C}_\alpha(\zeta)}^{\operatorname{end} \mathcal{C}_\alpha(\zeta)}.
\end{equation}
In Ecalle's formalism, $\overset{\triangledown}{\iota}_\alpha\defeq\int_{f^{-1}(\zeta)}\frac{\nu}{df}$ and $\hat{\iota}_\alpha$ are respectively a major and a minor of the singularity and they differ by an holomorphic function (we will see this in the examples Section Airy, Bessel). 


We can rewrite our Taylor series for $\nu$ as
\begin{align*}
\nu & = \sum_{k \ge 0} b_n^\alpha [2(f - \zeta_\alpha)]^{k/2}\,\frac{df}{[2(f - \zeta_\alpha)]^{1/2}} \\
& = \sum_{k \ge 0} b_n^\alpha [2(f - \zeta_\alpha)]^{(k - 1)/2}\,df,
\end{align*}
taking the positive branch of the square root on $\mathcal{C}^+_\alpha$ and the negative branch on $\mathcal{C}^-_\alpha$. Plugging this into our expression for $\hat{\iota}_\alpha$, we learn that
\begin{align*}
\hat{\iota}_\alpha(\zeta) & = \left[ \sum_{k \ge 0} b_k^\alpha [2(f - \zeta_\alpha)]^{(k - 1)/2} \right]_{\operatorname{start} \mathcal{C}_\alpha(\zeta)}^{\operatorname{end} \mathcal{C}_\alpha(\zeta)} \\
& = \sum_{k \ge 0} b_n^\alpha \Big( [2(\zeta - \zeta_\alpha)]^{(k - 1)/2} - (-1)^{k-1}[2(\zeta - \zeta_\alpha)]^{(k - 1)/2} \Big) \\
& = \sum_{k \ge 0} 2 b_{2k}^\alpha [2(\zeta - \zeta_\alpha)]^{k - 1/2} \\
& = \sum_{k \ge 0} 2^{k+1/2} b_{2k}^\alpha (\zeta - \zeta_\alpha)^{k - 1/2} \\
& = \mathcal{B}_{\zeta_\alpha} \tilde{I}_\alpha.
\end{align*}
We have now shown that the sum of $\mathcal{B}_{\zeta_\alpha} \tilde{I}_\alpha$ is actually equal to $\hat{\iota}_\alpha$ as $\zeta\in\zeta_\alpha+[0,e^{i\theta}\infty)$.
\end{proof}

\begin{remark}
Different choices of admissible $\theta$ correspond to different choices of thimbles $[\mathcal{C}_{\alpha}]\in H_n^{B}(X,zf)$, but the Borel transform of $\tilde{I}_{\alpha}$ does not depend on $\theta$. However, if $\theta_*\defeq\arg(\zeta_{\alpha}-\zeta_{\beta})$ and $\theta_{\pm}\defeq\theta_*\pm\delta$ for small $\delta$, then $I_{\alpha}(z)$ jumps on the intersection between $\operatorname{Re}(e^{i\theta_+}z)>0$ and $\operatorname{Re}(e^{i\theta_-}z)>0$. This is known as the Stokes phenomenon (see Section resurgence thimbles integrals).  
\end{remark}

\subsubsection{$3/2$ derivative formula}

In Theorem \ref{thm:maxim} we have seen that the asymptotic behaviour of $I_\alpha(z)$ has a fractional power contribution, namely \[\tilde{I}_{\alpha}(z)=e^{-z\zeta_\alpha}z^{-1/2}\sqrt{2\pi}\sum_{k\geq 0}a_{\alpha,k}z^{-k},\] hence we have used the extended notion of Borel transform to deal with fractional powers. Now we will focus on the formal series $\tilde{\Phi}_\alpha(z)\defeq e^{-z\zeta_\alpha}\sqrt{2\pi}\sum_{k\geq 0}a_{\alpha,k}z^{-k}=z^{1/2}\tilde{I}_\alpha(z)$ which does not contain any fractional power and we prove a fractional derivative formula which relates the Borel transforms $\hat{\varphi}_\alpha(\zeta)$ and $\hat{\iota}_{\alpha}(\zeta)$. Moreover we show that the $\hat{\varphi}_{\alpha}(\zeta)$ depends on $\nu$ and $df$ as well as $\hat{\iota}_{\alpha}(\zeta)$ does. 

\begin{corollary}\label{int:deriv-formula} 
Under the same assumptions of Theorem \ref{thm:maxim}, for any $\zeta$ on the ray going rightward from $\zeta_\alpha$ in the direction of $\theta$, we have
\begin{multline}\label{formula1}
\hat{\varphi}_{\alpha}(\zeta)=\partial^{3/2}_{\zeta \text{ from }\zeta_\alpha} \left( \int_{\mathcal{C}_\alpha(\zeta)}\nu \right)={\left(\tfrac{\partial}{\partial \zeta}\right)^2}\,\frac{1}{\Gamma\big(\tfrac{1}{2}\big)} \int_{\zeta_\alpha}^\zeta (\zeta-\zeta')^{-1/2} {\left( \int_{\mathcal{C}_\alpha(\zeta')} \nu \right)}\,d\zeta',
\end{multline}
where $\mathcal{C}_\alpha(\zeta)$ is the part of $\mathcal{C}_\alpha$ that goes through $e^{-i\theta}f^{-1}([\zeta_\alpha, \zeta ])$. Notice that $\mathcal{C}_\alpha(\zeta)$ starts and ends in $e^{-i\theta}f^{-1}(\zeta)$. \textbf{[Be careful about the orientation of $\mathcal{C}_\alpha$.]}
\end{corollary}

\begin{proof}
Theorem~\ref{thm:frac-diff-borel} tells us that
\begin{align*}
\mathcal{B}_{\zeta_\alpha} \tilde{I}_\alpha  = \mathcal{B}_{\zeta_\alpha} z^{-1/2} \tilde{\varphi}_\alpha = \partial^{-1/2}_{\zeta \text{ from } \zeta_\alpha} \mathcal{B} \tilde{\varphi}_\alpha = \partial^{-1/2}_{\zeta \text{ from } \zeta_\alpha} \hat{\varphi}_\alpha.
\end{align*}
It follows, from the proof of part $3$ of Theorem \ref{thm:maxim}, that
\begin{equation}\label{shifted-resum-valid}
\hat{\iota}_\alpha(\zeta) = \partial^{-1/2}_{\zeta \text{ from } \zeta_\alpha} \hat{\varphi}_\alpha.
\end{equation}
Since fractional integrals form a semigroup, equation~\eqref{shifted-resum-valid} implies that
\[ \partial^{-1}_{\zeta \text{ from } \zeta_\alpha} \hat{\iota}_\alpha(\zeta) = \partial^{-3/2}_{\zeta \text{ from } \zeta_\alpha} \hat{\varphi}_\alpha. \]
Rewriting equation~\eqref{thimble-difference} as
\[ \hat{\iota}_\alpha(\zeta) = \partial_\zeta \left( \int_{\mathcal{C}_\alpha(\zeta)} \nu \right), \]
we can see that
\[ \partial^{-1}_{\zeta \text{ from } \zeta_\alpha} \hat{\iota}_\alpha(\zeta) = \int_{\mathcal{C}_\alpha(\zeta)} \nu - \int_{\mathcal{C}_\alpha(0)} \nu. \]
The initial value term vanishes, because the path $\mathcal{C}_\alpha(0)$ is a point. Hence,
\[ \int_{\mathcal{C}_\alpha(\zeta)} \nu = \partial^{-3/2}_{\zeta \text{ from } \zeta_\alpha} \hat{\varphi}_\alpha(\zeta). \]
Recalling that the Riemann-Liouville fractional derivative is a left inverse of the fractional integral, we conclude that
\[ \partial^{3/2}_{\zeta \text{ from } \zeta_\alpha} \left( \int_{\mathcal{C}_\alpha(\zeta)} \nu \right) = \hat{\varphi}_\alpha(\zeta). \]
\end{proof}


\subsubsection{Singularities} 
From equation \eqref{shifted-resum-valid} we see that singularities of $\hat{\iota}_{\alpha}(\zeta)$ in the Borel plane comes from either poles of $\nu$ or zeros of $df$. Instead, the fractional derivatives formula tells that singularities of $\hat{\varphi}_\alpha$ are given by convolutions of $\zeta^{-1/2}/\Gamma(1/2)$ with $\hat{\iota}_{\alpha}$. Since $\zeta^{-1/2}/\Gamma(1/2)$ is singular at $\zeta=0$ the set of singularities of $\hat{\varphi}_{\alpha}(\zeta)$ is exactly the same as the one of $\hat{\iota}_{\alpha}(\zeta)$. However, the type of singularities will change and we expect $\hat{\varphi}_{\alpha}(\zeta)$ to have only simple singularities.

In the examples we noticed that $\hat{\varphi}_{\alpha}(\zeta)$ is always an hypergeometric function. In particular when there are only two critical values (see Airy, Bessel) the $\hat{\varphi}_{\alpha}(\zeta)$ is a Gaussian hypergeometric function ${}_2F_1\left(a,b;c;\tfrac{\zeta}{\zeta_\alpha}\right)$ with $c=2$ and $a+b=c+1$. Whereas, in the generalized Airy example (see Section ??) we get generalized hypergeometric functions ${}_3F_2\left(\mathbf{a};\mathbf{b};(\tfrac{\zeta}{\zeta_\alpha}-1)^2\right)$ and ${}_3F_2\left(\mathbf{a}_0;\mathbf{b}_0;(\tfrac{\zeta}{\zeta_\alpha})^2\right)$ with $|\mathbf{a}|=|\mathbf{b}|+1$. This behaviour reflects the resurgence properties of $\hat{\varphi}_{\alpha}$ (as well as the one of $\hat
{\iota}_{\alpha}$), meaning the analytic continuation of $\hat{\varphi}_{\alpha}(\zeta)$ at $\zeta_\alpha$ is given in terms of $\hat{\varphi}_{\beta}(\zeta)$, $\zeta_\beta\neq\zeta_\alpha$ when $\hat{\varphi}_{\alpha}(\zeta), \hat{\varphi}_{\beta}(\zeta)$ are hypergeometric functions of the previous type.

\begin{lemma}
Let us assume $f$ has only two critical values $\zeta_\alpha=-\zeta_\beta$ and let $\hat{\varphi}_{\alpha}(\zeta)={}_2F_1(a,b;2;\tfrac{\zeta}{\zeta_\alpha})$ with $a+b=c+1$, then across the branch cut 
\begin{equation}
\hat{\varphi}_{\alpha}(\zeta+i0)-\hat{\varphi}_{\alpha}(\zeta-i0)=C\,\,{}_2F_1\left(a,b;2;1+\tfrac{\zeta}{\zeta_{\beta}}\right)
\end{equation}
\begin{equation}
\hat{\varphi}_{\beta}(\zeta+i0)-\hat{\varphi}_{\beta}(\zeta-i0)=-C\,\,{}_2F_1\left(a,b;2;1+\tfrac{\zeta}{\zeta_{\alpha}}\right)
\end{equation}
\end{lemma}
\begin{proof}
It follows from DLMF eq. 15.2.2. 
\end{proof}

It would be interesting to further investigate the relationship between the properties of resurgent functions (with finitely many singularities in the Borel plane) and hypergeometric functions.   

\subsubsection{Contour argument}

As noticed in proof of Theorem \ref{thm:maxim}, the integral $I_{\alpha}(z)$ can be written as 
\begin{enumerate}
\item[$(i)$] the Laplace transform of $\hat{\iota}_{\alpha}(\zeta)$
\item[$(ii)$] the Hankel contour integral of the major $\overset{\triangledown}{\iota}_\alpha(\zeta)$
\end{enumerate}
and $\overset{\triangledown}{\iota}_\alpha(\zeta)=\hat{
\iota}_{\alpha}(\zeta)+\text{hol.fct.}$. In the applications we have evidence that $\overset{\triangledown}{\iota}_\alpha(\zeta)$ is an algebraic hypergeometric function and when there are only two critical values, it decomposes as a sum of two germs of holomorphic functions at each critical values respectively (see {\tt airy-resurgence} Section 6.1, 6.3). 

\section{Stokes phenomenon for thimble integrals}

\section{Examples}   

\bibliographystyle{amsplain}
\bibliography{airy-resurgence} 

\end{document}