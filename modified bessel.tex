\documentclass{article}

\usepackage{url}
\usepackage[hmargin=1.5in]{geometry}
\usepackage{amsmath}
\usepackage{amssymb}
\usepackage{graphicx}
\usepackage{xcolor}
% convenience aliases
\newcommand{\maps}{\colon}

% symbology
\newcommand{\Z}{\mathbb{Z}}
\newcommand{\R}{\mathbb{R}}
\newcommand{\C}{\mathbb{C}}
\newcommand{\laplace}{\mathcal{L}}
%\renewenvironment{proof}{{\scshape Proof.}}{\qed}

\makeatletter
\newenvironment{proofof}[1]{\par
  \pushQED{\qed}%
  \normalfont \topsep6\p@\@plus6\p@\relax
  \trivlist
  \item[\hskip3\labelsep
        \itshape
    Proof of #1\@addpunct{.}]\ignorespaces
}{%
  \popQED\endtrivlist\@endpefalse
}
\makeatother

% Def
%\def\be{\begin{equation}}    
%\def\ee{\end{equation}}
\def\into{\hookrightarrow}
\def\onto{\twoheadrightarrow}
\def\isom{\cong}  
\def\ra{\rightarrow}
\def\lra{\longrightarrow}
\def\surj{\twoheadrightarrow}
\def\Var{\mathrm{Var}}
\def\Sch{\mathrm{Sch}}
\def\Sets{\mathrm{Sets}}
\def\Def{\mathsf{Def}}
\def\KS{\mathsf{KS}}
\def\ad{\mathsf{ad}}
\def\St{\mathrm{St}}
\def\st{\mathrm{st}}

\def\L{\mathbb L}
\def\A{\mathcal A}
\def\B{\mathcal B}
\def\R{\mathbb R}
\def\C{\mathbb C}
\def\D{\mathbb D}
\def\P{\mathbb P}
\def\Q{\mathbb Q}
\def\G{\mathbb G}
\def\L{\mathbb{L}}
\def\SS{\mathcal S}
\def\RR{\mathbf R}
\def\X{\mathcal X}
\def\E{\mathcal E}
\def\Z{\mathbb Z}
\def\N{\mathbb N}
\def\ext{\mathrm{ext}}
\def\FF{\mathscr{F}}

\def\HS{\mathsf{HS}}
\def\O{\mathscr O}
\def\DDT{\mathsf{DT}}
\def\PPT{\mathsf{PT}}
\def\LL{\mathsf{L}}
\def\NN{\mathsf{N}}
\def\sc{\textrm{sc}}
\def\dcr{\textrm{d-crit}}
\def\loc{\textrm{loc}}
\def\Ad{\textrm{Ad}}
\def\reg{\textrm{reg}}
\def\red{\textrm{red}}
\def\relvir{\textrm{relvir}}
\def\pur{\textrm{pur}}
\def\vd{\mathrm{vd}}
\def\pure{\textrm{pure}}
\def\MF{\mathsf{MF}}
\def\WW{\mathsf{W}}
\def\HH{\mathsf{H}}
\def\h{\mathfrak{h}}
\def\at{\mathsf A}
\def\pt{\mathrm{pt}}

\def\CC{\mathrm{C}}
\def\KK{\mathrm{K}}
\DeclareMathOperator{\Mod}{Mod}
\DeclareMathOperator{\op}{op}
\DeclareMathOperator{\Tor}{Tor}
\DeclareMathOperator{\Mor}{Mor}
\DeclareMathOperator{\Fun}{Fun}
\DeclareMathOperator{\Vect}{Vect}
\DeclareMathOperator{\FDVect}{FDVect}
\DeclareMathOperator{\Rings}{Rings}
\DeclareMathOperator{\ev}{ev}
\DeclareMathOperator{\Quot}{Quot}
\DeclareMathOperator{\DD}{D}
\DeclareMathOperator{\Hilb}{Hilb}
\DeclareMathOperator{\Chow}{Chow}
\DeclareMathOperator{\Orb}{Orb}
\DeclareMathOperator{\Ob}{Ob}
\DeclareMathOperator{\ob}{ob}
\DeclareMathOperator{\Jac}{Jac}
\DeclareMathOperator{\ch}{ch}
\DeclareMathOperator{\Td}{Td}
\DeclareMathOperator{\tr}{tr}
\DeclareMathOperator{\id}{id}
\DeclareMathOperator{\Pic}{Pic}
\DeclareMathOperator{\codet}{codet}
\DeclareMathOperator{\Rep}{Rep}
\DeclareMathOperator{\Bl}{Bl}
\DeclareMathOperator{\ord}{ord}
\DeclareMathOperator{\aff}{aff}
\DeclareMathOperator{\vir}{vir}
\DeclareMathOperator{\QCoh}{QCoh}
\DeclareMathOperator{\Coh}{Coh}
\DeclareMathOperator{\Span}{Span}
\DeclareMathOperator{\mult}{mult}
\DeclareMathOperator{\Spec}{Spec\,}
\DeclareMathOperator{\Proj}{Proj\,}
\DeclareMathOperator{\Supp}{Supp\,}
\DeclareMathOperator{\coker}{coker}
\DeclareMathOperator{\Cone}{Cone}
\DeclareMathOperator{\Perf}{Perf}
\DeclareMathOperator{\im}{im}
\DeclareMathOperator{\DT}{DT}
\DeclareMathOperator{\PT}{PT}
\DeclareMathOperator{\RRR}{R}
\DeclareMathOperator{\GL}{GL}
\DeclareMathOperator{\SL}{SL}
\DeclareMathOperator{\dd}{d}
\DeclareMathOperator{\Tr}{Tr}
\DeclareMathOperator{\NCHilb}{NCHilb}
\DeclareMathOperator{\Sym}{Sym}
\DeclareMathOperator{\Aut}{Aut}
\DeclareMathOperator{\Ext}{Ext}
\DeclareMathOperator{\lExt}{{\mathscr Ext}}
\DeclareMathOperator{\Hom}{Hom}
\DeclareMathOperator{\lHom}{{\mathscr Hom}}
\DeclareMathOperator{\catA}{{\mathscr A}}
\DeclareMathOperator{\catB}{{\mathscr B}}
\DeclareMathOperator{\catC}{{\mathcal C}}
\DeclareMathOperator{\catD}{{\mathcal D}}
\DeclareMathOperator{\catT}{{\mathscr T}}
\DeclareMathOperator{\catF}{{\mathscr F}}
\DeclareMathOperator{\End}{End}
\DeclareMathOperator{\Eu}{Eu}
\DeclareMathOperator{\Exp}{Exp}
\DeclareMathOperator{\rk}{rk}
\DeclareMathOperator{\Nil}{Nil}
\DeclareMathOperator{\Tot}{Tot}
\DeclareMathOperator{\length}{length}
\DeclareMathOperator{\codim}{codim}
\DeclareMathOperator{\pr}{pr}
%\DeclareMathOperator{\at}{at}
\DeclareMathOperator{\Art}{Art}
\DeclareMathOperator{\uC}{\underline{\mathcal C}}
\DeclareMathOperator{\uA}{\underline{\mathscr A}}
\DeclareMathOperator{\F}{\mathcal F}
\DeclareMathOperator{\hh}{H}%Da togliere quando corregger� il capitolo 4
\DeclareMathOperator{\Der}{Der}
\DeclareMathOperator{\Ab}{Ab}


%%%%%%%%%%%%%%%%
%\theoremstyle{definition}
%
%\newtheorem*{lemma*}{Lemma}
%\newtheorem*{theorem*}{Theorem}
%\newtheorem*{example*}{Example}
%\newtheorem*{fact*}{Fact}
%\newtheorem*{notation*}{Notation}
%\newtheorem*{definition*}{Definition}
%\newtheorem*{prop*}{Proposition}
%\newtheorem*{remark*}{Remark}
%\newtheorem*{corollary*}{Corollary}
%\newtheorem*{conventions*}{Conventions}
%\newtheorem*{caution*}{Caution}

%\newtheorem{definition}{Definition}[section]
%\newtheorem{problem}[definition]{Problem}
%\newtheorem{example}[definition]{Example}
%\newtheorem{fact}[definition]{Fact}
%\newtheorem{aside}[definition]{Aside}
%%\newtheorem{prop}[definition]{Proposition}
%\newtheorem{question}[definition]{Question}
%\newtheorem{remark}[definition]{Remark}
%\newtheorem{theorem}[definition]{Theorem}
%%\newtheorem{corollary}[definition]{Corollary}
%\newtheorem{lemma}[definition]{Lemma}
%%\newtheorem{conjecture}[definition]{Conjecture}
%\newtheorem{claim}[definition]{Claim}
%\newtheorem{exercise}[definition]{Exercise}

%\newtheoremstyle{thm} % <name> % (ambienti con dimostrazione)
%        {3mm}% <Space above>
%        {3mm}% <Space below>
%        {\slshape}% <Body font> % 
%        {0mm}% <Indent amount>
%        {\bfseries}% <Theorem head font>
%        {.}% <Punctuation after theorem head>
%        {1mm}% <Space after theorem head>
%        {}% <Theorem head spec (can be left empty, meaning 'normal')> 
%\theoremstyle{thm}
%\newtheorem{theorem}[definition]{Theorem}
%\newtheorem{corollary}[definition]{Corollary}
%\newtheorem{lemma}[definition]{Lemma}
%\newtheorem{prop}[definition]{Proposition}
%\newtheorem{thm}{Theorem}
%\newtheorem{notation}{Notation}
%\renewcommand*{\thethm}{\Alph{thm}}



%\newtheoremstyle{sol} % <name> % (ambienti con dimostrazione)
%        {3mm}% <Space above>
%        {3mm}% <Space below>
%        {\normalfont}% <Body font> % 
%        {0mm}% <Indent amount>
%        {\scshape}% <Theorem head font>
%        {.}% <Punctuation after theorem head>
%        {1mm}% <Space after theorem head>
%        {}% <Theorem head spec (can be left empty, meaning 'normal')> 
%\theoremstyle{sol}
%\newtheorem{slogan}[definition]{Slogan}
%\newtheorem{assumption}[definition]{Assumption}
%%\newtheorem{claim}[definition]{Claim}
%\newtheorem{notation}[definition]{Notation}
%\newtheorem*{ssolution*}{Solution (sketch)}
%\newtheorem*{solution*}{Solution}


%%%%%%%%%%%%%%%%%%%%%%%%%

\usepackage{tikz}
\usepackage{tikz-cd}
\usepackage{rotating}
\newcommand*{\isoarrow}[1]{\arrow[#1,"\rotatebox{90}{\(\sim\)}"
]}
\usetikzlibrary{matrix,shapes,arrows,decorations.pathmorphing}
\tikzset{commutative diagrams/arrow style=math font}
\tikzset{commutative diagrams/.cd,
mysymbol/.style={start anchor=center,end anchor=center,draw=none}}
\newcommand\MySymb[2][\square]{%
  \arrow[mysymbol]{#2}[description]{#1}}
\tikzset{
shift up/.style={
to path={([yshift=#1]\tikztostart.east) -- ([yshift=#1]\tikztotarget.west) \tikztonodes}
}
}

\DeclareMathAlphabet{\mathpzc}{OT1}{pzc}{m}{it}

\newcommand*{\defeq}{\mathrel{\vcenter{\baselineskip0.5ex \lineskiplimit0pt
                     \hbox{\scriptsize.}\hbox{\scriptsize.}}}%
                     =}
\newcommand*{\defeqin}{\mathrel{\vcenter{\lineskiplimit0pt\baselineskip0.5ex
                     \hbox{\scriptsize.}\hbox{\scriptsize.}}}%
                     =}                     


\title{Resurgence of modified Bessel functions of second kind}
\author{Veronica Fantini}

\begin{document}
\maketitle

\section{Modified Bessel function of second kind}

The modified Bessel function of the second kind $K_\mu(z)$ is a solution of the equation
\begin{equation}\label{Bessel_nu}
w''+\frac{w'}{z}-w-\frac{\mu^2}{z^2}w=0
\end{equation}
such that $K_{\mu}(z)\sim\sqrt{\pi/(2z)}e^{-z}$ as $z\to\infty$ in $|\arg z|<\frac{3\pi}{2}$\footnote{A system of solution of Bessel equation is given by $I_\mu(z)$ and $K_\mu(z)$. In particular, their asymptotic behaviour as $z\to\infty$ is given by \begin{align}
\tilde{I}_\mu(z)&=\frac{1}{\sqrt{2\pi}}e^zz^{-1/2}\sum_{k\geq 0}\frac{\left(\frac{1}{2}-\mu\right)_k\left(\frac{1}{2}+\mu\right)_k}{2^kk!}z^{-k}\\
\tilde{K}_{\mu}(z)&=\sqrt{\frac{\pi}{2}}e^{-z}z^{-1/2}\sum_{k\geq 0}\frac{\left(\frac{1}{2}-\mu\right)_k\left(\frac{1}{2}+\mu\right)_k}{(-2)^kk!}z^{-k}
\end{align}}. It has a branch point at $z=0$ for every $\mu\in\C$ and  the principal branch is analytic in $\C\setminus(-\infty,0]$. 
\subsection{Differential equation}
From the general theory of ODE, the formal integral solution of \eqref{Bessel_nu} is two parameters family 
\begin{equation}
\tilde{w}(z)=U_1e^{-z}z^{-1/2}\tilde{w}_{\mu,+}(z)+U_2e^{z}z^{-1/2}\tilde{w}_{\mu,-}(z)
\end{equation} 
where $\tilde{w}_{\mu,\pm}=\sum_{j\geq 0}a_{\pm,j}z^{-j}\in\C[\![z^{-1}]\!]$ are unique formal solutions of
\begin{align*}
\tilde{w}_{\mu,+}''-2\tilde{w}_{\nu,+}'+\frac{\tilde{w}_{\mu,+}}{4z^2}-\frac{\mu^2}{z^2}\tilde{w}_{\mu,+}=0\\
\tilde{w}_{\mu,-}''+2\tilde{w}_{\mu,-}'+\frac{\tilde{w}_{\mu,-}}{4z^2}-\frac{\mu^2}{z^2}\tilde{w}_{\mu,-}=0
\end{align*} 
In particular, $\tilde{K}_{\mu}(z)=\sqrt{\frac{\pi}{2}}U_1e^{-z}z^{-1/2}\tilde{w}_{\mu,+}(z)$ and $\tilde{I}_\mu(z)=\frac{1}{\sqrt{2\pi}}U_2 e^zz^{-1/2}\tilde{w}_{\mu,-}(z)$ (we assume $a_{\pm,0}=1$) for some constants $U_1, U_2$. We now compute the Borel transform of $\tilde{w}_+(z)$\footnote{We do not consider constant term of $\tilde{w}_{\mu,\pm}$, i.e. $\mathcal{B}:\C[\![z^{-1}]\!]\to \C[\zeta]$.}: it is a solution of 
\begin{align*}
&\zeta^2\hat{w}_{\mu,+}+2t\hat{w}_{\mu,+}+\left(\frac{1}{4}-\nu^2\right)\int_0^{\zeta}(\zeta-s)\hat{w}_{\nu,+}(s)ds=0 &\\
&\zeta^2\hat{w}_{\mu,+}''+2\zeta\hat{w}_+''+4\zeta\hat{w}_{\mu,+}'+\left(\frac{9}{4}-\mu^2\right)\hat{w}_{\mu,+}=0 & \\
&t(1-t)\hat{w}_{\mu,+}''+(2-4t)\hat{w}_{\mu,+}'-\left(\frac{9}{4}-\mu^2\right)\hat{w}_{mu,+}=0 & t=-\frac{\zeta}{2}
\end{align*} 
therefore $\hat{w}_{\mu,+}(\zeta)$ is an hypergeometric function
\begin{equation}
\hat{w}_{\mu,+}(\zeta)=c_{\mu,+} {}_2F_1\left(\frac{3}{2}-\mu,\frac{3}{2}+\mu;2;-\frac{\zeta}{2}\right)
\end{equation}
and it has a branch point singularities at $\zeta=-2$. By the same reasoning, 
\begin{equation}
\hat{w}_{\mu,-}(\zeta)=c_{\mu,-} {}_2F_1\left(\frac{3}{2}-\mu,\frac{3}{2}+\mu;2;\frac{\zeta}{2}\right)
\end{equation} 
and it has branch point at $\zeta=2$. 


\subsection{Exponential integral}

Let $X=\C^*$, $f(x)=x+\frac{1}{x}$ and for every $\mu\in [0,+\infty)$ let $\nu=\left(x^\mu+x^{-\mu}\right) \tfrac{dx}{x}$, then  

\begin{equation}
I(z;m)\defeq\int_{0}^{\infty}e^{-zf}\nu
\end{equation}

In particular, on the universal cover $\pi\colon\tilde{C}\to \C^*$ setting $x=e^u$ \textbf{[DLMF, Identity~10.32.9]}
\begin{equation}
I(\tfrac{z}{2};\mu)=2\int_{-\infty}^{\infty}e^{-z\cosh(u)}\cosh(\mu u)\,du=4K_\mu(z)\,\quad |\arg(z)|<\pi/2
\end{equation} 
where $K_\mu(z)$ is the second kind modified Bessel function with parameter $\mu$. 

The critical points of $\pi^* f$ are at $u=k i \pi$, for $k\in\Z$ and we denote $\tilde{I}_{1}(z;\mu)$ the asymptotic expansion of $I(\tfrac{z}{2};\mu)$ at $u=0$ and $\tilde{I}_{-1}(z;\mu)$ the expansion at $u=i \pi$. They are respectively multiple of $\tilde{K}_{\mu}$ and $\tilde{I}_{\mu}$, because they solve \eqref{Bessel_nu} and they have the same leading order asymptotic of $\tilde{K}_{\mu}, \tilde{I}_{\mu}$ which are a basis.  

Notice that $I(z;\mu)$ differs from $I(z;0)$ only in $\pi^*(\nu)$ while $\pi^*(f)$ stays the same for every $\mu\in [0,\infty)$. Hence we can adapt part of the argument used in Bessel example (see ), and apply the $3/2$-derivative formula: let $\zeta=\cosh(u)$ and $\mathcal{C}_0(\zeta)\colon\theta\in\R\to\cosh(\theta)\in\C_\zeta$

\begin{align*}
\int_{\mathcal{C}_0(\zeta)}\pi^*(\nu)&=\int_{\mathcal{C}_0(\zeta)}\cosh(\mu u)du\\
&=\frac{1}{\mu}\Big[\sinh(\mu u)\Big]_{\mathrm{start}\mathcal{C}_0(\zeta)}^{\mathrm{end}\mathcal{C}_0(\zeta)}\\
&=\frac{1}{\mu}\left(\sinh\left(\mu\,\mathrm{acosh}\left(\zeta\right)\right)-\sinh\left(-\mu\,\mathrm{acosh}\left(\zeta\right)\right)\right)\\
&=\frac{2}{\mu}\sinh\left(\mu\,\mathrm{acosh}\left(\zeta\right)\right)
\end{align*}


The we set $\xi=\frac{1}{2}\left(\zeta-1\right)$, thanks to identity 15.4.16 \textbf{DLMF}

\begin{align*}
\sinh(\tau) {}_2F_1\left(\frac{1}{2}-\mu,\frac{1}{2}+\mu;\frac{3}{2};-\sinh^2(\tau)\right)&=\frac{1}{2\mu}\sinh(2\mu\tau) & \\
\xi^{1/2} {}_2F_1\left(\frac{1}{2}-\mu,\frac{1}{2}+\mu;\frac{3}{2};-\xi\right)&=\frac{1}{2\mu}\sinh(2\mu\tau)  & \sinh^2(\tau)=\xi \\
\xi^{1/2} {}_2F_1\left(\frac{1}{2}-\mu,\frac{1}{2}+\mu;\frac{3}{2};-\xi\right)&=\frac{1}{2\mu}\sinh(\mu\,\,\mathrm{acosh}\left(\zeta\right))  & \cosh(2\tau)=\zeta \\
&=\frac{1}{4}\int_{\mathcal{C}_0(\zeta)}\pi^*(\nu) &\\
\end{align*}

Thus we take $3/2$-derivative based at $\zeta=1$

\begin{align*}
\partial_{\zeta}^{3/2}\left(\int_{\mathcal{C}_0(\zeta)}\pi^*(\nu)\right)&=\partial_\zeta^2\left(\frac{1}{\Gamma\left(\frac{1}{2}\right)}\int_1^\zeta(\zeta-\zeta')^{-1/2}\left(\int_{\mathcal{C}_0(\zeta')}\pi^*(\nu)\right)d\zeta'\right)\\
&=4\partial_\zeta^2\left[\frac{1}{\Gamma\left(\frac{1}{2}\right)}\int_0^\xi\frac{1}{\sqrt{2}}(\xi-\xi')^{-1/2}(\xi')^{1/2} {}_2F_1\left(\frac{1}{2}-\mu,\frac{1}{2}+\mu;\frac{3}{2};-\xi'\right) 2\,\,d\xi'\right]\\
&=\frac{8}{\sqrt{2}}\partial_\zeta^2\left[\Gamma\left(\frac{3}{2}\right)\xi\,\,{}_2F_1\left(\frac{1}{2}-\mu,\frac{1}{2}+\mu;2;-\xi\right)\right] \\
&=\frac{8}{\sqrt{2}}\frac{\sqrt{\pi}}{2}\frac{1}{4}\partial_\xi^2\left[\xi\,\,{}_2F_1\left(\frac{1}{2}-\mu,\frac{1}{2}+\mu;2;-\xi\right)\right]\\
&=-\frac{\sqrt{\pi}}{\sqrt{2}}\partial_{\xi}\,\, {}_2F_1\left(\frac{1}{2}-\mu,\frac{1}{2}+\mu;1;-\xi\right)\\
&=\frac{\sqrt{\pi}}{\sqrt{2}}\Gamma\left(\frac{1}{2}-\mu\right)\Gamma\left(\frac{1}{2}+\mu\right){}_2F_1\left(\frac{3}{2}-\mu,\frac{3}{2}+\mu;2;-\xi\right)\\
&=\frac{\sqrt{\pi}}{\sqrt{2}}\frac{\pi}{\cos(\mu \pi)}{}_2F_1\left(\frac{3}{2}-\mu,\frac{3}{2}+\mu;2;\frac{1}{2}-\frac{\zeta}{2}\right)
\end{align*}



Let us now consider the integral whose asymptotic behvaior is given in terms of $\hat{w}_{\mu,-}(z)$:

\textbf{I have to check the correct form of the integral I which correspond to the path $C_\pi$. I suspect a scaling factor of $\cos(\pi \mu)$ that will adjust the Stokes factor computations.}

set $\zeta=-\cosh(u)$, 

\begin{align*}
\int_{\mathcal{C}_\pi(\zeta)}\pi^*(\nu)&=\int_{\mathcal{C}_\pi(\zeta)}\cosh(\mu u)du\\
&=\frac{1}{\mu}\Big[\sinh(\mu u)\Big]_{\mathrm{start}\mathcal{C}_\pi(\zeta)}^{\mathrm{end}\mathcal{C}_\pi(\zeta)}\\
&=\frac{1}{\mu}\left(\sinh\left(\mu\,\mathrm{acosh}\left(-\zeta\right)\right)-\sinh\left(-\mu\,\mathrm{acosh}\left(-\zeta\right)\right)\right)\\
&=\frac{2}{\mu}\sinh\left(\mu\,\mathrm{acosh}\left(-\zeta\right)\right)
\end{align*}


The we set $\xi=\frac{1}{2}\left(\zeta+1\right)$, thanks to identity 15.4.16 \textbf{DLMF}

\begin{align*}
\sinh(\tau) {}_2F_1\left(\frac{1}{2}-\mu,\frac{1}{2}+\mu;\frac{3}{2};-\sinh^2(\tau)\right)&=\frac{1}{2\mu}\sinh(2\mu\tau) & \\
(-\xi)^{1/2} {}_2F_1\left(\frac{1}{2}-\mu,\frac{1}{2}+\mu;\frac{3}{2};\xi\right)&=\frac{1}{2\mu}\sinh(2\mu\tau)  & \sinh^2(\tau)=-\xi \\
(-\xi)^{1/2} {}_2F_1\left(\frac{1}{2}-\mu,\frac{1}{2}+\mu;\frac{3}{2};\xi\right)&=\frac{1}{2\mu}\sinh(\mu\,\,\mathrm{acosh}\left(-\zeta\right))  & \cosh(2\tau)=-\zeta \\
&=\frac{1}{4}\int_{\mathcal{C}_\pi(\zeta)}\pi^*(\nu) &\\
\end{align*}

Thus we take $3/2$-derivative based at $\zeta=-1$


\begin{align*}
\partial_{\zeta}^{3/2}\left(\int_{\mathcal{C}_\pi(\zeta)}\pi^*(\nu)\right)&=\partial_\zeta^2\left(\frac{1}{\Gamma\left(\frac{1}{2}\right)}\int_{-1}^\zeta(\zeta-\zeta')^{-1/2}\left(\int_{\mathcal{C}_\pi(\zeta')}\pi^*(\nu)\right)d\zeta'\right)\\
&=4\partial_\zeta^2\left(\frac{1}{\Gamma\left(\frac{1}{2}\right)}\int_0^\xi\frac{1}{\sqrt{2}}(\xi-\xi')^{-1/2}(-\xi')^{1/2} {}_2F_1\left(\frac{1}{2}-\mu,\frac{1}{2}+\mu;\frac{3}{2};\xi'\right) 2\,\,d\xi'\right)\\
&=-i\frac{8}{\sqrt{2}}\partial_\zeta^2\left(\Gamma\left(\tfrac{3}{2}\right)(-\xi) \,\,{}_2F_1\left(\frac{1}{2}-\mu,\frac{1}{2}+\mu;2;\xi\right)\right) \\
&=i\frac{8}{\sqrt{2}}\frac{\sqrt{\pi}}{2}\frac{1}{4}\partial_\xi^2\left(\xi\,\,{}_2F_1\left(\frac{1}{2}-\mu,\frac{1}{2}+\mu;2;\xi\right)\right)\\
&=i\frac{\sqrt{\pi}}{\sqrt{2}}\partial_{\xi}\,\, {}_2F_1\left(\frac{1}{2}-\mu,\frac{1}{2}+\mu;1;\xi\right)\\
&=i\frac{\sqrt{\pi}}{\sqrt{2}}\Gamma\left(\frac{1}{2}-\mu\right)\Gamma\left(\frac{1}{2}+\mu\right){}_2F_1\left(\frac{3}{2}-\mu,\frac{3}{2}+\mu;2;\xi\right)\\
&=i\frac{\sqrt{\pi}}{\sqrt{2}}\frac{\pi}{\cos(\mu \pi)}{}_2F_1\left(\frac{3}{2}-\mu,\frac{3}{2}+\mu;2;\frac{1}{2}-\frac{\zeta}{2}\right)
\end{align*}

\color{black}

\subsection{Stokes factors}

Thanks to explicit formula for the analytic continuation of hypergeomtric functions (see \cite{dlmf} 15.2.3) and using the constants prescribed by the fractional derivative formula we are able ot compute the Stokes constants: set $\hat{w}_{+,\mu}(\zeta)=\textcolor{red}{\frac{\pi\sqrt{\pi}}{\cos(\mu \pi)}} {}_2F_1\left(\tfrac{3}{2}-\mu,\tfrac{3}{2}-\mu;2,1-\tfrac{\zeta}{2}\right)$ and $\hat{w}_{-,\mu}(\zeta)\defeq \textcolor{red}{i\tfrac{\pi\sqrt{\pi}}{\sqrt{2}\cos(\mu \pi)}} {}_2F_1\left(\tfrac{3}{2}-\mu,\tfrac{3}{2}-\mu;2,1+\tfrac{\zeta}{2}\right)$

\begin{align*}
\hat{w}_{\mu,+}(\zeta+i0)-\hat{w}_{\mu,+}(\zeta-i0)&=\textcolor{red}{\frac{\pi\sqrt{\pi}}{\sqrt{2}\cos(\mu \pi)}}\frac{2\pi i}{\Gamma(\tfrac{3}{2}-\mu)\Gamma(\tfrac{3}{2}+\mu)}\big(-\frac{\zeta}{2}-1\big)^{-1}{}_2F_1\left(\frac{1}{2}+\mu,\frac{1}{2}-\mu;0;1+\frac{\zeta}{2}\right) & \zeta>-2 \\
&=-\textcolor{red}{\frac{\pi\sqrt{\pi}}{\sqrt{2}\cos(\mu \pi)}}\frac{2\pi i}{\Gamma(\tfrac{3}{2}-\mu)\Gamma(\tfrac{3}{2}+\mu)}\sum_{k\geq 0}\frac{\left(\tfrac{1}{2}-\mu\right)_k\left(\tfrac{1}{2}+\mu\right)_k}{\Gamma(k)k!}\left(1+\frac{\zeta}{2}\right)^{k-1} & \\
&=-\textcolor{red}{\frac{\pi\sqrt{\pi}}{\sqrt{2}\cos(\mu \pi)}}\frac{2\pi i}{\Gamma(\tfrac{3}{2}-\mu)\Gamma(\tfrac{3}{2}+\mu)}\sum_{k\geq 1}\frac{\left(\tfrac{1}{2}-\mu\right)_k\left(\tfrac{1}{2}+\mu\right)_k}{\Gamma(k)k!}\left(1+\frac{\zeta}{2}\right)^{k-1} & \\
&=-\textcolor{red}{\frac{\pi\sqrt{\pi}}{\sqrt{2}\cos(\mu \pi)}}\frac{2\pi i}{\Gamma(\tfrac{3}{2}-\mu)\Gamma(\tfrac{3}{2}+\mu)}\frac{1}{\Gamma\left(\frac{1}{2}-\mu\right)\Gamma\left(\frac{1}{2}+\mu\right)}\cdot\\
& \qquad\cdot \sum_{k\geq 1}\frac{\Gamma\left(\tfrac{1}{2}-\mu+k\right)\Gamma\left(\tfrac{1}{2}+\mu+k\right)}{\Gamma(k)k!}\left(1+\frac{\zeta}{2}\right)^{k-1} & \\
&=-\textcolor{red}{\frac{\pi\sqrt{\pi}}{\sqrt{2}\cos(\mu \pi)}}\frac{2\pi i}{\Gamma(\tfrac{3}{2}-\mu)\Gamma(\tfrac{3}{2}+\mu)}\frac{1}{\Gamma\left(\frac{1}{2}-\mu\right)\Gamma\left(\frac{1}{2}+\mu\right)}\cdot \\
&\qquad\cdot \sum_{k\geq 0}\frac{\Gamma\left(\tfrac{3}{2}-\mu+k\right)\Gamma\left(\tfrac{3}{2}+\mu+k\right)}{\Gamma(k+1)(k+1)!}\left(1+\frac{\zeta}{2}\right)^{k} & \\
&=-\textcolor{red}{\frac{\pi\sqrt{\pi}}{\sqrt{2}\cos(\mu \pi)}}\frac{2\pi i}{\Gamma\left(\frac{1}{2}-\mu\right)\Gamma\left(\frac{1}{2}+\mu\right)}\,\, {}_2F_1\left(\frac{3}{2}-\mu,\frac{3}{2}+\mu;2;1+\frac{\zeta}{2}\right) & \\
&=-2\pi i\textcolor{red}{\frac{\sqrt{\pi}}{2}}\,\, {}_2F_1\left(\frac{3}{2}-\mu,\frac{3}{2}+\mu;2;1+\frac{\zeta}{2}\right)\\
&=\mathbf{-2}\cos(\pi\mu)\hat{w}_{-,\mu}(\zeta+2)
\end{align*} 
and for $\hat{w}_{\mu,-}(\zeta)$
\begin{align*}
\hat{w}_{\mu,-}(\zeta+i0)-\hat{w}_{\mu,-}(\zeta-i0)&=\textcolor{red}{i\frac{\pi\sqrt{\pi}}{\sqrt{2}\cos(\mu \pi)}}\frac{2\pi i}{\Gamma(\tfrac{3}{2}-\mu)\Gamma(\tfrac{3}{2}+\mu)}\big(\frac{\zeta}{2}-1\big)^{-1}{}_2F_1\left(\frac{1}{2}-\mu,\frac{1}{2}+\mu;0;1-\frac{\zeta}{2}\right) & \zeta<2 \\
&=-\textcolor{red}{i\frac{\pi\sqrt{\pi}}{\sqrt{2}\cos(\mu \pi)}}\frac{2\pi i}{\Gamma(\tfrac{3}{2}-\mu)\Gamma(\tfrac{3}{2}+\mu)}\sum_{k\geq 0}\frac{\left(\tfrac{1}{2}-\mu\right)_k\left(\tfrac{1}{2}+\mu\right)_k}{\Gamma(k)k!}\left(1-\frac{\zeta}{2}\right)^{k-1} & \\
&=-2i\cos(\mu \pi)\,\textcolor{red}{\frac{i\pi\sqrt{\pi}}{\sqrt{2}\cos(\mu \pi)}}\,\, {}_2F_1\left(\frac{3}{2}-\mu,\frac{3}{2}+\mu;2;1-\frac{\zeta}{2}\right) & \\
&=\mathbf{+2}\cos(\mu \pi)\hat{w}_{\mu,+}(\zeta-2)
\end{align*}

Therefore we have shown that Stokes constants are independent on $\mu$ and equal to $\pm 2$.
\subsection{Why the Stokes factors are non-integer}
The infinite dihedral group $D_{2\infty} \isom \Z \ltimes \Z/2\Z$ has a nice action on $\tilde{C}$ that leaves $e^{-z \cosh(u)}$ invariant. The action of $1 \in \Z$ adds $2\pi i$ to the value of $u$, and the actions of $\{\pm 1\} \isom \Z/2$ multiply the value of $u$ by $\pm 1$.

Consider exponential integrals of the form
\[ \langle \Gamma, e^{-z \cosh(u)}\,\alpha\rangle := \int_\Gamma e^{-z \cosh(u)}\,\alpha, \]
where the variety $\tilde{C}$ and the twisting factor $e^{-z \cosh(u)}$ are fixed, while the integration path $\Gamma$ and the twisted 1-form $e^{-z \cosh(u)}\,\alpha$ are allowed to vary. We can apply the action of $D_{2\infty}$ to these integrals by pushing forward the integration contour, or equivalently by pulling back the twisted 1-form. This makes the vector space of exponential integrals into a representation of $D_{2\infty}$.

Let's see what the action does to $I(\tfrac{z}{2}; \mu) = \langle -\infty \to \infty, e^{-z \cosh(u)}\,\nu \rangle$, recalling that $\nu = \cosh(\mu u)\,du$. Pulling $\nu$ back along $1 \in \Z$ gives
\begin{align*}
\cosh(\mu u + 2\pi i \mu)\,du & = \big[\cosh(2\pi i \mu) \cosh(\mu u) + \sinh(2\pi i \mu) \sinh(\mu u)\big]\,du \\
& = \cos(2\pi \mu)\,\nu + i\sin(2\pi \mu) \sinh(\mu u)\,du.
\end{align*}
and pulling it back along $\pm 1 \in \Z/2\Z$ gives $\pm\nu$. Hence, $1 \in \Z$ sends
\[ I(\tfrac{z}{2}; \mu) = 2\int_{-\infty}^{\infty} e^{-z\cosh(u)}\,\nu \]
to
\begin{align*}
& 2\int_{-\infty}^{\infty} e^{-z\cosh(u)} \big[\cos(2\pi \mu)\,\nu + i\sin(2\pi \mu) \sinh(\mu u)\,du\big] \\
& = 2\cos(2\pi \mu) \int_{-\infty}^{\infty} e^{-z\cosh(u)}\,\nu + 2i\sin(2\pi \mu) \int_{-\infty}^{\infty} e^{-z\cosh(u)} \sinh(\mu u)\,du \\
& = \cos(2\pi \mu)\,I(\tfrac{z}{2}; \mu),
\end{align*}
with the second term vanishing in the last step because $\sinh(\mu u)$ is odd. Also, $\pm 1 \in \Z/2\Z$ sends $I(\tfrac{z}{2}; \mu)$ to $\pm I(\tfrac{z}{2}; \mu)$. Now we see that in the space of exponential integrals, considered as a representation of $D_{2\infty}$, the span of $I(\tfrac{z}{2}; \mu)$ is a one-dimensional subrepresentation, giving the character \textbf{[correct word?]}
\begin{align*}
D_{2\infty} & \to \C^\times \\
1 \in \Z & \mapsto \cos(2\pi \mu) \\
\pm 1 \in \Z/2\Z & \mapsto \pm 1.
\end{align*}

\textbf{[The other invariant subspace should be generated by the path $-\infty + \pi i \to \infty + \pi i$.]}

Another way to think of this: if we choose bases for the space of (relative homology classes of) integration paths and the space of (relative cohomology classes of) twisted 1-forms, we can write a period matrix for $\langle\;,\;\rangle$. The entries of the period matrix form a basis for the space of exponential integrals. The action of $D_{2\infty}$ on exponential integrals can be seen as an action on the period matrix. When $D_{2\infty}$ acts on integration paths, it acts on the period matrix by left multiplication; when it acts on 1-forms, it acts on the period matrix by right multiplication. The the left and right action matrices are adjoints with respect to $\langle\;,\;\rangle$. The argument above shows that for well-chosen bases, we can make the action matrices block-diagonal, with two blocks.

When $\mu = 1/n$ for $n \in \N$, the space of integration paths and the space of twisted 1-forms are both $n$-dimensional, and \textbf{[...]}

\bibliographystyle{utphys}
\bibliography{airy-resurgence}
\end{document}