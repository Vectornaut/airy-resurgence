\documentclass{article}

\usepackage{url}
\usepackage[hmargin=1.5in]{geometry}
\usepackage{amsmath}
\usepackage{amssymb}
\usepackage{graphicx}

% convenience aliases
\newcommand{\maps}{\colon}

% symbology
\newcommand{\Z}{\mathbb{Z}}
\newcommand{\R}{\mathbb{R}}
\newcommand{\C}{\mathbb{C}}
\newcommand{\laplace}{\mathcal{L}}

\title{Resurgence of modified Bessel functions of second kind}
\author{Veronica Fantini}

\begin{document}
\maketitle

\section{Modified Bessel function of second kind}

The modified Bessel function of the second kind $K_\nu(z)$ is a solution of the equation
\begin{equation}\label{Bessel_nu}
w''+\frac{w'}{z}-w-\frac{\nu^2}{z^2}=0
\end{equation}
such that $K_{\nu}(z)\sim\sqrt{\pi/(2z)}e^{-z}$ as $z\to\infty$ in $|\arg z|<\frac{3\pi}{2}$\footnote{A system of solution of Bessel equation is given by $I_\nu(z)$ and $K_\nu(z)$. In particular, their asymptotic behaviour as $z\to\infty$ is given by \begin{align}
\tilde{I}_\nu(z)&=\frac{1}{\sqrt{2\pi}}e^zz^{-1/2}\sum_{k\geq 0}\frac{\left(\frac{1}{2}-\nu\right)_k\left(\frac{1}{2}+\nu\right)_k}{2^kk!}z^{-k}\\
\tilde{K}_{\nu}(z)&=\sqrt{\frac{\pi}{2}}e^{-z}z^{-1/2}\sum_{k\geq 0}\frac{\left(\frac{1}{2}-\nu\right)_k\left(\frac{1}{2}+\nu\right)_k}{(-2)^kk!}z^{-k}
\end{align}}. It has a branch point at $z=0$ for every $\nu\in\C$ and  the principal branch is analytic in $\C\setminus(-\infty,0]$. 
\subsection{Differential equation}
From the general theory of ODE, the formal integral solution of \eqref{Bessel_nu} is two parameters family 
\begin{equation}
\tilde{w}(z)=U_1e^{-z}z^{-1/2}\tilde{w}_{\nu,+}(z)+U_2e^{z}z^{-1/2}\tilde{w}_{\nu,-}(z)
\end{equation} 
where $\tilde{w}_{\nu,\pm}=\sum_{j\geq 0}a_{\pm,j}z^{-j}\in\C[\![z^{-1}]\!]$ are unique formal solutions of
\begin{align*}
\tilde{w}_{\nu,+}''-2\tilde{w}_{\nu,+}'+\frac{\tilde{w}_{\nu,+}}{4z^2}-\frac{\nu^2}{z^2}\tilde{w}_{\nu,+}=0\\
\tilde{w}_{\nu,-}''+2\tilde{w}_{\nu,-}'+\frac{\tilde{w}_{\nu,-}}{4z^2}-\frac{\nu^2}{z^2}\tilde{w}_{\nu,-}=0
\end{align*} 
In particular, $\tilde{K}_{\nu}(z)=\sqrt{\frac{\pi}{2}}e^{-z}z^{-1/2}\tilde{w}_{\nu,+}(z)$ and $\tilde{I}_\nu(z)=\frac{1}{\sqrt{2\pi}}e^zz^{-1/2}\tilde{w}_{\nu,-}(z)$ (once we choose $a_{\pm,0}=1$). We now compute the Borel transform of $\tilde{w}_+(z)$\footnote{We do not consider constant term of $\tilde{w}_{\nu,\pm}$, i.e. $\mathcal{B}:\C[\![z^{-1}]\!]\to \C[\zeta]$.} it is a solution of 
\begin{align*}
&\zeta^2\hat{w}_{\nu,+}+2t\hat{w}_{\nu,+}+\left(\frac{1}{4}-\nu^2\right)\int_0^{\zeta}(\zeta-s)\hat{w}_{\nu,+}(s)ds=0 &\\
&\zeta^2\hat{w}_{\nu,+}''+2\zeta\hat{w}_+''+4\zeta\hat{w}_{\nu,+}'+\left(\frac{9}{4}-\nu^2\right)\hat{w}_{\nu,+}=0 & \\
&t(1-t)\hat{w}_{\nu,+}''+(2-4t)\hat{w}_{\nu,+}'-\left(\frac{9}{4}-\nu^2\right)\hat{w}_{\nu,+}=0 & t=-\frac{\zeta}{2}
\end{align*} 
therefore $\hat{w}_{\nu,+}(\zeta)$ is an hypergeometric function
\begin{equation}
\hat{w}_{\nu,+}(\zeta)={}_2F_1\left(\frac{3}{2}-\nu,\frac{3}{2}+\nu;2;-\frac{\zeta}{2}\right)
\end{equation}
and it has a branch point singularities at $\zeta=-2$. By the same reasoning, 
\begin{equation}
\hat{w}_{\nu,-}(\zeta)={}_2F_1\left(\frac{3}{2}-\nu,\frac{3}{2}+\nu;2;\frac{\zeta}{2}\right)
\end{equation} 
and it has branch point at $\zeta=2$. Thanks to explicit formula for the analytic continuation of hypergeomtric functions (see \cite{dlmf} 15.2.3)
\begin{align*}
\hat{w}_{\nu,+}(\zeta+i0)-\hat{w}_{\nu,+}(\zeta-i0)&=\frac{2\pi i}{\Gamma(\tfrac{3}{2}-\nu)\Gamma(\tfrac{3}{2}+\nu)}\left(-\frac{\zeta}{2}-1\right)^{-1}{}_2F_1\left(\frac{1}{2}+\nu,\frac{1}{2}-\nu;0;1+\frac{\zeta}{2}\right) & \zeta>-2 \\
&=-\frac{2\pi i}{\Gamma(\tfrac{3}{2}-\nu)\Gamma(\tfrac{3}{2}+\nu)}\sum_{k\geq 0}\frac{\left(\tfrac{1}{2}-\nu\right)_k\left(\tfrac{1}{2}+\nu\right)_k}{\Gamma(k)k!}\left(1+\frac{\zeta}{2}\right)^{k-1} & \\
&=-\frac{2\pi i}{\Gamma(\tfrac{3}{2}-\nu)\Gamma(\tfrac{3}{2}+\nu)}\sum_{k\geq 1}\frac{\left(\tfrac{1}{2}-\nu\right)_k\left(\tfrac{1}{2}+\nu\right)_k}{\Gamma(k)k!}\left(1+\frac{\zeta}{2}\right)^{k-1} & \\
&=-\frac{2\pi i}{\Gamma(\tfrac{3}{2}-\nu)\Gamma(\tfrac{3}{2}+\nu)}\frac{1}{\Gamma\left(\frac{1}{2}-\nu\right)\Gamma\left(\frac{1}{2}+\nu\right)}\sum_{k\geq 1}\frac{\Gamma\left(\tfrac{1}{2}-\nu+k\right)\Gamma\left(\tfrac{1}{2}+\nu+k\right)}{\Gamma(k)k!}\left(1+\frac{\zeta}{2}\right)^{k-1} & \\
&=-\frac{2\pi i}{\Gamma(\tfrac{3}{2}-\nu)\Gamma(\tfrac{3}{2}+\nu)}\frac{1}{\Gamma\left(\frac{1}{2}-\nu\right)\Gamma\left(\frac{1}{2}+\nu\right)}\sum_{k\geq 0}\frac{\Gamma\left(\tfrac{3}{2}-\nu+k\right)\Gamma\left(\tfrac{3}{2}+\nu+k\right)}{\Gamma(k+1)(k+1)!}\left(1+\frac{\zeta}{2}\right)^{k} & \\
&=-\frac{2\pi i}{\Gamma\left(\frac{1}{2}-\nu\right)\Gamma\left(\frac{1}{2}+\nu\right)}\,\, {}_2F_1\left(\frac{3}{2}-\nu,\frac{3}{2}+\nu;2;1+\frac{\zeta}{2}\right) & \\
&=-2i\cos(\nu \pi)\hat{w}_{\nu,-}(\zeta+2)
\end{align*} 
and for $\hat{w}_{\nu,-}(\zeta)$
\begin{align*}
\hat{w}_{\nu,-}(\zeta+i0)-\hat{w}_{\nu,-}(\zeta-i0)&=-\frac{2\pi i}{\Gamma(\tfrac{3}{2}-\nu)\Gamma(\tfrac{3}{2}+\nu)}\left(\frac{\zeta}{2}-1\right)^{-1}{}_2F_1\left(\frac{1}{2}-\nu,\frac{1}{2}+\nu;0;1-\frac{\zeta}{2}\right) & \zeta<2 \\
&=\frac{2\pi i}{\Gamma(\tfrac{3}{2}-\nu)\Gamma(\tfrac{3}{2}+\nu)}\sum_{k\geq 0}\frac{\left(\tfrac{1}{2}-\nu\right)_k\left(\tfrac{1}{2}+\nu\right)_k}{\Gamma(k)k!}\left(1-\frac{\zeta}{2}\right)^{k-1} & \\
&=2i\cos(\nu \pi)\,\, {}_2F_1\left(\frac{3}{2}-\nu,\frac{3}{2}+\nu;2;1-\frac{\zeta}{2}\right) & \\
&=2i\cos(\nu \pi)\hat{w}_{\nu,+}(\zeta-2)
\end{align*}
In addition, the previous relations computes the Stokes constants which are funtions of $\nu$ and are given by $\pm 2i\cos(\nu\pi)$.


\subsection{Exponential integral}
As showed by Aaron, if $T_n(u)$ and $U_n(u)$ denote the Chebyschev polynomials \footnote{$T_n(\cos(t))=\cos(nt)$ and $U_n(\cos(t))\sin(t)=\sin((n+1)t)$.} 

\begin{align*}
K_{\nu}(z)&=\frac{1}{2 i\sin(\nu \pi)}\int_{\mathcal{C}_{\alpha}}e^{z\cosh(t)}\sinh(\nu t) dt & u=\cosh(\nu t)\\
&=-\frac{1}{2i\nu\sin(\nu\pi)}\int_{\mathcal{C}_{\alpha}}e^{zT_{\frac{1}{\nu}}(u)}du & \zeta=T_{\frac{1}{\nu}}(u)\\
&=\frac{1}{2i\sin(\nu\pi)}\int_{\mathcal{H}_{\alpha}}e^{-\zeta z}\frac{d\zeta}{U_{\frac{1}{\nu}-1}(u)}\\
&=-\frac{1}{2i\sin(\nu\pi)}\int_{\mathcal{H}_{\alpha}}e^{-\zeta z}{}_2F_1\left(\frac{1-\nu}{2},\frac{1+\nu}{2};\frac{3}{2};1-\zeta^2\right)d\zeta
\end{align*}
where $\mathcal{C}_{\alpha}$ \textbf{has to be checked, but I guess } $\alpha=1$

Indentity 15.10.17 from \cite{dlmf} splits the integrand above into
\begin{align*}
{}_2F_1\left(\frac{1-\nu}{2},\frac{1+\nu}{2};\frac{3}{2};1-\zeta^2\right)&=C_1 \, {}_2F_1\left(\frac{1-\nu}{2},\frac{1+\nu}{2};\frac{1}{2};\zeta^2\right)+C_2\zeta \,{}_2F_1\left(1-\frac{\nu}{2},1+\frac{\nu}{2};\frac{3}{2};\zeta^2\right)\\
&=\tilde{C}_1\left[{}_2F_1\left({1-\nu},{1+\nu};\frac{3}{2};\frac{1}{2}-\frac{\zeta}{2}\right)+{}_2F_1\left({1-\nu},{1+\nu};\frac{3}{2};\frac{1}{2}+\frac{\zeta}{2}\right)\right]+\\
&\qquad+\tilde{C}_2\left[{}_2F_1\left({1-\nu},{1+\nu};\frac{3}{2};\frac{1}{2}-\frac{\zeta}{2}\right)+{}_2F_1\left({1-\nu},{1+\nu};\frac{3}{2};\frac{1}{2}+\frac{\zeta}{2}\right)\right]
\end{align*}
therefore, collecting the contrubutions together we have 
\begin{multline}
K_{\nu}(z)=\frac{i}{4\sin(\nu\pi)}\int_{\mathcal{H}_{\alpha}}e^{-\zeta z}\left[{}_2F_1\left({1-\nu},{1+\nu};\frac{3}{2};\frac{1}{2}-\frac{\zeta}{2}\right)+{}_2F_1\left({1-\nu},{1+\nu};\frac{3}{2};\frac{1}{2}+\frac{\zeta}{2}\right)\right] d\zeta
\end{multline}

Since ${}_2F_1\left({1-\nu},{1+\nu};\frac{3}{2};\frac{1}{2}+\frac{\zeta}{2}\right) $ is singular at $\zeta=1$, the inverse Laplace transform of $K_\nu(z)$ is 
\[
\hat{K}_{\nu}(\zeta)=\frac{i}{4\sin(\nu\pi)}{}_2F_1\left({1-\nu},{1+\nu};\frac{3}{2};\frac{1}{2}-\frac{\zeta}{2}\right)
\] 
\bibliographystyle{utphys}
\bibliography{airy-resurgence}
\end{document}