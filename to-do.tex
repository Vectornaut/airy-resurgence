\documentclass{article}
\usepackage{xcolor}
\newcommand{\singexp}[2]{\mathcal{H}L^\infty_{#1, #2}}
\newcommand{\singexpalg}[1]{\singexp{#1}{\bullet}}
\newcommand{\dualsingexp}[2]{\widehat{\mathcal{H}}L^\infty_{#1, #2}}
\newcommand{\dualsingexpalg}[1]{\dualsingexp{#1}{\bullet}}
\title{To-do}
\date{}

\begin{document}
\maketitle
\section{Borel regularity paper}
\subsection{Intro}
\begin{itemize}
\item Finish new version of ``Why does Borel summation work for solutions of level 1 ODEs?''
\end{itemize}
\subsection{The Laplace and Borel transforms}
\begin{itemize}
\item Finish ``Regularity and decay properties''
\end{itemize}
\subsection{Proof of main results}
\begin{itemize}
    \item Finish new version of ''Borel regularity for ODEs''
\end{itemize}
\color{gray}
\subsection{Laplace and Borel transforms}
\begin{itemize}
\item Harmonize layout of Laplace and Borel properties.
\begin{itemize}
  \item Maybe split properties into two parts.
  \item Geometric properties: change of chart, change of fiber
  \item Laplace and Borel transforms as algebra homomorphisms (``The Laplace/Borel transform as an algebra homomorphism'')
\end{itemize}
\color{black}
\item \textbf{(Aaron)} Understand why \texttt{lem:frac-deriv-Borel} works without boundary terms, and explain this in text.
\item \textbf{(Aaron)} Clarify what the Borel transforms with respect to different charts are doing.
\item Talk about higher Laplace transform at singular point? (Low-priority. Maybe just mention in a footnote.)
\end{itemize}
\subsection{Examples}
\begin{itemize}
\item Revise Borel regularity argument for Airy-Lucas. Instead of assuming that the Poincar\'{e} solutions are Borel-summable, follow the general argument more closely, starting from the analytic position corner and using the asymptotics implied by the regularity and decay properties.
\item Rewrite thimble projection reasoning for Airy-Lucas functions
\item Explain equation of tringular cantilver
\end{itemize}
\subsection{Figures}
\begin{itemize}
\item For some reason all figures move at the end of the document.
\item Add the picture of the Borel plane as translation surface as tikzpicture and not as pdf.
\end{itemize}
\section{Part-time projects}
\begin{itemize}
\item Try to come up with a concrete goal and plan for a short project that we can do February--June
\begin{itemize}
\item \textbf{(Veronica)} Look into the Braaksma generalization project
\item \textbf{(Aaron)} Look into the algebraic hypergeometric project
\item \textbf{(Both)} Look into Varchenko's project
\end{itemize}
\item Look into projects that Veronica could continue with other collaborators
\begin{itemize}
\item Connection to two-variable exact WKB formalism (with Nikita)
\end{itemize}
\end{itemize}

\section{brainstorm}
\begin{itemize}
    \item Section~\ref{borel_reg-ODE} and maybe other subsections--revise the proof of the main result for ODE taking into account the results in our paper
    \item Do flagged revisions in Section~\ref{borel-reg-thimble}
    \item Section~\ref{contour-argument-AL}---revise the general contour argument
    \item Revise Section 2, using the formalism of $\singexp{\sigma}{\Lambda}(\Omega)$
    \begin{itemize}
    \item Replace formalism in Section~\ref{sec:reg-decay}
    \end{itemize}
    \item Check which examples are still in the framework of our theorem
    \begin{itemize}
    \item Start a storage document for shelved examples
    \end{itemize}
    \item Section~\ref{resurgence-AL}---Draw intersection points figure
    \item Section~\ref{sec:mod-bessel-lift}---pin down contour
    \item Explain contour figure in Airy example (explain what the regions indicate and what the color)
    \begin{itemize}
    \item Figure out how to release source code. How much do we need to clean it up? Is it okay that it relies on Shadertoy?
    \end{itemize}
    \item Look at ``Plan of the paper'' appendix and confirm that we've included everything we want
    \begin{itemize}
    \item Put future goals in storage document
    \end{itemize}
    \item Veronica doesn't like $\gamma_z$ as notation for Hankel contour. I'll suggest $\mathcal{H}_{1,z}$ where $1$ is the point where the path is centered. 
\end{itemize}
\end{document}