\DeclareSymbolFont{AMSb}{U}{msb}{m}{n}
\documentclass[11pt,a4paper,twoside,leqno,noamsfonts]{amsart}
           \usepackage{setspace}
\linespread{1.34}           
           %\onehalfspacing
\usepackage[english]{babel}
\usepackage[dvipsnames]{xcolor}
\definecolor{britishracinggreen}{rgb}{0.0, 0.26, 0.15}
\definecolor{cobalt}{rgb}{0.0, 0.28, 0.67}
\usepackage[utopia]{mathdesign}
    \DeclareSymbolFont{usualmathcal}{OMS}{cmsy}{m}{n}
    \DeclareSymbolFontAlphabet{\mathcal}{usualmathcal}
\usepackage[a4paper,top=4cm,bottom=3cm,left=3.5cm,
           right=3.5cm,bindingoffset=5mm]{geometry}
\usepackage[utf8]{inputenc}
\usepackage{braket,caption,comment,mathtools,stmaryrd}
\usepackage{multirow,booktabs,microtype}
\usepackage{latexsym}
\usepackage{todonotes}
\usepackage{fancyhdr}
%\renewcommand{\sectionmark}[1]{\markboth{\thesection\ #1}{}}
\pagestyle{fancy}
% Clear the header and footer
\fancyhead{}
\fancyfoot{}
% Set the right side of the footer to be the page number
\fancyfoot[R]{\thepage}
\addtolength{\headheight}{\baselineskip}
%\fancyhead[RE]{\rightmark}
%\fancyhead[RE]{}
\usepackage{soul} % per testo barrato
\usepackage[colorlinks,bookmarks]{hyperref} %
\hypersetup{colorlinks,%
            citecolor=britishracinggreen,%
            filecolor=black,%
            linkcolor=cobalt,%
            urlcolor=black}
\setcounter{tocdepth}{2}
%\setcounter{section}{-1}
\numberwithin{equation}{section}

% Veronica's custom commands
%\renewenvironment{proof}{{\scshape Proof.}}{\qed}

\makeatletter
\newenvironment{proofof}[1]{\par
  \pushQED{\qed}%
  \normalfont \topsep6\p@\@plus6\p@\relax
  \trivlist
  \item[\hskip3\labelsep
        \itshape
    Proof of #1\@addpunct{.}]\ignorespaces
}{%
  \popQED\endtrivlist\@endpefalse
}
\makeatother

% Def
%\def\be{\begin{equation}}    
%\def\ee{\end{equation}}
\def\into{\hookrightarrow}
\def\onto{\twoheadrightarrow}
\def\isom{\cong}  
\def\ra{\rightarrow}
\def\lra{\longrightarrow}
\def\surj{\twoheadrightarrow}
\def\Var{\mathrm{Var}}
\def\Sch{\mathrm{Sch}}
\def\Sets{\mathrm{Sets}}
\def\Def{\mathsf{Def}}
\def\KS{\mathsf{KS}}
\def\ad{\mathsf{ad}}
\def\St{\mathrm{St}}
\def\st{\mathrm{st}}

\def\L{\mathbb L}
\def\A{\mathcal A}
\def\B{\mathcal B}
\def\R{\mathbb R}
\def\C{\mathbb C}
\def\D{\mathbb D}
\def\P{\mathbb P}
\def\Q{\mathbb Q}
\def\G{\mathbb G}
\def\L{\mathbb{L}}
\def\SS{\mathcal S}
\def\RR{\mathbf R}
\def\X{\mathcal X}
\def\E{\mathcal E}
\def\Z{\mathbb Z}
\def\N{\mathbb N}
\def\ext{\mathrm{ext}}
\def\FF{\mathscr{F}}

\def\HS{\mathsf{HS}}
\def\O{\mathscr O}
\def\DDT{\mathsf{DT}}
\def\PPT{\mathsf{PT}}
\def\LL{\mathsf{L}}
\def\NN{\mathsf{N}}
\def\sc{\textrm{sc}}
\def\dcr{\textrm{d-crit}}
\def\loc{\textrm{loc}}
\def\Ad{\textrm{Ad}}
\def\reg{\textrm{reg}}
\def\red{\textrm{red}}
\def\relvir{\textrm{relvir}}
\def\pur{\textrm{pur}}
\def\vd{\mathrm{vd}}
\def\pure{\textrm{pure}}
\def\MF{\mathsf{MF}}
\def\WW{\mathsf{W}}
\def\HH{\mathsf{H}}
\def\h{\mathfrak{h}}
\def\at{\mathsf A}
\def\pt{\mathrm{pt}}

\def\CC{\mathrm{C}}
\def\KK{\mathrm{K}}
\DeclareMathOperator{\Mod}{Mod}
\DeclareMathOperator{\op}{op}
\DeclareMathOperator{\Tor}{Tor}
\DeclareMathOperator{\Mor}{Mor}
\DeclareMathOperator{\Fun}{Fun}
\DeclareMathOperator{\Vect}{Vect}
\DeclareMathOperator{\FDVect}{FDVect}
\DeclareMathOperator{\Rings}{Rings}
\DeclareMathOperator{\ev}{ev}
\DeclareMathOperator{\Quot}{Quot}
\DeclareMathOperator{\DD}{D}
\DeclareMathOperator{\Hilb}{Hilb}
\DeclareMathOperator{\Chow}{Chow}
\DeclareMathOperator{\Orb}{Orb}
\DeclareMathOperator{\Ob}{Ob}
\DeclareMathOperator{\ob}{ob}
\DeclareMathOperator{\Jac}{Jac}
\DeclareMathOperator{\ch}{ch}
\DeclareMathOperator{\Td}{Td}
\DeclareMathOperator{\tr}{tr}
\DeclareMathOperator{\id}{id}
\DeclareMathOperator{\Pic}{Pic}
\DeclareMathOperator{\codet}{codet}
\DeclareMathOperator{\Rep}{Rep}
\DeclareMathOperator{\Bl}{Bl}
\DeclareMathOperator{\ord}{ord}
\DeclareMathOperator{\aff}{aff}
\DeclareMathOperator{\vir}{vir}
\DeclareMathOperator{\QCoh}{QCoh}
\DeclareMathOperator{\Coh}{Coh}
\DeclareMathOperator{\Span}{Span}
\DeclareMathOperator{\mult}{mult}
\DeclareMathOperator{\Spec}{Spec\,}
\DeclareMathOperator{\Proj}{Proj\,}
\DeclareMathOperator{\Supp}{Supp\,}
\DeclareMathOperator{\coker}{coker}
\DeclareMathOperator{\Cone}{Cone}
\DeclareMathOperator{\Perf}{Perf}
\DeclareMathOperator{\im}{im}
\DeclareMathOperator{\DT}{DT}
\DeclareMathOperator{\PT}{PT}
\DeclareMathOperator{\RRR}{R}
\DeclareMathOperator{\GL}{GL}
\DeclareMathOperator{\SL}{SL}
\DeclareMathOperator{\dd}{d}
\DeclareMathOperator{\Tr}{Tr}
\DeclareMathOperator{\NCHilb}{NCHilb}
\DeclareMathOperator{\Sym}{Sym}
\DeclareMathOperator{\Aut}{Aut}
\DeclareMathOperator{\Ext}{Ext}
\DeclareMathOperator{\lExt}{{\mathscr Ext}}
\DeclareMathOperator{\Hom}{Hom}
\DeclareMathOperator{\lHom}{{\mathscr Hom}}
\DeclareMathOperator{\catA}{{\mathscr A}}
\DeclareMathOperator{\catB}{{\mathscr B}}
\DeclareMathOperator{\catC}{{\mathcal C}}
\DeclareMathOperator{\catD}{{\mathcal D}}
\DeclareMathOperator{\catT}{{\mathscr T}}
\DeclareMathOperator{\catF}{{\mathscr F}}
\DeclareMathOperator{\End}{End}
\DeclareMathOperator{\Eu}{Eu}
\DeclareMathOperator{\Exp}{Exp}
\DeclareMathOperator{\rk}{rk}
\DeclareMathOperator{\Nil}{Nil}
\DeclareMathOperator{\Tot}{Tot}
\DeclareMathOperator{\length}{length}
\DeclareMathOperator{\codim}{codim}
\DeclareMathOperator{\pr}{pr}
%\DeclareMathOperator{\at}{at}
\DeclareMathOperator{\Art}{Art}
\DeclareMathOperator{\uC}{\underline{\mathcal C}}
\DeclareMathOperator{\uA}{\underline{\mathscr A}}
\DeclareMathOperator{\F}{\mathcal F}
\DeclareMathOperator{\hh}{H}%Da togliere quando corregger� il capitolo 4
\DeclareMathOperator{\Der}{Der}
\DeclareMathOperator{\Ab}{Ab}


%%%%%%%%%%%%%%%%
\theoremstyle{definition}

\newtheorem*{lemma*}{Lemma}
\newtheorem*{theorem*}{Theorem}
\newtheorem*{example*}{Example}
\newtheorem*{fact*}{Fact}
\newtheorem*{notation*}{Notation}
\newtheorem*{definition*}{Definition}
\newtheorem*{prop*}{Proposition}
\newtheorem*{remark*}{Remark}
\newtheorem*{corollary*}{Corollary}
\newtheorem*{conventions*}{Conventions}
\newtheorem*{caution*}{Caution}

\newtheorem{definition}{Definition}[section]
\newtheorem{problem}[definition]{Problem}
\newtheorem{example}[definition]{Example}
\newtheorem{fact}[definition]{Fact}
\newtheorem{aside}[definition]{Aside}
\newtheorem{prop}[definition]{Proposition}
\newtheorem{question}[definition]{Question}
\newtheorem{remark}[definition]{Remark}
\newtheorem{theorem}[definition]{Theorem}
\newtheorem{corollary}[definition]{Corollary}
\newtheorem{lemma}[definition]{Lemma}
%\newtheorem{conjecture}[definition]{Conjecture}
\newtheorem{claim}[definition]{Claim}
%\newtheorem{exercise}[definition]{Exercise}

%\newtheoremstyle{thm} % <name> % (ambienti con dimostrazione)
%        {3mm}% <Space above>
%        {3mm}% <Space below>
%        {\slshape}% <Body font> % 
%        {0mm}% <Indent amount>
%        {\bfseries}% <Theorem head font>
%        {.}% <Punctuation after theorem head>
%        {1mm}% <Space after theorem head>
%        {}% <Theorem head spec (can be left empty, meaning 'normal')> 
%\theoremstyle{thm}
%\newtheorem{theorem}[definition]{Theorem}
%\newtheorem{corollary}[definition]{Corollary}
%\newtheorem{lemma}[definition]{Lemma}
%\newtheorem{prop}[definition]{Proposition}
%\newtheorem{thm}{Theorem}
%\newtheorem{notation}{Notation}
%\renewcommand*{\thethm}{\Alph{thm}}



%\newtheoremstyle{sol} % <name> % (ambienti con dimostrazione)
%        {3mm}% <Space above>
%        {3mm}% <Space below>
%        {\normalfont}% <Body font> % 
%        {0mm}% <Indent amount>
%        {\scshape}% <Theorem head font>
%        {.}% <Punctuation after theorem head>
%        {1mm}% <Space after theorem head>
%        {}% <Theorem head spec (can be left empty, meaning 'normal')> 
\theoremstyle{sol}
%\newtheorem{slogan}[definition]{Slogan}
\newtheorem{assumption}[definition]{Assumption}
%%\newtheorem{claim}[definition]{Claim}
%\newtheorem{notation}[definition]{Notation}
%\newtheorem*{ssolution*}{Solution (sketch)}
%\newtheorem*{solution*}{Solution}


%%%%%%%%%%%%%%%%%%%%%%%%%

\usepackage{tikz}
\usepackage{tikz-cd}
\usepackage{rotating}
\newcommand*{\isoarrow}[1]{\arrow[#1,"\rotatebox{90}{\(\sim\)}"
]}
\usetikzlibrary{matrix,shapes,arrows,decorations.pathmorphing}
\tikzset{commutative diagrams/arrow style=math font}
\tikzset{commutative diagrams/.cd,
mysymbol/.style={start anchor=center,end anchor=center,draw=none}}
\newcommand\MySymb[2][\square]{%
  \arrow[mysymbol]{#2}[description]{#1}}
\tikzset{
shift up/.style={
to path={([yshift=#1]\tikztostart.east) -- ([yshift=#1]\tikztotarget.west) \tikztonodes}
}
}

\DeclareMathAlphabet{\mathpzc}{OT1}{pzc}{m}{it}

\newcommand*{\defeq}{\mathrel{\vcenter{\baselineskip0.5ex \lineskiplimit0pt
                     \hbox{\scriptsize.}\hbox{\scriptsize.}}}%
                     =}
\newcommand*{\defeqin}{\mathrel{\vcenter{\lineskiplimit0pt\baselineskip0.5ex
                     \hbox{\scriptsize.}\hbox{\scriptsize.}}}%
                     =}                     


% symbology
\newcommand{\blankbox}{{\fboxsep 0pt \colorbox{lightgray}{\phantom{$h$}}}}
\newcommand{\maps}{\colon}
\newcommand{\van}{\mathfrak{m}}
\newcommand{\laplace}{\mathcal{L}}
\newcommand{\borel}{\mathcal{B}}
\newcommand{\laplacepde}{\mathcal{D}}
\DeclareMathOperator{\Ai}{Ai}

\DeclareRobustCommand{\subtitle}[1]{\\#1}

\title{Resurgence of the Airy function \\ and other exponential integrals}
\author{Veronica Fantini and Aaron Fenyes}

\begin{document}
\maketitle
\section{Hypergeometric functions as Borel transform of second order ODE (\textit{series normales de \emph{I}er ordre})}

Let us consider the following linear second order ODE

\begin{equation}\label{eq}
\big[P(\tfrac{\partial}{\partial z})+\tfrac{1}{z}Q(\tfrac{\partial}{\partial z})+\tfrac{1}{z}R(\tfrac{1}{z})\big]f(z)=0
\end{equation}
with $\deg P=2$, $\deg Q=1$ and $R=O(\tfrac{1}{z})$. We denote by $\alpha_1,\alpha_2$ the roots of $P(\lambda)$ and we assume they are distinct. Furthermore we assume $\tau_j:=\frac{Q(-\alpha_j)}{P'(-\alpha_j)}\neq 0$. The latter assumption guarantees the formal solution $\tilde{f}$ being slight, while the former assumption implies there will be two independent solutions. 

Under the previous assumptions we prove that the Borel transformed solution $\hat{f}(\zeta_j)$ is a Gauss hypergeometric function, $\zeta_j=\zeta-\alpha_j$.

\begin{prop}
Let $P(\lambda)=\lambda^2+a_1\lambda+a_0$, $Q(\lambda)=b_1\lambda+b_0$ and $R(\tfrac{1}{z})=\tfrac{c_1}{z}$ satisfying the previous assumptions. Then 
\begin{align}
\hat{f}(\zeta_1)&=\left(1-\frac{\zeta_1}{\alpha_2-\alpha_1}\right)^{c-1} \, {}_2F_1\left(a,b;c;1-\frac{\zeta_1}{\alpha_2-\alpha_1} \right)\\
\hat{f}(\zeta_2)&=\left(1-\frac{\zeta_2}{\alpha_1-\alpha_2}\right)^{c-1} \, {}_2F_1\left(a,b;c;1-\frac{\zeta_2}{\alpha_1-\alpha_2} \right)
\end{align}
where the coefficients $a,b,c$ depend on the parameter of $P,Q,R$. 
\end{prop}

\begin{proof}
We start by taking the Borel transform of \eqref{eq}:
\begin{align}\label{eq:B}
(\zeta^2-a_1\zeta+a_0)\hat{f}(\zeta)+\int_0^\zeta b_1(-\zeta')\hat{f}(\zeta')d\zeta'+b_0\int_0^\zeta\hat{f}(\zeta')d\zeta'+c_1\int_0^\zeta(\zeta-\zeta')\hat{f}(\zeta')d\zeta'=0
\end{align} 
then we differentiate twice in order to have a differential equation which can be easier recognized as a hypergeometric equation. Since $\tilde{F}$ is slight and locally integrable at $0$ by assumption, Proposition 1 {\tt Resurgent Airy doc by Aaron} tells we are not loosing information taking derivatives, and that $\hat{f}(\zeta)$ is a solution of \eqref{eq:B} if and only if it is a solution of \eqref{eq:B_deriv}
\begin{align}
\label{eq:B_deriv}
\big[(\zeta^2-a_1\zeta+a_0)\partial_\zeta^2+(4\zeta-b_1\zeta-2a_1+b_0)\partial_\zeta+(c_1+2-b_1)\big]\hat{f}(\zeta)=0
\end{align}
We introduce some notation to simplify the computations, we denote by $\beta_1=4-b_1$, $\beta_0=b_0-2a_1$, $\gamma=c_1+2-b_1$ so \eqref{eq:B_deriv} turns into
\begin{align*}
\big[(\zeta-\alpha_1)(\zeta-\alpha_2)\partial_\zeta^2+(\beta_1\zeta+\beta_0)\partial_\zeta+\gamma\big]\hat{f}(\zeta)=0
\end{align*}
We consider the following change of coordinates $\zeta=\alpha_2-(\alpha_2-\alpha_1)\xi$\footnote{$\partial_\zeta=(\alpha_1-\alpha_2)^{-1}\partial_\xi$ and $\partial_\zeta^2=(\alpha_1-\alpha_2)^{-2}\partial_\xi^2$}
\begin{align*}
&\big[(\alpha_2-(\alpha_2-\alpha_1)\xi-\alpha_1)(\alpha_2-(\alpha_2-\alpha_1)\xi-\alpha_2)(\alpha_1-\alpha_2)^{-2}\partial_\xi^2+(\beta_1(\alpha_2-(\alpha_2-\alpha_1)\xi)+\beta_0)(\alpha_1-\alpha_2)^{-1}\partial_\xi+\gamma\big]\hat{f}(\xi)=0\\
&\big[(\alpha_2-\alpha_1)(1-\xi)(\alpha_1-\alpha_2)\xi(\alpha_1-\alpha_2)^{-2}\partial_\xi^2+(\beta_1\alpha_2-\beta_1(\alpha_2-\alpha_1)\xi+\beta_0)(\alpha_1-\alpha_2)^{-1}\partial_\xi+\gamma\big]\hat{f}(\xi)=0\\
&\big[-(1-\xi)\xi\partial_\xi^2+((\beta_1\alpha_2+\beta_0)(\alpha_1-\alpha_2)^{-1}+\beta_1\xi)\partial_\xi+\gamma\big]\hat{f}(\xi)=0\\
&\big[(1-\xi)\xi\partial_\xi^2-((\beta_1\alpha_2+\beta_0)(\alpha_1-\alpha_2)^{-1}+\beta_1\xi)\partial_\xi-\gamma\big]\hat{f}(\xi)=0
\end{align*}

The latter equation is an hypergeometric equation of parameters 
\begin{align*}
&C=(\beta_1\alpha_2+\beta_0)(\alpha_2-\alpha_1)^{-1}\\
&A+B+1=\beta_1=4-b_1 \Rightarrow A+B=3-b_1\\
&AB=\gamma=c_1+2-b_1
\end{align*}
and a solution is given by 
\begin{align*}
%\hat{f}(\xi)=(1-\xi)^{C-A-B} {}_2F_1(C-A,C-B;C;\xi)\\
\hat{f}(\xi)&=\xi^{1-C} {}_2F_1(A-C+1,B-C+1;2-C;\xi)\\
&=\left(\frac{\alpha_2-\zeta}{\alpha_2-\alpha_1}\right)^{1-C} {}_2F_1\left(A-C+1,B-C+1;2-C;\frac{\alpha_2-\zeta}{\alpha_2-\alpha_1}\right)\\
&=\left(1-\frac{\zeta_1}{\alpha_2-\alpha_1}\right)^{1-C} {}_2F_1\left(A-C+1,B-C+1;2-C;1-\frac{\zeta_1}{\alpha_2-\alpha_1}\right)
\end{align*} 

\end{proof}

\bibliographystyle{plain}
\bibliography{airy-resurgence}
\end{document}