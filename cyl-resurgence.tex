\documentclass{article}

\usepackage{url}
\usepackage[hmargin=1.5in]{geometry}
\usepackage{amsmath}
\usepackage{amssymb}
\usepackage{graphicx}
\usepackage[svgnames]{xcolor} %% for revision coloring

% convenience aliases
\newcommand{\maps}{\colon}

% symbology
\newcommand{\Z}{\mathbb{Z}}
\newcommand{\R}{\mathbb{R}}
\newcommand{\C}{\mathbb{C}}
\newcommand{\laplace}{\mathcal{L}}

\title{Resurgence of a parabolic cylinder function}
\author{Aaron Fenyes}

\begin{document}
\maketitle
\section{Hypergeometric functions in trigonometry}
\subsection{Chebyshev sine formulas}
By definition,
\begin{align*}
\cos(n\phi) & = T_n(\cos(\phi)) \\
\sin(n\phi) & = U_{n-1}(\cos(\phi))\,\sin(\phi).
\end{align*}
Furthermore,
\begin{align*}
\cos(n\phi - \tfrac{n}{2}\pi) & = T_n(\cos(\phi - \tfrac{\pi}{2})) \\
\cos(n\phi) \cos(\tfrac{n}{2}\pi) + \sin(n\phi) \sin(\tfrac{n}{2}\pi) & = T_n(\sin(\phi)).
\end{align*}
Thus, for $n = 2k + 1$, we have
\[ (-1)^k \sin(n\phi) = T_n(\sin(\phi)), \]
implying also that
\begin{align*}
(-1)^k n \cos(n\phi) & = T_n'(\sin(\phi)) \cos(\phi) \\
(-1)^k n T_n(\cos(\phi)) & = nU_{n-1}(\sin(\phi)) \cos(\phi) \\
(-1)^k T_n(\cos(\phi)) & = U_{n-1}(\sin(\phi)) \cos(\phi).
\end{align*}
\subsection{A hypergeometric identity}
Let $a = \tfrac{1}{2} - \tfrac{1}{2n}$, giving $1 - 2a = \tfrac{1}{n}$.
\subsubsection{All orders}\label{all-orders}
From \cite[equation~15.4.16]{dlmf},
\[ F(a, 1-a, \tfrac{3}{2}; \sin(\theta)^2) = \frac{\sin((1-2a)\theta)}{(1-2a)\sin(\theta)} \]
on the principal branch of $F$ when $|\theta| < \frac{\pi}{2}$. Hence, we have
\begin{align*}
F\left(\tfrac{1}{2} - \tfrac{m}{2n}, \tfrac{1}{2} + \tfrac{m}{2n}, \tfrac{3}{2}; \sin(\theta)^2\right) & = \frac{\sin(\tfrac{m}{n}\,\theta)}{\tfrac{m}{n} \sin(\theta)} \\
& = \frac{n}{m} \cdot \frac{\sin(m\,\theta/n)}{\sin(\theta/n)} \cdot \frac{\sin(\theta/n)}{\sin(\theta)} \\
& = \frac{n}{m} \cdot \frac{U_{m-1}(\cos(\theta/n))}{U_{n-1}(\cos(\theta/n))}
\end{align*}
under the same conditions on $F$ and $\theta$. Letting $u = \cos(\theta/n)$, we get the identity
\[ F\left(\tfrac{1}{2} - \tfrac{m}{2n}, \tfrac{1}{2} + \tfrac{m}{2n}, \tfrac{3}{2}; 1 - T_n(u)^2\right) = \frac{n}{m} \cdot \frac{U_{m-1}(u)}{U_{n-1}(u)}. \]
%%From our reasoning so far, we can only conclude the identity holds when $\pm T_n(u)$ is in the right half-plane.

Identity~15.10.17 \textcolor{magenta}{[or, better, 15.8.4]} from \cite{dlmf} splits the left-hand side above into
\begin{align*}
& \frac{\Gamma(\tfrac{1}{2}) \Gamma(\tfrac{3}{2})}{\Gamma(1 - \tfrac{m}{2n})\Gamma(1 + \tfrac{m}{2n})} F(\tfrac{1}{2} - \tfrac{m}{2n}, \tfrac{1}{2} + \tfrac{m}{2n}, \tfrac{1}{2}; T_n(u)^2) \\
\pm & \frac{\Gamma(-\tfrac{1}{2}) \Gamma(\tfrac{3}{2})}{\Gamma(\tfrac{1}{2} - \tfrac{m}{2n})\Gamma(\tfrac{1}{2} + \tfrac{m}{2n})}\,T_n(u)\,F(1 - \tfrac{m}{2n}, 1 + \tfrac{m}{2n}, \tfrac{3}{2}; T_n(u)^2),
\end{align*}
\textcolor{DarkCyan}{where the sign must be chosen so that $\pm T_n(u)$ is in the right half-plane (?)}. Applying identityies 15.8.27--28 from \cite{dlmf} turns this into
\begin{align*}
& \tfrac{1}{2} F(1 - \tfrac{m}{n}, 1 + \tfrac{m}{n}, \tfrac{3}{2}, \tfrac{1}{2} \mp \tfrac{1}{2}T_n(u)) + \tfrac{1}{2} F(1 - \tfrac{m}{n}, 1 + \tfrac{m}{n}, \tfrac{3}{2}, \tfrac{1}{2} \pm \tfrac{1}{2}T_n(u)) \\
+ & \tfrac{1}{2} F(1 - \tfrac{m}{n}, 1 + \tfrac{m}{n}, \tfrac{3}{2}, \tfrac{1}{2} \mp \tfrac{1}{2}T_n(u)) - \tfrac{1}{2} F(1 - \tfrac{m}{n}, 1 + \tfrac{m}{n}, \tfrac{3}{2}, \tfrac{1}{2} \pm \tfrac{1}{2}T_n(u)).
\end{align*}
After cancellation, we conclude that
\[ F(1 - \tfrac{m}{n}, 1 + \tfrac{m}{n}, \tfrac{3}{2}, \tfrac{1}{2} \mp \tfrac{1}{2}T_n(u)) = \frac{n}{m} \cdot \frac{U_{m-1}(u)}{U_{n-1}(u)} \]
\textcolor{DarkCyan}{when $\pm T_n(u)$ is in the right half-plane (?)}.
%%which simplifies\footnote{Calculate $\Gamma(-\tfrac{1}{2}) \Gamma(\tfrac{3}{2}) = -\pi$ and $[\Gamma(\tfrac{1}{2} - \tfrac{1}{2n})\Gamma(\tfrac{1}{2} + \tfrac{1}{2n})]^{-1} = \tfrac{1}{\pi} \sin\big((\tfrac{1}{2} + \tfrac{1}{2n})\pi\big) = \tfrac{1}{\pi} \cos\big(\tfrac{\pi}{2n}\big)$. Similarly, $\Gamma(\tfrac{1}{2}) \Gamma(\tfrac{3}{2}) = \tfrac{\pi}{2}$ and $[\Gamma(1 - \tfrac{1}{2n})\Gamma(1 + \tfrac{1}{2n})]^{-1} = [\Gamma(1 - \tfrac{1}{2n})\,\tfrac{1}{2n} \Gamma(\tfrac{1}{2n})]^{-1} = \tfrac{2n}{\pi} \sin\big(\tfrac{\pi}{2n}\big)$.} to \textcolor{magenta}{[$m = 1$]}
%%\begin{align*}
%%& n \sin\big(\tfrac{\pi}{2n}\big)\,F(\tfrac{1}{2} - \tfrac{1}{2n}, \tfrac{1}{2} + \tfrac{1}{2n}, \tfrac{1}{2}; T_n(u)^2) \\
%%- & \cos\big(\tfrac{\pi}{2n}\big)\,|T_n(u)|\,F(1 - \tfrac{1}{2n}, 1 + \tfrac{1}{2n}, \tfrac{3}{2}; T_n(u)^2).
%%\end{align*}
\subsubsection{Odd order}
From \cite[equation~15.4.14]{dlmf},
\[ F(a, 1-a, \tfrac{1}{2}; \sin(\theta)^2) = \frac{\cos((1-2a)\theta)}{\cos(\theta)}. \]
so we have
\begin{align*}
F\left(\tfrac{1}{2} - \tfrac{1}{2n}, \tfrac{1}{2} + \tfrac{1}{2n}, \tfrac{1}{2}; \sin(\theta)^2\right) & = \frac{\cos(\theta/n)}{\cos(\theta)} \\
& = \frac{\cos(\theta/n)}{T_n(\cos(\theta/n))}.
\end{align*}
Let $u = \sin(\theta/n)$. If $n = 2k + 1$, we get the identity
\[ F\left(\tfrac{1}{2} - \tfrac{1}{2n}, \tfrac{1}{2} + \tfrac{1}{2n}, \tfrac{1}{2}; T_n(u)^2\right) = \frac{(-1)^k}{U_{n-1}(u)}. \]
\section{The Weber equation}
\subsection{Basics}
The Weber equation is
\[ \left[\big(\tfrac{\partial}{\partial y}\big)^2 - \left(\tfrac{1}{4}y^2 + a\right)\right] \psi = 0. \]
Setting $a$ to zero, we get
\begin{equation}\label{eqn:weber}
\left[\big(\tfrac{\partial}{\partial y}\big)^2 - \tfrac{1}{4}y^2\right] \psi = 0.
\end{equation}
One solution is given by the parabolic cylinder function~\cite[equation~12.7.10]{dlmf}
\[ U(0, y) = \tfrac{1}{\sqrt{2\pi}}\,y^{1/2}\,K_{1/4}\big(\tfrac{1}{4} y^2\big). \]

Combining the identity \cite[equation~10.27.4]{dlmf}
\[ K_\nu(z) = \frac{\pi}{2 \sin(\nu \pi)}\big(I_{-\nu}(z) - I_\nu(z)\big) \]
and the integral \cite[equation~10.32.12]{dlmf}
\[ I_\nu(z) = \frac{1}{2\pi i} \int_{\mathcal{H}} e^{z \cosh(\phi)} e^{-\nu \phi}\;d\phi, \]
where $\mathcal{H}$ runs clockwise around the rectangle
\begin{align*}
0 & < \operatorname{Re}(\phi) & |\operatorname{Im}(\phi)| & < \pi,
\end{align*}
we learn that
\begin{align*}
I_{-\nu}(z) - I_\nu(z) & = \frac{1}{\pi i} \int_{\mathcal{H}} e^{z \cosh(\phi)} \sinh(\nu \phi)\;d\phi \\
K_\nu(z) & = \frac{1}{2i\,\sin(\nu \pi)} \int_{\mathcal{H}} e^{z \cosh(\phi)} \sinh(\nu \phi)\;d\phi.
\end{align*}
In particular, we have
\[ K_{1/4}(z) = \frac{1}{i\sqrt{2}} \int_{\mathcal{H}} e^{z \cosh(\phi)} \sinh(\phi/4)\;d\phi. \]
Setting $u = \cosh(\phi/4)$ and getting $\cosh(\phi) = 8u^4 - 8u^2 + 1$ from a table of Chebyshev polynomials, we can write
\begin{equation}\label{integral:watson}
K_{1/4}(z) = \frac{2\sqrt{2}}{i} \int_\Gamma \exp\left[z \left(8u^4 - 8u^2 + 1\right)\right]\,du,
\end{equation}
where $\Gamma$ is a path that comes from $\infty$ at $-45^\circ$ and goes to $\infty$ at $45^\circ$. \textbf{[This feels dubious; check using definition of $\mathcal{H}$.]}
\subsection{Contour argument}\label{contour-argument}
We can recast integral~\ref{integral:watson} into $\hat{\C}$ by setting $-\zeta = 8u^4 - 8u^2 + 1$. Projecting $\Gamma$ to a contour $\gamma$ in $\hat{\C}$ and choosing the branch of $u$ that lifts $\gamma$ back to $\Gamma$, we have
\begin{equation}\label{integral:watson-zeta}
K_{1/4} = \frac{i}{\sqrt{2}} \int_\gamma e^{-z\zeta}\frac{d\zeta}{8u^3 - 4u}.
\end{equation}
The integrand has poles at $u = \pm\tfrac{1}{\sqrt{2}}$, and $\gamma$ runs counterclockwise around $\big[\tfrac{1}{\sqrt{2}}, \infty\big)$.

Using the identity from Section~\ref{all-orders}, we can rewrite integral~\ref{integral:watson-zeta} as
\[ K_{1/4} = \frac{i}{\sqrt{2}} \int_{\gamma_z} e^{-z\zeta} F\big(\tfrac{3}{8}, \tfrac{5}{8}; \tfrac{3}{2}; 1 - \zeta^2\big)\;d\zeta. \]
\bibliographystyle{utphys}
\bibliography{airy-resurgence}
\end{document}