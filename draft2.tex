\DeclareSymbolFont{AMSb}{U}{msb}{m}{n}
\documentclass[11pt,a4paper,twoside,leqno,noamsfonts]{amsart}
           \usepackage{setspace}
\linespread{1.34}           
           %\onehalfspacing
\usepackage[english]{babel}
\usepackage[dvipsnames]{xcolor}
\definecolor{britishracinggreen}{rgb}{0.0, 0.26, 0.15}
\definecolor{cobalt}{rgb}{0.0, 0.28, 0.67}
\usepackage[utopia]{mathdesign}
    \DeclareSymbolFont{usualmathcal}{OMS}{cmsy}{m}{n}
    \DeclareSymbolFontAlphabet{\mathcal}{usualmathcal}
\usepackage[a4paper,top=4cm,bottom=3cm,left=3.5cm,
           right=3.5cm,bindingoffset=5mm]{geometry}
\usepackage[utf8]{inputenc}
\usepackage{braket,caption,comment,mathtools,stmaryrd}
\usepackage{multirow,booktabs,microtype}
\usepackage{latexsym}
\usepackage{todonotes}
\usepackage{fancyhdr}
%\renewcommand{\sectionmark}[1]{\markboth{\thesection\ #1}{}}
\pagestyle{fancy}
% Clear the header and footer
\fancyhead{}
\fancyfoot{}
% Set the right side of the footer to be the page number
\fancyfoot[R]{\thepage}
\addtolength{\headheight}{\baselineskip}
%\fancyhead[RE]{\rightmark}
%\fancyhead[RE]{}
\usepackage{soul} % per testo barrato
\usepackage[colorlinks,bookmarks]{hyperref} %
\hypersetup{colorlinks,%
            citecolor=britishracinggreen,%
            filecolor=black,%
            linkcolor=cobalt,%
            urlcolor=black}
\setcounter{tocdepth}{2}
%\setcounter{section}{-1}
\numberwithin{equation}{section}

% Veronica's custom commands
%\renewenvironment{proof}{{\scshape Proof.}}{\qed}

\makeatletter
\newenvironment{proofof}[1]{\par
  \pushQED{\qed}%
  \normalfont \topsep6\p@\@plus6\p@\relax
  \trivlist
  \item[\hskip3\labelsep
        \itshape
    Proof of #1\@addpunct{.}]\ignorespaces
}{%
  \popQED\endtrivlist\@endpefalse
}
\makeatother

% Def
%\def\be{\begin{equation}}    
%\def\ee{\end{equation}}
\def\into{\hookrightarrow}
\def\onto{\twoheadrightarrow}
\def\isom{\cong}  
\def\ra{\rightarrow}
\def\lra{\longrightarrow}
\def\surj{\twoheadrightarrow}
\def\Var{\mathrm{Var}}
\def\Sch{\mathrm{Sch}}
\def\Sets{\mathrm{Sets}}
\def\Def{\mathsf{Def}}
\def\KS{\mathsf{KS}}
\def\ad{\mathsf{ad}}
\def\St{\mathrm{St}}
\def\st{\mathrm{st}}

\def\L{\mathbb L}
\def\A{\mathcal A}
\def\B{\mathcal B}
\def\R{\mathbb R}
\def\C{\mathbb C}
\def\D{\mathbb D}
\def\P{\mathbb P}
\def\Q{\mathbb Q}
\def\G{\mathbb G}
\def\L{\mathbb{L}}
\def\SS{\mathcal S}
\def\RR{\mathbf R}
\def\X{\mathcal X}
\def\E{\mathcal E}
\def\Z{\mathbb Z}
\def\N{\mathbb N}
\def\ext{\mathrm{ext}}
\def\FF{\mathscr{F}}

\def\HS{\mathsf{HS}}
\def\O{\mathscr O}
\def\DDT{\mathsf{DT}}
\def\PPT{\mathsf{PT}}
\def\LL{\mathsf{L}}
\def\NN{\mathsf{N}}
\def\sc{\textrm{sc}}
\def\dcr{\textrm{d-crit}}
\def\loc{\textrm{loc}}
\def\Ad{\textrm{Ad}}
\def\reg{\textrm{reg}}
\def\red{\textrm{red}}
\def\relvir{\textrm{relvir}}
\def\pur{\textrm{pur}}
\def\vd{\mathrm{vd}}
\def\pure{\textrm{pure}}
\def\MF{\mathsf{MF}}
\def\WW{\mathsf{W}}
\def\HH{\mathsf{H}}
\def\h{\mathfrak{h}}
\def\at{\mathsf A}
\def\pt{\mathrm{pt}}

\def\CC{\mathrm{C}}
\def\KK{\mathrm{K}}
\DeclareMathOperator{\Mod}{Mod}
\DeclareMathOperator{\op}{op}
\DeclareMathOperator{\Tor}{Tor}
\DeclareMathOperator{\Mor}{Mor}
\DeclareMathOperator{\Fun}{Fun}
\DeclareMathOperator{\Vect}{Vect}
\DeclareMathOperator{\FDVect}{FDVect}
\DeclareMathOperator{\Rings}{Rings}
\DeclareMathOperator{\ev}{ev}
\DeclareMathOperator{\Quot}{Quot}
\DeclareMathOperator{\DD}{D}
\DeclareMathOperator{\Hilb}{Hilb}
\DeclareMathOperator{\Chow}{Chow}
\DeclareMathOperator{\Orb}{Orb}
\DeclareMathOperator{\Ob}{Ob}
\DeclareMathOperator{\ob}{ob}
\DeclareMathOperator{\Jac}{Jac}
\DeclareMathOperator{\ch}{ch}
\DeclareMathOperator{\Td}{Td}
\DeclareMathOperator{\tr}{tr}
\DeclareMathOperator{\id}{id}
\DeclareMathOperator{\Pic}{Pic}
\DeclareMathOperator{\codet}{codet}
\DeclareMathOperator{\Rep}{Rep}
\DeclareMathOperator{\Bl}{Bl}
\DeclareMathOperator{\ord}{ord}
\DeclareMathOperator{\aff}{aff}
\DeclareMathOperator{\vir}{vir}
\DeclareMathOperator{\QCoh}{QCoh}
\DeclareMathOperator{\Coh}{Coh}
\DeclareMathOperator{\Span}{Span}
\DeclareMathOperator{\mult}{mult}
\DeclareMathOperator{\Spec}{Spec\,}
\DeclareMathOperator{\Proj}{Proj\,}
\DeclareMathOperator{\Supp}{Supp\,}
\DeclareMathOperator{\coker}{coker}
\DeclareMathOperator{\Cone}{Cone}
\DeclareMathOperator{\Perf}{Perf}
\DeclareMathOperator{\im}{im}
\DeclareMathOperator{\DT}{DT}
\DeclareMathOperator{\PT}{PT}
\DeclareMathOperator{\RRR}{R}
\DeclareMathOperator{\GL}{GL}
\DeclareMathOperator{\SL}{SL}
\DeclareMathOperator{\dd}{d}
\DeclareMathOperator{\Tr}{Tr}
\DeclareMathOperator{\NCHilb}{NCHilb}
\DeclareMathOperator{\Sym}{Sym}
\DeclareMathOperator{\Aut}{Aut}
\DeclareMathOperator{\Ext}{Ext}
\DeclareMathOperator{\lExt}{{\mathscr Ext}}
\DeclareMathOperator{\Hom}{Hom}
\DeclareMathOperator{\lHom}{{\mathscr Hom}}
\DeclareMathOperator{\catA}{{\mathscr A}}
\DeclareMathOperator{\catB}{{\mathscr B}}
\DeclareMathOperator{\catC}{{\mathcal C}}
\DeclareMathOperator{\catD}{{\mathcal D}}
\DeclareMathOperator{\catT}{{\mathscr T}}
\DeclareMathOperator{\catF}{{\mathscr F}}
\DeclareMathOperator{\End}{End}
\DeclareMathOperator{\Eu}{Eu}
\DeclareMathOperator{\Exp}{Exp}
\DeclareMathOperator{\rk}{rk}
\DeclareMathOperator{\Nil}{Nil}
\DeclareMathOperator{\Tot}{Tot}
\DeclareMathOperator{\length}{length}
\DeclareMathOperator{\codim}{codim}
\DeclareMathOperator{\pr}{pr}
%\DeclareMathOperator{\at}{at}
\DeclareMathOperator{\Art}{Art}
\DeclareMathOperator{\uC}{\underline{\mathcal C}}
\DeclareMathOperator{\uA}{\underline{\mathscr A}}
\DeclareMathOperator{\F}{\mathcal F}
\DeclareMathOperator{\hh}{H}%Da togliere quando corregger� il capitolo 4
\DeclareMathOperator{\Der}{Der}
\DeclareMathOperator{\Ab}{Ab}


%%%%%%%%%%%%%%%%
\theoremstyle{definition}

\newtheorem*{lemma*}{Lemma}
\newtheorem*{theorem*}{Theorem}
\newtheorem*{example*}{Example}
\newtheorem*{fact*}{Fact}
\newtheorem*{notation*}{Notation}
\newtheorem*{definition*}{Definition}
\newtheorem*{prop*}{Proposition}
\newtheorem*{remark*}{Remark}
\newtheorem*{corollary*}{Corollary}
\newtheorem*{conventions*}{Conventions}
\newtheorem*{caution*}{Caution}

\newtheorem{definition}{Definition}[section]
\newtheorem{problem}[definition]{Problem}
\newtheorem{example}[definition]{Example}
\newtheorem{fact}[definition]{Fact}
\newtheorem{aside}[definition]{Aside}
\newtheorem{prop}[definition]{Proposition}
\newtheorem{question}[definition]{Question}
\newtheorem{remark}[definition]{Remark}
\newtheorem{theorem}[definition]{Theorem}
\newtheorem{corollary}[definition]{Corollary}
\newtheorem{lemma}[definition]{Lemma}
%\newtheorem{conjecture}[definition]{Conjecture}
\newtheorem{claim}[definition]{Claim}
%\newtheorem{exercise}[definition]{Exercise}

%\newtheoremstyle{thm} % <name> % (ambienti con dimostrazione)
%        {3mm}% <Space above>
%        {3mm}% <Space below>
%        {\slshape}% <Body font> % 
%        {0mm}% <Indent amount>
%        {\bfseries}% <Theorem head font>
%        {.}% <Punctuation after theorem head>
%        {1mm}% <Space after theorem head>
%        {}% <Theorem head spec (can be left empty, meaning 'normal')> 
%\theoremstyle{thm}
%\newtheorem{theorem}[definition]{Theorem}
%\newtheorem{corollary}[definition]{Corollary}
%\newtheorem{lemma}[definition]{Lemma}
%\newtheorem{prop}[definition]{Proposition}
%\newtheorem{thm}{Theorem}
%\newtheorem{notation}{Notation}
%\renewcommand*{\thethm}{\Alph{thm}}



%\newtheoremstyle{sol} % <name> % (ambienti con dimostrazione)
%        {3mm}% <Space above>
%        {3mm}% <Space below>
%        {\normalfont}% <Body font> % 
%        {0mm}% <Indent amount>
%        {\scshape}% <Theorem head font>
%        {.}% <Punctuation after theorem head>
%        {1mm}% <Space after theorem head>
%        {}% <Theorem head spec (can be left empty, meaning 'normal')> 
\theoremstyle{sol}
%\newtheorem{slogan}[definition]{Slogan}
\newtheorem{assumption}[definition]{Assumption}
%%\newtheorem{claim}[definition]{Claim}
%\newtheorem{notation}[definition]{Notation}
%\newtheorem*{ssolution*}{Solution (sketch)}
%\newtheorem*{solution*}{Solution}


%%%%%%%%%%%%%%%%%%%%%%%%%

\usepackage{tikz}
\usepackage{tikz-cd}
\usepackage{rotating}
\newcommand*{\isoarrow}[1]{\arrow[#1,"\rotatebox{90}{\(\sim\)}"
]}
\usetikzlibrary{matrix,shapes,arrows,decorations.pathmorphing}
\tikzset{commutative diagrams/arrow style=math font}
\tikzset{commutative diagrams/.cd,
mysymbol/.style={start anchor=center,end anchor=center,draw=none}}
\newcommand\MySymb[2][\square]{%
  \arrow[mysymbol]{#2}[description]{#1}}
\tikzset{
shift up/.style={
to path={([yshift=#1]\tikztostart.east) -- ([yshift=#1]\tikztotarget.west) \tikztonodes}
}
}

\DeclareMathAlphabet{\mathpzc}{OT1}{pzc}{m}{it}

\newcommand*{\defeq}{\mathrel{\vcenter{\baselineskip0.5ex \lineskiplimit0pt
                     \hbox{\scriptsize.}\hbox{\scriptsize.}}}%
                     =}
\newcommand*{\defeqin}{\mathrel{\vcenter{\lineskiplimit0pt\baselineskip0.5ex
                     \hbox{\scriptsize.}\hbox{\scriptsize.}}}%
                     =}                     


% symbology
\newcommand{\blankbox}{{\fboxsep 0pt \colorbox{lightgray}{\phantom{$h$}}}}
\newcommand{\maps}{\colon}
\newcommand{\van}{\mathfrak{m}}
\newcommand{\laplace}{\mathcal{L}}
\newcommand{\borel}{\mathcal{B}}
\DeclareMathOperator{\Ai}{Ai}

\DeclareRobustCommand{\subtitle}[1]{\\#1}
\title[Exponential Integrals]{Exponential Integrals\\ [1ex]
  }

\author{
Veronica Fantini 
}
%\address{SISSA Trieste, Via Bonomea 265, 34136 Trieste, Italy}
%\email{ vfantini@sissa.it}
%
\begin{document}
\hbadness=150
\vbadness=150
%
%\begin{abstract}
%We show how wall-crossing formulas in coupled $2d$-$4d$ systems, introduced by Gaiotto, Moore and Neitzke, can be interpreted geometrically in terms of the deformation theory of holomorphic pairs, given by a complex manifold together with a holomorphic vector bundle. The main part of the paper studies the relation between scattering diagrams and deformations of holomorphic pairs, building on recent work by Chan, Conan Leung and Ma.      
%\end{abstract}
%
\maketitle
%
%
%
{
\hypersetup{linkcolor=black}
\tableofcontents}

\section{Introduction}
\section{Fractional derivatives and Borel transform}
For $\nu \in (-\infty, 1)$, the fractional integral $\partial^{\nu-1}_{x \text{ from } 0}$ is defined by
\[ \partial^{\nu-1}_{x \text{ from } 0} f(x) \defeq \frac{1}{\Gamma(1-\nu)} \int_0^x (x-x')^{-\nu} f(x')\,dx'. \]
It obeys the expected semigroup law \textbf{[Lazarevi\'{c}, \S 1.3]}
\begin{align*}
\partial^{\lambda}_{x \text{ from } 0}\,\partial^{\mu}_{x \text{ from } 0} & = \partial^{\lambda+\mu}_{x \text{ from } 0} & \lambda, \mu \in (-\infty, 0),
\end{align*}
and agrees with ordinary repeated integration when $\nu$ is an integer \textbf{[Lazarevi\'{c}, equation~35]}.

For $\alpha \in (0, 1)$ and integers $n \ge 0$, fractional derivatives $\partial^{n+\alpha}_{x \text{ from } 0}$ are defined by composing $\partial^{\alpha-1}_{x \text{ from } 0}$ with powers of $\tfrac{\partial}{\partial x}$. However, $\partial^{\alpha-1}_{x \text{ from } 0}$ and $\tfrac{\partial}{\partial x}$ don't commute: their commutator is an initial value operator \textbf{[check, clarify]}. Various ordering conventions give various definitions of $\partial^{n+\alpha}_{x \text{ from } 0} f(x)$, which differ by operators that act on the germ of $f$ at zero \textbf{[Lazarevi\'{c}, \S 1.3---original source Podlubny]}. We'll use the {\em Riemann-Liouville} convention.
\begin{definition}
For $\alpha \in (0, 1)$ and integers $n \ge 0$, the {\em Riemann-Liouville fractional derivative} $\partial^{n+\alpha}_{x \text{ from } 0}$ is defined by
\[ \partial^{n+\alpha}_{x \text{ from } 0} \defeq \left(\tfrac{\partial}{\partial x}\right)^{n+1} \partial^{\alpha-1}_{x \text{ from } 0}. \]
\end{definition}
The symbol $\partial^\mu_{\zeta \text{ from } 0}$ is now defined for any $\mu \in \R \smallsetminus \{0, 1, 2, 3, \ldots\}$. It denotes a fractional integral when $\mu$ is negative, and a fractional derivative when $\mu$ is a positive non-integer.

The Riemann-Liouville fractional derivative is a left inverse of the fractional integral, in the sense that $\partial^\lambda_{x \text{ from } 0}\,\partial^{-\lambda}_{x \text{ from } 0}=\text{Id}$ for all $\lambda \in (0, \infty)$. This extends the semigroup law:
\begin{align*}
\partial^{\lambda}_{x \text{ from } 0}\,\partial^{\mu}_{x \text{ from } 0} & = \partial^{\lambda+\mu}_{x \text{ from } 0} & \lambda \in \R \smallsetminus \{0, 1, 2, 3, \ldots\},\quad\mu \in (-\infty, 0).
\end{align*}

Each convention for the fractional derivative brings its own annoyances to interactions with the Borel transform.\footnote{See Remark~\ref{rmk:caputo} for examples.} The Riemann-Liouville derivative will be the least annoying for our purposes. Here's what's nice about it.

\subsection{Borel transform}

We breifly recall some basic facts on Borel transform and its properties (more details can be founded in \cite{MS}, \cite{MDiablerets}, \cite{}, \cite{}). 

The Borel transform $\mathcal{B}$ is a linear map from formal power seires in the time domain $\C_z$ to the space domain $\C_{\zeta}$ such that  \begin{align*}\mathcal{B}(z^{-n-1})&\defeq\tfrac{\zeta^n}{n!} \qquad, \text{ if } n\geq 0 \\
\mathcal{B}(1)&\defeq\delta 
\end{align*} 

where $\delta$ is the element $(1,0)\in\C\times\C[\![\zeta]\!]$\footnote{Sometimes, in physics it is common to adopt a differnt convention, i.e. $\mathcal{B}(z^{-n})\defeq \tfrac{\zeta^n}{n!}$ (see\cite{MDiablerets}\cite{SGM}, for example). However, we find natural for our pourposes the mathematical convention, as the Laplace transform is the inverse of $\mathcal{B}$ under suitable growth conditions. }, and $\mathcal{B}$ extends formally by linearity to  

\begin{align*}
\mathcal{B}\colon \C[\![z^{-1}]\!] & \to \C\delta\oplus\C[\![\zeta\!]] \\
\sum_{n\geq 0}a_n z^{-n} & \to  a_0\delta+\sum_{n\geq 0}a_{n+1}\frac{\zeta^n}{n!}
\end{align*}

Notice that we are at the level of formal series, so there are no convergence assumptions. However, there is a special class of formal series that behaves well under Borel transform, meaning that its Borel transform gives a germ of holomorphic fucntions at the origin in $\C_{\zeta}$. These formal series are the so called Gevrey-$1$ series: $\tilde{\Phi}(z)=\sum_{n\geq 0}a_nz^{-n}$ is Gevrey-$1$ if there exists $A>0$ such that $|a_n|\leq A^n n!$ for all $n\geq 0$.\footnote{In asymptotic analysis Gevrey-$k$ series $\sum_{n\geq 0}a_nz^{-n}$ have coefficients that grow as $(n!)^{k}$, i.e. there exists $A>0$ such that $|a_n|\leq A^n (n!)^k$ for every $n\geq 0$. The Borel transform can be generalized in order to obatin germs of holomorphic function for higher Gevrey series. }  

\begin{notation*}
We want to distinghuish between formal series and holomorphic functions, as well as between the Borel plane (spatial domain) and the $z$-plane (time domain). Therefore we adopt the following notation:
\begin{itemize}
\item $\Phi(z)$: upper-case letters are holomorphic functions in the $z$-plane;
\item $\tilde{\Phi}(z)$: \textit{tilde} stands for formal series, so an upper-case letter with \textit{tilde} is a formal series in the $z$-plane;
\item $\phi(\zeta)$: lower-case letter are holomorphic functions in the Borel plane (time domain). We follow the convention for Laplace transform of holomorphic functions, namely $\mathcal{L}(\phi(\zeta)((z)=\Phi(z)$. 
\item $\tilde{\phi}(\zeta)$: lower-case letter with \textit{tilde} are formal series in the Borel plane. As we will see, the Borel transform of $\tilde{\Phi}(z)$ is $\mathcal{B}(\tilde{\Phi})(\zeta)=:\tilde{\phi}(\zeta)$; 
\item $\hat{\phi}(\zeta)$: lower-case letters with \textit{hat} are the sum of the formal series $\tilde{\phi}(\zeta)$, when it exists.  
\end{itemize}
\end{notation*}  

\begin{lemma}
Let $\tilde{\Phi}(z)=\sum_{n\geq 0}a_nz^{-n}$, it is Gevrey-$1$ if and only if \[\hat{\phi}(\zeta)=\tilde{\phi}(\zeta)\defeq\mathcal{B}(\tilde{\Phi})\in\C\lbrace\zeta\rbrace.\]  
\end{lemma}
In particular, we deduce from the lemma that the Borel transform is an isomorphism between Gevrey-$1$ series and $\C\delta\oplus\C\lbrace\zeta\rbrace$. 

For what follows, it is convenient to extend the definition of $\mathcal{B}$ to fractional power of $z$, replacing the factorial with the Gamma function: 
\begin{align*}
\mathcal{B}(z^{-\alpha})\defeq \frac{\zeta^{\alpha-1}}{\Gamma(\alpha)} & \qquad \text{ if } \alpha\in\R\setminus\Z_{\leq 0}
\end{align*}
However, if $\tilde{\Phi}(z)=\sum_{n\geq 0}a_nz^{-n}$ is a Gevrey-$1$ series, and $\alpha$ is a rational number with denominator $q$, then $\mathcal{B}(z^{-\alpha}\tilde{\Phi})(\zeta)$ is a germ of meromorphic function at the oringin of the Riemann surface of $\zeta^{1/q}$. Indeed,
\begin{align*}
\mathcal{B}(z^{-\alpha}\tilde{\Phi})(\zeta)&=\frac{\zeta^{\alpha-1}}{\Gamma(\alpha)}\ast\left(a_0\delta+\sum_{n\geq 0}a_{n+1}\frac{\zeta^n}{n!}\right)\\
&=a_0\frac{\zeta^{\alpha-1}}{\Gamma(\alpha)}+\sum_{n\geq 0}a_{n+1}\frac{\zeta^{n+\alpha}}{\Gamma(1+n+\alpha)}.
\end{align*}
If $\alpha > 1$, then $\mathcal{B}(z^{-\alpha}\tilde{\Phi})(\zeta)$ is actually a germ of holomorphic functions on the Riemann surface of $\zeta
^{1/q}$. We're interested in the case where $\alpha=\tfrac{1}{2}$, so the Borel transform is meromorphic on the Riemann surface of $\zeta^{1/2}$, and has a pole at $\zeta = 0$ unless $a_0 = 0$.
\subsubsection{Properties of the Borel transform}
We now recall some properties of the Borel transform: let $\tilde{\Phi}(z)=\sum_{n\geq 0}a_nz^{-n}$ and denote by $\tilde{\phi}(\zeta)\defeq\mathcal{B}(\tilde{\Phi}(z))$

\begin{itemize}
\item[(i)] \emph{[fractional derivative]} if $\alpha\in\R_{\geq 0}$, $\mathcal{B}\left(z^\alpha \tilde{\Phi}(z)\right)=\partial_{\zeta}^{\alpha}\tilde{\phi}(\zeta)$
\item[(ii)] \emph{[fractional integral]} if $\alpha\in\R_{<0}\setminus\Z_{\geq 0}$, $\mathcal{B}\left(z^\alpha \tilde{\Phi}(z)\right)=\partial_{\zeta}^{\alpha}\tilde{\phi}(\zeta)$
\item[(iii)] if $n\in\Z$ and $z^nf(z)\in z^{-1}\C[\![z^{-1}]\!]$, then $\mathcal{B}(z^n\tilde{\Phi}(z))=\partial_\zeta^n\tilde{\phi}(\zeta)$ 
\item[(iv)] $\mathcal{B}\left(\partial_z^{n} \tilde{\Phi}(z)\right)=(-\zeta)^n\tilde{\phi}(\zeta)$, for every $n\geq 0$
\item[(v)] $\mathcal{B}(\tilde{\Phi}(z-c))=e^{-c\zeta}\tilde{\phi}(\zeta)$
\item[(vi)] if $\tilde{\Psi}(z)\in z^{-1}\C[\![z^{-1}]\!]$, then $\mathcal{B}(\tilde{\Phi}(z)\tilde{\Psi}(z))=\int_0^{\zeta}d\zeta' \tilde{\phi}(\zeta-\zeta')\tilde{\psi}(\zeta')=:\tilde{\phi}\ast \tilde{\psi}$, where $\tilde{\psi}(\zeta)=\mathcal{B}(\tilde{\Psi}(z))$.
\end{itemize} 


\begin{lemma}\label{lem:frac-deriv-Borel}
For any non-integer $\mu \in (0, \infty)$ and any integer $k \ge 0$,
\[ \partial^\mu_{\zeta \text{ from } 0} \left[ \borel \left(z^{-(k+1)}\right)(\zeta) \right] =  \borel \left(z^\mu z^{-(k+1)}\right)(\zeta). \]
\end{lemma}
\begin{proof}
We'll show that for any $\alpha \in (0, 1)$ and any integer $n \ge 0$, the claim holds with $\mu = n + \alpha$. First, evaluate
\begin{align*}
\partial^{\alpha-1}_{\zeta \text{ from } 0} \left[ \borel \left(z^{-(k+1)}\right)(\zeta) \right] & = \frac{1}{\Gamma(1-\alpha)} \int_0^\zeta (\zeta-\zeta')^{-\alpha} \frac{{\zeta'}^k}{\Gamma(k+1)}\,d\zeta' \\
& = \frac{1}{\Gamma(1-\alpha)\,\Gamma(k+1)} \int_0^1 (\zeta-\zeta t)^{-\alpha} (\zeta t)^k\,\zeta\,dt \\
& = \frac{\zeta^{k-(\alpha-1)}}{\Gamma(1-\alpha)\,\Gamma(k+1)} \int_0^1 (1-t)^{-\alpha} t^k\,dt \\
& = \frac{\zeta^{k-(\alpha-1)}}{\Gamma\big(k-(\alpha-1)+1\big)}
\end{align*}
by reducing the integral to Euler's beta function \textbf{[DLMF 5.12.1]}. This establishes that
\begin{equation}\label{frac-diff-laplace}
\left(\tfrac{\partial}{\partial \zeta}\right)^{n+1}\,\partial^{\alpha-1}_{\zeta \text{ from } 0} \left[ \borel \left(z^{-(k+1)}\right)(\zeta) \right] = \frac{\zeta^{k-(n+\alpha)}}{\Gamma\big(k-(n+\alpha)+1\big)}
\end{equation}
for $n = -1$. If \eqref{frac-diff-laplace} holds for $n = m$, it also holds for $n = m+1$, because
\begin{align*}
\tfrac{\partial}{\partial \zeta} \left(\tfrac{\partial}{\partial \zeta}\right)^{m+1}\,\partial^{\alpha-1}_{\zeta \text{ from } 0} \left[ \borel \left(z^{-(k+1)}\right)(\zeta) \right] & = \tfrac{\partial}{\partial \zeta} \left( \frac{\zeta^{k-(m+\alpha)}}{\big(k-(m+\alpha)\big)\,\Gamma\big(k-(m+\alpha)\big)} \right) \\
& = \frac{\zeta^{k-(m+1+\alpha)}}{\Gamma\big(k-(m+\alpha)\big)}
\end{align*}
Hence, \eqref{frac-diff-laplace} holds for all $n \ge -1$, and the desired result quickly follows. The condition $\alpha \in (0, 1)$ saves us from the trouble we'd run into if $k-(m+\alpha)$ were in $\Z_{\le 0}$. This is how we avoid the initial value corrections that appear in ordinary derivatives of Borel transforms.
\end{proof}




\begin{proof} 

We are going to prove properties (i)--(vi). 

(i) follows from Lemma \ref{lem:frac-deriv-Borel}. 

(ii) Notice that for $\alpha\in\R\setminus\Z_{\geq 0}$ the fractional integral $\partial_\zeta^{\alpha}\zeta^{k}=\zeta^{k-\alpha}\tfrac{k!}{\Gamma(k-\alpha+1)}$, hence \[\borel (z^\alpha \tilde{\Phi}(z))=\sum_{k\geq 0}a_k\tfrac{\zeta^{k-\alpha}}{\Gamma(k-\alpha+1)}=\sum_{k\geq 0}a_k\tfrac{\zeta^{k-\alpha}}{k!}\tfrac{k!}{\Gamma(k-\alpha+1)}=\sum_{k\geq 0}a_k \tfrac{1}{k!}\partial_\zeta^{\alpha}\zeta^k=\partial_\zeta^\alpha\tilde{\phi}(\zeta).\]

(iii) $z^n\tilde{\Phi}(z)=\sum_{k\geq 0}a_kz^{-(k-n)-1} $ and by assumption $k-n\geq 0$, hence by defintion \[\borel (z^n \tilde{\Phi}(z))=\sum_{k\geq n} a_k\tfrac{\zeta^{k-n}}{(k-n)!}=\sum_{k\geq n} a_k\tfrac{1}{k!}\tfrac{k!\zeta^{k-n}}{(k-n)!}=\sum_{k\geq n} a_k\tfrac{1}{k!}\partial_\zeta^n\zeta^k=\partial_\zeta^n\tilde{\phi}(\zeta).\] 

(vi) $\partial_z^n \tilde{\Phi}(z)=\sum_{k\geq 0}a_k(-1)^{n}z^{-k-n-1}\tfrac{\Gamma(k+n+1)}{k!}$, hence \[\borel (z^\alpha \tilde{\Phi}(z))=\sum_{k\geq 0}a_k(-1)^{n}\tfrac{\zeta^{k+n}}{\Gamma(k+n+1)}\tfrac{\Gamma(k+n+1)}{k!}=(-\zeta)^{n}\tilde{\phi}(\zeta).\]

(v) see Lemma 5.10 \cite{MS16}. 

(vi) see Definition 5.12 and Lemma 5.14 \cite{MS16}. 
\end{proof}

\begin{remark}
We notice that properties (i) and (ii) are special cases of property (vi), indeed we can use the convolution product 
\begin{align*}
\borel (z^\alpha \tilde{\Phi}(z))&=\borel (z^\alpha)\ast \tilde{\phi}(\zeta)\\
&=\frac{\zeta^{-\alpha-1}}{\Gamma(-\alpha)}\ast\tilde{\phi}(\zeta)\\
&=\int_0^\zeta \frac{(\zeta')^{-\alpha-1}}{\Gamma(-\alpha)}\sum_{k\geq 0}a_k\frac{(\zeta-\zeta')^k}{k!}d\zeta'\\
&=\sum_{k\geq 0}\frac{a_k}{k!}\frac{1}{\Gamma(-\alpha)}\int_0^1(t\zeta)^{-\alpha-1}(\zeta-t\zeta)^k\zeta dt\\
&=\sum_{k\geq 0}\frac{a_k}{k!}\frac{1}{\Gamma(-\alpha)}\zeta^{k-\alpha}\int_0^1 t^{-\alpha-1}(1-t)^k dt\\
&=\sum_{k\geq 0}\frac{a_k}{k!}\frac{1}{\Gamma(-\alpha)}\zeta^{k-\alpha}\frac{\Gamma(k+1)\Gamma(-\alpha)}{\Gamma(k-\alpha+1)}\\
&=\sum_{k\geq 0}\frac{a_k}{k!}\partial_\zeta^{\alpha} \zeta^k \\
&=\partial_\zeta^\alpha\tilde{\phi}(\zeta)
\end{align*}
\end{remark}

\section{The geometry of the Laplace transform}
\subsection{Introduction}
Classically, the Laplace transform turns functions on the position domain into functions on the frequency domain. In the study of Borel summation and resurgence, it's useful to see the position domain as a {\em translation surface} $B$, and the frequency domain as one of its cotangent spaces. Roughly speaking, the Laplace transform lifts holomorphic functions on $B$ to holomorphic functions on $T^* B$.
\subsection{Translation surfaces, briefly}
\subsubsection{Definition}
A translation surface is a Riemann surface $B$ carrying a holomorphic $1$-form $\lambda$~\textbf{[Zorich, ``Flat Surfaces''?]}. A translation chart is a local coordinate $\zeta$ with $d\zeta = \lambda$. The standard metric on $\C$ pulls back along translation charts to a flat metric on $B$, with a conical singularity of angle $2\pi n$ wherever $\lambda$ has a zero of order $n-1 > 0$. We'll require $B$ to be finite-type and $\lambda$ to have a pole at each puncture. This kind of translation surface has a ``cylindrical end'' \textbf{(figure)} at each puncture where $\lambda$ has order $-1$, and a ``$|2n|$-planar end'' \textbf{(figure)} at each puncture where $\lambda$ has order $n-1 < -1$~\textbf{[Gupta, ``Meromorphic quadratic differentials with half-plane structures,'' \S 2.5] (or cite Aaron's article, which will hopefully present the same background in the translation surface context)}.
\subsubsection{Direction}\label{transl:dir}
The translation structure gives $B$ a notion of direction as well as distance. Away from the zeros of $\lambda$, which we'll call {\em branch points}, we can talk about moving upward, rightward, or at any angle, just as we would on $\C$. At a branch point of cone angle $2\pi n$, we can also talk about moving upward, rightward, or at any angle in $\R/2\pi\Z$, but here there are $n$ directions that fit each description. To make this more concrete, note that around any point $b \in B$, there's a unique holomorphic function $\zeta_b$ that vanishes at $b$ and has $d\zeta_b = \lambda$. \textcolor{VioletRed}{[If we define ``translation parameter'' earlier, we can say:] there's a unique translation parameter $\zeta_b$ that vanishes at $b$.} This function is a translation chart when $b$ is an ordinary point, and an $n$-fold branched covering when $b$ is a branch point of cone angle $2\pi n$. In either case, $\zeta_b \in e^{i\theta} [0, \infty)$ is a ray or a set of rays leaving $b$ at angle $\theta \in \R/2\pi\Z$.

Near each branch point $b$, let's fix a coordinate $\omega_b$ with $\zeta_b = \tfrac{1}{n} \omega_b^n$, where $2\pi n$ is the cone angle at $b$. This lets us label each direction at $b$ with an ``extended angle'' in $\R/2\pi n\Z$. Of course, there are $n$ different choices for $\omega_b$.
\subsubsection{Frequency}\label{transl-freq}
The translation structure also gives us an isomorphism $z \maps T^*B_b \to \C$ when $b \in B$ is an ordinary point, and an isomorphism $z \maps T^*B_b^{\otimes n} \to \C$ when $b$ is a branch point of cone angle $2\pi n$. At an ordinary point, we can define $z$ simply as the map
\begin{align*}
z \maps T^*B_b & \to \C \\
\lambda\big|_b & \mapsto 1.
\end{align*}
To get a definition that generalizes to branch points, though, it's worth taking a fancier point of view. Recall that $T^*B_b = \van_b / \van_b^2$, where $\van_b$ is the ideal of holomorphic functions that vanish at $b$. Observing that $(f + \van_b)^n$ lies within $f^n + \van_b^{n+1}$ for any $f \in \van_b$, we can identify $T^*B_b^{\otimes n}$ with $\van_b^n / \van_b^{n+1}$ for $n \ge 1$. When $b$ is an ordinary point, the function $\zeta_b$ defined in Section~\ref{transl:dir} represents a nonzero element of $\van_b / \van_b^2$: the cotangent vector $\lambda\big|_b$. In general, $\zeta_b$ represents a nonzero element of $\van_b^n / \van_b^{n+1}$, where $2\pi n$ is the cone angle at $b$. We define $z$ as the isomorphism
\begin{align*}
z \maps \van_b^n / \van_b^{n+1} & \to \C \\
\zeta_b + \van^{n+1} & \mapsto 1.
\end{align*}

When $b$ is a branch point, the coordinate $\omega_b$ we chose in Section~\ref{transl:dir} gives us an isomorphism
\begin{align*}
w_b \maps T^*B_b & \to \C \\
\omega_b + \van^2 & \mapsto 1
\end{align*}
that makes the diagram
\begin{center}
\begin{tikzcd}
T^*B_b^{\otimes n} \arrow[r,"z"] & \C \\
T^*B_b \arrow[u,"\blankbox^n"] \arrow[r,"w"'] & \C \arrow[u,"\blankbox^n"']
\end{tikzcd}
\end{center}
commute.
\subsubsection{Boundary}
\textbf{Discuss the visual boundary, citing Lemma~3.1 of Dankwart's thesis \textit{On the large-scale geometry of flat surfaces} for the description of geodesics.}
\subsection{The Laplace transform}
\subsubsection{Over an ordinary point}%%\label{laplace:ordinary}
Pick a local holomorphic function $\zeta$ on $B$ with $d\zeta = \lambda$, and an extended angle $\theta \in \R$. \textcolor{VioletRed}{[If we define ``translation parameter'' earlier, we can say:] Pick a translation parameter $\zeta$.} The {\em Laplace transform} $\laplace_\zeta^\theta$ turns a local holomorphic function $f$ on $B$ into a local holomorphic function on $T^*B$. When $b \in B$ is an ordinary point, $\laplace_\zeta^\theta f$ is defined on $T^*B_b$ by the formula
\begin{equation}\label{laplace:int}
\laplace_\zeta^\theta f\big|_b = \int_{\Gamma_b^\theta} e^{-z\zeta} f\,d\zeta,
\end{equation}
where $z$ is the frequency function defined in Section~\ref{transl:freq} and $\Gamma_b^\theta$ is the ray that leaves $b$ at angle $\theta$.

To make sense of this formula, we ask for the following conditions.
\begin{itemize}
\item The base point $b$ is in the domain of $\zeta$. Once we have this, we can continue $\zeta$ along the whole ray $\Gamma_b^\theta$.
\item The ray $\Gamma_b^\theta$ avoids the branch points after leaving $b$.
\item The integral converges. We guarantee this by asking for a pair of simpler conditions.
\begin{itemize}
\item With respect to the flat metric, $f$ grows subexponentially along $\Gamma_b^\theta$~\textbf{[define]}, and is locally integrable throughout.
\item The value of $z$ is in the half-plane $H_{-\theta}$ centered around the ray $e^{-i\theta} [0, \infty)$.
%%defined as $e^{-i\theta}$ times the right half-plane.
\end{itemize}
\end{itemize}
%%When $b \in B$ is an ordinary point, there's a unique ray $\Gamma_b^\theta$ leaving $b$ at angle $\theta$. Thanks to the labeling choices we made at the end of Section~\ref{transl:dir}, we can say the same when $b$ is a branch point.

%%Pick a point $b \in B$ and an extended angle $\theta \in \R/2\pi n\Z$, where $2\pi n$ is the cone angle at $b$. Pick a translation chart $\zeta$ whose domain includes the beginning of the ray that leaves $b$ at angle $\theta$. Since $b$ might be a branch point, we don't require the domain of $\zeta$ to include $b$ itself. Make sure the ray $\Gamma_b^\theta$ described above doesn't hit a branch point.
\subsubsection{Over a branch point}
When $b$ is a branch point, we can still use formula~\ref{laplace:int} to define $\laplace_\zeta^\theta f$ on $T^*B_b$, as long as we take care of a few subtleties. Thanks to the labeling choices we made at the end of Section~\ref{transl:dir}, the extended angle $\theta \in \R$ still picks out a ray $\Gamma_b^\theta$. The function $z$ is defined on $T^*B_b^{\otimes n}$, where $2\pi n$ is cone angle at $b$, so we pull it back to $T^*B_b$ along the $n$th-power map. This amounts to substituting $w_b^n$ for $z$ in formula~\ref{laplace:int}. The half-plane $z \in H_{-\theta}$ in $T^*B_b^{\otimes n}$ pulls back to $n$ sectors of angle $\pi/n$ in $T^*B_b$. We only define $\laplace_\zeta^\theta f$ on one of them: the one centered around the ray $w_b \in e^{-i\theta/n}[0, \infty)$.

%%We only define $\laplace_\zeta^\theta f$ on one of them: the sector $w_b \in \R\exp\Big(-\tfrac{\theta}{n} + \big(-\tfrac{\pi}{2n}, \tfrac{\pi}{2n}\big)\Big)$.

%%the sector where the phase of $w_b$ is in $\tfrac{1}{n}\big(-\theta + (-\frac{\pi}{2}, \frac{\pi}{2})\big)$.

%%$w_b \in \R \exp\big(\frac{1}{n}(-\theta - \frac{\pi}{2}, -\theta + \frac{\pi}{2})\big)$

%%We define $\laplace_\zeta^\theta f$ on the one where $e^{i\theta/n} w_b$ is closest to the positive real axis.
\section{Borel--Laplace summability and Resurgence}

\[
\begin{tikzcd}
\C\oplus z^{-1}\C[\![z^{-1}]\!]_1\arrow[rr,"\borel"]& & \C\delta\oplus \C\lbrace\zeta\rbrace\cap\mathcal{N}\arrow[dl,"\mathcal{L}"]\\
& s \arrow[lu, "\text{Gevrey asymptotic}"] & 
\end{tikzcd}
\]

\subsection{Resurgence}

Alternative to Borel--Laplace summability, J. Ecalle introduced the theory of resurgence: divergent power series that are Gevrey-$1$ become germs of holomorphic functions at the origin in the Borel plane. Being divergent, they have singularities in Borel plane and the aim of resurgence theory is to investigate the type of singularities and the analytic continuation of the germ. Indeed a formal series is resurgent if it admits an \textit{endless} analytic continuation.

\begin{definition}
A germ of analytic functions $\hat{f}$ at the origin is \emph{endlessly continuable }on $\C$ if for all $L>0$, there exists a finite set $\Omega\subset\C$ of singularities, such that $\hat{f}$ can be analytically continued along all paths whose length is less than $L$, avoiding the singularities $\Omega$.
\end{definition}
 

\[
\begin{tikzcd}
\C\oplus z^{-1}\C[\![z^{-1}]\!]_1\arrow[r,"\borel"]& \C\delta\oplus \C\lbrace\zeta\rbrace\arrow[rr,"\text{analytic cont.} "] & & \mathcal{R}_{\Omega}\arrow[r, "\pi"]& \widetilde{\C\setminus\Omega}
\end{tikzcd}
\]


Among resurgent series, the class of simple resurgent series is caracterized by having a special type of singularities: 

\begin{definition}
A holomorphic function $\hat{f}$ in an open disk $D\subset \C_{\zeta}$ has a simple singularity at $\omega$, adherent to $D$, if there exist $\alpha\in\C$ and a germs of analytic functions at
the origin $\hat{\Phi}(\zeta)\in\C\lbrace\zeta\rbrace$, such that
\begin{equation}
\hat{f}(\zeta)=\frac{\alpha}{2\pi i(\zeta-\omega)}+\frac{1}{2\pi i}\log(\zeta-\omega)\hat{\phi}(\zeta-\omega)+\text{hol. fct.}
\end{equation}
for all $\zeta\in D$ close enough to $\omega$. The constant $\alpha$ is called the residuum and $\hat{\phi}$ the minor.
\end{definition}

The holomorphic function $\hat{\phi}$ associated with the logarithmic singularity can be obtained by considering the analytic continuation of $\hat{f}$ across the logarithmic branch cut
\begin{equation}
\hat{\phi}(\zeta)=\hat{f}(\zeta+\omega)-\hat{f}(\zeta e^{-2\pi i}+\omega)
\end{equation}
where with $\hat{f}(\zeta e^{-2\pi i}+\omega)$ is the analytic continuation of $\hat{f}$ along the circular
path $\omega+\zeta e^{-2\pi i t}$ with $t\in [0,1]$.
 
\begin{definition}[\cite{Sauzin_notes} Definition 7]
A simple resurgent function is a resurgent function $c\delta +\hat{f}(\zeta)$ such that, for each $\omega\in\Omega$ and for each path $\gamma$ which starts from the origin $0\in\C_{\zeta}$, lies in $\C\setminus\Omega$ and has its extremity in the disc of radius $\pi$ centred at $\omega$, the branch $\text{cont}_\gamma\hat{f}$ has a simple singularity at $\omega$.
\end{definition}  
 
%\color{orange}
%
%
%\begin{theorem}\label{thm:frac-diff-borel}
%Given a Gevrey-$1$ formal series $\varphi(z) = \sum_{k \ge 0}a_kz^{-(k+1)}$, we have
%\[ \partial^\mu_{\zeta \text{ from } 0} \left[ \borel \varphi \right](\zeta) = \left[ \borel z^\mu \varphi \right](\zeta) \]
%for any $\mu \in \R \smallsetminus \{0, 1, 2, 3, \ldots\}$.
%\end{theorem}

%\color{MidnightBlue}
%\begin{definition}
%Let $\alpha\in (0,1)$ and $n\in\N$, then the $n+\alpha$-Caputo's derivative of a smooth fucntion $f$ is defined as
%\begin{equation}
%\partial_x^{n+\alpha}f(x)\defeq\frac{1}{\Gamma(1-\alpha)}\int_0^x(x-s)^{-\alpha}f^{(n+1)}(s)ds
%\end{equation}
%\end{definition}
%
%In particular, this definition is well suited for the differential calculus in the convolutive model $\left(\C[\![\zeta]\!], \ast\right)$. Let $\varphi(z)\defeq\sum_{k\geq 0}a_kz^{-k-1}\in\C[\![z^{-1}]\!]$ be Gevrey $1$, then assuming $a_k=0$ for every $k<n$, the Borel transform of $z^{n+\alpha}\varphi(z)$ can be computed in two different ways: \textbf{[Can we do this when $\varphi$ isn't in $o(z^n)$?]}
%\begin{multline}
%\label{mod1}
%\mathcal{B}\left(z^{n+\alpha}\varphi(z)\right)(\zeta)=\mathcal{B}(z^{\alpha+n})\ast\hat{\varphi}(\zeta)=\int_0^{\zeta}\frac{(\zeta-s)^{-1-n-\alpha}}{(-1-n-\alpha)!}\sum_{k\geq 0}\frac{a_k}{k!}s^{k}ds\\
%=\frac{1}{(-\alpha)!}\int_0^{\zeta}(\zeta-s)^{-\alpha}\sum_{k\geq 0}\frac{a_k}{(k-n-1)!}s^{k-n-1}ds=\partial_{\zeta}^{n+\alpha}\hat{\varphi}(\zeta)
%\end{multline}
%\begin{equation}
%\label{mod2}
%\mathcal{B}\left(z^{n+\alpha}\varphi(z)\right)(\zeta)=\mathcal{B}\left(\sum_{k\geq 0}a_kz^{-k-1+n+\alpha}\right)(\zeta)=\sum_{k>n}\frac{a_k}{(k-n-\alpha)!}\zeta^{k-n-\alpha}
%\end{equation}
%
%and computitng the integral which defines the $n+\alpha$-derivative in \eqref{mod1} we get exaclty the same result as \eqref{mod2}.
\color{black}
\section{Exponential integrals}
\subsection{Borel }
Let $X$ be a $N-\dim$ manifold, $f\colon X\to\C$ be a holomorphic Morse function with only simple critical points, and $\nu\in\Gamma(X,\Omega^N)$, and set
\begin{equation}
I(z)\defeq\int_{\mathcal{C}}e^{-zf}\nu
\end{equation}
where $\mathcal{C}$ is a suitable countur such that the integral is well defined.  
For any Morse cirtial points $x_\alpha$ of $f$, the saddle point approximation gives the following formal series 
\begin{equation}
I_{\alpha}(z)\defeq\int_{\mathcal{C}_\alpha}e^{-zf}\nu\sim \tilde{I}_{\alpha}\defeq e^{-zf(x_\alpha)}(2\pi)^{N/2} z^{-N/2}\sum_{n\geq 0}a_{\alpha,n}z^{-n} \qquad \text{ as } z\to\infty
\end{equation}
where $\mathcal{C}_\alpha$ is a steepest descent path through the critical point $x_\alpha$. Notice that $f \circ \mathcal{C}_\alpha$ lies in the ray $\zeta_\alpha +[0, \infty)$, where $\zeta_\alpha := f(x_\alpha)$.
\begin{theorem}\label{thm:maxim} Let $N=1$. Let $\tilde{\varphi}_{\alpha}(z)\defeq e^{-zf(x_\alpha)}(2\pi)^{1/2} \sum_{n\geq 0}a_{\alpha,n}z^{-n}$ and assume $f''(x_\alpha)\neq 0$ for every critical point $x_\alpha$. Then:
\begin{enumerate}
\item\label{int:series-gevrey} The series $\tilde{\varphi}_\alpha$ is Gevrey-1.
\item\label{int:resum-converges} The series $\hat{\varphi}_\alpha(\zeta)\defeq\mathcal{B}(\tilde{\varphi})$ converges near $\zeta=\zeta_{\alpha}$.
\item\label{int:resum-valid} If you continue the sum of $\hat{\phi}_\alpha$ along the ray going rightward from $\zeta_\alpha$, and take its Laplace transform along that ray, you'll recover $z^{1/2} I_\alpha$.
\item\label{int:deriv-formula} For any $\zeta$ on the ray going rightward from $\zeta_\alpha$, we have
\begin{align*}\label{formula1}
\hat{\varphi}_{\alpha}(\zeta) & = \partial^{\textcolor{red}{3/2}}_{\zeta \text{ from }\zeta_\alpha} \left( \int_{\mathcal{C}_\alpha(\zeta)} \nu \right) \\
& = \textcolor{red}{\left(\tfrac{\partial}{\partial \zeta}\right)^2}\,\frac{1}{\Gamma\big(\tfrac{1}{2}\big)} \int_{\zeta_\alpha}^\zeta (\zeta-\zeta')^{-1/2} \textcolor{red}{\left( \int_{\mathcal{C}_\alpha(\zeta')} \nu \right)}\,d\zeta',
\end{align*}
where $\mathcal{C}_\alpha(\zeta)$ is the part of $\mathcal{C}_\alpha$ that goes through $f^{-1}([\zeta_\alpha, \zeta])$. Notice that $\mathcal{C}_\alpha(\zeta)$ starts and ends in $f^{-1}(\zeta)$. \textbf{[Be careful about the orientation of $\mathcal{C}_\alpha$.]}
\end{enumerate}
\end{theorem}
\begin{proof}
Part~\eqref{int:series-gevrey}: Let's write $\approx$ when two functions are asymptotic (at all orders around the base point \textbf{[is this the right condition?]}), and $\sim$ when a function is asymptotic to a formal power series (at the truncation order of each partial sum).

Since $f$ is Morse, we can find a holomorphic chart $\tau$ around $x_\alpha$ with $\tfrac{1}{2} \tau^2 = f - \zeta_\alpha$. Let $\mathcal{C}^-_\alpha$ and $\mathcal{C}^+_\alpha$ be the parts of $\mathcal{C}_\alpha$ that go from the past to $x_\alpha$ and from $x_\alpha$ to the future, respectively. We can arrange for $\tau$ to be valued in $(-\infty, 0]$ and $[0, \infty)$ on $\mathcal{C}^-_\alpha$ and $\mathcal{C}^+_\alpha$, respectively. \textbf{[We should explicitly spell out and check the conditions that make this possible. I think we're implicitly orienting $\mathcal{C}_\alpha$ so that $\tau$ in the upper half-plane.]} Since $\nu$ is holomorphic, we can express it as a Taylor series
\[ \nu = \sum_{n \ge 0} b_n^\alpha \tau^n\,d\tau \]
that converges in some disk $|\tau| < \varepsilon$.

By the steepest descent method,
\[ e^{-z\zeta_\alpha} I_\alpha(z) \approx \int_{\tau \in [-\varepsilon, \varepsilon]} e^{-z\tau^2/2} \nu \]
as $z \to \infty$. \textbf{[I need to learn how this works! Do we get asymptoticity at all orders? ---Aaron]} Plugging in the Taylor series above, we get
\begin{align*}
e^{-z\zeta_\alpha} I_\alpha(z) & \approx \int_{-\varepsilon}^\varepsilon e^{-z\tau^2/2} \sum_{n \ge 0} b_n^\alpha \tau^n\,d\tau \\
& = \int_{-\varepsilon}^\varepsilon e^{-z\tau^2/2} \sum_{n \ge 0} b_{2n}^\alpha \tau^{2n}\,d\tau.
\end{align*}
By the dominated convergence theorem,\footnote{Notice that the sum over $k$ is empty when $n = 0$. Following convention, we extend the double factorial to all odd integers by its recurrence relation, giving $(-1)!! = 1$.}
\begin{align*}
e^{-z\zeta_\alpha} I_\alpha(z) & \approx \sum_{n \ge 0} b_{2n}^\alpha \int_{-\varepsilon}^\varepsilon e^{-z\tau^2/2} \tau^{2n}\,d\tau \\
& = \sum_{n \ge 0} (2n-1)!!\,b_{2n}^\alpha \left[ \sqrt{2\pi}\,z^{-(n+1/2)} \operatorname{erf}\big(\varepsilon \sqrt{z/2}\big) - 2e^{-z\varepsilon^2/2} \sum_{\substack{k \in \N_+ \\ k \le n}} \frac{\varepsilon^{2k-1}}{(2k-1)!!} z^{n-k+1} \right].
\end{align*}

The annoying $e^{-z\varepsilon^2/2}$ correction terms are dwarfed by their $z^{-(n+1/2)}$ counterparts when $z$ is large. These terms are crucial, however, for the convergence of the sum. To see why, consider their absolute sum $C_\text{exp}$. When $z \in [0, \infty)$,
\begin{align*}
C_\text{exp} & = 2e^{-\operatorname{Re}(z)\varepsilon^2/2} \sum_{n \ge 1} (2n-1)!!\,\left| b_{2n}^\alpha \sum_{\substack{k \in \N_+ \\ k \le n}} \frac{\varepsilon^{2k-1}}{(2k-1)!!} z^{n-k+1} \right| \\
& = 2e^{-z\varepsilon^2/2} \sum_{n \ge 1} (2n-1)!!\,\left|b_{2n}^\alpha\right| \sum_{\substack{k \in \N_+ \\ k \le n}} \frac{\varepsilon^{2k-1}}{(2k-1)!!} z^{n-k+1} \\
& \ge -2\varepsilon e^{-z\varepsilon^2/2} \sum_{n \ge 1} (2n-1)!!\,\left|b_{2n}^\alpha\right| z^n,
\end{align*}
which diverges for typical $f$ and $\nu$. \textbf{[Does it? Veronica points out that we expect $b_{2n}$ to shrink at least as fast as $(n!)^{-1}$.]}

This argument suggests that no matter how tiny the correction terms get, we can't expect to swat them all aside. We can, however, set aside any finite set of them. \color{violet}\textbf{[Use Miller's proof of Watson's lemma in place of the following argument, which has a few soft spots. See also Loday-Richaud, \S 5.1.5, Theorem~5.1.3]} For each cutoff $N$, the tail
\[ \sum_{n \ge N} b_{2n}^\alpha \int_{-\varepsilon}^\varepsilon e^{-z\tau^2/2} \tau^{2n}\,d\tau \]

\color{CarnationPink}
For each cutoff $N$, the tail error \textbf{[check]}
\begin{align*}
\left| \sum_{n \ge N} b_{2n}^\alpha \int_{-\varepsilon}^\varepsilon e^{-z\tau^2/2} \tau^{2n}\,d\tau \right| & \le \sum_{n \ge N} \left| b_{2n}^\alpha \right| \int_{-\varepsilon}^\varepsilon e^{-|z|\tau^2/2} \tau^{2n}\,d\tau \\
& \le \sum_{n \ge N} \left| b_{2n}^\alpha \right| \int_{-\infty}^\infty e^{-|z|\tau^2/2} \tau^{2n}\,d\tau \\
& = \sqrt{2\pi} \sum_{n \ge N} (2n-1)!!\,\left| b_{2n}^\alpha \right| |z|^{-(n+1/2)} \\
& \lesssim \sum_{n \ge N} (2n-1)!!\,\varepsilon^{-2n} |z|^{-(n+1/2)} \\
& = \varepsilon \sum_{n \ge N} (2n-1)!!\,\big(\varepsilon^{-1}\big)^{2n+1} \big(|z|^{-1/2}\big)^{2n+1} \\
& = \varepsilon \sum_{n \ge N} (2n-1)!!\,\big(\varepsilon^{-1} |z|^{-1/2}\big)^{2n+1} \\
& = \textbf{uh-oh!}
\end{align*}
\color{violet}
is in $o_{z \to \infty}(z^{-N})$ \textbf{[check]}, and the absolute sum
\begin{align*}
C_\text{exp}^N & = 2e^{-\operatorname{Re}(z)\varepsilon^2/2} \sum_{n = 1}^{N-1} (2n-1)!!\,\left| b_{2n}^\alpha \sum_{\substack{k \in \N_+ \\ k \le n}} \frac{\varepsilon^{2k-1}}{(2k-1)!!} z^{n-k+1} \right| \\
& \le 2e^{-\operatorname{Re}(z)\varepsilon^2/2} \sum_{n = 1}^{N-1} (2n-1)!!\,\left|b_{2n}^\alpha\right| \sum_{\substack{k \in \N_+ \\ k \le n}} \frac{\varepsilon^{2k-1}}{(2k-1)!!} |z|^{n-k+1} \\
& \ge -2\varepsilon e^{-z\varepsilon^2/2} \sum_{n \ge 1} (2n-1)!!\,\left|b_{2n}^\alpha\right| z^n,
\end{align*}
is in $o_{z \to \infty}(z^{-m})$ for every $m$ \textbf{[check]}.\color{black} Hence,
\[ e^{-z\zeta_\alpha} I_\alpha(z) \sim \sqrt{2\pi} \sum_{n \ge 0} (2n-1)!!\,b_{2n}^\alpha\,z^{-(n+1/2)} \operatorname{erf}\big(\varepsilon \sqrt{z/2}\big). \]
The differences $1 - \operatorname{erf}\big(\varepsilon \sqrt{z/2}\big)$ shrink exponentially as $z$ grows, allowing the simpler estimate
\[ e^{-z\zeta_\alpha} I_\alpha(z) \sim \sqrt{2\pi} \sum_{n \ge 0} (2n-1)!!\,b_{2n}^\alpha\,z^{-(n+1/2)}. \]
Call the right-hand side $\tilde{I}_\alpha$. We now see that $a_{\alpha,n} = (2n-1)!!\,b_{2n}^\alpha$ in the statement of the theorem. \textbf{[Resolve discrepancy with previous calculation.]} Note that \textbf{[explain formally what it means to center at $\zeta_\alpha$]}
\begin{align*}
\mathcal{B}_{\zeta_\alpha} \tilde{I}_\alpha & = \sqrt{2\pi} \sum_{n \ge 0} \frac{2^n}{\sqrt{\pi}} \Gamma\big(n+\tfrac{1}{2}\big)\,b_{2n}^\alpha\,\frac{(\zeta - \zeta_\alpha)^{n-1/2}}{\Gamma\big(n+\tfrac{1}{2}\big)} \\
& = \sum_{n \ge 0} 2^{n+1/2}\,b_{2n}^\alpha\,(\zeta - \zeta_\alpha)^{n-1/2}.
\end{align*}

We know from the definition of $\varepsilon$ that $\left|b_n^\alpha\right| \varepsilon^n \lesssim 1$. Recalling that $(2n - 1)!! \approx (\pi n)^{-1/2}\,2^n\,n!$ as $n \to \infty$, we deduce that $|a_{\alpha,n}| \lesssim \left(\tfrac{2}{\varepsilon^2}\right)^n\,n!\,$, showing that $\tilde{\varphi}_\alpha$ is Gevrey-1.

Part~\eqref{int:resum-converges}: \begin{align*}
\hat{\varphi}_\alpha(\zeta)=\mathcal{B}\left(e^{-zf(x_\alpha)}(2\pi)^{1/2} \sum_{n\geq 0}a_{\alpha,n}z^{-n}\right)(\zeta)=T_{f(x_\alpha)}(2\pi)^{1/2} \left(\delta a_0+\sum_{n\geq 0}a_{n+1}\frac{\zeta^n}{n!}\right)\\
(2\pi)^{1/2} \left(\delta(f_{x_\alpha}) a_0+\sum_{n\geq 0}a_{n+1}\frac{(\zeta-f(x_\alpha))^n}{n!}\right)
\end{align*}
Since $a_{n}\leq CA^nn!$, the series $\sum_{n\geq 0}a_{n+1}\frac{(\zeta-f(x_\alpha))^n}{n!}$ has a finite radius of convergence. 

Part~\eqref{int:resum-valid}: Let's recast the integral $I_\alpha$ into the $f$ plane. As $\zeta$ goes rightward from $\zeta_\alpha$, the start and end points of $\mathcal{C}_\alpha(\zeta)$ sweep backward along $\mathcal{C}^-_\alpha(\zeta)$ and forward along $\mathcal{C}^+_\alpha(\zeta)$, respectively. Hence, we have
\begin{align*}
I_\alpha(z) & = \int_{\mathcal{C}_{\alpha}} e^{-zf} \nu \\
& = \int_{\zeta_\alpha}^\infty e^{-z\zeta} \left[\frac{\nu}{df}\right]_{\operatorname{start} \mathcal{C}_\alpha(\zeta)}^{\operatorname{end} \mathcal{C}_\alpha(\zeta)}\,d\zeta.
\end{align*}
Noticing that the right-hand side is a Laplace transform, we learn that
\begin{equation}\label{thimble-difference}
\hat{I}_\alpha(\zeta) = \left[\frac{\nu}{df}\right]_{\operatorname{start} \mathcal{C}_\alpha(\zeta)}^{\operatorname{end} \mathcal{C}_\alpha(\zeta)}.
\end{equation}

We can rewrite our Taylor series for $\nu$ as
\begin{align*}
\nu & = \sum_{n \ge 0} b_n^\alpha [2(f - \zeta_\alpha)]^{n/2}\,\frac{df}{[2(f - \zeta_\alpha)]^{1/2}} \\
& = \sum_{n \ge 0} b_n^\alpha [2(f - \zeta_\alpha)]^{(n - 1)/2}\,df,
\end{align*}
taking the positive branch of the square root on $\mathcal{C}^+_\alpha$ and the negative branch on $\mathcal{C}^-_\alpha$. Plugging this into our expression for $\hat{I}_\alpha$, we learn that
\begin{align*}
\hat{I}_\alpha(\zeta) & = \left[ \sum_{n \ge 0} b_n^\alpha [2(f - \zeta_\alpha)]^{(n - 1)/2} \right]_{\operatorname{start} \mathcal{C}_\alpha(\zeta)}^{\operatorname{end} \mathcal{C}_\alpha(\zeta)} \\
& = \sum_{n \ge 0} b_n^\alpha \Big( [2(\zeta - \zeta_\alpha)]^{(n - 1)/2} - (-1)^{n-1}[2(\zeta - \zeta_\alpha)]^{(n - 1)/2} \Big) \\
& = \sum_{n \ge 0} 2 b_{2n}^\alpha [2(\zeta - \zeta_\alpha)]^{n - 1/2} \\
& = \sum_{n \ge 0} 2^{n+1/2} b_{2n}^\alpha (\zeta - \zeta_\alpha)^{n - 1/2} \\
& = \mathcal{B}_{\zeta_\alpha} \tilde{I}_\alpha.
\end{align*}
We already knew, from the general theory of the Borel transform, that the sum of $\mathcal{B}_{\zeta_\alpha} \tilde{I}_\alpha$ would be asymptotic to $\hat{I}_\alpha$. We've now shown that the sum of $\mathcal{B}_{\zeta_\alpha} \tilde{I}_\alpha$ is actually equal to $\hat{I}_\alpha$.

Theorem~\ref{thm:frac-diff-borel} tells us that
\begin{align*}
\mathcal{B}_{\zeta_\alpha} \tilde{I}_\alpha & \defeq \mathcal{B}_{\zeta_\alpha} z^{-1/2} \tilde{\varphi}_\alpha \\
& = \partial^{-1/2}_{\zeta \text{ from } \zeta_\alpha} \mathcal{B} \tilde{\varphi}_\alpha \\
& = \partial^{-1/2}_{\zeta \text{ from } \zeta_\alpha} \hat{\varphi}_\alpha.
\end{align*}
It follows, from our conclusion above, that
\begin{equation}\label{shifted-resum-valid}
\hat{I}_\alpha(\zeta) = \partial^{-1/2}_{\zeta \text{ from } \zeta_\alpha} \hat{\varphi}_\alpha.
\end{equation}
Taking the Laplace transform of both sides and applying \textbf{the inverse of Theorem~\ref{thm:frac-diff-borel} that works for shifted analytic functions}, we see that
\begin{align*}
I_\alpha(z) & = \laplace_{\zeta, \zeta_\alpha} \left[ \partial^{-1/2}_{\zeta \text{ from } \zeta_\alpha} \hat{\varphi}_\alpha \right] \\
& = z^{-1/2} \laplace_{\zeta, \zeta_\alpha} \hat{\varphi}_\alpha,
\end{align*}
as we claimed.

Part~\eqref{int:deriv-formula}: Since fractional integrals form a semigroup, equation~\eqref{shifted-resum-valid} implies that
\[ \partial^{-1}_{\zeta \text{ from } \zeta_\alpha} \hat{I}_\alpha(\zeta) = \partial^{-3/2}_{\zeta \text{ from } \zeta_\alpha} \hat{\varphi}_\alpha. \]
Rewriting equation~\eqref{thimble-difference} as
\[ \hat{I}_\alpha(\zeta) = \partial_\zeta \left( \int_{\mathcal{C}_\alpha(\zeta)} \nu \right), \]
we can see that
\[ \partial^{-1}_{\zeta \text{ from } \zeta_\alpha} \hat{I}_\alpha(\zeta) = \int_{\mathcal{C}_\alpha(\zeta)} \nu - \int_{\mathcal{C}_\alpha(0)} \nu. \]
The initial value term vanishes, because the path $\mathcal{C}_\alpha(0)$ is a point. Hence,
\[ \int_{\mathcal{C}_\alpha(\zeta)} \nu = \partial^{-3/2}_{\zeta \text{ from } \zeta_\alpha} \hat{\varphi}_\alpha(\zeta). \]
Recalling that the Riemann-Liouville fractional derivative is a left inverse of the fractional integral, we conclude that
\[ \partial^{3/2}_{\zeta \text{ from } \zeta_\alpha} \left( \int_{\mathcal{C}_p(\zeta)} \nu \right) = \hat{\varphi}_p(\zeta). \]
\end{proof}



\section{Airy exponential integral}
By definition,
\[ \Ai(x) \defeq \frac{1}{2\pi i}\int_{-\infty e^{-i\frac{\pi}{3}}}^{\infty e^{i\frac{\pi}{3}}} e^{t^3/3 - xt}\,dt. \]
Define $I(z)$ by the change of coordinates $z=x^{3/2}$, $I(z)=-2\pi iz^{-1/3}\Ai(x)$. This new function satisfies the ODE\footnote{$\Ai(x)$ solves the Airy equation $y''=xy$.}
\begin{equation}\label{eq:I}
I''(z)-\frac{4}{9}I(z)+\frac{I'(z)}{z}-\frac{1}{9}\frac{I(z)}{z^2} = 0.
\end{equation}
A formal solution of \eqref{eq:I} can be computed by making the following ansatz 
\begin{equation}
\tilde{I}(z)=\sum_{k\in\N^2}U^ke^{-\lambda\cdot k z}z^{-\tau\cdot k}w_k(z)
\end{equation}
with $U^{(k_1,k_2)}=U_1^{k_1}U_2^{k_2}$ and $U_1,U_2\in\C$ are constant parameter, $\lambda=(\frac{2}{3},-\frac{2}{3})$, $\tau=(\frac{1}{2},\frac{1}{2})$, and $\tilde{w}_k(z)\in\C[[z^{-1}]]$. In addition, we can check that the only non zero $\tilde{w}_k(z)$ occurs at $k=(1,0)$ and $k=(0,1)$, therefore
\begin{equation}
\tilde{I}(z)=U_1e^{-2/3z}z^{-1/2}\tilde{w}_{+}(z)+U_2e^{2/3z}z^{-1/2}\tilde{w}_{-}(z)
\end{equation}  
where from now on we denote $\tilde{w}_+(z)=\tilde{w}_{(1,0)}(z)$ and $\tilde{w}_-(z)=\tilde{w}_{(0,1)}(z)$. In particular, $\tilde{w}_+(z)$ and $\tilde{w}_-(z)$ are formal solution of 
\begin{align}
\label{eq:w+} \tilde{w}_+''-\frac{4}{3}\tilde{w}_+'+\frac{5}{36}\frac{\tilde{w}_+}{z^2}=0\\
\label{eq:w-} \tilde{w}_-''+\frac{4}{3}\tilde{w}_-'+\frac{5}{36}\frac{\tilde{w}_-}{z^2}=0
\end{align}
Taking the Borel transform of \eqref{eq:w+}, \eqref{eq:w-} we get
\begin{align*}
&\zeta^2\hat{w}_{+}(\zeta)+\frac{4}{3}\zeta\hat{w}_{+}+\frac{5}{36}\zeta\ast\hat{w}_{+}=0\\
&\zeta^2\hat{w}_{+}(\zeta)+\frac{4}{3}\zeta\hat{w}_{+}+\frac{5}{36}\int_0^\zeta(\zeta-\zeta')\hat{w}_{+}(\zeta')d\zeta'=0
\end{align*}
\begin{align*}
&\zeta^2\hat{w}_{-}(\zeta)-\frac{4}{3}\zeta\hat{w}_{-}+\frac{5}{36}\zeta\ast\hat{w}_{-}=0\\
&\zeta^2\hat{w}_{-}(\zeta)-\frac{4}{3}\zeta\hat{w}_{-}+\frac{5}{36}\int_0^\zeta(\zeta-\zeta')\hat{w}_{-}(\zeta')d\zeta'=0
\end{align*}
and taking derivatives we get
\begin{align*}
&\zeta(\frac{4}{3}+ \zeta)\hat{w}_{+}''+(\frac{8}{3}+4\zeta)\hat{w}_+'+\frac{77}{36}\hat{w}_{+}=0 & \\
&\frac{4}{3}\zeta( 1+ \frac{3}{4}\zeta)\hat{w}_{+}''+(\frac{8}{3}+4\zeta)\hat{w}_+'+\frac{77}{36}\hat{w}_{+}=0 & \\
&\qquad u(1-u)\hat{w}_+''(u)+(2-4u)\hat{w}_+'(u)-\frac{77}{36}\hat{w}_+(u)=0 & u=-\frac{3}{4}\zeta\\
%&\qquad u(1-u)\hat{w}_-''(u)+(2-4u)\hat{w}_-'(u)-\frac{77}{36}\hat{w}_-(u) & u=\frac{3}{4}\zeta
\end{align*} 
\begin{align*}
&\zeta(-\frac{4}{3}+ \zeta)\hat{w}_{-}''+(-\frac{8}{3}+4\zeta)\hat{w}_-'+\frac{77}{36}\hat{w}_{-}=0 & \\
&\frac{4}{3}\zeta(-1+ \frac{3}{4}\zeta)\hat{w}_{-}''+(-\frac{8}{3}+4\zeta)\hat{w}_-'+\frac{77}{36}\hat{w}_{-}=0 & \\
&\qquad u(1-u)\hat{w}_-''(u)+(2-4u)\hat{w}_-'(u)-\frac{77}{36}\hat{w}(u)=0 & u=\frac{3}{4}\zeta
\end{align*} 
Notice that the latter equations are hypergeometric, hence a solution is given by 
\begin{align}
\label{hat+}\hat{w}_+(\zeta)=\mathit{c}_1 \, {}_{1}F_{2}\left(\frac{7}{6},\frac{11}{6},2,-\frac{3}{4}\zeta\right)\\
\label{hat-}\hat{w}_-(\zeta)=\mathit{c}_2 \, {}_{1}F_{2}\left(\frac{7}{6},\frac{11}{6},2,\frac{3}{4}\zeta\right)
\end{align}
for some constants $\mathit{c}_1, \mathit{c}_2\in\C$ (see DLMF 15.10.2). In addition $\hat{w}_{\pm}(\zeta)$ have a log singularity respectively at $\zeta=\mp\frac{4}{3}$, therefore they are $\lbrace\mp\frac{4}{3}\rbrace $-resurgent functions.\footnote{The solution we find are equal to the ones in DLMF $\mathsection 9.7$. They also agree with slide 10 of Maxim's talk for ERC and with the series (6.121) in Sauzin's book. However they do not agree with (2.16) in Mari\~no's Diablerets.}



%\begin{remark}
%In \cite{Marino-diableret}, Mari\~{n}o studies the example of the Airy function and its resurgent properties. According to his notation
%\begin{equation}
%Ai(x)=\frac{1}{2x^{1/4}\sqrt{\pi}}e^{-\frac{2}{3}x^{3/2}}\varphi_1(x^{-3/2})
%\end{equation}
%\begin{equation}
%Bi(x)=\frac{1}{2x^{1/4}\sqrt{\pi}}e^{\frac{2}{3}x^{3/2}}\varphi_2(x^{-3/2})
%\end{equation}
%with \[\varphi_{1,2}(z)=\sum_{k\geq 0}\frac{1}{2\pi}\left(\mp\frac{3}{4}\right)^k\frac{\Gamma\left(k+\frac{1}{6}\right)\Gamma\left(k+\frac{5}{6}\right)}{k!}z^k.\]
%In particular, the Borel transform of $\varphi_1(z)$ and $\varphi_2(z)$ are respectively 
%\begin{align}
%\label{Marino_hat}
%\hat{\varphi}_1(\zeta)&={}_1F_2\left(\frac{1}{6},\frac{5}{6};1;-\frac{3}{4}\zeta\right)\\
%\label{Marino_hat2}\hat{\varphi}_2(\zeta)&={}_1F_2\left(\frac{1}{6},\frac{5}{6};1;\frac{3}{4}\zeta\right)
%\end{align} 
% In particular, we notice that $\hat{w}_+(\zeta)$ and $\hat{w}_-(\zeta)$ in \eqref{eq:w+}, \eqref{eq:w-} are (up to a constant) the first derivatives of
%\begin{align*}
%\hat{w}_+(\zeta)&\propto\frac{\dd}{\dd\zeta}\hat{\varphi}_1(\zeta)\\
%\hat{w}_-(\zeta)&\propto\frac{\dd}{\dd\zeta}\hat{\varphi}_2(\zeta)
%\end{align*} 
%\end{remark}

Our next goal is to prove that the Borel transform of $\tilde{I}(z)$ can be written in terms of $1/f'(f^{-1}(\zeta))$, namely formula \eqref{formula1}. It is convenient to consider the two asymptotic formal solutions separately, namely we define

\begin{align}
\tilde{I}_{-1}(z)\defeq e^{-2/3z}z^{-1/2}\tilde{w}_+(z)=\colon z^{-1/2}\tilde{u}_+(z) \\
\tilde{I}_{1}(z)\defeq e^{2/3z}z^{-1/2}\tilde{w}_-(z)=\colon z^{-1/2}\tilde{u}_-(z)
\end{align}
 
In particular, $\tilde{u}_{\pm}(z)$ are solutions of 

\begin{equation}\label{eq:u}
\tilde{u}''(z)-\frac{4}{9}\tilde{u}(z)+\frac{5}{36}\frac{\tilde{u}(z)}{z^2}=0
\end{equation} 

with asymptotic behaviour $\tilde{u}_\pm(z)\sim O(e^{\pm 2/3 z})$ as $z\to\infty$. 

The Borel transforms $\hat{u}_{\pm}(\zeta)$ solve the same equation
\begin{align*}
&\zeta^2\hat{u}-\frac{4}{9}\hat{u}+\frac{5}{36}\zeta\ast\hat{u}\\
&\zeta^2\hat{u}-\frac{4}{9}\hat{u}+\frac{5}{36}\int_0^\zeta(\zeta-\zeta')\hat{u}(\zeta')d\zeta'\\
&\text{taking derivatives is equivalent to} \\
&(\zeta^2-\frac{4}{9})\hat{u}''(\zeta)+4\zeta\hat{u}'(\zeta)+\frac{77}{36}\hat{u}(\zeta)=0
\end{align*}

and Mathematica gives the following solutions
\begin{align*}
\hat{u}(\zeta)&=c_1 \, {}_{1}F_{2}\left(\frac{7}{12},\frac{11}{12},\frac{1}{2},\frac{9}{4}\zeta^2\right) +\frac{3i}{2}\zeta c_2 \, {}_{1}F_{2}\left(\frac{13}{12},\frac{17}{12},\frac{3}{2},\frac{9}{4}\zeta^2\right)= &\\
&=c_1\frac{\Gamma(\frac{13}{12})\Gamma(\frac{17}{12})}{2\sqrt{\pi}}\left({}_{1}F_{2}\left(\frac{7}{12},\frac{11}{12},\frac{1}{2},\frac{1}{2}-\frac{3}{4}\zeta\right)- {}_{1}F_{2}\left(\frac{7}{12},\frac{11}{12},\frac{1}{2},\frac{1}{2}+\frac{3}{4}\zeta\right)\right) & \text{see DLMF 15.8.27} \\
&\qquad +\frac{3i}{2}\zeta c_2 \left(\frac{\Gamma(\frac{7}{12})\Gamma(\frac{11}{12})}{3\zeta\Gamma(-\frac{1}{2})\Gamma(2)}\right)\left( {}_{1}F_{2}\left(\frac{7}{6},\frac{11}{6},{2},\frac{1}{2}-\frac{3}{4}\zeta\right)-{}_{1}F_{2}\left(\frac{7}{6},\frac{11}{6},{2},\frac{1}{2}+\frac{3}{4}\zeta\right)\right)  & \text{see DLMF 15.8.28}\\
&=\left(c_1\frac{\Gamma(\frac{13}{12})\Gamma(\frac{17}{12})}{2\sqrt{\pi}} -c_2i\frac{\Gamma(\frac{7}{12})\Gamma(\frac{11}{12})}{4\sqrt{\pi}}\right)\, {}_{1}F_{2}\left(\frac{7}{6},\frac{11}{6},2,\frac{1}{2}-\frac{3}{4}\zeta\right)  + & \\
&\qquad\qquad +\left(c_1\frac{\Gamma(\frac{13}{12})\Gamma(\frac{17}{12})}{2\sqrt{\pi}} +c_2i\frac{\Gamma(\frac{7}{12})\Gamma(\frac{11}{12})}{4\sqrt{\pi}}\right)\, {}_{1}F_{2}\left(\frac{7}{6},\frac{11}{6},2,\frac{1}{2}+\frac{3}{4}\zeta\right) &  
\end{align*}
Since $\hat{u}_+$ has a simple singularity at $\zeta=-2/3$ and $\hat{u}_-$ has a simple singularity at $\zeta=2/3$, we have
\begin{align*}
\hat{u}_+(\zeta)&=C_1 T_{-\tfrac{2}{3}}\;{}_{2}F_{1}\left(\frac{7}{6},\frac{11}{6},2,-\frac{3}{4}\zeta\right)=C_1 T_{-\tfrac{2}{3}}\;\hat{w}_+(\zeta)\\
\hat{u}_-(\zeta)&= C_2 T_{\tfrac{2}{3}}\;{}_{2}F_{1}\left(\frac{7}{6},\frac{11}{6},2,\frac{3}{4}\zeta\right)= C_2 T_{\tfrac{2}{3}}\;\hat{w}_-(\zeta)
\end{align*}

\begin{lemma}
The following identity holds true
\begin{equation}
{}_2F_1\left(\frac{1}{3},\frac{2}{3};\frac{1}{2};\frac{9}{4}\zeta^2\right)=\frac{1}{1-u^2}\qquad \zeta=\frac{u^3}{3}-u
\end{equation}
\end{lemma}
\begin{proof}From the special case of hypergeometric function (see 15.4.14 DLMF) we have the following identity:
\begin{align*}
{}_2F_1\left(\frac{1}{3},\frac{2}{3};\frac{1}{2};\frac{9}{4}\zeta^2\right) &= \frac{\cos(y)}{\cos(3y)} & 3y=\arcsin\left(\frac{3}{2}\zeta\right)\\
&=\frac{\cos(y)}{\cos(2y)\cos(y)-\sin(2y)\sin(y)} & \\
&=\frac{1}{\cos(2y)-2\sin^2(y)} & \\
&=\frac{1}{1-4\sin^2(y)} & \zeta=2\sin(y)-\frac{8}{3}\sin^3(y)
\end{align*}
Therefore, if $u\defeq -2\sin(y)$, we have $\zeta=\frac{u^3}{3}-u=f(u)$ and \[{}_2F_1\left(\frac{1}{3},\frac{2}{3};\frac{1}{2};\frac{9}{4}\zeta^2\right)=\frac{1}{1-u^2}=-\frac{1}{f'(u)}\]
\end{proof}

Then equations \eqref{formula1} is equivalent to: \textbf{[Now that we've switched to the Riemann-Liouville derivative, the claim that was referenced here no longer involves $\nu/df$. The calculation above is still useful, but it should go somewhere else.]}
 
\begin{claim}\label{claim 2}
\textbf{[There should be a way to predict the correct normalization of the RHS.]}
\[ \partial^{3/2}_{\zeta \text{ from } 2/3} \left( \int_{\mathcal{C}_+(\zeta)} \nu \right) = i \tfrac{\sqrt{\pi}}{8}\,\tfrac{5}{12}\,\hat{w}_+(\zeta-2/3), \]
consistent with Theorem~\ref{thm:maxim}~\eqref{int:deriv-formula}.
\end{claim}
\begin{proof}
With the substitution $t = -2ux^{1/2}$, we can rewrite the Airy integral as
\[ \Ai(x) = x^{1/2}\;\frac{i}{\pi} \int_{x^{-1/2} \mathcal{C}_+} \exp\left[-\tfrac{2}{3}x^{3/2} \left(4u^3 - 3u\right)\right]\,du, \]
where $\mathcal{C}_+$ is the path $\theta \mapsto \cosh(\theta - \tfrac{2}{3}\pi i)$. When $x \in [0, \infty)$, this leads to the expression
\[ I_+(z) = \int_{\mathcal{C}_+} \exp\left[-\tfrac{2}{3} z \left(4u^3 - 3u\right)\right]\,du. \]
In our general picture of exponential integrals, $f = \tfrac{2}{3}(4u^3 - 3u)$ and $\nu = du$. Hence,
\begin{align*}
\int_{\mathcal{C}_+(\zeta)} \nu & = \int_{\mathcal{C}_+(\zeta)} du \\
& = u \Big|_{\operatorname{start} \mathcal{C}_+(\zeta)}^{\operatorname{end} \mathcal{C}_+(\zeta)}.
\end{align*}
Since $4u^3 - 3u$ is the third Chebyshev polynomial, and $\cosh$ is $2\pi$-periodic in the imaginary direction, the start and end points of $\mathcal{C}_+(\zeta)$ are characterized by
\begin{align*}
u & = \cosh(\mp\theta - \tfrac{2}{3}\pi i) \\
\zeta & = \tfrac{2}{3} \cosh(3\theta),
\end{align*}
so
\begin{align*}
\int_{\mathcal{C}_+(\zeta)} \nu & = \cosh(\theta - \tfrac{2}{3}\pi i) - \cosh(-\theta - \tfrac{2}{3}\pi i) \\
& = \big[\cosh(\theta) \cosh(-\tfrac{2}{3}\pi i) + \sinh(\theta) \sinh(-\tfrac{2}{3}\pi i)\big] \\
& \quad - \big[\cosh(-\theta) \cosh(-\tfrac{2}{3}\pi i) + \sinh(-\theta) \sinh(-\tfrac{2}{3}\pi i)\big] \\
& = 2\sinh(\theta) \sinh(-\tfrac{2}{3}\pi i) \\
& = -i\sqrt{3}\,\sinh(\theta)
\end{align*}
with $\tfrac{3}{2} \zeta = \cosh(3\theta)$. Let $\xi = \tfrac{1}{2}(1 - \tfrac{3}{2}\zeta)$, and notice that $\xi = -\sinh(\tfrac{3}{2} \theta)^2$ at the start and end points. The identity \textbf{[DLMF~15.4.16]}
\[ \sinh(\theta) = \tfrac{2}{3} \sinh(\tfrac{3}{2} \theta)\,F\big(\tfrac{1}{6}, \tfrac{5}{6}; \tfrac{3}{2}; -\sinh(\tfrac{3}{2} \theta)^2\big) \]
then shows us that
\[ \frac{i}{\sqrt{3}} \int_{\mathcal{C}_+(\zeta)} \nu = \tfrac{2}{3} (-\xi)^{1/2}\,F\big(\tfrac{1}{6}, \tfrac{5}{6}; \tfrac{3}{2}; \xi\big). \]

Now we can evaluate the half-integral of $\int_{\mathcal{C}_+} \nu$ using Bateman's fractional integral formula for hypergeometric functions~\textbf{[Koornwinder, \S 4.1]}.
\begin{align*}
\partial^{-1/2}_{\zeta \text{ from } 2/3} \left( \int_{\mathcal{C}_+(\zeta)} \nu \right) & = \frac{1}{\Gamma\big(\tfrac{1}{2}\big)} \int_{2/3}^\zeta (\zeta - \zeta')^{-1/2} \left( \int_{\mathcal{C}_+(\zeta')} \nu \right)\,d\zeta' \\
& = \frac{1}{\Gamma\big(\tfrac{1}{2}\big)} \int_0^\xi \tfrac{\sqrt{3}}{2} (\xi' - \xi)^{-1/2} \Big[ -{i}{\sqrt{3}}\,\tfrac{2}{3} (-\xi)^{1/2}\,F\big(\tfrac{1}{6}, \tfrac{5}{6}; \tfrac{3}{2}; \xi\big) \Big] \,\big( -\tfrac{4}{3}\,d\xi' \big) \\
& = -i \frac{4}{3} \frac{\Gamma\big(\tfrac{3}{2}\big)}{\Gamma(2)} (-\xi)\,F\big(\tfrac{1}{6}, \tfrac{5}{6}; 2; \xi\big) \\
& = i \frac{2}{3} \sqrt{\pi}\,\xi\,F\big(\tfrac{1}{6}, \tfrac{5}{6}; 2; \xi\big).
\end{align*}
Finally, we differentiate twice using \textbf{[DLMF 15.5.4]} and \textbf{[DLMF 15.5.1]}.
\begin{align*}
\partial^{3/2}_{\zeta \text{ from } 2/3} \left( \int_{\mathcal{C}_+(\zeta)} \nu \right) & = \left(-\tfrac{3}{4} \tfrac{\partial}{\partial \xi}\right)^2 \left[ i \frac{2}{3} \sqrt{\pi}\,\xi\,F\big(\tfrac{1}{6}, \tfrac{5}{6}; 2; \xi\big) \right] \\
& = i \tfrac{3\sqrt{\pi}}{8} \left(\tfrac{\partial}{\partial \xi}\right)^2 \left[ \xi\,F\big(\tfrac{1}{6}, \tfrac{5}{6}; 2; \xi\big) \right] \\
& = i \tfrac{3\sqrt{\pi}}{8}\,\tfrac{\partial}{\partial \xi} \left[ F\big(\tfrac{1}{6}, \tfrac{5}{6}; 1; \xi\big) \right] \\
& = i \tfrac{\sqrt{\pi}}{8}\,\tfrac{5}{12}\,F\big(\tfrac{7}{6}, \tfrac{11}{6}; 2; \xi\big).
\end{align*}
\textbf{[Check comparison with Mari\~{n}o's result more carefully?]}
\end{proof}
Analogously, it can be verified for $\hat{w}_-(\zeta+2/3)$ for $\zeta\in(-\infty,-2/3)$.

\begin{claim}\label{claim 3}

\[ \partial^{3/2}_{\zeta \text{ from } -2/3} \left( \int_{\mathcal{C}_-(\zeta)} \nu \right) = - \tfrac{\sqrt{\pi}}{8}\,\tfrac{5}{12}\,\hat{w}_-(\zeta+2/3), \]
consistent with Theorem~\ref{thm:maxim}~\eqref{int:deriv-formula}.
\end{claim}
\begin{proof}
As before, with the substitution $t = -2ux^{1/2}$, we can rewrite the Airy integral as
\[ \Ai(x) = x^{1/2}\;\frac{i}{\pi} \int_{x^{-1/2} \mathcal{C}_-} \exp\left[-\tfrac{2}{3}x^{3/2} \left(4u^3 - 3u\right)\right]\,du, \]
where $\mathcal{C}_-$ is the path $\theta \mapsto -\cosh(\theta - \tfrac{2}{3}\pi i)$. When $x \in [0, \infty)$, this leads to the expression
\[ I_-(z) = \int_{\mathcal{C}_-} \exp\left[-\tfrac{2}{3} z \left(4u^3 - 3u\right)\right]\,du. \]
In our general picture of exponential integrals, $f = \tfrac{2}{3}(4u^3 - 3u)$ and $\nu = du$. Hence,
\begin{align*}
\int_{\mathcal{C}_-(\zeta)} \nu & = \int_{\mathcal{C}_-(\zeta)} du \\
& = u \Big|_{\operatorname{start} \mathcal{C}_-(\zeta)}^{\operatorname{end} \mathcal{C}_-(\zeta)}.
\end{align*}
The start and end points of $\mathcal{C}_-(\zeta)$ are characterized by
\begin{align*}
u & = -\cosh(\mp\theta - \tfrac{2}{3}\pi i) \\
\zeta & = -\tfrac{2}{3} \cosh(3\theta),
\end{align*}
so
\begin{align*}
\int_{\mathcal{C}_-(\zeta)} \nu & =- \cosh(\theta - \tfrac{2}{3}\pi i) + \cosh(-\theta - \tfrac{2}{3}\pi i) \\
& =- \big[\cosh(\theta) \cosh(-\tfrac{2}{3}\pi i) + \sinh(\theta) \sinh(-\tfrac{2}{3}\pi i)\big] \\
& \quad + \big[\cosh(-\theta) \cosh(-\tfrac{2}{3}\pi i) + \sinh(-\theta) \sinh(-\tfrac{2}{3}\pi i)\big] \\
& = 2\sinh(\theta) \sinh(\tfrac{2}{3}\pi i) \\
& = i\sqrt{3}\,\sinh(\theta)
\end{align*}
with $\tfrac{3}{2} \zeta = -\cosh(3\theta)$. Let $\xi = \tfrac{1}{2}(1 + \tfrac{3}{2}\zeta)$, and notice that $\xi =- \sinh(\tfrac{3}{2} \theta)^2$ at the start and end points. The identity \textbf{[DLMF~15.4.16]}
\[ \sinh(\theta) = \tfrac{2}{3} \sinh(\tfrac{3}{2} \theta)\,F\big(\tfrac{1}{6}, \tfrac{5}{6}; \tfrac{3}{2}; -\sinh(\tfrac{3}{2} \theta)^2\big) \]
then shows us that
\[ -\frac{i}{\sqrt{3}} \int_{\mathcal{C}_-(\zeta)} \nu = \tfrac{2}{3} (-\xi)^{1/2}\,F\big(\tfrac{1}{6}, \tfrac{5}{6}; \tfrac{3}{2}; \xi\big). \]

Now we can evaluate the half-integral of $\int_{\mathcal{C}_-} \nu$ using Bateman's fractional integral formula for hypergeometric functions~\textbf{[Koornwinder, \S 4.1]}.
\begin{align*}
\partial^{-1/2}_{\zeta \text{ from } -2/3} \left( \int_{\mathcal{C}_-(\zeta)} \nu \right) & = \frac{1}{\Gamma\big(\tfrac{1}{2}\big)} \int_{-2/3}^\zeta (\zeta - \zeta')^{-1/2} \left( \int_{\mathcal{C}_-(\zeta')} \nu \right)\,d\zeta' \\
& = \frac{1}{\Gamma\big(\tfrac{1}{2}\big)} \int_0^\xi \tfrac{\sqrt{3}}{2} (\xi - \xi')^{-1/2} \Big[{i}{\sqrt{3}}\,\tfrac{2}{3} (-\xi')^{1/2}\,F\big(\tfrac{1}{6}, \tfrac{5}{6}; \tfrac{3}{2}; \xi' \big) \Big] \,\big( \tfrac{4}{3}\,d\xi' \big) \\
& =  \frac{4}{3} \frac{\Gamma\big(\tfrac{3}{2}\big)}{\Gamma(2)} (-\xi)\,F\big(\tfrac{1}{6}, \tfrac{5}{6}; 2; \xi\big) \\
& = - \frac{2}{3} \sqrt{\pi}\,\xi\,F\big(\tfrac{1}{6}, \tfrac{5}{6}; 2; \xi\big).
\end{align*}
Finally, we differentiate twice using \textbf{[DLMF 15.5.4]} and \textbf{[DLMF 15.5.1]}.
\begin{align*}
\partial^{3/2}_{\zeta \text{ from } 2/3} \left( \int_{\mathcal{C}_+(\zeta)} \nu \right) & = \left(\tfrac{3}{4} \tfrac{\partial}{\partial \xi}\right)^2 \left[ - \frac{2}{3} \sqrt{\pi}\,\xi\,F\big(\tfrac{1}{6}, \tfrac{5}{6}; 2; \xi\big) \right] \\
& = - \tfrac{3\sqrt{\pi}}{8} \left(\tfrac{\partial}{\partial \xi}\right)^2 \left[ \xi\,F\big(\tfrac{1}{6}, \tfrac{5}{6}; 2; \xi\big) \right] \\
& = - \tfrac{3\sqrt{\pi}}{8}\,\tfrac{\partial}{\partial \xi} \left[ F\big(\tfrac{1}{6}, \tfrac{5}{6}; 1; \xi\big) \right] \\
& = - \tfrac{\sqrt{\pi}}{8}\,\tfrac{5}{12}\,F\big(\tfrac{7}{6}, \tfrac{11}{6}; 2; \xi\big).
\end{align*}
\textbf{[Check comparison with Mari\~{n}o's result more carefully?]}
\end{proof}

In particular, from the previous computations we get the right normalization constants for $\hat{w}_{\pm}(\zeta)$, which allows to compute the Stokes factors from the analytic continuation of $\hat{w}_{\pm}(\zeta)$ at the branch cut: from \cite{DLMF}~15.2.3

\begin{align*}
\textcolor{red}{\left(i\tfrac{\sqrt{\pi}}{8}\tfrac{5}{12}\right)}\left[\hat{w}_+(\zeta+i0)-\hat{w}_+(\zeta-i0)\right]&=\textcolor{red}{\left(i\tfrac{\sqrt{\pi}}{8}\tfrac{5}{12}\right)}\left(-\tfrac{36}{5}i(-\frac{3}{4}\zeta-1)^{-1}\sum_{n\geq 0}\tfrac{(5/6)_n(1/6)_n}{\Gamma(n)n!}(1+\tfrac{3}{4}\zeta)^n\right) &\zeta<-\tfrac{4}{3}\\
&=\textcolor{red}{\left(i\tfrac{\sqrt{\pi}}{8}\tfrac{5}{12}\right)}\tfrac{36}{5}i\sum_{n\geq 0}\tfrac{(5/6)_n(1/6)_n}{\Gamma(n)n!}(1+\tfrac{3}{4}\zeta)^{n-1} & \\
&=-\textcolor{red}{\left(i\tfrac{\sqrt{\pi}}{8}\tfrac{5}{12}\right)}\tfrac{36}{5}i(-\tfrac{3}{4}\zeta-1)^{-1}\left(\tfrac{5}{144} (4 + 3 \zeta)\left(1+{}_{1}F_{2}\left(\tfrac{7}{6},\tfrac{11}{6},2,1+\tfrac{3}{4}\zeta\right)\right)\right) &\\
&=\textcolor{red}{\left(i\tfrac{\sqrt{\pi}}{8}\tfrac{5}{12}\right)}i \,\,{}_{1}F_{2}\left(\tfrac{7}{6},\tfrac{11}{6},2,1+\tfrac{3}{4}\zeta\right) &\\
&=\textcolor{red}{\left(-\tfrac{\sqrt{\pi}}{8}\tfrac{5}{12}\right)}\,\,{}_{1}F_{2}\left(\tfrac{7}{6},\tfrac{11}{6},2,1+\tfrac{3}{4}\zeta\right) & \\
&=\mathbf{+1}\hat{w}_{-}(\zeta+\tfrac{4}{3}) &
\end{align*}
Anolougusly, $\hat{w}_-(\zeta)$ is Laplace summable along the negative real axis, and it jumps across the branch cut $\frac{4}{3}\R_{\geq 0}$ as 
\begin{align*}
\textcolor{red}{\left(-\tfrac{\sqrt{\pi}}{8}\tfrac{5}{12}\right)}\left[\hat{w}_-(\zeta+i0)-\hat{w}_-(\zeta-i0)\right]&=\textcolor{red}{\left(-\tfrac{\sqrt{\pi}}{8}\tfrac{5}{12}\right)}\left(-\tfrac{36}{5}i(\tfrac{3}{4}\zeta-1)^{-1}\sum_{n\geq 0}\frac{(5/6)_n(1/6)_n}{\Gamma(n)n!}(1-\tfrac{3}{4}\zeta)^n\right) & \zeta>\tfrac{4}{3}\\
&=-\textcolor{red}{\left(-\tfrac{\sqrt{\pi}}{8}\tfrac{5}{12}\right)}\tfrac{36}{5}i(\tfrac{3}{4}\zeta-1)^{-1}\left(-\tfrac{5}{144}(-4+3\zeta){}_{2}F_{1}\left(\tfrac{7}{6},\tfrac{11}{6},2,1-\tfrac{3}{4}\zeta\right)\right)  &\\
&=i\textcolor{red}{\left(-\tfrac{\sqrt{\pi}}{8}\tfrac{5}{12}\right)}\,\,{}_{2}F_{1}\left(\tfrac{7}{6},\tfrac{11}{6},2,1-\tfrac{3}{4}\zeta\right) &\\
&=-\textcolor{red}{\left(i\tfrac{\sqrt{\pi}}{8}\tfrac{5}{12}\right)}\,\,{}_{2}F_{1}\left(\tfrac{7}{6},\tfrac{11}{6},2,1-\tfrac{3}{4}\zeta\right) &\\
&=\mathbf{-1}\hat{w}_{+}(\zeta-\tfrac{4}{3}) &
\end{align*}
These relations manifest the resurgence property of $\tilde{I}_{\pm 1}$, indeed near the singularities in the Borel plane of either $\hat{w}_+$ or $\hat{w}_-$, $\hat{w}_-$ and $\hat{w}_+$ respectively contribute to the jump of the former solution.

\color{pink}
\begin{remark}
$\hat{w}_{+}(\zeta)$ is Laplace summable along the positive real axis, and it can be analyticaly continued on $\C\setminus -\frac{4}{3}\R_{\leq 0}$ with (see 15.2.3 DLMF)
\begin{align*}
\hat{w}_+(\zeta+i0)-\hat{w}_+(\zeta-i0)&=-\frac{36}{5}i(-\frac{3}{4}\zeta-1)^{-1}\sum_{n\geq 0}\frac{(5/6)_n(1/6)_n}{\Gamma(n)n!}(1+\frac{3}{4}\zeta)^n &\zeta<-\frac{4}{3}\\
&=\frac{36}{5}i\sum_{n\geq 0}\frac{(5/6)_n(1/6)_n}{\Gamma(n)n!}(1+\frac{3}{4}\zeta)^{n-1} & \\
&=-\frac{36}{5}i(-\frac{3}{4}\zeta-1)^{-1}\left(\frac{5}{144} (4 + 3 \zeta)\left(1+{}_{1}F_{2}\left(\frac{7}{6},\frac{11}{6},2,1+\frac{3}{4}\zeta\right)\right)\right) &\\
&=\mathbf{i}\,\,{}_{1}F_{2}\left(\frac{7}{6},\frac{11}{6},2,1+\frac{3}{4}\zeta\right) &\\
&=\mathbf{i}\hat{w}_{-}(\zeta+\frac{4}{3}) &
\end{align*}
Anolougusly, $\hat{w}_-(\zeta)$ is Laplace summable along the negative real axis, and it jumps across the branch cut $\frac{4}{3}\R_{\geq 0}$ as 
\begin{align*}
\hat{w}_-(\zeta+i0)-\hat{w}_-(\zeta-i0)&=\frac{36}{5}i(\frac{3}{4}\zeta-1)^{-1}\sum_{n\geq 0}\frac{(5/6)_n(1/6)_n}{\Gamma(n)n!}(1-\frac{3}{4}\zeta)^n & \zeta>\frac{4}{3}\\
&=\frac{36}{5}i(\frac{3}{4}\zeta-1)^{-1}\left(-\frac{5}{144}(-4+3\zeta){}_{1}F_{2}\left(\frac{7}{6},\frac{11}{6},2,1-\frac{3}{4}\zeta\right)\right)  &\\
&=-\mathbf{i}\pi\,\,{}_{1}F_{2}\left(\frac{7}{6},\frac{11}{6},2,1-\frac{3}{4}\zeta\right) &\\
&=-\mathbf{i}\hat{w}_{+}(\zeta-\frac{4}{3}) &
\end{align*}
These relations manifest the resurgence property of $\tilde{I}$, indeed near the singularities in the Borel plane of either $\hat{w}_+$ or $\hat{w}_-$, $\hat{w}_-$ and $\hat{w}_+$ respectively contribute to the jump of the former solution.  
\end{remark}

\color{black}
\subsection{Comparison with other Airy examples}

\subsubsection{Different Borel transform convention} In physics, sometimes it happens that authors find more convenient a different definition of the Borel transform which does not involve the \textit{delta} element for the unit: they define $\mathcal{B}_{\textit{phys}}\colon \C [\![ z^{-1}]\!] \to \C \lbrace \zeta\rbrace$ such that $\mathcal{B}_{\textit{phys}}(z^{-n})\defeq \tfrac{\zeta^n}{n!}$. It is also commun to deal with formal power series in small parameters, like $\Phi(\hbar)=\sum_{n\geq 0} a_n\hbar^n$ as $\hbar\to 0$. Then the Borel transform of $\Phi$ is defined as $\phi(\zeta)=\sum_{n\geq 0}a_n\tfrac{\zeta^n}{n!}$. In \cite{Diablerets}, the author study the resurgent properties of the Airy functions: his starting point is the formal solution of Airy differential equation
\begin{align*}
\tilde{\Phi}_{\mathrm{Ai}}(x)&=\frac{1}{2\sqrt{\pi}}x^{-1/4}e^{-\tfrac{2}{3}x^{3/2}}\tilde{W}_1(x^{-3/2})\\
\tilde{\Phi}_{\mathrm{Bi}}(x)&=\frac{1}{2\sqrt{\pi}}x^{-1/4}e^{\tfrac{2}{3}x^{3/2}}\tilde{W}_2(x^{-3/2})
\end{align*}  
where 
\begin{align*}
\tilde{W}_{1,2}(\hbar)=\sum_{n=0}^{\infty}\frac{1}{2\pi}\left(\mp\frac{3}{4}\right)^{n}\frac{\Gamma(n+\frac{5}{6})\Gamma(n+\frac{1}{6})}{n!}\hbar^n
\end{align*}
Notice that $\tilde{W}_{1,2}(\hbar)$ are proportional to $\tilde{W}_{\pm}(z)$ with $z=\hbar^{-1}$. However, their Borel transforms are two different hypergeometric functions:
\begin{align*}
w_{1,2}(\zeta)&\defeq\mathcal{B}_{\textit{phys}}(\tilde{W}_{1,2})(\zeta)\\
&=\sum_{n=0}^{\infty}\frac{1}{2\pi}\left(\mp\frac{3}{4}\right)^{n}\frac{\Gamma(n+\frac{5}{6})\Gamma(n+\frac{1}{6})}{n!}\frac{\zeta^n}{n!}\\
&={}_2F_1\left(\frac{1}{6},\frac{5}{6};1;\mp\frac{3}{4}\zeta\right) \\
\mathcal{B}(\tilde{W}_{1,2})(\zeta)&=\frac{1}{2\pi}\delta+\sum_{n=1}^{\infty} \frac{1}{2\pi}\left(\mp\frac{3}{4}\right)^{n}\frac{\Gamma(n+\frac{5}{6})\Gamma(n+\frac{1}{6})}{n!}\frac{\zeta^{n-1}}{(n-1)!}\\
&=\frac{1}{2\pi}\delta+\sum_{n=0}^{\infty} \frac{1}{2\pi}\left(\mp\frac{3}{4}\right)^{n+1}\frac{\Gamma(n+1+\frac{5}{6})\Gamma(n+1+\frac{1}{6})}{(n+1)!}\frac{\zeta^{n}}{n!}\\
&=\frac{1}{2\pi}\delta\mp\frac{3}{4}\sum_{n=0}^{\infty} \frac{1}{2\pi}\left(\mp\frac{3}{4}\right)^{n}\frac{\Gamma(n+\frac{11}{6})\Gamma(n+\frac{7}{6})}{\Gamma(n+2)}\frac{\zeta^{n}}{n!}\\
&=\frac{1}{2\pi}\delta\mp\frac{5}{48} {}_2F_1\left(\frac{7}{6},\frac{11}{6};2;\mp\frac{3}{4}\zeta\right)%\\
%&=\frac{1}{2\pi}\delta\mp \frac{5}{48}\frac{1}{c_{1,2}}w_{\pm}(\zeta)
%w_{\pm}(\zeta)&=c_{1,2}\,\,{}_2F_1\left(\frac{7}{6},\frac{11}{6};2;\mp\frac{3}{4}\zeta\right) &\qquad \text{see \eqref{eq:hat+}\eqref{eq:hat-}}
\end{align*}
%That ambiguity comes from the different definitions of Borel transforms: 
%\begin{align*}
%\mathcal{B}(\tilde{W}_{1,2})(\zeta)&=\frac{1}{2\pi}\delta+\sum_{n=1}^{\infty} \frac{1}{2\pi}\left(\mp\frac{3}{4}\right)^{n}\frac{\Gamma(n+\frac{5}{6})\Gamma(n+\frac{1}{6})}{n!}\frac{\zeta^{n-1}}{(n-1)!}\\
%&=\frac{1}{2\pi}\delta+\sum_{n=0}^{\infty} \frac{1}{2\pi}\left(\mp\frac{3}{4}\right)^{n+1}\frac{\Gamma(n+1+\frac{5}{6})\Gamma(n+1+\frac{1}{6})}{(n+1)!}\frac{\zeta^{n}}{n!}\\
%&=\frac{1}{2\pi}\delta\mp\frac{3}{4}\sum_{n=0}^{\infty} \frac{1}{2\pi}\left(\mp\frac{3}{4}\right)^{n}\frac{\Gamma(n+\frac{11}{6})\Gamma(n+\frac{7}{6})}{\Gamma(n+2)}\frac{\zeta^{n}}{n!}\\
%&=\frac{1}{2\pi}\delta\mp\frac{5}{48} {}_2F_1\left(\frac{7}{6},\frac{11}{6};2;\mp\frac{3}{4}\zeta\right)%\\
%%&=\frac{1}{2\pi}\delta\mp \frac{5}{48}\frac{1}{c_{1,2}}w_{\pm}(\zeta)
%\end{align*}  

Comparing $\mathcal{B}(\tilde{W}_{1,2})(\zeta)$ with $w_{\pm}$ in \eqref{hat+}\eqref{hat-}, they differ only by a constant which multiplies the $\delta$.

\subsubsection{Integral formula for hypergeometric functions}

In \cite{MS16} the author studies summability and resurgent properties of solutions of the Airy equation. He defines the formal series $\tilde{\Phi}(z)\defeq \sum_{n=0}^{\infty}\frac{1}{2\pi}\left(\mp\frac{1}{2}\right)^{n}\frac{\Gamma(n+\frac{5}{6})\Gamma(n+\frac{1}{6})}{n!}z^{-n}$ such that 

\begin{equation}
\Phi_{\mathrm{Ai}}(w)=\frac{1}{2\sqrt{\pi}}w^{-1/4}e^{-\tfrac{2}{3}w^{3/2}}\mathcal{L}\circ\mathcal{B}\tilde{\Phi}\left(\tfrac{2}{3}w^{3/2}\right)
\end{equation}

Notice that $\tilde{W}_1(\tfrac{2}{3}\hbar)=\tilde{\Phi}(z)$ for $\hbar=z^{-1}$, hence 

\begin{align*}
\tilde{\phi}(\zeta)=\mathcal{B}\tilde{W}_1(\tfrac{2}{3}\zeta)=\frac{1}{2\pi}\delta-\frac{5}{48} {}_2F_1\left(\frac{7}{6},\frac{11}{6};2;-\frac{\zeta}{2}\right) 
\end{align*}

However, the author adopts a different approach to compute $\tilde{\phi}(\zeta)$: he argues that $\Phi_{\mathrm{Ai}}$ is a solution of Airy equation if and only if 

\begin{align*}
\tilde{\phi}(\zeta)=\delta+\frac{d}{d\zeta}\tilde{\chi}(\zeta) & \qquad \text{ where }\,\, \chi(\zeta)=\frac{2^{1/6}}{\Gamma(1/6)\Gamma(5/6)}(2\zeta+\zeta^2)^{-1/6}\ast \zeta^{-5/6}
\end{align*}

The function $\chi$ is an hypergeometric function:

\begin{align*}
\chi(\zeta)&=\frac{2^{1/6}}{\Gamma(1/6)\Gamma(5/6)}(2\zeta+\zeta^2)^{-1/6}\ast \zeta^{-5/6}\\
&=\frac{2^{1/6}}{\Gamma(1/6)\Gamma(5/6)}\int_0^{\zeta}(2\zeta'+\zeta'^2)^{-1/6} (\zeta-\zeta')^{-5/6}d\zeta'\\
&=\frac{2^{1/6}}{\Gamma(1/6)\Gamma(5/6)}\int_0^{1}(\zeta t)^{-1/6}(2+\zeta t)^{-1/6} (\zeta-\zeta t)^{-5/6} \zeta dt\\
&=\frac{2^{1/6}}{\Gamma(1/6)\Gamma(5/6)}\int_0^{1} t^{-1/6} 2^{-1/6}(1+\zeta t)^{-1/6} (1-t)^{-5/6}d\zeta'\\
&=\frac{1}{\Gamma(1/6)\Gamma(5/6)}\int_0^{1} t^{-1/6} (1+\zeta t)^{-1/6} (1-t)^{-5/6}d\zeta'\\
&={}_2F_1\left(\frac{1}{6},\frac{5}{6};1;-\frac{\zeta}{2}\right)
\end{align*}
where in the last step we use the Euler formula for hypergeometric functions (see \eqref{Euler formula}). Finally, we take the derivative 

\begin{align*}
\tilde{\phi}(\zeta)=\delta-\frac{1}{2}\frac{5}{36}\,\, {}_2F_1\left(\frac{7}{6},\frac{11}{6};2;-\frac{\zeta}{2}\right)=\delta-\frac{2}{3}\frac{5}{48} {}_2F_1\left(\frac{7}{6},\frac{11}{6};2;-\frac{\zeta}{2}\right).
\end{align*}  

\section{Bessel 0}

Let $X=\C^*$, $f(x)=x+\frac{1}{x}$ and $\nu=\frac{dx}{x}$, then the ciritcal points of $f$ are $x=\pm1$ and 
\begin{equation}
I(z)\defeq\int_0^{\infty}e^{-zf(x)}\frac{dx}{x}.
\end{equation}

Let $\pi\colon\tilde{\C}\to \C^*$ be the universal cover of $\C^*$, where $\pi(u)=e^u$, then on $\tilde{\C}$, $I(z)$ turns into
\begin{align*}
I(z)=\int_{-\infty}^{\infty}e^{-2z\cosh(u)}du=2\int_0^{\infty}e^{-2z\cosh(u)}du=2K_0(2z) \,\qquad |\arg z|<\frac{\pi}{2}
\end{align*}
where $K_0(z)$ is the modified Bessel function (see definition 10.32.10 DLMF). In particular, since $K_0(z)$ solves 
\begin{equation}
\frac{d^2}{dz^2}w(z)+\frac{1}{z}\frac{d}{dz}w(z)-w(z)=0
\end{equation}
and $K_0(z)\sim\left(\frac{\pi}{2}\right)^{1/2}e^{-z}z^{-1/2}\sum_{k\geq 0}\frac{(1/2)_k(1/2)_k}{(-2)^kk!}z^{-k}$ as $z\to\infty$ (see DLMF 10.40.2), then $I(z)$ is a solution of 
\begin{equation}
\label{eq:IB}
\frac{d^2}{dz^2}I(z)+\frac{1}{z}\frac{d}{dz}I(z)-4I(z)=0.
\end{equation} 
The formal integral of \eqref{eq:IB} is given by a two parameter formal solution $\tilde{I}(z)$
\begin{equation}
\tilde{I}(z)=\sum_{\mathbf{k}\in\N^2}U^ke^{-\mathbf{k}\cdot\lambda z}z^{-\tau\cdot \mathbf{k}}\tilde{w}_{\mathbf{k}}(z)
\end{equation}
where $\lambda=(2,-2)$, $\tau=(-\frac{1}{2},-\frac{1}{2})$, $U^k:=U_1^{k_1}U_2^{k_2}$ with $k=(k_1,k_2)$ and $U_1,U_2\in\C$, and $\tilde{w}_{\mathbf{k}}(z)\in\C[[z^{-1}]]$ is a formal solution of 
\begin{multline}
\tilde{w}_{\mathbf{k}}''(z)-4(k_1-k_2)\tilde{w}_{\mathbf{k}}'(z)+4(1-(k_1-k_2)^2)\tilde{w}_{\mathbf{k}}(z)+\frac{(k_1+k_2-1)}{z}\tilde{w}_{\mathbf{k}}'(z)+\\
-2(k_1-k_2)\frac{(k_1+k_2-1)}{z}\tilde{w}_{\mathbf{k}}(z)+\frac{(k_1+k_2)^2}{4z^2}\tilde{w}_{\mathbf{k}}(z)=0
\end{multline}
The only non zero $\tilde{w}_{\mathbf{k}}(z)$ occurs for $\mathbf{k}=(1,0)$ and $\mathbf{k}=(0,1)$, hence 
\begin{equation}
\tilde{I}(z)=U_1 e^{-2z}z^{-1/2}\tilde{w}_{(1,0)}(z)+U_2e^{2z}z^{-1/2}\tilde{w}_{(0,1)}(z)
\end{equation}
and we define
\begin{align}
\label{IB+} &\tilde{I}_{1}(z)\defeq e^{-2z}z^{-1/2}\tilde{w}_{(1,0)}(z)\\
\label{IB-} &\tilde{I}_{-1}(z)\defeq e^{2z}z^{-1/2}\tilde{w}_{(0,1)}(z).
\end{align}
We set $\tilde{w}_{(1,0)}=\tilde{w}_+$ and $\tilde{w}_{(0,1)}=\tilde{w}_-$, then their Borel transforms are solutions respectively of the following equations 
\begin{align}
\zeta^2\hat{w}_+(\zeta)+4\zeta\hat{w}_+(\zeta)+\frac{1}{4}\zeta\ast\hat{w}_+(\zeta)=0\\
\zeta^2\hat{w}_-(\zeta)-4\zeta\hat{w}_-(\zeta)+\frac{1}{4}\zeta\ast\hat{w}_-(\zeta)=0
\end{align}
taking twice derivative in $\zeta$ we get respectively for $\hat{w}_+(\zeta)$ and $\hat{w}_-(\zeta)$
\begin{align*}
&(\zeta^2+4\zeta)\frac{d^2}{d\zeta^2}\hat{w}_++4(\zeta-1)\frac{d}{d\zeta}\hat{w}_++\frac{9}{4}\hat{w}_+=0 &\\
(+)\quad &\xi(1-\xi)\frac{d^2}{d\xi^2}\hat{w}_++(1-4\xi)\frac{d}{d\xi}\hat{w}_+-\frac{9}{4}\hat{w}_+=0 & \xi=-\frac{\zeta}{4} 
\end{align*}
\begin{align*}
&(\zeta^2-4\zeta)\frac{d^2}{d\zeta^2}\hat{w}_-+4(\zeta+1)\frac{d}{d\zeta}\hat{w}_++\frac{9}{4}\hat{w}_-=0 &\\
(-)\quad &\xi(1-\xi)\frac{d^2}{d\xi^2}\hat{w}_-+(1-4\xi)\frac{d}{d\xi}\hat{w}_--\frac{9}{4}\hat{w}_-=0 & \xi=\frac{\zeta}{4}
\end{align*}
Since equation $(+), (-)$ are hypergeometric, the fundamental solutions are respectively (see DLMF 15.10.2)
\begin{align}
\label{wB+}&\hat{w}_+(\zeta)=c_1\,\, {}_2F_1\left(\frac{3}{2},\frac{3}{2};2;-\frac{\zeta}{4}\right)\\
\label{wB-}&\hat{w}_-(\zeta)=c_2\,\, {}_2F_1\left(\frac{3}{2},\frac{3}{2};2;\frac{\zeta}{4}\right)
\end{align}

Now we show that the fractional derivative formula holds in this example: first we write a parametrization of the integration path in the Borel plane. Let $\mathcal{C}_p=\R$, $p\in \pi\Z$, be the contour of integration in $\tilde{\C}$ which is parametrized as 

\begin{align*}
\mathcal{C}_p\colon \R & \to \tilde{\C} \\
\theta & \to \theta+i p
\end{align*} 

Then in the Borel plane $\C_\zeta$, where $\zeta=2\cosh(u)$, the path $\mathcal{C}_0$ is parametrized as  
\begin{align*}
\mathcal{C}_0(\zeta)\colon\R &\to \C_\zeta \\
\theta & \to 2\cosh(\theta)
\end{align*}

\begin{figure}
\caption{Integration path in the $u$-plane and in the Borel plane $\zeta=2\cosh(u)$. }
\end{figure} 

\textbf{Check Aaron notation}

\begin{equation}
\int_{\mathcal{C}_0(\zeta)}\pi^*(\nu)=\int_{\mathcal{C}_0(\zeta)}du=\Big[ u\Big]^{\mathrm{end}\mathcal{C}_0(\zeta)}_{\mathrm{start}\mathcal{C}_0(\zeta)}=2\mathrm{arc}\cosh\left(\frac{\zeta}{2}\right)
\end{equation}

We can now write $\int_{\mathcal{C}_0(\zeta)}\pi^*(\nu)$ as an hypergeometric function thanks to identity 14.4.4 \cite{DLMF}: set $\xi=\frac{1}{2}\left(\frac{1}{2}\zeta-1\right)=\frac{1}{2}\left(\cosh(\theta)-1\right)=-\sinh^2\left(\frac{\theta}{2}\right)$, then 

\begin{align*}
\sinh\left(\frac{\theta}{2}\right)\,\,{}_2F_1\left(\frac{1}{2},\frac{1}{2};\frac{3}{2};\sinh^2\left(\frac{\theta}{2}\right)\right)&=i\frac{\theta}{2} \\
(-\xi)^{1/2}{}_2F_1\left(\frac{1}{2},\frac{1}{2};\frac{3}{2};-\xi\right)&=i\frac{\theta}{2}\\
 &=\frac{i}{2}\mathrm{arcosh}\left(\frac{\zeta}{2}\right)\\
 &=\frac{i}{4}\int_{\mathcal{C}_0(\zeta)}\pi^*(\nu)
\end{align*}

The $3/2$-derivative of $\int_{\mathcal{C}_0(\zeta)}\pi^*(\nu)$ can be computed as follows: we compute the $-1/2$-derivatve and then we differentiate twice

\begin{align*}
\partial_{\zeta}^{-1/2}\left(\int_{\mathcal{C}_0(\zeta)}\pi^*(\nu)\right)&=\frac{1}{\Gamma\left(\frac{1}{2}\right)}\int_2^\zeta (\zeta-\zeta')^{-1/2}\left(\int_{\mathcal{C}_0(\zeta)}\pi^*(\nu)\right)d\zeta'\\
&=\frac{1}{\Gamma\left(\frac{1}{2}\right)}\int_0^{\xi}\frac{1}{2}(\xi-\xi')^{-1/2}(-4i)(-\xi')^{1/2} {}_2F_1\left(\frac{1}{2},\frac{1}{2};\frac{3}{2};-\xi'\right) 4d\xi'\\
&=-8i (i)\frac{\Gamma\left(\frac{3}{2}\right)}{\Gamma(2)}\xi \,\,{}_2F_1\left(\frac{1}{2},\frac{1}{2};2,-\xi\right)\\
&=4\sqrt{\pi} \xi \,{}_2F_1\left(\frac{1}{2},\frac{1}{2};2,-\xi\right)
\end{align*} 

\begin{align*}
\partial_\zeta^{3/2}\left(\int_{\mathcal{C}_0(\zeta)}\pi^*(\nu)\right)&=\partial_\zeta^2\left(4\sqrt{\pi}\xi \,\,{}_2F_1\left(\frac{1}{2},\frac{1}{2};2,-\xi\right)\right) & \\
&=\frac{1}{16}\partial_\xi^2\left(4\sqrt{\pi}\xi \,\,{}_2F_1\left(\frac{1}{2},\frac{1}{2};2,-\xi\right)\right) & \partial_\xi=4\partial_\zeta\\
&=-\frac{\sqrt{\pi}}{4}\Gamma\left(\frac{3}{2}\right)\partial_\xi\left({}_2F_1\left(\frac{1}{2},\frac{1}{2};1,-\xi\right)\right) & \textbf{DLMF} 15.5.4\\
&=\frac{\sqrt{\pi}}{16}{}_2F_1\left(\frac{3}{2},\frac{3}{2};2,-\xi\right) & \\
&=\frac{\sqrt{\pi}}{16}{}_2F_1\left(\frac{3}{2},\frac{3}{2};2,\frac{1}{2}-\frac{\zeta}{4}\right) & \\
\end{align*}


Analougusly, it can be verified for $\hat{w}_-(\zeta+2)$ for $\zeta\in(-\infty,-2)$. The path $\mathcal{C}_\pi$ is parametrized as  
\begin{align*}
\mathcal{C}_\pi\colon\R &\to \C_\zeta \\
\theta & \to 2\cosh(\theta-i\pi)=-2\cosh(\theta)
\end{align*}


\begin{equation}
\int_{\mathcal{C}_\pi(\zeta)}\pi^*(\nu)=\int_{\mathcal{C}_\pi(\zeta)}du=\Big[ u\Big]^{\mathrm{end}\mathcal{C}_\pi(\zeta)}_{\mathrm{start}\mathcal{C}_\pi(\zeta)}=2\mathrm{arc}\cosh\left(-\frac{\zeta}{2}\right)
\end{equation}

We can now write $\int_{\mathcal{C}_\pi(\zeta)}\pi^*(\nu)$ as an hypergeometric function thanks to identity 15.4.4 \cite{DLMF}: set $\xi=\frac{1}{2}\left(\frac{1}{2}\zeta+1\right)=\frac{1}{2}\left(-\cosh(\theta)+1\right)=\sinh^2\left(\frac{\theta}{2}\right)$, then 

\begin{align*}
\sinh\left(\frac{\theta}{2}\right)\,\,{}_2F_1\left(\frac{1}{2},\frac{1}{2};\frac{3}{2};\sinh^2\left(\frac{\theta}{2}\right)\right)&=i\frac{\theta}{2} \\
(\xi)^{1/2}{}_2F_1\left(\frac{1}{2},\frac{1}{2};\frac{3}{2};\xi\right)&=i\frac{\theta}{2}\\
 &=\frac{i}{2}\mathrm{arcosh}\left(-\frac{\zeta}{2}\right)\\
 &=\frac{i}{4}\int_{\mathcal{C}_\pi(\zeta)}\pi^*(\nu)
\end{align*}

The $3/2$-derivative of $\int_{\mathcal{C}_\pi(\zeta)}\pi^*(\nu)$ is computed in two steps: first we compute the $-1/2$-derivatve and then we differentiate twice

\begin{align*}
\partial_{\zeta}^{-1/2}\left(\int_{\mathcal{C}_\pi(\zeta)}\pi^*(\nu)\right)&=\frac{1}{\Gamma\left(\frac{1}{2}\right)}\int_{-2}^\zeta (\zeta-\zeta')^{-1/2}\left(\int_{\mathcal{C}_\pi(\zeta)}\pi^*(\nu)\right)d\zeta'\\
&=\frac{1}{\Gamma\left(\frac{1}{2}\right)}\int_0^{\xi}\frac{1}{2}(\xi-\xi')^{-1/2}(-4i)\xi'^{1/2} {}_2F_1\left(\frac{1}{2},\frac{1}{2};\frac{3}{2};\xi'\right) 4d\xi'\\
&=-8i\frac{\Gamma\left(\frac{3}{2}\right)}{\Gamma(2)}\xi \,\,{}_2F_1\left(\frac{1}{2},\frac{1}{2};2,\xi\right)
\end{align*} 

\begin{align*}
\partial_\zeta^{3/2}\left(\int_{\mathcal{C}_0(\zeta)}\pi^*(\nu)\right)&=\partial_\zeta^2\left(-8i\frac{\Gamma\left(\frac{3}{2}\right)}{\Gamma(2)}\xi \,\,{}_2F_1\left(\frac{1}{2},\frac{1}{2};2,\xi\right)\right) & \\
&=\frac{1}{16}\partial_\xi^2\left(-8i\frac{\Gamma\left(\frac{3}{2}\right)}{\Gamma(2)}\xi \,\,{}_2F_1\left(\frac{1}{2},\frac{1}{2};2,\xi\right)\right) & \partial_\xi=4\partial_\zeta\\
&=-\frac{i}{2}\Gamma\left(\frac{3}{2}\right)\partial_\xi\left({}_2F_1\left(\frac{1}{2},\frac{1}{2};1,\xi\right)\right) & \textbf{DLMF} 15.5.4\\
&=-\frac{i}{8}\Gamma\left(\frac{3}{2}\right){}_2F_1\left(\frac{3}{2},\frac{3}{2};2,\xi\right) & \\
&=-\frac{i\sqrt{\pi}}{16}{}_2F_1\left(\frac{3}{2},\frac{3}{2};2,\frac{1}{2}+\frac{\zeta}{4}\right) & 
\end{align*} 



Once we have determined the right constants $c_1,c_2$ we can compute the Stokes constants. 
\color{blue}
First, we notice that taking the series expansion of $\hat{w}_+$ and $\hat{w}_-$ at the critical point we get numerically that
\begin{align*}
\hat{w}_+(\zeta-4)=\frac{1}{\pi}\log(z)\hat{w}_-(z)+\phi_{\text{reg}}\\
\hat{w}_-(\zeta+4)=\frac{1}{\pi}\log(z)\hat{w}_+(z)+\psi_{\text{reg}}
\end{align*}
and analytically (thanks to ~15.2.3 DLMF)

\color{black}

Let us redefine $\hat{w}_{+}(\zeta)\defeq -i\tfrac{\sqrt{\pi}}{16} {}_2F_1\left(\tfrac{3}{2},\tfrac{3}{2};2;\tfrac{1}{2}-\tfrac{\zeta}{4}\right)$ and $\hat{w}_-(\zeta)\defeq\tfrac{\sqrt{\pi}}{16} {}_2F_1\left(\tfrac{3}{2},\tfrac{3}{2};2;\tfrac{\zeta}{4}+\tfrac{1}{2}\right)$. From equation ~15.2.3 in \cite{DLMF} 
\begin{align*}
\hat{w}_+(\zeta+i0)-\hat{w}_+(\zeta-i0)&=\textcolor{red}{-\tfrac{i\sqrt{\pi}}{16}}\left({}_2F_1\left(\tfrac{3}{2},\tfrac{3}{2};2;-\tfrac{\zeta}{4}+i0\right)-{}_2F_1\left(\tfrac{3}{2},\tfrac{3}{2};2;-\tfrac{\zeta}{4}-i0\right)\right) &\qquad\zeta<-4\\
&=-8i\textcolor{red}{\left(\tfrac{\sqrt{\pi}}{16}\right)}\left(-\tfrac{\zeta}{4}-1\right)^{-1}\sum_{n\geq 0}\tfrac{(1/2)_n(1/2)_n}{n!\Gamma(n)}\left(\tfrac{\zeta}{4}+1\right)^n\\
&=8i\textcolor{red}{\left(\tfrac{\sqrt{\pi}}{16}\right)}\sum_{n\geq 0}\frac{(1/2)_n(1/2)_n}{n!\Gamma(n)}\left(\tfrac{\zeta}{4}+1\right)^{n-1} & \\
&=8i\textcolor{red}{\left(\tfrac{\sqrt{\pi}}{16}\right)}\sum_{n\geq 0}\tfrac{(1/2)_{n+1}(1/2)_{n+1}}{(n+1)!\Gamma(n+1)}\left(\tfrac{\zeta}{4}+1\right)^{n} &\\
&={2i}\textcolor{red}{\left(\tfrac{\sqrt{\pi}}{16}\right)}\sum_{n\geq 0}\tfrac{(3/2)_{n}(3/2)_{n}}{(n)!\Gamma(n+2)}\left(\tfrac{\zeta}{4}+1\right)^{n} & \\
&=\mathbf{-2}\textcolor{red}{\left(-\tfrac{i\sqrt{\pi}}{16}\right)}{}_2F_1\left(\tfrac{3}{2},\tfrac{3}{2};2;\tfrac{\zeta}{4}+1\right)\\
&=\mathbf{-2}\hat{w}_-(\zeta+2)
\end{align*}
\begin{align*}
\hat{w}_-(\zeta+i0)-\hat{w}_-(\zeta-i0)&=\textcolor{red}{\left(-\tfrac{i\sqrt{\pi}}{16}\right)}\left[{}_2F_1\left(\tfrac{3}{2},\tfrac{3}{2};2;\tfrac{\zeta}{4}+i0\right)-{}_2F_1\left(\tfrac{3}{2},\tfrac{3}{2};2;\tfrac{\zeta}{4}-i0\right)\right]& \qquad\zeta>4\\
&=-8i\textcolor{red}{\left(-\tfrac{i\sqrt{\pi}}{16}\right)}\left(\tfrac{\zeta}{4}-1\right)^{-1}\sum_{n\geq 0}\tfrac{(1/2)_n(1/2)_n}{n!\Gamma(n)}\left(\tfrac{\zeta}{4}-1\right)^n\\
&=8i\textcolor{red}{\left(-\tfrac{i\sqrt{\pi}}{16}\right)}\sum_{n\geq 0}\tfrac{(1/2)_n(1/2)_n}{n!\Gamma(n)}\left(1-\tfrac{\zeta}{4}\right)^{n-1}\\
&=8i\textcolor{red}{\left(-\tfrac{i\sqrt{\pi}}{16}\right)}\sum_{n\geq 0}(-1)^n\frac{(1/2)_{n+1}(1/2)_{n+1}}{(n+1)!\Gamma(n+1)}\left(1-\tfrac{\zeta}{4}\right)^{n}\\
&=2i\textcolor{red}{\left(-\tfrac{i\sqrt{\pi}}{16}\right)}\sum_{n\geq 0}(-1)^n\tfrac{(3/2)_{n}(3/2)_{n}}{(n)!\Gamma(n+2)}\left(1-\tfrac{\zeta}{4}\right)^{n}\\
&=\mathbf{+2}\textcolor{red}{\left(\tfrac{\sqrt{\pi}}{16}\right)}{}_2F_1\left(\frac{3}{2},\frac{3}{2};2;1-\frac{\zeta}{4}\right)\\
&=\mathbf{+2}\hat{w}_+(\zeta-2)
\end{align*}
These are evidence of the resurgent properties of $\tilde{I}_{\pm 1}(z)$. \textbf{With the correct normalization, the latter identities show that the Stokes constants can be computed via Alien calculus and they are equal to $\pm 2$.  }  



\section{Useful identities for Gauss hypergeomtric functions}

\begin{multline}
\label{w1w3w5}
{}_2F_1\left(a,b;c;z\right)=e^{b\pi i}\frac{\Gamma(c)\Gamma(a-c+1)}{\Gamma(a+b-c+1)\Gamma(c-b)}{}_2F_1\left(a,b;c;1-z\right)+\\
-e^{(a+b-c)\pi i}\frac{\Gamma(c)\Gamma(a-c+1)}{\Gamma(b)\Gamma(a-b+1)}|z|^{-a}{}_2F_1\left(a,a-c+1;a-b+1;\frac{1}{z}\right)
\end{multline}
%\begin{multline}
%\label{4.1}
%\int_0^x |y|^{a-\mu-1}{}_2F_1\left(a,b;c,y\right)|x-y|^{\mu-1}dy=\frac{\Gamma(\mu)\Gamma(a-\mu)}{\Gamma(a)}|x|^{a-1}{}_2F_1\left(a-\mu,b;c;x\right)\\
%x\in(-\infty,0)\cup(0,1), \, \Re a>\Re\mu>0
%\end{multline}

\begin{multline}
\label{4.5}
\int_0^x |y|^{a-\mu-1}{}_2F_1\left(a,b;c,y\right)|x-y|^{\mu-1}dy=\frac{\Gamma(\mu)\Gamma(a-\mu)}{\Gamma(a)}|x|^{a-1}{}_2F_1\left(a-\mu,b;c;x\right)\\
x\in(-\infty,0)\cup(0,1), \, \Re a>\Re\mu>0
\end{multline}
which can be rewritten as \textbf{(\texttt{arXiv:1504.08144}, formula~4.8)}
\begin{multline}
\label{4.8}
\int_{y>x}|y|^{-a}|x-y|^{\mu-1}{}_2F_1\left(a,b;c;y^{-1}\right)dy=\frac{\Gamma(\mu)\Gamma(a-\mu)}{\Gamma(a)}|x|^{-a+\mu}{}_2F_1\left(a-\mu,b;c;x^{-1}\right)\\
x\in(-\infty,0)\cup(1,\infty), \, \Re a>\Re\mu>0
\end{multline}
%\textcolor{red}{We should check that for Hypergeomtric functions the following relation holds true
%\[{}_2F_1(a,b;c;z-1)\propto\int_1^z(z-t)^{-1/2}\left[{}_2F_1\left(a+\frac{1}{2},b+\frac{1}{2};c+\frac{1}{2};1-t\right)-{}_2F_1\left(a+\frac{1}{2},b+\frac{1}{2};c+\frac{1}{2};t\right)\right]dt\]
%}
%
%The following identities hold true
%\begin{multline}
%\label{a+b+c+}
%\int_{0<\frac{y}{x}<1}|y|^{c-1}(1-y)^{a+b-c}{}_2F_1\left(a,b;c;y\right)\frac{|x-y|^{\mu-1}}{\Gamma(\mu)}dy=\\
%\frac{\Gamma(c)}{\Gamma(c+\mu)}|x|^{c-1+\mu}(1-x)^{a+b-c+\mu}{}_2F_1\left(a+\mu,b+\mu;c+\mu;x\right)\\
% x\in (-\infty,0)\cup(0,1), \,\Re c>0,\,\Re\mu>0
%\end{multline}
%
%\begin{multline}\label{a-b-c-}
%\int_{-\infty}^x{}_2F_1\left(a,b;c;y\right) (x-y)^{\mu-1}dy=\Gamma(\mu)\frac{\Gamma(a-\mu)}{\Gamma(a)}\frac{\Gamma(b-\mu)}{\Gamma(b)}\frac{\Gamma(c)}{\Gamma(c-\mu)}{}_2F_1\left(a-\mu,b-\mu;c-\mu;x\right)\\
%x<1, \Re a,\Re b>\Re\mu>0
%\end{multline}

\begin{equation}\label{Euler formula}
{}_{2}F_1\left(a,b;c;x\right)=\frac{\Gamma(c)}{\Gamma(b)\Gamma(c-b)}\int_0^1 t^{b-1}(1-t)^{c-b-1}(1-xt)^{-a}dt
\end{equation}

\section{Resurgence for degree $3$ polynomials}

Let $f$ be a degree $3$ polynomial, and $t_{1},t_{2}$ its critical points (not necessarily distinguished):
\begin{enumerate}
\item if $t_{1}\neq t_{2}$, then 
\[I(z)=\int_{\mathcal{C}_{j}}e^{-zf}dt\]
is a solution of 
\begin{equation}
\label{eq:I for p not 0}
 I''+a I'+bI+c\frac{I'}{z}+\frac{d}{z}I+\frac{e}{z^2}I=0
\end{equation}
where $a,b,c,d,e$ are determined in terms of $f$.
\item if $t_{1}=t_2$, then 
\[I(z)=\int_{\mathcal{C}_{1}}e^{-zf}dt\]
is a solution of a first order ODE
\begin{equation}
I'+\left(a_4-\frac{a_2^3}{27a_1^2}+\frac{1}{3z}\right)I=0
\end{equation}
\end{enumerate}

\begin{proof}
Let $f(t)=a_1t^3+a_2t^2+a_3t+a_4$ with $a_1\neq 0$, 
\begin{align*}
\int_{\mathcal{C}_j}e^{-fz}dt&=\int_{\mathcal{C}_j+\frac{a_2}{3a_1}}e^{-(a_1t^3+pt+q)z}dt & t\to t-\frac{a_2}{3a_1}
\end{align*}
where $p=a_3-\frac{a_2^2}{3a_1}$ and $q=a_4-\frac{a_2a_3}{3a_1}+\frac{2a_2^3}{27a_1^2}$.

Case $(1)$: if $p\neq 0$,
\begin{align*}
I(z)=\int t(3a_1t^2+p)ze^{-fz}=\int(3a_1t^3+pt)ze^{-ft}=\\
\int2a_1t^3ze^{-fz}+\int(a_1t^3+pt+q)ze^{-fz}-qzI\\
2z\int a_1t^3e^{-fz}-zI'-qzI
\end{align*}

\begin{align*}
2z\int a_1t^3e^{-fz}=2z^2\int\frac{t^4}{4}a_1(3a_1t^2+p)e^{-fz}=\frac{z^2}{2}\int (3a_1^2t^6+pa_1t^4)e^{-fz}=\\
\frac{z^2}{2}\int(3a_1^2t^6+6pa_1t^4+3q^2+3p^2t^2+6pqt+6a_1qt^3)e^{-fz}+\\
+\frac{z^2}{2}\int(pa_1t^4-6pa_1t^4)e^{-fz}-\frac{z^2}{2}\int(3q^2+3p^2t^2+6pqt+6a_1qt^3)e^{-fz}\\
=\frac{3z^2}{2}I''+\frac{3z^2}{2}q^2I+3qz^2I'-\frac{z^2}{2}p\int(3a_1t^4+pt^2)e^{-fz}-z^2p\int(a_1t^4+pt^2)e^{-fz}\\
=\frac{3z^2}{2}I''+\frac{3z^2}{2}q^2I+3qz^2I'-\frac{5}{3}zp\int te^{-fz}-\frac{2}{3}z^2p^2\int t^2e^{-fz}
\end{align*}

hence 

\begin{align}
I=-zI'-qzI+\frac{3z^2}{2}I''+\frac{3z^2}{2}q^2I+3qz^2I'-\frac{5}{3}zp\int te^{-fz}-\frac{2}{3}z^2p^2\int t^2e^{-fz}\\
\frac{3z^2}{2}\left(I''+q^2I+2qI'-\frac{2}{3z}I'-\frac{2q}{3z}I-\frac{2}{3z^2}I-\frac{10}{9z}p\int te^{-fz}dt-\frac{4}{9}p^2\int t^2e^{-fz}\right)=0 
\end{align}

Notice that 
\begin{align*}
\frac{4}{9}p^2\int t^2e^{-fz}=\frac{4}{27a_1}p^2\int(3a_1t^2+p)e^{-fz}-\frac{4}{27a_1}p^3I=-\frac{4}{27a_1}p^3I
\end{align*}

\begin{align*}
-\frac{10}{9z}p\int te^{-fz}dt=\frac{5}{9z}\int pte^{-fz}-\frac{5}{3z}\int(pt+q)e^{-fz}+\frac{5}{3z}qI=\\
\frac{5}{9z}\int pte^{-fz}-\frac{5}{3z}\int(pt+q+a_1t^3)e^{-fz}+\frac{5}{3z}\int a_1t^3e^{-fz}+\frac{5}{3z}qI=\\
\frac{5}{9z}\int t(3a_1t^2+p)e^{-fz}+\frac{5}{3z}I'+\frac{5}{3z}qI\\
=\frac{5}{9z^2}I+\frac{5}{3z}I'+\frac{5}{3z}qI
\end{align*}
thefore, collecting all the contributions together we find 

\begin{align*}
I''+q^2I+2qI'-\frac{2}{3z}I'-\frac{2q}{3z}I-\frac{2}{3z^2}I+\frac{5}{9z^2}I+\frac{5}{3z}I'+\frac{5}{3z}qI+\frac{4}{27a_1}p^3I=0\\
I''+2qI'+\left(\frac{4p^3}{27a_1}+q^2\right)I+\frac{1}{z}I'+\frac{q}{z}I-\frac{1}{9z^2}I=0
\end{align*}

Case $(2)$: if $p=0$, then integrating by part we have
\begin{align*}
I(z)&=\int_{\mathcal{C}_1+\frac{a_2}{3a_1}}e^{-(a_1t^3+q)z}dt\\
&=\left[te^{-(a_1t^3+q)z}\right]_{\mathcal{C}_1+\frac{a_2}{3a_1}}+\int_{\mathcal{C}_1+\frac{a_2}{3a_1}}3a_1t^3ze^{-(a_1t^3+q)z}dt\\
&=3z\int_{\mathcal{C}_1+\frac{a_2}{3a_1}}(a_1t^3+q)e^{-(a_1t^3+q)z}dt-3qz\int_{\mathcal{C}_1+\frac{a_2}{3a_1}}e^{-(a_1t^3+q)z}dt\\
&=-3zI'(z)-3qzI(z)
\end{align*}
\end{proof}
%  \[I(z)=\int_{\mathcal{C}_{\alpha}}e^{-zf}dt\] is a solution of a second order ODE with regular/irregular singularity at zero/infinity. In particular, only two cases can occur: let $f(t)=a_1t^3+a_2t^2+a_3t+a_4$
%\begin{enumerate}
%\item if $a_2=0$ and $a_3\neq 0$, \begin{equation} I''+2a_4 I'+\left(a_4^2+\frac{4}{27}\frac{a_3^3}{a_1}\right)I+\frac{I'}{z}+\frac{a_4}{z}I-\frac{1}{9z^2}=0
%\end{equation}
%\item if $a_2=a_3=0$, $I$ solves
%\begin{equation}\label{eq:I,a,e}
%I''+aI'+(aa_4-a_4^2)I+\frac{4+9e}{3}\frac{I'}{z}+\frac{a+2a_4+9a_4e}{3}\frac{I}{z}+\frac{e}{z^2}I=0
%\end{equation}
%for every $a,e\in\C$; in particular if $I$ is a solution of
%\begin{equation}
%I'+\left(a_4+\frac{1}{3z}\right)I=0
%\end{equation}
%then it solves equation \eqref{eq:I,a,e} for every $a,e\in\C$.   
%\end{enumerate}  
%We notice that in the first case (of which Airy is an example), since $a_3\neq 0$ $f(t)$ has always two distinct critical points. Wherease, if $a_3=0$ there is a single critical point at $t=0$.
 
We would like to verify that for every cubic function $f$, the Borel transform of the exponential integral can be expressed by an hypergeometric function and hence deduce its resurgent properties in full generality. If $p\neq 0$, $I(z)$ is a solution of

\begin{equation}
I''+2qI'+\left(\frac{4p^3}{27a_1}+q^2\right)I+\frac{1}{z}I'+\frac{q}{z}I-\frac{1}{9z^2}I=0
\end{equation}

hence a formal solution as $z\to \infty$ is given (up to constants $U_1,U_2\in\C$) by 
\begin{align}
\tilde{I}_{+}(z)&\defeq U_1e^{-(q+\sqrt{\frac{4p^3}{27a_1}})z}z^{1/2}\tilde{w}_+(z)\\
\tilde{I}_-(z)&\defeq U_2 e^{-(q-\sqrt{\frac{4p^3}{27a_1}})z}z^{1/2}\tilde{w}_-(z)
\end{align}
where $\tilde{w}_{\pm}(z)\in\C[\![z^{-1}]\!]$ is the formal solution of
\begin{equation}\label{w_general}
\tilde{w}_{\pm}''\mp 2\sqrt{\frac{4p^3}{27a_1}}\tilde{w}_{\pm}'+\frac{5}{36}\frac{\tilde{w}_{\pm}}{z^2}=0
\end{equation}
with $\tilde{w}_{\pm}(z)=1+\sum_{k\geq 1}a_{\pm,k}z^{-k}$.

We can now compute the Borel transform of \eqref{w_general}: for $\tilde{w}_+(z)$

\begin{align*}
\zeta^2\hat{w}-2\sqrt{\frac{4p^3}{27a_1}}\zeta\hat{w}+\frac{5}{36}\int_0^{\zeta}(\zeta-\zeta')\hat{w}(\zeta')d\zeta'&=0 &\\
\zeta^2\hat{w}'+2\zeta\hat{w}+2\sqrt{\frac{4p^3}{27a_1}}\hat{w}+2\sqrt{\frac{4p^3}{27a_1}}\zeta\hat{w}'+\frac{5}{36}\int_0^\zeta\hat{w}(\zeta')&=0 &\\
\left(\zeta^2+2\sqrt{\frac{4p^3}{27a_1}}\zeta\right)\hat{w}''+4\left(\zeta+\sqrt{\frac{4p^3}{27a_1}}\right)\hat{w}'+\frac{77}{36}\hat{w}&=0 &\\
t(1-t)\hat{w}''+(2-4t)\hat{w}'-\frac{77}{36}\hat{w}&=0 &\zeta=-2t\sqrt{\frac{4p^3}{27a_1}}
\end{align*}

hence \[\hat{w}_+(\zeta)=c_1 \,\, {}_2F_1\left(\frac{7}{6},\frac{11}{6};2;-\frac{3}{4p}\sqrt{\frac{3a_1}{p}}\zeta\right)\] and analougusly, \[\hat{w}_-(\zeta)=c_2\,\, {}_2F_1\left(\frac{7}{6},\frac{11}{6};2;\frac{3}{4p}\sqrt{\frac{3a_1}{p}}\zeta\right).\] Notice that $\hat{w}_{\pm}(\zeta)$ has a branch cut singularity respectively at $\zeta=\zeta_{\pm}\defeq\pm \sqrt{\frac{16p^3}{27a_1}}$, and thanks to the well known formulas for the analytic continuation of hypergeometric functions (see 15.2.3 DLMF), if we assume the branch cut is from $\zeta_{\pm}$ to $+\infty$

\begin{align*}
\hat{w}_-(\zeta+i0)-\hat{w}_-(\zeta-i0)&=c_2 \frac{2\pi i}{\Gamma(7/6)\Gamma(11/6)}\left(\frac{\zeta}{\zeta_+}-1\right)^{-1}\sum_{k\geq 0}\frac{(5/6)_k(1/6)_k}{\Gamma(k)k!}\left(1-\frac{\zeta}{\zeta_+}\right)^k & \zeta\in\left(\zeta_+,+\infty\right) & \\
&=-c_2\frac{2\pi i}{\Gamma(7/6)\Gamma(11/6)} \sum_{k\geq 0}\frac{(5/6)_k(1/6)_k}{\Gamma(k)k!}\left(1-\frac{\zeta}{\zeta_+}\right)^{k-1} & \\
&={-i}c_2\frac{1}{\Gamma(7/6)\Gamma(11/6)}\sum_{k\geq 0}\frac{\Gamma\left(\tfrac{7}{6}+k\right)\Gamma\left(\tfrac{11}{6}+k\right)}{\Gamma(2+k)\Gamma(k+1)}\left(1-\frac{\zeta}{\zeta_+}\right)^k & \\
&={-i}c_2 \,\, {}_2F_1\left(\frac{7}{6},\frac{11}{6};2;1-\frac{\zeta}{\zeta_+}\right) & \\
&={-i}\frac{c_2}{c_1}\hat{w}_+(\zeta-\zeta_+)
\end{align*}

Similarly, 

\begin{align*}
\hat{w}_+(\zeta+i0)-\hat{w}_+(\zeta-i0)&=c_1\frac{2\pi i}{\Gamma(7/6)\Gamma(11/6)}\left(\frac{\zeta}{\zeta_-}-1\right)^{-1}\sum_{k\geq 0}\frac{(5/6)_k(1/6)_k}{\Gamma(k)k!}\left(1-\frac{\zeta}{\zeta_-}\right)^k & \zeta\in\left(-\infty,\zeta_-\right) & \\
&=-c_1\frac{2\pi i}{\Gamma(7/6)\Gamma(11/6)} \sum_{k\geq 0}\frac{(5/6)_k(1/6)_k}{\Gamma(k)k!}\left(1-\frac{\zeta}{\zeta_-}\right)^{k-1} & \\
&=-ic_1\frac{1}{\Gamma(7/6)\Gamma(11/6)}\sum_{k\geq 0}\frac{\Gamma\left(\tfrac{7}{6}+k\right)\Gamma\left(\tfrac{11}{6}+k\right)}{\Gamma(2+k)\Gamma(k+1)}\left(1-\frac{\zeta}{\zeta_-}\right)^k & \\
&=-ic_1\,\, {}_2F_1\left(\frac{7}{6},\frac{11}{6};2;1-\frac{\zeta}{\zeta_-}\right) & \\
&=-i\frac{c_1}{c_2}\hat{w}_-(\zeta-\zeta_-)
\end{align*}

therefore we see that the Stokes factors are given by $\pm i$ (as for Airy). 

\textbf{I think it will be nice to add the geometric interpretation of Maxim in term of Lefschetz thimbles}

The situation is quite different if we consider the degenerate case, where we have only a singular point: indeed there is a one parameter family of solutions of 
\begin{equation}
I'(z)+\left(\frac{1}{3z}+q\right)I(z)=0
\end{equation}
namely for $U\in\C$
\begin{equation}
\tilde{I}(z)=Ue^{-qz}z^{1/3}\tilde{w}(z)\,\qquad \text{ where } \tilde{w}(z)\in\C[\![z^{-1}]\!.]
\end{equation} 
The Borel transform of $\tilde{w}$ is a solution of 
\begin{equation}
\zeta\hat{w}'+\frac{\hat{w}}{3}=0
\end{equation}
hence, up to rescaling by a constant, \[\hat{w}(\zeta)\propto \zeta^{-1/3}=\,\, {}_2F_1\left(a,\frac{1}{3};a;1-\zeta\right)\] for every $a\in \C$. In the degenerate case we get an hypergeometric function as well, but the resurgent structure is trivial, i.e. $\hat{w}(\zeta)$ is holomorphic on the Riemann surface of $\zeta^{1/3}$. 
 

\subsubsection{Alternative computation of the Borel transform of I}

Let us first compute the Borel transform of \eqref{eq:I for p not 0} (indeed as in the proof of Theorem \ref{thm:maxim} we know that \eqref{eq:I for p not 0} admits a formal solution which is Gevrey-$1$)

\begin{align*}
\zeta^2\hat{I}-a\zeta\hat{I}+b\hat{I}-\int_0^{\zeta}\zeta'\hat{I}(\zeta')+d\int_0^{\zeta}\hat{I}(\zeta')-\frac{1}{9}\int_0^{\zeta}(\zeta-\zeta')\hat{I}(\zeta')=0\\
2\zeta\hat{I}+\zeta^2\hat{I}'-a\hat{I}-a\zeta\hat{I}'+b\hat{I}'-\zeta\hat{I}+d\hat{I}-\frac{1}{9}\int\hat{I}(\zeta')=0\\
(\zeta^2-a\zeta+b)\hat{I}''+(3\zeta-2a+d)\hat{I}'+\frac{8}{9}\hat{I}=0
\end{align*}

Now we denote by $\lambda_1,\lambda_2$ the distinguished (we assume that $p\neq 0$) roots of $\zeta^2-a\zeta+b$, then 

\begin{align}
& (\zeta-\lambda_1)(\zeta-\lambda_2)\hat{I}''+(3\zeta-2a+d)\hat{I}'+\frac{8}{9}\hat{I}=0 & \\
&(t+\lambda_2-\lambda_1)t\hat{I}''+(3t+3\lambda_2-2a+d)\hat{I}'+\frac{8}{9}=0 & t=\zeta-\lambda_2 \\
& \label{eq:I hyper} s(1-s)\hat{I}''-\left(3s+\frac{3\lambda_2-2a+d}{\lambda_1-\lambda_2}\right)\hat{I}'-\frac{8}{9}\hat{I}=0 & t=(\lambda_1-\lambda_2)s
\end{align}

where \eqref{eq:I hyper} is an hypergeometric equation\footnote{Notice that $\lambda_{1,2}=q\pm \frac{2i}{3}p\sqrt{\frac{p}{3a_1}}$, $a=2q$ and $d=q$. Hence \[\frac{2a-d-3\lambda_2}{\lambda_1-\lambda_2}=\frac{4q-q-3q-2ip\sqrt{\frac{p}{3a_1}}}{-\frac{4i}{3}p\sqrt{\frac{p}{3a_1}}}=\frac{3}{2}\]} and a solution is given by

\begin{equation}
\hat{I}_{\lambda_1}(\zeta;U_1,U_2)=U_1\,{}_2F_1\left(\frac{2}{3},\frac{4}{3};\frac{3}{2};\frac{\zeta-\lambda_2}{\lambda_1-\lambda_2}\right)+U_2\left(\frac{\zeta-\lambda_2}{\lambda_1-\lambda_2}\right)^{-1/2}{}_2F_1\left(\frac{1}{6},\frac{5}{6};\frac{1}{2};\frac{\zeta-\lambda_2}{\lambda_1-\lambda_2}\right)
\end{equation}  

which has a branch cut at $\zeta=\lambda_1$, where $U_1,U_2$ are constants. Of course, reversing the role of $\lambda_1$ and $\lambda_2$ we find  

\begin{equation}
\hat{I}_{\lambda_2}(\zeta,;U_1,U_2)=U_1{}_2F_1\left(\frac{2}{3},\frac{4}{3};\frac{3}{2};\frac{\zeta-\lambda_1}{\lambda_2-\lambda_1}\right)+U_2\left(\frac{\zeta-\lambda_1}{\lambda_2-\lambda_1}\right)^{-1/2}{}_2F_1\left(\frac{1}{6},\frac{5}{6};\frac{1}{2};\frac{\zeta-\lambda_1}{\lambda_2-\lambda_1}\right)
\end{equation} 

is the Borel transform of $\tilde{I}_{\lambda_2}(z)$ and it has a branch cut singularity at $\zeta=\lambda_2$. It is remarkable that  the dependece on the function $f$ is only on the location of the singularities, but it is always an hypergeometric function with the same parameters. In addition, we can compute the Stokes constants thanks to the well known formula for analyitic continuation of hypergeometric (see 15.2.3 in DLMF) 

\begin{align*}
\hat{I}_{\lambda_1}(\zeta+i0;U_1,0)-\hat{I}_{\lambda_1}(\zeta-i0;U_1,0)&=-U_1\frac{2\pi i}{\Gamma\left(\frac{2}{3}\right)\Gamma\left(\frac{4}{3}\right)}\left(\frac{\lambda_1-\zeta}{\lambda_2-\lambda_1}\right)^{-1/2}{}_2F_1\left(\frac{1}{6},\frac{5}{6};\frac{1}{2};\frac{\zeta-\lambda_1}{\lambda_2-\lambda_1}\right)\\
&=\mathbf{-i}U_1\frac{2\pi i}{\Gamma\left(\frac{2}{3}\right)\Gamma\left(\frac{4}{3}\right)}\left(\frac{\zeta-\lambda_1}{\lambda_2-\lambda_1}\right)^{-1/2}{}_2F_1\left(\frac{1}{6},\frac{5}{6};\frac{1}{2};\frac{\zeta-\lambda_1}{\lambda_2-\lambda_1}\right)\\
&=\mathbf{-i} \hat{I}_{\lambda_2}\left(\zeta;0,\frac{U_1}{\Gamma(2/3)\Gamma(4/3)}\right)
\end{align*}
    
\end{document}
