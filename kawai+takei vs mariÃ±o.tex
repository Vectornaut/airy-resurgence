\documentclass{article}

\usepackage{ajf}
\usepackage{amsthm}

\theoremstyle{definition}
\newtheorem{defn}{Definition}
\theoremstyle{plain}
\newtheorem{prop}{Proposition}
\newtheorem{thm}{Theorem}

\DeclareMathOperator{\Ai}{Ai}

\title{Airy function: Kawai+Takei vs. Mari\~{n}o}
\author{Aaron Fenyes}

\begin{document}
\maketitle
Kawai and Takei want to solve
\[ \left[\left(\frac{d}{dx}\right)^2 - \eta^2 x \right] \psi(x, \eta) = 0. \]
They define $\psi_B(x, y)$ as the inverse Laplace transform of $\psi(x, \eta)$ with respect to $\eta$.

With $w = x \eta^{2/3}$, the equation above is equivalent to
\[ \left[\left(\frac{d}{dw}\right)^2 - w \right] \psi(w \eta^{-2/3}, \eta) = 0. \]
Proof: substitute back to get
\begin{align*}
\left[\eta^{-4/3} \left(\frac{d}{dx}\right)^2 - \eta^{2/3} x \right] \psi(x, \eta) = 0 \\
\left[\eta^{-4/3} \left(\frac{d}{dx}\right)^2 - \eta^{-4/3} \eta^2 x \right] \psi(x, \eta) = 0 \\
\eta^{-4/3} \left[\left(\frac{d}{dx}\right)^2 - \eta^2 x \right] \psi(x, \eta) = 0.
\end{align*}
Hence, $\psi(w \eta^{-2/3}, \eta) = k(\eta) \Ai(w)$ is a solution for any holomorphic function $k$.

%y = x eta^(2/3)
%d/dy^2 = eta^(-4/3) d/dx^2
%eta^(4/3) d/dy^2 = d/dx^2
%
%eta^2 x = eta^2 y eta^(-2/3)
%        = eta^(4/3) y
%
%[(d/dy)^2 - y] psi(x, eta) = 0
%
%psi(x, eta) = Ai(y) is a solution
%psi(eta^(-2/3) y, eta) = Ai(y)
\end{document}