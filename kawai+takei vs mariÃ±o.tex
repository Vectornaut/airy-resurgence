\documentclass{article}

%\usepackage{ajf}
\usepackage{amsthm}

\theoremstyle{definition}
\newtheorem{defn}{Definition}
\theoremstyle{plain}
\newtheorem{prop}{Proposition}
\newtheorem{lemma}{Lemma}
\newtheorem{thm}{Theorem}
\usepackage{url}
\usepackage[hmargin=1.5in]{geometry}
\usepackage{amsmath}
\usepackage{amssymb}
\usepackage{graphicx}
\usepackage{xcolor}
\usepackage{tikz}
\usetikzlibrary{cd}
%\DeclareMathOperator{\Ai}{Ai}

\title{Airy function: Kawai+Takei vs. Mari\~{n}o}
\author{Aaron Fenyes, Veronica Fantini}

\begin{document}
\maketitle
Kawai and Takei want to solve
\[ \left[\left(\frac{d}{dx}\right)^2 - \eta^2 x \right] \psi(x, \eta) = 0. \]
They define $\psi_B(x, y)$ as the inverse Laplace transform of $\psi(x, \eta)$ with respect to $\eta$.

With $w = x \eta^{2/3}$, the equation above is equivalent to
\[ \left[\left(\frac{d}{dw}\right)^2 - w \right] \psi(w \eta^{-2/3}, \eta) = 0. \]
Proof: substitute back to get
\begin{align*}
\left[\eta^{-4/3} \left(\frac{d}{dx}\right)^2 - \eta^{2/3} x\right] \psi(x, \eta) = 0 \\
\left[\eta^{-4/3} \left(\frac{d}{dx}\right)^2 - \eta^{-4/3}\eta^2 x \right] \psi(x, \eta) = 0 \\
\eta^{-4/3} \left[\left(\frac{d}{dx}\right)^2 - \eta^2 x\right] \psi(x, \eta) = 0.
\end{align*}
Hence, $\psi(w \eta^{-2/3}, \eta) = k(\eta) \mathsf{Ai}(w)$ is a solution for any holomorphic function $k$.


\section{Veronica's change of coordinates}

Kawai and Takei study the WKB analysis of the equation

\begin{equation}
\label{WKB_Airy} 
\left[\left(\frac{d}{dx}\right)^2 - \eta^2 x \right] \psi(x, \eta) = 0 
\end{equation}
as $\eta\to\infty$. They define $\psi_B(x, y)$ as the inverse Laplace transform of $\psi(x, \eta)$ with respect to $\eta$. In the coordinates $t=yx^{-3/2}$ they find an explicit formula for $\psi_B(x,y)$ in terms of Gauss hypergeometric functions:
\begin{align*}
\psi_{+,B}(x,y)&=\frac{1}{x}\phi_+(t)=\frac{\sqrt{3}}{2\sqrt{\pi}}\frac{1}{x}s^{-1/2}\, {}_2F_1\left(\frac{1}{6},\frac{5}{6};\frac{1}{2};s\right)\\
\psi_{-,B}(x,y)&=\frac{1}{x}\phi_-(t)=\frac{\sqrt{3}}{2\sqrt{\pi}}\frac{1}{x}(1-s)^{-1/2}\, {}_2F_1\left(\frac{1}{6},\frac{5}{6};\frac{1}{2};1-s\right)
\end{align*}
where $s=3t/4+1/2$. 
The same hypergeometric functions have been computed in Section ?? as the Borel transform of the formal solutions of the Airy equation

\begin{equation}
\label{Airy}
\left[\left(\frac{d}{dw}\right)^2 -  w \right] f(w) = 0.
\end{equation}

Although the two equations look closely related (they are equivalent by the change of coordinates $w=x\eta^{2/3}$), the Borel transform of $\psi$ is computed with respect to $\eta x^{3/2}$ (which is the conjugate variable of $t$) while the Borel transform of $f(w)$ is computed with respect to $w$. So we need to find a different change of coordinates to explain why the Borel transforms of $\psi(x,\eta)$ and $f(w)$ are given by the same hypergeometric function. 

First of all notice that if $\eta$ and $y$ are conjugate variables under Borel transform, meaning 
\begin{align*}
\sum_{n\geq 0}a_n\eta^{-n-1}  \overset{\mathcal{B}}{\longrightarrow} \sum_{n\geq 0}\frac{a_n}{n!} y^{n} 
\end{align*} 
then $t=yx^{-3/2}$ is the conjugate variable of $q=\eta x^{3/2}$ up to correction by a factor of $x^{-3/2}$
\begin{align*}
\sum_{n\geq 0}a_nq^{-n-1}=\sum_{n\geq 0}a_nx^{-3/2(n+1)}\eta^{-n-1}  \overset{\mathcal{B}}{\longrightarrow} \sum_{n\geq 0}\frac{a_nx^{-3/2(n+1)}}{n!} y^{n}=x^{-3/2}\sum_{n\geq 0}\frac{a_n}{n!} t^{n}. 
\end{align*}
In addition, $\psi_{B,\pm}(x,y)=\frac{1}{x}\phi_{\pm}(t)$, therefore we expect that $\psi(x,\eta)=x^{1/2}\Phi(q)$. Assume that $\psi(x,y)$ is a solution of \eqref{WKB_Airy}, then $\Phi(q)$ solves 
\begin{equation}
\label{eq_Phi}
\left[\left(\frac{d}{dx}\right)^2+x^{-1}\frac{d}{dx}-\frac{1}{4}x^{-2} - \eta^2 x \right] \Phi(q) = 0
\end{equation}

\begin{proof}
\begin{align*}
&\left[\left(\frac{d}{dx}\right)^2 - \eta^2 x \right] \psi(x, \eta) = 0\\
&\left[\left(\frac{d}{dx}\right)^2 - \eta^2 x \right] x^{1/2}\Phi(q) = 0\\
&\frac{d}{dx}\left[\frac{1}{2}x^{-1/2}\Phi+x^{1/2}\frac{d}{dx}\Phi\right]-\eta^2x^{3/2}\Phi=0\\
&-\frac{1}{4}x^{-3/2}\Phi+\frac{1}{2}x^{-1/2}\frac{d}{dx}\Phi+\frac{1}{2}x^{-1/2}\frac{d}{dx}\Phi+x^{1/2}\left(\frac{d}{dx}\right)^2\Phi-\eta^2x^{3/2}\Phi=0\\
&\left[x^{1/2}\left(\frac{d}{dx}\right)^2+x^{-1/2}\frac{d}{dx}-\frac{1}{4}x^{-3/2}-\eta^2x^{3/2}\right]\Phi=0\\
&\left[\left(\frac{d}{dx}\right)^2+x^{-1}\frac{d}{dx}-\frac{1}{4}x^{-2}-\eta^2x\right]\Phi=0
\end{align*}
\end{proof}
Now rewrite \eqref{eq_Phi} in the coordinates $q=\eta x^{3/2}$: 
\begin{align*}
&\left[\left(\frac{d}{dx}\right)^2+x^{-1}\frac{d}{dx}-\frac{1}{4}x^{-2}-\eta^2x\right]\Phi=0\\
&\left[\frac{9}{4}\eta^2x\left(\frac{d}{dq}\right)^2+\frac{3}{4}\eta x^{-1/2}\frac{d}{dq}+x^{-1}\cdot\frac{3}{2}\, \eta\,  x^{1/2}\frac{d}{dq}-\frac{1}{4}x^{-2}-\eta^2x\right]\Phi=0\\
&\left[\eta^2x\left(\frac{d}{dq}\right)^2+\frac{1}{3}\eta x^{-1/2}\frac{d}{dq}+\frac{2}{3}\, \eta\,  x^{-1/2}\frac{d}{dq}-\frac{1}{9}x^{-2}-\frac{4}{9}\eta^2x\right]\Phi=0\\
&\left[\eta^2\left(\frac{d}{dq}\right)^2+\eta x^{-3/2}\frac{d}{dq}-\frac{1}{9}x^{-3}-\frac{4}{9}\eta^2\right]\Phi=0\\
&\left[\left(\frac{d}{dq}\right)^2+\eta^{-1} x^{-3/2}\frac{d}{dq}-\frac{1}{9}\eta^{-2}x^{-3}-\frac{4}{9}\right]\Phi=0\\
&\left[\left(\frac{d}{dq}\right)^2+q^{-1}\frac{d}{dq}-\frac{1}{9}q^{-2}-\frac{4}{9}\right]\Phi=0
\end{align*}

therefore $\Phi(q)$ is a solution of the transform Airy equation (see draft2).  

%y = x eta^(2/3)
%d/dy^2 = eta^(-4/3) d/dx^2
%eta^(4/3) d/dy^2 = d/dx^2
%
%eta^2 x = eta^2 y eta^(-2/3)
%        = eta^(4/3) y
%
%[(d/dy)^2 - y] psi(x, eta) = 0
%
%psi(x, eta) = Ai(y) is a solution
%psi(eta^(-2/3) y, eta) = Ai(y)

\section{Weber equation: WKB vs modifield Bessel ODE}

In [Takei] the author studied the Borel summation of WKB solutions of the harmonic oscillator 

\begin{equation}\label{eq:weber}
\left[\frac{d^2}{dx^2}-\eta^2x^2-\lambda\eta\right]\psi=0
\end{equation}

with a parameter $\lambda$. He proved that the WKB solutions
\begin{equation}
\psi_{\pm}(x,\eta)=e^{\pm\eta x^2/2}\sum_{n=0}^{\infty}\frac{\psi_{\pm,n}}{x^{2n+(1\pm\lambda)/2}}\eta^{-\left(\frac{1}{2}\pm\frac{\lambda}{4}+n\right)}
\end{equation}
 where $\psi_{\pm,n}$ are constants independent of $x$ and $\eta$ have Borel transform (in the variable $y$ conjugate to $\eta$) 
 \begin{align}
 \psi_{+,B}(x,y)&=\frac{x^{-3/2}}{\Gamma\left(\frac{1}{2}+\frac{\lambda}{4}\right)}s^{-1/2+\lambda/4} F\left(\frac{1}{4}+\frac{\lambda}{4},\frac{3}{4}+\frac{\lambda}{4};\frac{1}{2}+\frac{\lambda}{4};s\right)\\
 \psi_{-,B}(x,y)&=\frac{x^{-3/2}}{\Gamma\left(\frac{1}{2}+\frac{\lambda}{4}\right)}(s-1)^{-1/2+\lambda/4} F\left(\frac{1}{4}+\frac{\lambda}{4},\frac{3}{4}+\frac{\lambda}{4};\frac{1}{2}+\frac{\lambda}{4};1-s\right)
 \end{align}
where $s=y/x^2+1/2$. 

\begin{lemma}
Set $\lambda=0$. Let $\eta=2zx^{-2}$ then $\psi_-(x,\eta)=\frac{1}{2\sqrt{\pi}}x^{1/2}\tilde{K}_{1/4}(z)$ is a solution of \eqref{eq:weber} if and only if $K_{1/4}(z)$ is a solution of the modified Bessel equation

\begin{equation}
\left[\frac{d^2}{dz^2}-1+\frac{1}{z}\frac{d}{dz}-\frac{1}{16 z^2}\right]\tilde{K}_{1/4}(z)=0
\end{equation}

\end{lemma}

\begin{proof}
We start with the Weber equation \eqref{eq:weber}

\begin{align*}
&\left[\frac{d^2}{dx^2}-\eta^2x^2\right]\psi_-=0\\
&\left[\frac{d^2}{dx^2}-\eta^2x^2\right]x^{1/2}\tilde{K}_{1/4}(z)=0\\
&\frac{d}{dx}\left(\frac{1}{2}x^{-1/2}\tilde{K}_{1/4}(z)+x^{1/2}\frac{d}{dx}\tilde{K}_{1/4}(z)\right)-\eta^2x^{5/2}\tilde{K}_{1/4}(z)=0\\
&-\frac{1}{4}x^{-3/2}\tilde{K}_{1/4}(z)+x^{-1/2}\frac{d}{dx}\tilde{K}_{1/4}(z)+x^{1/2}\frac{d^2}{dx^2}\tilde{K}_{1/4}(z)-\eta^2x^{5/2}\tilde{K}_{1/4}(z)=0\\
&-\frac{1}{4}x^{-2}\tilde{K}_{1/4}(z)+x^{-1}\frac{d}{dx}\tilde{K}_{1/4}(z)+\frac{d^2}{dx^2}\tilde{K}_{1/4}(z)-\eta^2x^{2}\tilde{K}_{1/4}(z)=0
\end{align*}

Under the change of coordinates $\eta=2zx^{-2}$ we have
\begin{align*}
\frac{d}{dx}=\eta x\frac{d}{dz} & \qquad \qquad \frac{d^2}{dx^2}=\eta\frac{d}{dz}+\eta^2 x^2\frac{d^2}{dz^2}
\end{align*}
therefore

\begin{align*}
&-\frac{1}{4}x^{-2}\tilde{K}_{1/4}(z)+x^{-1}\frac{d}{dx}\tilde{K}_{1/4}(z)+\frac{d^2}{dx^2}\tilde{K}_{1/4}(z)-\eta^2x^{2}\tilde{K}_{1/4}(z)=0\\
&-\frac{1}{4}x^{-2}\tilde{K}_{1/4}(z)+\eta\frac{d}{dz}K_{1/4}(z)+\left(\eta\frac{d}{dz}+\eta^2 x^2\frac{d^2}{dz^2}\right)\tilde{K}_{1/4}(z)-\eta^2x^{2}\tilde{K}_{1/4}(z)=0\\
&-\frac{1}{4}\frac{\eta}{2z}\tilde{K}_{1/4}(z)+\eta\frac{d}{dz}\tilde{K}_{1/4}(z)+\eta\frac{d}{dz}\tilde{K}_{1/4}(z)+2z\eta \frac{d^2}{dz^2}\tilde{K}_{1/4}(z)-2z\eta \tilde{K}_{1/4}(z)=0\\
&-\frac{1}{4}\frac{1}{4z^2}K_{1/4}(z)+\frac{1}{z}\frac{d}{dz}\tilde{K}_{1/4}(z)+ \frac{d^2}{dz^2}K_{1/4}(z)- \tilde{K}_{1/4}(z)=0\\
&\qquad\qquad\qquad\qquad\qquad \left[\frac{d^2}{dz^2}-1+\frac{1}{z}\frac{d}{dz}-\frac{1}{16 z^2}\right]\tilde{K}_{1/4}(z)=0
\end{align*}


\end{proof}



\begin{lemma}
Set $\lambda=0$. Let $\eta=2zx^{-2}$ then \[\psi_{-,B}(x,y)=x^{-3/2}\hat{\kappa}_{1/4}(\zeta)\] where $\hat{\kappa}_{1/4}(\zeta)=\mathcal{B}(\tilde{K}_{1/4})(\zeta)$. 
\end{lemma}

\begin{proof}
We move to the Borel plane: let $\zeta$ be the variable conjugate to $z$ and let $y$ the conjugate to $\eta$, i.e.
\begin{align*}
\mathcal{B}\colon & \sum_{n\geq 0}a_nz^{-n-1}\to \sum_{n\geq 0}a_n\frac{\zeta^n}{n!}\\
\mathcal{B}\colon&\sum_{n\geq 0}a_n\eta^{-n-1}\to \sum_{n\geq 0}a_n\frac{y^n}{n!}
\end{align*} 

if $\eta=2zx^{-2}$ then $y=\frac{\zeta}{2}x^2$, indeed 


\begin{tikzcd}
\sum_{n\geq 0}a_n\eta^{-n-1} \arrow[d,equal]\arrow[rr,"\mathcal{B}"] & & \sum_{n\geq 0}a_n\frac{y^n}{n!}\\
\sum_{n\geq 0}a_n\left(\frac{2}{x^2}\right)^{-n-1}z^{-n-1}\arrow[r,equal]&\frac{x^2}{2}\sum_{n\geq 0}a_n\left(\frac{2}{x^2}\right)^{-n}z^{-n-1}\arrow[r,"\mathcal{B}"] &\frac{x^2}{2}\sum_{n\geq 0}a_n\left(\frac{x^2}{2}\right)^{n}\frac{\zeta^n}{n!}
\end{tikzcd}

In particular, $\tilde{K}_{1/4}(z)$ is known as the asymptotics of the modified Bessel function $K_{1/4}(z)$ which is equal to 

\begin{align*}
\tilde{K}_{1/4}(z)&=\sqrt{\pi}e^{-z}\sum_{n= 0}^{\infty}(-1)^n\frac{\left(\frac{1}{4}\right)_n\left(\frac{3}{4}\right)_n}{n!}\left(\frac{z}{2}\right)^{-n-1/2}
\end{align*}

Then we can compute its Borel transform: 

\begin{align*}
\hat{\kappa}_{1/4}&=\sqrt{\pi}\sum_{n\geq 0}\frac{(-1)^n\left(\frac{1}{4}\right)_n\left(\frac{3}{4}\right)_n}{\Gamma(n+1/2) n!}\left(\frac{\zeta-1}{2}\right)^{n-1/2}\\
&=\sqrt{\pi}\left(\frac{\zeta-1}{2}\right)^{-1/2}\sum_{n\geq 0}\frac{\left(\frac{1}{4}\right)_n\left(\frac{3}{4}\right)_n}{\Gamma(n+1/2) n!}\left(\frac{1-\zeta}{2}\right)^{n}\\
&=\sqrt{2}(\zeta-1)^{-1/2}F\left(\frac{1}{4},\frac{3}{4};\frac{1}{2};\frac{1-\zeta}{2}\right)
\end{align*}

hence 

\begin{tikzcd}
\psi_-(x,\eta) \arrow[d,equal]\arrow[r,"\mathcal{B}"]   & \psi_{-,B}(x,y)=\frac{x^{-3/2}}{\sqrt{\pi}}(s-1)^{-1/2+\lambda/4} F\left(\frac{1}{4},\frac{3}{4};\frac{1}{2}+;1-s\right)\arrow[d,equal,red,"\cdot \frac{x^2}{2}"]\\
\frac{1}{2\sqrt{\pi}}x^{1/2}\tilde{K}_{1/4}(z)\arrow[r,"\mathcal{B}"] &\frac{1}{2\sqrt{\pi}}x^{1/2}\sqrt{2}(\zeta-1)^{-1/2}F\left(\frac{1}{4},\frac{3}{4};\frac{1}{2};\frac{1-\zeta}{2}\right)
\end{tikzcd}

\end{proof}


\end{document}