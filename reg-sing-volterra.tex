\documentclass{article}

\usepackage{url}
\usepackage[hmargin=1.5in]{geometry}
\usepackage{amsmath}
\usepackage{amssymb}
\usepackage{amsthm}
\usepackage{eqnarray}
\usepackage{graphicx}
\usepackage[svgnames]{xcolor} %% for revisions
\usepackage{xparse} %% for squiggly underlines
\usepackage{tikz-cd} %% for term norm scratch work

%%\theoremstyle{definition}
\newtheorem{defn}{Definition}
\theoremstyle{plain}
\newtheorem{prop}{Proposition}
\newtheorem{thm}{Theorem}

% convenience aliases
\newcommand{\maps}{\colon}

% group action
\newcommand{\acts}{\mathbin{\raisebox{\depth}{\rotatebox{-90}{$\circlearrowright$}}}}

% symbology
\newcommand{\Z}{\mathbb{Z}}
\newcommand{\R}{\mathbb{R}}
\newcommand{\C}{\mathbb{C}}
\let\Re\relax
\DeclareMathOperator{\Re}{Re}
\newcommand{\laplace}{\mathcal{L}}
\newcommand{\series}[1]{\tilde{#1}}
\newcommand{\fracderiv}[3]{\partial^{#1}_{#2, #3}}
\newcommand{\cont}{\mathcal{C}}
\newcommand{\holo}{\mathcal{H}}
\newcommand{\singexp}[2]{\mathcal{H}L^\infty_{#1, #2}}
\newcommand{\holoL}[1]{\mathcal{H}L^{#1}} %% may no longer be needed
\newcommand{\expHoloL}[2]{\mathcal{H}L^{#1}_{#2}} %% may no longer be needed
\DeclareMathOperator{\Ai}{Ai}

% drafting environments
\newenvironment{verify}{\color{ForestGreen}}{\color{black}}

\title{Regular singular Volterra equations on complex domains}
\author{Aaron Fenyes}
\date{}

\begin{document}
\maketitle
\section{Introduction}
\subsection{Motivation}\label{motivation}
In its most basic form, the Laplace transform $\laplace$ turns $L^1$ functions of a real ``position'' variable $\zeta$ into holomorphic functions of a complex ``frequency'' variable $z$. Through identities like
\begin{align*}
\frac{\partial}{\partial z} \laplace \varphi & = \laplace(z\varphi) \\
\laplace k\;\laplace \varphi & = \laplace(k * \varphi) \\
z^{-\lambda} \laplace \varphi & = \laplace\,\fracderiv{-\lambda}{}{} \varphi,
\end{align*}
where $\fracderiv{-\lambda}{}{}$ is the Riemann-Liouville fractional integral of order $\lambda \in (0, \infty)$, the Laplace transform pulls differential operators on the frequency domain back to Volterra integral operators on the position domain. The favorable regularity properties and comprehensive theory of Volterra equations can thus be brought to bear on differential equations.

If we complexify the position variable $\zeta$, the functions on the position domain whose Laplace transforms satisfy a given differential equation will often extend holomorphically. \textbf{[...]}
\subsection{Main result}
%%\begin{defn}\label{fin-order}
%%Consider a Volterra operator $\mathcal{V}$ of the form
%%\[ [\mathcal{V}\varphi](a) = m(a)\,\varphi(a) + \int_{\zeta = 0}^{a} k(a, \cdot)\,\varphi\,d\zeta. \]
%%We'll say $\mathcal{V}$ is {\em effectively finite-order} over a domain $\Omega$ if there are some $c, c' \in (0, \infty)$ for which
%%\begin{itemize}
%%\item $|m(a)| \lesssim e^{c|\zeta(a)|}$ over all $a \in \Omega$
%%\item $|k(a, a')| \lesssim e^{c|\zeta(a)| + c'|\zeta(a')|}$ over all $a, a' \in \Omega$
%%\end{itemize}
%%and at one of the following holds:
%%\begin{enumerate}
%%\item\label{case:0th-order} $m$ isn't the zero function.
%%\item\label{case:pos-order} $m$ is the zero function, and there's some $\lambda \in (0, \infty)$ for which
%%\[ |k(a, a')| \in O_\text{\rm diag}\big( |\zeta(a) - \zeta(a')|^{\lambda-1} \big), \]
%%where {\rm ``diag''} means the diagonal in $\Omega^2$.
%%\end{enumerate}
%%We'll say $\mathcal{V}$ has order $0$ in case~\ref{case:0th-order}, and order at least $\lambda$ in case~\ref{case:pos-order}.
%%\end{defn}
\textcolor{DarkTurquoise}{The result below might not be stated perfectly, but it gives at least a rough sense of what I expect. It's an improved version of the result stated in Section~\ref{frac_int_exist}, which is copied from {\tt airy-resurgence}. Note that {\tt airy-resurgence} uses the opposite sign convention for the parameter $\sigma$ in $\holoL{\infty, \sigma}$.}
\begin{thm}
Let $\Omega$ be an acute-angled open sector around $\zeta = 0$ in the complex plane. \textcolor{DarkTurquoise}{I don't think it's really necessary for $\Omega$ to have a sharp point at $0$, but the condition of being an acute-angled sector is sufficient and easy to state.} Over $\Omega$, consider a Volterra operator
\[ \mathcal{V} = p + \partial^{-1} \circ q + \mathcal{R}, \]
where
\begin{itemize}
\item $p$ is a holomorphic function on $\Omega$ that extends holomorphically over $\zeta = 0$, with
\begin{itemize}
\item a simple zero at $\zeta = 0$,
\item no zeros in $\Omega$ (as the next condition implies), and
\item a positive lower bound $C_p |z|^d \le |p|$;
\end{itemize}
\item $q$ is a holomorphic function on $\Omega$ that extends holomorphically over $\zeta = 0$, with
\begin{itemize}
\item an upper bound $|q| \le C_q |z|^{d-1}$; and
\end{itemize}
\item $\mathcal{R}$ is an integral kernel operator
\[ [\mathcal{R}\varphi](a) = \int_{\zeta = 0}^{a} k(a, \cdot)\,\varphi\,d\zeta, \]
where $k$ is a holomorphic function on $\Omega^2$, with
\begin{itemize}
\item an exponential upper bound $k(a, a') \le C_k e^{A|\zeta(a) - \zeta(a')|}$.
\end{itemize}
\end{itemize}
Let $p_1$ and $q_0$ be the values of $\tfrac{\partial}{\partial \zeta} p$ and $q$, respectively, at $\zeta = 0$, and let $\tau = q_0 / p_1$.

For any $\epsilon > 0$, when the constant $\Lambda > A$ is large enough, the regular singular Volterra equation
\[ \mathcal{V}f = 0 \]
has a unique solution in the affine subspace
\[ \zeta^{\tau-1} + \holoL{\infty, \tau-1+\epsilon}_\Lambda(\Omega) \]
of the function space $\holoL{\infty, \tau-1}_\Lambda(\Omega)$, defined in Section~\ref{fn-spaces}.
\end{thm}
%%\begin{thm}
%%Let $\Omega$ be an acute-angled sector around $0$ in the complex plane. \textcolor{DarkTurquoise}{I don't think it's really necessary for $\Omega$ to have a sharp point at $0$, but the condition of being an acute-angled sector is sufficient and easy to state.}

%%Consider a Volterra operator
%%\[ \mathcal{V} = p + \partial^{-1} \circ q + \mathcal{R} \]
%%over $\Omega$ which can be decomposed, as shown above, into the sum of a zeroth-order part $p$, a first-order part $\partial^{-1} \circ q$, and an effectively higher-order part $\mathcal{R}$. To be precise, $\mathcal{R}$ is supposed to be an integral operator of the form
%%\[ [\mathcal{R}\varphi](a) = \int_{\zeta = 0}^{a} k(a, \cdot)\,\varphi\,d\zeta \]
%%which is effectively finite-order over $\Omega$ in the sense of Definition~\ref{fin-order}, with order at least $1 + \epsilon$ for some $\epsilon \in (0, \infty)$.

%%Assume, furthermore, that:
%%\begin{itemize}
%%\item There's some $c \in (0, \infty)$ for which
%%$|p(a)| \lesssim e^{c|\zeta(a)|}$ and $|q(a)| \lesssim e^{c|\zeta(a)|}$ over all $a \in \Omega$.
%%\item $p$ and $q$ extend holomorphically over $\zeta = 0$.
%%\item $p$ has a simple zero at $\zeta = 0$.
%%\end{itemize}
%%Let $p_1$ and $q_0$ be the values of $\tfrac{\partial}{\partial \zeta} p$ and $q$, respectively, at $\zeta = 0$.

%%Under these conditions, let's try to solve the regular singular Volterra equation
%%\[ \mathcal{V}f = 0. \]
%%The leading-order part of the equation,
%%\[ \big[p_1 \zeta + \partial^{-1} \circ q_0\big]f = 0, \]
%%has $\zeta^{\tau-1}$ as a solution, where $\tau = p_1/q_0$. If we guess that the function spaces defined in Section~\ref{fn-spaces} are well suited for this problem, we might hope for the full equation to have a solution of the form
%%\[ f = \zeta^{\tau-1} + f_*, \]
%%where $f_* \in \holoL{\infty, 1-\tau-\epsilon}_Z(\Omega)$ for some $Z \in \C$. When $Z\Omega$ is in the right half-plane and $|Z|$ is large enough, there is in fact a unique solution of this form.
%%\end{thm}
\section{Function spaces for holomorphic Volterra operators}\label{fn-spaces}
\subsection{Weighted holomorphic $L^\infty$ spaces}
Throughout this paper, as described in Section~\ref{motivation}, the ``position'' variable $\zeta$ will be the standard coordinate on $\C$. Take a simply connected open set $\Omega \subset \C$ that touches but doesn't contain $\zeta = 0$. Let $\cont(\Omega)$ be the space of continuous complex-valued functions on $\Omega$. Give $\cont(\Omega)$ the compact-open topology, recalling that this is the finest topology in which the seminorm $f \mapsto \sup_K |f|$ is continuous for every compact subset $K \subset \Omega$~\cite[Example~2.6 and \S 4 notes]{fnl-cpx-anal}. The holomorphic functions form a closed subspace $\holo(\Omega) \subset \cont(\Omega)$~\cite[Proposition~3.14]{fnl-cpx-anal}\textcolor{orange}{[Luecking \& Rubel]}.

Fixing a real constant $\Lambda$, let's restrict our attention to holomorphic functions on $\Omega$ which are bounded by constant multiples of $e^{\Lambda|\zeta|}$. One might describe these functions as being uniformly of exponential type $\Lambda$. They form a space $\singexp{0}{\Lambda}(\Omega)$, which we'll equip with the norm $\|f\|_{0,\Lambda} = \sup_\Omega e^{-\Lambda|\zeta|}\,|f|$. With respect to the seminorm on $\holo(\Omega)$ given by a compact set $K \subset \Omega$, the inclusion map $\singexp{0}{\Lambda}(\Omega) \hookrightarrow \holo(\Omega)$ has norm $\sup_K e^{\Lambda |\zeta|}$. That means the inclusion is continuous.
\begin{prop}\label{exp-complete}
The space $\singexp{0}{\Lambda}(\Omega)$ is complete.
\end{prop}
\begin{proof}
\textcolor{orange}{[Check!]} Take a Cauchy sequence $f_1, f_2, f_3, \ldots \in \singexp{0}{\Lambda}(\Omega)$. The inclusion map $\singexp{0}{\Lambda}(\Omega) \hookrightarrow \holo(\Omega)$ is bounded with respect to each of the seminorms on $\holo(\Omega)$, as discussed above, so our sequence is Cauchy in $\holo(\Omega)$ too. Since $\holo(\Omega)$ is complete~\cite[Proposition~3.5]{fnl-cpx-anal}, our sequence converges to a function $f$ there.

The Cauchy property in $\singexp{0}{\Lambda}(\Omega)$ tells us that for any $r > 0$, we can find some $n$ for which $e^{-\Lambda |\zeta|} |f_k - f_n| \le r$ whenever $k \ge n$. Since convergence in $\holo(\Omega)$ implies pointwise convergence, we can see as $k$ grows that $e^{-\Lambda |\zeta|} |f - f_n| \le r$. This shows that our sequence converges to $f$ in the norm $\|\cdot\|_{0,\Lambda}$. We can also use the Cauchy property in $\singexp{0}{\Lambda}(\Omega)$ to bound the norms $\|f_n\|_{0,\Lambda}$. Then, from the pointwise convergence argument above, we can draw the additional conclusion that $f$ is in $\singexp{0}{\Lambda}(\Omega)$.
\end{proof}

Now, let's relax our norm to allow both exponential growth at infinity and a power-law singularity at $\zeta = 0$. Let $\singexp{\sigma}{\Lambda}(\Omega)$ be the space of holomorphic functions on $\Omega$ which are bounded by constant multiples of $|\zeta|^\sigma e^{\Lambda|\zeta|}$. Give it the norm $\|f\|_{\sigma,\Lambda} = \sup_\Omega |\zeta|^{-\sigma} e^{-\Lambda|\zeta|}\,|f|$. Reprising the arguments from above, we can show that the inclusion $\singexp{\sigma}{\Lambda}(\Omega) \hookrightarrow \holo(\Omega)$ is continuous, and we can generalize Proposition~\ref{exp-complete}:
\begin{prop}
The space $\singexp{\sigma}{\Lambda}(\Omega)$ is complete.
\end{prop}
%%Let $\holoL{\infty}(\Omega)$ be the space of bounded holomorphic functions on $\Omega$ with the supremum norm $\|\cdot\|_\infty$. For any $\sigma \in \R$ and $Z \in \C$, multiplying by $\zeta^\sigma e^{\Re(Z\zeta)}$ maps $\holoL{\infty}(\Omega)$ isomorphically onto another space of holomorphic functions on $\Omega$. We'll call this space $\expHoloL{\infty, \sigma}{Z}(\Omega)$ and give it the norm $\|f\|_{\infty, \sigma; Z} = \|\zeta^{-\sigma} e^{-\Re(Z\zeta)} f\|_\infty$, so that
%%\begin{align*}
%%\holoL{\infty}(\Omega) & \to \expHoloL{\infty, \sigma}{Z}(\Omega) \\
%%\varphi & \mapsto \zeta^\sigma e^{\Re(Z\zeta)} \varphi
%%\end{align*}
%%is an isometry. More generally,
%%\begin{align*}
%%\expHoloL{\infty, \rho}{Z}(\Omega) & \to \expHoloL{\infty, \rho+\delta}{Z}(\Omega) \\
%%f & \mapsto \zeta^\delta f
%%\end{align*}
%%is an isometry for all $\rho, \delta \in \R$ and $Z \in \C$. This reduces to the previous statement when $\rho = 0$ and $Z = 0$.
\subsection{Graded algebra structure}
For each $\delta \in [0, \infty)$, the functions in $\holoL{\infty, \rho}(\Omega)$ belong to $\holoL{\infty, \rho-\delta}(\Omega)$ too, and the inclusion map $\holoL{\infty, \rho}(\Omega) \hookrightarrow \holoL{\infty, \rho-\delta}(\Omega)$ has norm $\|\zeta^\delta\|_\infty$.
\begin{verify}
\begin{align*}
\|f\|_{\infty, \rho+\delta, Z} & = \|\zeta^{-(\rho-\delta)} e^{-\Re(Z\zeta)} f\|_\infty \\
& = \|\zeta^\delta \zeta^{-\rho} e^{-\Re(Z\zeta)} f\|_\infty \\
& = \|\zeta^\delta\|_\infty\,\|\zeta^{-\rho} e^{-\Re(Z\zeta)} f\|_\infty & \text{(Banach algebra)} \\
& \le \|\zeta^\delta\|_\infty\,\|f\|_{\infty, \rho; Z}
\end{align*}
\end{verify}
Since $\holoL{\infty}(\Omega)$ is a Banach algebra, the function space $\expHoloL{\infty, -\infty}{Z}(\Omega) := \bigcup_{\sigma \in \R} \expHoloL{\infty, \sigma}{Z}(\Omega)$ is a graded algebra, with a different norm on each grade. For each $\rho, \delta \in \R$, multiplication by a function $m \in \holoL{\infty, \delta}(\Omega)$ gives a map $\holoL{\infty, \rho}(\Omega) \to \holoL{\infty, \rho+\delta}(\Omega)$ with norm $\|m\|_{\infty, \delta}$.
\subsection{Norms of effectively finite-order operators}
\begin{itemize}
\item Say $|m(a)| \lesssim e^{c|\zeta(a)|}$ over all $a \in \Omega$. For any $f \in \expHoloL{\infty, \rho}{Z}$,
\begin{align*}
\|mf\|_{\infty, \sigma; Z} =
\end{align*}
\end{itemize}

which is effectively finite-order over $\Omega$ in the sense of Definition~\ref{fin-order}, with order at least $1 + \epsilon$ for some $\epsilon \in (0, \infty)$.
%------------------------------------------------------------------------
\color{SteelBlue}
\section{Integro-differential equations {[from {\tt airy-resurgence}]}}
\textcolor{DarkTurquoise}{\textbf{Note that {\tt airy-resurgence} uses the opposite sign convention for the parameter $\sigma$ in $\holoL{\infty, \sigma}$.}}
\subsection{Existence of solutions}
\subsubsection{Algebraic integral operators}
Take a simply connected open set $\Omega \subset \C$ that touches but doesn't contain $\zeta = 0$. Let $\holoL{\infty}(\Omega)$ be the space of bounded holomorphic functions on $\Omega$ with the supremum norm $\|\cdot\|_\infty$. For any $\sigma \in \R$, multiplying by $\zeta^{-\sigma}$ maps $\holoL{\infty}(\Omega)$ isomorphically onto another space of holomorphic functions on $\Omega$. We'll call this space $\holoL{\infty, \sigma}(\Omega)$ and give it the norm $\|f\|_{\infty, \sigma} = \|\zeta^\sigma f\|_\infty$, so that
\begin{align*}
\holoL{\infty}(\Omega) & \to \holoL{\infty, \sigma}(\Omega) \\
\phi & \mapsto \zeta^{-\sigma} \phi
\end{align*}
is an isometry. More generally,
\begin{align*}
\holoL{\infty, \rho}(\Omega) & \to \holoL{\infty, \rho+\delta}(\Omega) \\
f & \mapsto \zeta^{-\delta} f
\end{align*}
is an isometry for all $\rho \in \R$ and $\delta \in [0, \infty)$. This reduces to the previous statement when $\rho = 0$. For each $\delta \in [0, \infty)$, the functions in $\holoL{\infty, \rho}(\Omega)$ belong to $\holoL{\infty, \rho+\delta}(\Omega)$ too, and the inclusion map $\holoL{\infty, \rho}(\Omega) \hookrightarrow \holoL{\infty, \rho+\delta}(\Omega)$ has norm $\|\zeta^\delta\|_\infty$. Conceptually, $\|\zeta^\delta\|_\infty$ measures of the size of $\Omega$, so let's write it as $M^\delta$ with $M = \|\zeta\|_\infty$.

Since $\holoL{\infty}(\Omega)$ is a Banach algebra, the function space $\holoL{\infty, \infty}(\Omega) := \bigcup_{\sigma \in \R} \holoL{\infty, \sigma}(\Omega)$ is a graded algebra, with a different norm on each grade. For each $\rho, \delta \in \R$, multiplication by a function $m \in \holoL{\infty, \delta}(\Omega)$ gives a map $\holoL{\infty, \rho}(\Omega) \to \holoL{\infty, \rho+\delta}(\Omega)$ with norm $\|m\|_{\infty, \delta}$.

We'll study integral operators $\mathcal{G} \maps \holoL{\infty, \rho}(\Omega) \to \holoL{\infty, \sigma}(\Omega)$ of the form
\[ [\mathcal{G}f](a) = \int_{\zeta = 0}^{a} g(a, \cdot)\,f\,d\zeta, \]
where the kernel $g$ is an algebraic function over $\C^2$ which can be singular on $\Delta$, the diagonal.\footnote{Thanks to Alex Takeda for suggesting this.} To avoid ambiguity, we fix a branch of $g$ to use at the start of the integration path. The domain of $g$ is a covering of $\C^2$ which can be branched over $\Delta$. Continuing $g$ around $\Delta$ changes its phase by a root of unity, leaving its absolute value the same \textbf{[check]}. That makes $|g|$ a well-defined function on $\C^2 \smallsetminus \Delta$, which we can use to bound $\|\mathcal{G}\|$.

For each $a \in \C$, the expression $|g(a, \cdot)\,d\zeta|$ defines a {\em density} on $\Omega \smallsetminus \{a\}$---a norm on the tangent bundle which is compatible with the conformal structure. The square of a density is a Riemannian metric. Let $\ell^{\sigma, \rho}_{g, \Omega}(a)$ be the distance from $\zeta = 0$ to $a$ with respect to the density $|\zeta(a)^\sigma\,g(a, \cdot)\,\zeta^{-\rho}\,d\zeta|$ on $\Omega \smallsetminus \{a\}$. The bound
\begin{align*}
\big|[\zeta^\sigma \mathcal{G}f](a)\big| & \le \left| \zeta(a)^\sigma \int_{\zeta = 0}^{a} g(a, \cdot)\,f\,d\zeta \right| \\
& \le \int_{\zeta = 0}^{a} |\zeta^\rho f|\,|\zeta(a)^\sigma\,g(a, \cdot)\,\zeta^{-\rho}\,d\zeta| \\
& \le \|f\|_{\infty, \rho} \int_{\zeta = 0}^{a} |\zeta(a)^\sigma\,g(a, \cdot)\,\zeta^{-\rho}\,d\zeta|
\end{align*}
holds for any integration path. Taking the infimum over all paths, we see that
\[ \big|[\mathcal{G}f](a)\big| \le \ell^{\sigma, \rho}_{g, \Omega}(a)\,\|f\|_{\infty, \rho}. \]
so $\|\mathcal{G}\| \le \sup_{a \in \Omega} \ell^{\sigma, \rho}_{g, \Omega}(a)$. Crucially, we can always make $\|\mathcal{G}\|$ a contraction by restricting $\Omega$.
\subsubsection{The example of fractional integrals}
Setting $g(a, a') = (\zeta(a) - \zeta(a'))^{-\lambda-1}$ with $\lambda \in (-\infty, 0)$, we get the fractional integral $\partial^\lambda_{\zeta \text{ from } 0}$. The shortest path from $\zeta = 0$ to $a$ with respect to $|\zeta(a)^{\rho+\lambda}\,g(a, \cdot)\,\zeta^{-\rho}\,d\zeta|$ is the same as the shortest path with respect to $|d\zeta|$ \textcolor{magenta}{[check]}. It follows that
\begin{align*}
\ell^{\sigma, \rho}_{g, \Omega}(a) & = \int_0^{|\zeta(a)|} |\zeta(a)|^{\rho+\lambda}\,(|\zeta(a)| - r)^{-\lambda-1}\,r^{-\rho}\,dr \\
& = |\zeta(a)|^{\rho+\lambda} \int_0^1 \,(|\zeta(a)| - |\zeta(a)| t)^{-\lambda-1}\,(|\zeta(a)| t)^{-\rho}\,|\zeta(a)|\,dt \\
& = |\zeta(a)|^{\rho+\lambda-\lambda-1-\rho+1} \int_0^1 \,(1-t)^{-\lambda-1}\,t^{-\rho}\,dt \\
& = \int_0^1 \,(1-t)^{-\lambda-1}\,t^{-\rho}\,dt \\
& = B(-\lambda,\,1-\rho).
\end{align*}
The beta function $B$ can be written more explicitly as
\[ B(-\lambda,\,1-\rho) = \frac{\Gamma(-\lambda)\,\Gamma(1-\rho)}{\Gamma(1-\lambda-\rho)}. \]
Now we can see that for each $\lambda \in (-\infty, 0)$ and $\rho \in \R$, the fractional integral $\partial^\lambda_{\zeta \text{ from } 0}$ maps $\holoL{\infty, \rho}(\Omega)$ into $\holoL{\infty, \rho+\lambda}(\Omega)$, with norm $\|\partial^\lambda_{\zeta \text{ from } 0}\| \le B(-\lambda,\,1-\rho)$.
\subsubsection{Fractional integral equations near a regular singular point}\label{frac_int_exist}
\textcolor{SeaGreen}{\textbf{Angeliki:} Maybe Kato--Rellich perturbation theory can give existence immediately. It might not give uniqueness, though.}

Consider an integral operator $\mathcal{J}$ of the form
\[ p + \partial^{-1}_{\zeta \text{ from } 0} \circ q + \sum_{\lambda \in \Lambda} \partial^\lambda_{\zeta \text{ from } 0} \circ r_\lambda, \]
where:
\begin{itemize}
\item $p$ is a function in $\holoL{\infty, -1}(\Omega)$ that extends holomorphically over $\zeta = 0$, and its derivative at $\zeta = 0$ is non-zero.
\item $q$ is a function in $\holoL{\infty}(\Omega)$ that extends holomorphically over $\zeta = 0$.
\item $r_\lambda$ are functions in $\holoL{\infty}(\Omega)$.
\item $\Lambda$ is a countable subset of $(-\infty, -1)$ whose supremum is less than $-1$.
\end{itemize}
Our demand that $p$ and $q$ have convergent power series at $\zeta = 0$ can probably be relaxed; having convergent Novikov series, for example, should be enough. We could also probably replace $\partial^{-1}_{\zeta \text{ from } 0}$ with $\partial^{-1+\delta}_{\zeta \text{ from } 0} \circ \zeta^\delta$ for some $\delta \in [0, 1)$, or adjust the $\partial^\lambda_{\zeta \text{ from } 0}$ similarly.

We want to solve the equation $\mathcal{J}f = 0$. Let's look for a solution of the form $f = \zeta^{\tau-1} + \tilde{f}$ with $\tau \in (0, \infty)$ and $\tilde{f} \in \holoL{\infty, 1-\tau-\epsilon}(\Omega)$ for some $\epsilon \in (0, 1]$. When $\epsilon$ is small enough that $\Lambda \subset (-\infty, -1 - \epsilon]$, we'll see that we can always find such a solution, as long as we're willing to shrink $\Omega$. In fact, there's exactly one such solution. \textcolor{magenta}{[Add convergence conditions for $\partial^\lambda$ terms.]}

Let $p'_0$ and $q_0$ be the values of $\tfrac{\partial}{\partial \zeta} p$ and $q$, respectively, at $\zeta = 0$. We're assuming that $p$ and $q$ extend holomorphically over $\zeta = 0$, and the additional assumption that $p \in \holoL{\infty, -1}(\Omega)$ implies that $p$ has a first-order zero at $\zeta = 0$.

Since $p$ and $q$ extend holomorphically over $\zeta = 0$, and $p$ vanishes at $\zeta = 0$, we can write
\begin{alignat*}{2}
p & = p'_0 \zeta &\;+\;& \tilde{p} \\
q & = q_0 &\;+\;& \tilde{q}
\end{alignat*}
with $\tilde{p} \in \holoL{\infty, -2}(\Omega)$ and $\tilde{q} \in \holoL{\infty, -1}(\Omega)$. Then we have
\[ \mathcal{J} = p'_0\zeta + q_0\,\partial^{-1}_{\zeta \text{ from } 0} + \tilde{\mathcal{J}} \]
with
\[ \tilde{\mathcal{J}} = \tilde{p} + \partial^{-1}_{\zeta \text{ from } 0} \circ \tilde{q} + \sum_{\lambda \in \Lambda} \partial^\lambda_{\zeta \text{ from } 0} \circ r_\lambda \]
For any $\tau \in (0, \infty)$,
\[ \mathcal{J} \zeta^{\tau-1} = (p'_0 + q_0/\tau)\,\zeta^\tau + \tilde{\mathcal{J}} \zeta^{\tau-1}. \]
Setting $\tau = -q_0 / p'_0$ makes the first term vanish, leaving
\[ \mathcal{J} \zeta^{\tau-1} = \tilde{\mathcal{J}} \zeta^{\tau-1}. \]
Then the equation $\mathcal{J}f = 0$ becomes
\begin{align}
0 & = \tilde{\mathcal{J}}\zeta^{\tau-1} + \mathcal{J}\tilde{f} \nonumber \\
0 & = \tilde{\mathcal{J}}\zeta^{\tau-1} + \left[ p'_0\zeta + q_0\,\partial^{-1}_{\zeta \text{ from } 0} + \tilde{\mathcal{J}} \right]\tilde{f} \nonumber \\
-p'_0\zeta\tilde{f} & = \tilde{\mathcal{J}}\zeta^{\tau-1} + \left[ q_0\,\partial^{-1}_{\zeta \text{ from } 0} + \tilde{\mathcal{J}} \right]\tilde{f} \nonumber \\
\tilde{f} & = \left[ -\tfrac{1}{p'_0}\,\zeta^{-1} \circ \tilde{\mathcal{J}} \right]\,\zeta^{\tau-1} + \left[ \tau\,\zeta^{-1} \circ \partial^{-1}_{\zeta \text{ from } 0} - \tfrac{1}{p'_0}\,\zeta^{-1} \circ \tilde{\mathcal{J}} \right]\tilde{f}. \label{fixed-pt}
\end{align}

\color{Indigo}
\begin{center}
\begin{tikzcd}[column sep=25mm, row sep=2mm]
& & \holoL{\infty, \rho-1}(\Omega) \\
\holoL{\infty, \rho}(\Omega) \arrow[r, "\tilde{p}", "\|\tilde{p}\|_{\infty, -2}"'] & \holoL{\infty, \rho-2}(\Omega) \arrow[ru, hook, "\|\zeta\|_\infty"'] \arrow[rd, hook, "\|\zeta\|_\infty^{1-\epsilon}"'] \\
& & \holoL{\infty, \rho-1-\epsilon}(\Omega) \\
& & & \holoL{\infty, \rho-1}(\Omega) \\
\holoL{\infty, \rho}(\Omega) \arrow[r, "\tilde{q}", "\|\tilde{q}\|_{\infty, -1}"'] & \holoL{\infty, \rho-1}(\Omega) \arrow[r, "\partial^{-1}", "{B(1, 2-\rho) = \frac{1}{2-\rho}}"'] & \holoL{\infty, \rho-2}(\Omega) \arrow[ru, hook, "\|\zeta\|_\infty"'] \arrow[rd, hook, "\|\zeta\|_\infty^2"'] \\
& & & \holoL{\infty, \rho}(\Omega) \\
& & & \holoL{\infty, \rho-1}(\Omega) \\
\holoL{\infty, \rho}(\Omega) \arrow[r, "r_\lambda", "\|r_\lambda\|_\infty"'] & \holoL{\infty, \rho}(\Omega) \arrow[r, "\partial^\lambda", "{B(-\lambda, 1-\rho)}"'] & \holoL{\infty, \rho+\lambda}(\Omega) \arrow[ru, hook, "\|\zeta\|_\infty^{-1-\lambda}"'] \arrow[rd, hook, "\|\zeta\|_\infty^{-1-\epsilon-\lambda}"'] \\
& & & \holoL{\infty, \rho-1-\epsilon}(\Omega) \\
\holoL{\infty, \rho}(\Omega) \arrow[r, "\partial^{-1}", "{B(1, 1-\rho) = \frac{1}{1-\rho}}"'] & \holoL{\infty, \rho-1}(\Omega) \arrow[r, "\zeta^{-1}", "\|\zeta^{-1}\|_{\infty, 1} = 1"'] & \holoL{\infty, \rho}(\Omega) \\
\end{tikzcd}
\end{center}
\color{SteelBlue}

From \textbf{[our previous discussion]}, we can work out that
\begin{align*}
\tilde{\mathcal{J}} & \maps \holoL{\infty, 1-\tau}(\Omega) \to \holoL{\infty, -\tau-\epsilon}(\Omega)\;\text{with} \\
\|\tilde{\mathcal{J}}\| & \le \left(\|\tilde{p}\|_{\infty, -2} + \tfrac{1}{1+\tau}\,\|\tilde{q}\|_{\infty, -1} \right) M^{1-\epsilon} + \sum_{\lambda \in \Lambda} B(-\lambda, \tau)\,\|r_\lambda\|_{\infty}\,M^{-1-\epsilon-\lambda}
\end{align*}
and
\begin{align*}
\tilde{\mathcal{J}} & \maps \holoL{\infty, 1-\tau-\epsilon}(\Omega) \to \holoL{\infty, -\tau-\epsilon}(\Omega)\;\text{with} \\
\|\tilde{\mathcal{J}}\| & \le \left(\|\tilde{p}\|_{\infty, -2} + \tfrac{1}{1+\tau+\epsilon}\,\|\tilde{q}\|_{\infty, -1} \right) M + \sum_{\lambda \in \Lambda} B(-\lambda, \tau+\epsilon)\,\|r_\lambda\|_{\infty}\,M^{-1-\lambda}.
\end{align*}
Since $\|\zeta^{-1}\|_{\infty, 1} = 1$, it follows that
\begin{align}
\zeta^{-1} \circ \tilde{\mathcal{J}} & \maps \holoL{\infty, 1-\tau}(\Omega) \to \holoL{\infty, 1-\tau-\epsilon}(\Omega)\;\text{with} \nonumber \\
\|\zeta^{-1} \circ \tilde{\mathcal{J}}\| & \le \left(\|\tilde{p}\|_{\infty, -2} + \tfrac{1}{1+\tau}\,\|\tilde{q}\|_{\infty, -1} \right) M^{1-\epsilon} + \sum_{\lambda \in \Lambda} B(-\lambda, \tau)\,\|r_\lambda\|_{\infty}\,M^{-1-\epsilon-\lambda} \label{bound:mollify}
\end{align}
and
\begin{align}
\zeta^{-1} \circ \tilde{\mathcal{J}} & \acts \holoL{\infty, 1-\tau-\epsilon}(\Omega)\;\text{with} \nonumber \\
\|\zeta^{-1} \circ \tilde{\mathcal{J}}\| & \le \left(\|\tilde{p}\|_{\infty, -2} + \tfrac{1}{1+\tau+\epsilon}\,\|\tilde{q}\|_{\infty, -1} \right) M + \sum_{\lambda \in \Lambda} B(-\lambda, \tau+\epsilon)\,\|r_\lambda\|_{\infty}\,M^{-1-\lambda}. \label{bound:perturb}
\end{align}
We can also see that
\begin{align}
\tau\,\zeta^{-1} \circ \partial^{-1}_{\zeta \text{ from } 0} & \acts \holoL{\infty, 1-\tau-\epsilon}(\Omega)\;\text{with} \nonumber \\
\|\tau\,\zeta^{-1} \circ \partial^{-1}_{\zeta \text{ from } 0}\| & = \tfrac{\tau}{\tau+\epsilon} < 1 \label{bound:contract}.
\end{align}

Now, let's return to equation~\ref{fixed-pt}, which tells us that $f = \zeta^{\tau-1} + \tilde{f}$ satisfies $\mathcal{J}f = 0$ when \textcolor{magenta}{[and only when?]} $\tilde{f}$ is a fixed point of the affine map $\mathcal{A}(\cdot) + b$, where
\begin{align*}
\mathcal{A} & = \tau\,\zeta^{-1} \circ \partial^{-1}_{\zeta \text{ from } 0} - \tfrac{1}{p'_0}\,\zeta^{-1} \circ \tilde{\mathcal{J}} \\
b & = \left[ -\tfrac{1}{p'_0}\,\zeta^{-1} \circ \tilde{\mathcal{J}} \right]\,\zeta^{\tau-1}
\end{align*}
Choosing $\epsilon \in (0, 1]$ so that $\Lambda \subset (-\infty, -1 - \epsilon]$ has given us the domain and codomain statements in bounds \ref{bound:mollify} and \ref{bound:perturb}, which tell us that $\mathcal{A}(\cdot) + b$ sends $\holoL{\infty, 1-\tau-\epsilon}(\Omega)$ into itself. We'll show that when $\Omega$ is small enough, $\mathcal{A}(\cdot) + b$ contracts $\holoL{\infty, 1-\tau-\epsilon}(\Omega)$, and thus---by the contraction mapping theorem---has a unique fixed point.

An affine map is a contraction if and only if its linear part is a contraction. We know from bound~\ref{bound:contract} that $\tau\,\zeta^{-1} \circ \partial^{-1}_{\zeta \text{ from } 0}$ contracts $\holoL{\infty, 1-\tau-\epsilon}(\Omega)$. Since the supremum of $\Lambda$ is less than $-1$, all the powers of $M = \|\zeta\|_\infty$ in bound~\ref{bound:perturb} are positive. Thus, by shrinking $\Omega$, we can make the norm of $\zeta^{-1} \circ \tilde{\mathcal{J}}$ on $\holoL{\infty, 1-\tau-\epsilon}(\Omega)$ as small as we want---small enough to make $\mathcal{A}$ a contraction.
\end{document}