\documentclass{article}

\usepackage{ajf}
\usepackage{url}
\usepackage[hmargin=1.5in]{geometry}
\usepackage{amsthm}

\theoremstyle{definition}
\newtheorem{defn}{Definition}
\theoremstyle{plain}
\newtheorem{prop}{Proposition}

% symbology
\newcommand{\posreal}{\R_{> 0}}
\newcommand{\laplace}{\mathcal{L}}
\DeclareMathOperator{\Ai}{Ai}

\title{Resurgence of the Airy function}
\author{Aaron Fenyes}

\begin{document}
\maketitle
\section{The Laplace transform}
\subsection{Analytic version}
\subsubsection{Regularity and decay properties}
The Laplace transform is defined by
\[ [\mathcal{L}\hat{\varphi}](z) = \int_0^\infty e^{-z \zeta} \hat{\varphi}(\zeta)\;d\zeta. \]
For $a \in [0, \infty]$, recall that recall that $O_a(g)$ is the space of functions $\varphi$ on $\posreal$ with $|\varphi| \lesssim g$ in some neighborhood of $a$. A function is {\em subexponential} if it's in $O_\infty(e^{cz})$ for all $c \in \posreal$. Let $\mathcal{E}$ be the space of subexponential functions on $\posreal$ which are $L^1$ both locally and around zero. The Laplace transform maps $\mathcal{E}$ into the space of holomorphic functions on the right half-plane~\cite[\S 5.6]{diverg-resurg-i}. If a function is in $O_0(1)$, its Laplace transform is in $O_\infty(z^{-1})$ \cite[equation~1.8]{laplace-tfm}.\footnote{The argument cited still works in our generality. For holomorphic $\hat{\varphi}$, one can also use Equation 1.5 of {\em Borel-Laplace Transform and Asymptotic Theory} \textbf{(Sternin \& Shatalov)}.} More generally, if a function is in $O_0(\zeta^\alpha)$, with $\alpha > -1$, its Laplace transform is in $O_\infty(z^{-(\alpha + 1)})$.
\subsubsection{Action on differential operators}\label{L-diff-op}
When $\hat{\varphi} \in \mathcal{E}$, we can use differentiation under the integral to show that~\cite[Theorem~1.34]{laplace-tfm}
\begin{equation}\label{id:L-mult}
\laplace (\zeta^n \hat{\varphi}) = \big({-\tfrac{\partial}{\partial z}}\big)^n \laplace \hat{\varphi}.
\end{equation}

When $\hat{\varphi}$ is $n$ times differentiable, its $n$th derivative is in $\mathcal{E}$, and its zeroth through $(n - 1)$st derivatives extend continuously to zero, integration by parts gives the formula
\begin{align}\label{id:L-diff}
\laplace \big(\tfrac{\partial}{\partial \zeta}\big)^n \hat{\varphi} & = z^n \laplace \hat{\varphi} - \left( \hat{\varphi}(0)\,z^{n-1} + \hat{\varphi}'(0)\,z^{n-2} + \hat{\varphi}''(0)\,z^{n-3} + \ldots + \hat{\varphi}^{(n-1)}(0) \right) \\
& = z^n\,\laplace\left[ \hat{\varphi} - \left( \hat{\varphi}(0) + \hat{\varphi}'(0)\,\zeta + \tfrac{\hat{\varphi}''(0)}{2!}\,\zeta^2 + \ldots + \tfrac{\hat{\varphi}^{(n-1)}(0)}{(n-1)!}\,\zeta^{n-1} \right) \right]. \nonumber
\end{align}
Note that if a function's derivative is subexponential, so is the function itself.\footnote{Say $f' \in O(e^{c\zeta})$. Then \[ \left|\int_0^\zeta f'(\tau)\,d\tau\right| \le \int_0^\zeta |f'(\tau)|\,d\tau \lesssim \int_0^\zeta e^{c\tau}\,d\tau = \tfrac{1}{c}(e^{c\zeta} - 1) \lesssim e^{c\zeta}.\] Now we know the integral on the left-hand side converges, implying that $f$ extends continuously to zero, with $|f(\zeta) - f(0)| \lesssim e^{c\zeta}$.}

%%In the space of holomorphic functions on the right half-plane, $\C[z] + O(z^{-1})$ is a direct sum, so there's a unique linear projection $\Lambda \maps \C[z] + O(z^{-1}) \to O(z^{-1})$. Applying $\Lambda$ to both sides of the formula above reveals that
%%\[ \laplace \big(\tfrac{\partial}{\partial \zeta}\big)^n \hat{\varphi} = \Lambda (z^n\,\laplace \hat{\varphi}). \]

%%Say $A$ is an $n$th-order differential operator with polynomial coefficients. If we know that $A\hat{\varphi} = 0$, and we know the Taylor polynomial of $\hat{\varphi}$ at zero through degree $n - 1$, we can use the formulas above to find an inhomogeneous differential equation that $\laplace \hat{\varphi}$ must satisfy.
\subsection{Algebraic version}
\subsubsection{Definition}
Let $\mathcal{P}$ be the vector space spanned by $\zeta^\alpha$ for $\alpha \in \R \smallsetminus \Z_{< 0}$. Note that $\mathcal{P} \cap \mathcal{E}$ is $\mathcal{P}_{> -1}$, the subspace spanned by $\zeta^\alpha$ with $\alpha > -1$. Since $\laplace(\zeta^\alpha) = \Gamma(\alpha+1)\,z^{-(\alpha + 1)}$ for all $\alpha > -1$, let's use the same formula to extend $\laplace$ to all of $\mathcal{P}$. This defines $\laplace$ consistently on $\mathcal{E} + \mathcal{P}$.
\subsubsection{Action on differential operators}
Observe that
\[ \laplace(\zeta^{\alpha + 1}) = -\tfrac{\partial}{\partial z}\,\laplace(\zeta^\alpha) \]
for $\alpha \neq -1$. This extends identity~\ref{id:L-mult} to all of $\mathcal{P}$.

Observe that
\[ \laplace\tfrac{\partial}{\partial \zeta}(\zeta^\alpha) = \begin{cases}
z\,\laplace(\zeta^\alpha) & \alpha \neq 0 \\
0 & \alpha = 0,
\end{cases} \]
and that $0 = z\,\laplace(1) - 1$. This recovers identity~\ref{id:L-diff} for any function in $\mathcal{P}$ whose $n$th derivative is in $\mathcal{P}_{> -1}$. Although the functions in $\mathcal{P}_{< 0}$ are singular at zero, let's pretend they vanish at zero. With that convention, formula~\ref{id:L-diff} extends to all of $\mathcal{P}$.

Now we have the results of Section~\ref{L-diff-op} for all functions in $\mathcal{E} + \mathcal{P}$. Identity~\ref{id:L-diff} is particularly simple for functions that can be written as $\hat{\varphi}_\text{frac} + \hat{\varphi}_\text{reg}$, where $\hat{\varphi}_\text{frac} \in \mathcal{P}$ has only non-integer exponents, and the zeroth through $(n-1)$st derivatives of $\hat{\varphi}_\text{reg} \in \mathcal{E}$ vanish at zero. For these functions, all the initial values vanish, and we have
\[ \laplace \big(\tfrac{\partial}{\partial \zeta}\big)^n (\hat{\varphi}_\text{frac} + \hat{\varphi}_\text{reg}) = z^n \laplace (\hat{\varphi}_\text{frac} + \hat{\varphi}_\text{reg}). \]
\section{The Airy equation}
\subsection{Basics}
The Airy equation is
\begin{equation}\label{eqn:airy}
\left[\big(\tfrac{\partial}{\partial x}\big)^2 - x\right] \psi = 0.
\end{equation}
One solution is given by the Airy function,
\[ \Ai(x) = \frac{1}{2\pi i} \int_{\Gamma_0} \exp\left(\tfrac{1}{3}t^3 - xt\right)\,dt, \]
where $\Gamma_0$ is a path that comes from $\infty$ at $-60^\circ$ and goes to $\infty$ at $60^\circ$. With the substitution $t = 2ux^{1/2}$, we can rewrite the Airy integral as
\[ \Ai(x) = x^{1/2}\;\frac{1}{\pi i} \int_{x^{-1/2} \Gamma_0} \exp\left[\tfrac{2}{3}x^{3/2} \left(4u^3 - 3u\right)\right]\,du. \]
We've rescaled the contour by a factor of two, but it still approaches $\infty$ in the desired way. Note that $4u^3 - 3u$ is the third Chebyshev polynomial.
\subsection{Rewriting as a modified Bessel equation}
We can distill the most interesting part of the Airy function by writing
\[ \Ai(x) = \tfrac{1}{\pi\sqrt{3}}\,x^{1/2}\, K\big(\tfrac{2}{3} x^{3/2}\big), \]
where
\[ K(z) = \frac{\sqrt{3}}{i} \int_{z^{-1/3}\Gamma_0} \exp\left[z \left(4u^3 - 3u\right)\right]\,du. \]
Saying that $\Ai$ satisfies the Airy equation is equivalent to saying that $K$ satisfies the modified Bessel equation
\begin{equation}\label{eqn:mod-bessel}
\left[z^2 \big(\tfrac{\partial}{\partial z}\big)^2 + z \tfrac{\partial}{\partial z} - \big[\big(\tfrac{1}{3}\big)^2 + z^2\big]\right] \varphi = 0.
\end{equation}
In fact, $K$ is the modified Bessel function $K_{1/3}$~\cite[equation~9.6.1]{dlmf}.
\subsection{Going to the spatial domain}
Let's find a function $\hat{K}$ with $K = \laplace \hat{K}$, which is unique if it exists~\cite[Theorem~1.23]{laplace-tfm}. If a function $\hat{\varphi}$ satisfies the equation
\begin{equation}\label{eqn:spatial-mod-bessel}
\left[\big(\zeta^2 - 1\big) \big(\tfrac{\partial}{\partial \zeta}\big)^2 + 3\zeta \tfrac{\partial}{\partial \zeta} + \big[1 - \big(\tfrac{1}{3}\big)^2\big]\right] \hat{\varphi} = 0,
\end{equation}
its Laplace transform $\varphi = \laplace \hat{\varphi}$ satisfies the equation
\begin{align*}
\left[\big({-\tfrac{\partial}{\partial z}}\big)^2 - 1\right] \Big[z^2 \varphi - \Big(\hat{\varphi}(0)\,z + \hat{\varphi}'(0)\Big)\Big] + 3\big({-\tfrac{\partial}{\partial z}}\big)\Big[z\varphi - \hat{\varphi}(0)\Big] + \big[1 - \big(\tfrac{1}{3}\big)^2\big] \varphi & = 0 \\
\big(\tfrac{\partial}{\partial z}\big)^2 \big[z^2 \varphi\big] - \Big[z^2 \varphi - \Big(\hat{\varphi}(0)\,z + \hat{\varphi}'(0)\Big)\Big] - 3\big(\tfrac{\partial}{\partial z}\big)\big[z\varphi\big] + \big[1 - \big(\tfrac{1}{3}\big)^2\big] \varphi & = 0 \\
\Big[2 + 4z\tfrac{\partial}{\partial z} + z^2\big(\tfrac{\partial}{\partial z}\big)^2\Big]\varphi - \Big[z^2 \varphi - \Big(\hat{\varphi}(0)\,z + \hat{\varphi}'(0)\Big)\Big] - 3\Big[1 + z\tfrac{\partial}{\partial z}\Big]\varphi + \big[1 - \big(\tfrac{1}{3}\big)^2\big] \varphi & = 0,
\end{align*}
which simplifies to
\begin{equation}\label{eqn:inhomog-mod-bessel}
\left[z^2 \big(\tfrac{\partial}{\partial z}\big)^2 + z \tfrac{\partial}{\partial z} - \big[\big(\tfrac{1}{3}\big)^2 + z^2\big]\right] \varphi = -\Big(\hat{\varphi}(0)\,z + \hat{\varphi}'(0)\Big).
\end{equation}
Noticing that equation~\ref{eqn:inhomog-mod-bessel} is an inhomogeneous version of equation~\ref{eqn:mod-bessel}, we

%%This suggests that $\hat{K}$ could be a solution of equation~\ref{eqn:spatial-mod-bessel} that vanishes through first degree at $\zeta = 0$.

With the change of variable $\xi = \tfrac{1}{2}(1-\zeta)$, equation~\ref{eqn:spatial-mod-bessel} becomes the hypergeometric equation
\begin{equation}\label{eqn:hypergeom}
\left[\xi(1 - \xi) \big(\tfrac{\partial}{\partial \xi}\big)^2 + 3(\tfrac{1}{2} - \xi) \tfrac{\partial}{\partial \xi} - \big[1 - \big(\tfrac{1}{3}\big)^2\big]\right] \hat{\varphi} = 0.
\end{equation}
\textbf{[...]}

The hypergeometric function
\[ F\big(\tfrac{2}{3}, \tfrac{4}{3}; \tfrac{3}{2}; \xi\big) \]
satisfies equation~\ref{eqn:hypergeom} by definition. It has a singularity of type $\xi^{-1/2}$ at $\xi = 0$, and continues holomorphically to the rest of $\C$. Using DLMF~15.10.13, we can see that
\[ F\big(\tfrac{2}{3}, \tfrac{4}{3}; \tfrac{3}{2}; 1-\xi\big). \]
also satisies equation~\ref{eqn:hypergeom}. It has a singularity of type $(1 - \xi)^{-1/2}$ at $\xi = 1$, and continues holomorphically to the rest of $\C$.

Based on the calculations in the other sheet, we should have
\[ K(z) = i \frac{2}{\sqrt{3}} \int_{\text{Hankel at } 1} e^{-z\zeta} F\big(\tfrac{1}{3}, \tfrac{2}{3}; \tfrac{1}{2}; \zeta^2\big)\;d\zeta. \]
This is because the hypergeometric function in the integrand happens to be algebraic. It can be expressed in terms of Legendre functions and rearranged to get
\[ F\big(\tfrac{1}{3}, \tfrac{2}{3}; \tfrac{1}{2}; \zeta^2\big) = \frac{1}{4u^2 - 1} \]
with $\zeta = -4u^3 + 3u$.

The quadratic transformation identity DLMF~15.8.27 then shows that
\[ F\big(\tfrac{1}{3}, \tfrac{2}{3}; \tfrac{1}{2}; \zeta^2\big) \propto F\big(\tfrac{2}{3}, \tfrac{4}{3}; \tfrac{3}{2}; \xi\big) + F\big(\tfrac{2}{3}, \tfrac{4}{3}; \tfrac{3}{2}; 1-\xi\big), \]
so we have
\[ K(z) \propto \int_{\text{Hankel at } 1} e^{-z\zeta} \big[F\big(\tfrac{2}{3}, \tfrac{4}{3}; \tfrac{3}{2}; \xi\big) + F\big(\tfrac{2}{3}, \tfrac{4}{3}; \tfrac{3}{2}; 1-\xi\big)\big]\;d\zeta. \]
The first term in the integrand is regular on $\zeta \in [1, \infty)$, so it integrates to zero. The second term looks like $(1 - \zeta)^{-1/2}$ times a function which is regular for $\zeta \in [1, \infty)$, so integrating it along a Hankel contour is the same as integrating it times two times a direct contour, with its value continued from below the singularity. Hence,
\[ K(z) \propto \int^\infty_1 e^{-z\zeta} F\big(\tfrac{2}{3}, \tfrac{4}{3}; \tfrac{3}{2}; 1-\xi\big)\;d\zeta. \]
With $\zeta = 1 + \tilde{\zeta}$, we get $1 - \xi = 1 + \tfrac{1}{2} \tilde{\zeta}$, so
\begin{align*}
K(z) & = \int^\infty_0 e^{-z(1 + \tilde{\zeta})} F\big(\tfrac{2}{3}, \tfrac{4}{3}; \tfrac{3}{2}; 1 + \tfrac{1}{2}\tilde{\zeta}\big)\;d\tilde{\zeta} \\
e^z K(z) & = \int^\infty_0 e^{-z\tilde{\zeta}} F\big(\tfrac{2}{3}, \tfrac{4}{3}; \tfrac{3}{2}; 1 + \tfrac{1}{2}\tilde{\zeta}\big)\;d\tilde{\zeta}
\end{align*}
In other words, $\kappa(z) = e^z K(z)$ is the Laplace transform of $\hat{\kappa}(\tilde{\zeta}) = F\big(\tfrac{2}{3}, \tfrac{4}{3}; \tfrac{3}{2}; 1 + \tfrac{1}{2}\tilde{\zeta}\big)$. We've confirmed numerically that $\hat{\kappa}(\tilde{\zeta})$ seems to have a singularity of type $\tilde{\zeta}^{-1/2}$ at $\tilde{\zeta} = 0$, as expected. \textbf{Can we see this analytically?}

Using the DLMF identities 15.10.12 and 15.10.14, we can see that the functions
\begin{align*}
\xi^{-1/2} & F\big(\tfrac{1}{6}, \tfrac{5}{6}; \tfrac{1}{2}; \xi\big) \\
(1-\xi)^{-1/2} & F\big(\tfrac{1}{6}, \tfrac{5}{6}; \tfrac{1}{2}; 1-\xi\big)
\end{align*}
also satisfy equation~\ref{eqn:hypergeom}. Numerically, these functions appear to sum to a constant multiple of $F\big(\tfrac{1}{3}, \tfrac{2}{3}; \tfrac{1}{2}; \zeta^2\big)$.
\subsection{Correspondence with Mari\~{n}o's series}
Let $f_1(z)$ be the holomorphic function corresponding to Mari\~{n}o's formal power series $\varphi_1(z^{-1})$. The formal power series corresponding to $f$ will be written in the variable $z$.

\begin{align*}
\Ai(x) & = \tfrac{1}{2\sqrt{\pi}} x^{-1/4} e^{-z} \varphi_1\big(\tfrac{2}{3} z^{-1}\big) \\
& = \tfrac{1}{2\sqrt{\pi}} x^{-1/4} e^{-z} f_1\big(\tfrac{3}{2} z\big) \\
\Ai(x) & = \frac{1}{\pi\sqrt{3}} x^{1/2} K(\tfrac{2}{3} x^{3/2})
\end{align*}
Putting together,
\begin{align*}
\tfrac{1}{2\sqrt{\pi}} x^{-1/4} e^{-z} f_1\big(\tfrac{3}{2} z\big) & = \frac{1}{\pi\sqrt{3}} x^{1/2} K(\tfrac{2}{3} x^{3/2}) \\
\tfrac{\sqrt{3\pi}}{2}\;x^{-3/4} e^{-z} f_1\big(\tfrac{3}{2} z\big) & = K(\tfrac{2}{3} x^{3/2}) \\
\tfrac{\sqrt{3\pi}}{2}\;\big(\tfrac{3}{2} z)^{-1/2} e^{-z} f_1\big(\tfrac{3}{2} z\big) & = K(z) \\
\sqrt{\tfrac{\pi}{2}}\;z^{-1/2} e^{-z} f_1\big(\tfrac{3}{2} z\big) & = K(z) \\
\sqrt{\tfrac{\pi}{2}}\;\big[\laplace^{-1} z^{-1/2}\big] * \big[\laplace^{-1} f_1\big(\tfrac{3}{2} z\big)\big](\zeta - 1) & = \hat{K}(\zeta) \\
\sqrt{\tfrac{\pi}{2}}\;\left[\Gamma\big({-\tfrac{1}{2}}\big)^{-1} \zeta^{-1/2}\right] * \tfrac{2}{3} \hat{f}_1\big[\tfrac{2}{3}(\zeta - 1)\big] & = \hat{K}(\zeta) \\
-\tfrac{1}{3\sqrt{2}}\;\left[\zeta^{-1/2}\right] * \hat{f}_1\big[\tfrac{2}{3}(\zeta - 1)\big] & = \hat{K}(\zeta) \\
\end{align*}
Notice that if the hypergeometric differentiation formula holds for fractional derivatives,
\[ \big(\tfrac{\partial}{\partial \xi}\big)^{1/2}F\big(\tfrac{2}{3}, \tfrac{4}{3}; \tfrac{3}{2}; \xi\big) \propto F\big(\tfrac{7}{6}, \tfrac{11}{6}; 2; \xi\big) \]
\bibliographystyle{utphys}
\bibliography{airy-resurgence}
\end{document}