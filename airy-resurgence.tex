\documentclass{article}

\usepackage{ajf}
\usepackage{amsthm}

\theoremstyle{definition}
\newtheorem{defn}{Definition}
\theoremstyle{plain}
\newtheorem{prop}{Proposition}

% symbology
\newcommand{\posreal}{\R_{> 0}}
\newcommand{\laplace}{\mathcal{L}}
\DeclareMathOperator{\Ai}{Ai}

\title{Resurgence of the Airy function}
\author{Aaron Fenyes}

\begin{document}
\maketitle
\section{The Laplace transform}
\subsection{Regularity and decay properties}
The Laplace transform is defined by
\[ [\mathcal{L}\hat{\varphi}](z) = \int_0^\infty e^{-z \zeta} \hat{\varphi}(\zeta)\;d\zeta. \]
Recall that $O(g)$ is the space of functions $\varphi$ on $\posreal$ with $|\varphi| \lesssim g$. A function is {\em subexponential} if it's in $O(e^{cz})$ for all $c \in \posreal$. Let $\mathcal{E}$ be the space of subexponential functions on $\posreal$ which are $L^1$ both locally and around zero. Observe that $\laplace$ maps $\mathcal{E}$ into the space of holomorphic functions on the right half-plane \textbf{(Mitschi \& Sauzin, \S 5.6)} which restrict to $O(z^{-1})$ on $\posreal$ \textbf{(Schiff, equation~1.8---the argument still works in our generality)}.\footnote{For holomorphic $\hat{\varphi}$, one can also use Equation 1.5 of {\em Borel-Laplace Transform and Asymptotic Theory} \textbf{(Sternin \& Shatalov)}.}
\subsection{Action on initial value problems}
When $\hat{\varphi} \in \mathcal{E}$, we can use differentiation under the integral to show that \textbf{[Schiff, Theorem~1.34]}
\[ \laplace (\zeta^n \hat{\varphi}) = \big({-\tfrac{\partial}{\partial z}}\big)^n \laplace \hat{\varphi}. \]

When $\hat{\varphi} \in C^n$ and $\big(\tfrac{\partial}{\partial \zeta}\big)^n \hat{\varphi} \in \mathcal{E}$, integration by parts gives the formula
\begin{align*}
\laplace \big(\tfrac{\partial}{\partial \zeta}\big)^n \hat{\varphi} & = z^n \laplace \hat{\varphi} - \left( \hat{\varphi}(0)\,z^{n-1} + \hat{\varphi}'(0)\,z^{n-2} + \hat{\varphi}''(0)\,z^{n-3} + \ldots + \hat{\varphi}^{(n-1)}(0) \right) \\
& = z^n\,\laplace\left[ \hat{\varphi} - \left( \hat{\varphi}(0) + \hat{\varphi}'(0)\,\zeta + \tfrac{\hat{\varphi}''(0)}{2!}\,\zeta^2 + \ldots + \tfrac{\hat{\varphi}^{(n-1)}(0)}{(n-1)!}\,\zeta^{n-1} \right) \right].
\end{align*}
Note that if a function's derivative is subexponential, so is the function itself.\footnote{Say $f' \in O(e^{c\zeta})$. Since constant functions are subexponential, we can assume without loss of generality that $f(0) = 0$ \textbf{[watch out: $f$ may not extend continuously to $0$]}. Then \[ |f(\zeta)| = \left|\int_0^\zeta f(\tau)\,d\tau\right| \le \int_0^\zeta |f(\tau)|\,d\tau \lesssim \int_0^\zeta e^{c\tau}\,d\tau = \tfrac{1}{c}(e^{c\zeta} - 1) \lesssim e^{c\zeta}.\]}

In the space of holomorphic functions on the right half-plane, $\C[z] + O(z^{-1})$ is a direct sum, so there's a unique linear projection $\Lambda \maps \C[z] + O(z^{-1}) \to O(z^{-1})$. Applying $\Lambda$ to both sides of the formula above reveals that
\[ \laplace \big(\tfrac{\partial}{\partial \zeta}\big)^n \hat{\varphi} = \Lambda (z^n\,\laplace \hat{\varphi}). \]

Say $A$ is an $n$th-order differential operator with polynomial coefficients. If we know that $A\hat{\varphi} = 0$, and we know the Taylor polynomial of $\hat{\varphi}$ at zero through degree $n - 1$, we can use the formulas above to find an inhomogeneous differential equation that $\laplace \hat{\varphi}$ must satisfy.
\section{The Airy equation}
\subsection{Basics}
The Airy equation is
\[ \left[\big(\tfrac{\partial}{\partial x}\big)^2 - x\right] \varphi = 0. \]
One solution is given by the Airy function,
\[ \Ai(x) = \frac{1}{2\pi i} \int_{\Gamma_0} \exp\left(\tfrac{1}{3}t^3 - xt\right)\,dt, \]
where $\Gamma_0$ is a path that comes from $\infty$ at $-60^\circ$ and goes to $\infty$ at $60^\circ$. With the substitution $t = 2ux^{1/2}$, we can rewrite the Airy integral as
\[ \Ai(x) = x^{1/2}\;\frac{1}{\pi i} \int_{x^{-1/2} \Gamma_0} \exp\left[\tfrac{2}{3}x^{3/2} \left(4u^3 - 3u\right)\right]\,du. \]
We've rescaled the contour by a factor of two, but it still approaches $\infty$ in the desired way. Note that $4u^3 - 3u$ is the third Chebyshev polynomial.
\subsection{Rewriting as a modified Bessel equation}
We can distill the most interesting part of this expression by writing
\[ \Ai(x) = \tfrac{1}{\pi\sqrt{3}}\,x^{1/2}\, K\big(\tfrac{2}{3} x^{3/2}\big), \]
where
\[ K(z) = \frac{\sqrt{3}}{i} \int_{z^{-1/3}\Gamma_0} \exp\left[z \left(4u^3 - 3u\right)\right]\,du. \]
Saying that $\Ai$ satisfies the Airy equation is equivalent to saying that $K$ satisfies the modified Bessel equation
\[ \left[z^2 \big(\tfrac{\partial}{\partial z}\big)^2 + z \big(\tfrac{\partial}{\partial z}\big) - \big[\big(\tfrac{1}{3}\big)^2 + z^2\big]\right] K(z) = 0. \]

Since our rewritten Airy integral only depends on the variable $z = \tfrac{2}{3} x^{3/2}$,



we can \textbf{[...]}

In terms of

we have
\[ \Ai(x) = \frac{1}{\pi\sqrt{3}} x^{1/2} K(\tfrac{2}{3} x^{3/2}) \]
[DLMF~9.6.1].
\subsection{Taking the Laplace transform}
Tentatively (there's some details with initial values to work out),
\[ \big(\zeta^2 - 1\big) \hat{K}'' + 3\zeta \hat{K}' + \big[1 - \big(\tfrac{1}{3}\big)^2\big]\hat{K} = 0. \]
With the change of variable $\xi = \tfrac{1}{2}(1-\zeta)$, this becomes \textbf{[...]}

The hypergeometric function
\[ F\big(\tfrac{2}{3}, \tfrac{4}{3}; \tfrac{3}{2}; \xi\big) \]
satisfies the same equation as $\hat{K}$ by definition. Using DLMF~15.10.13, we can see that
\[ F\big(\tfrac{2}{3}, \tfrac{4}{3}; \tfrac{3}{2}; 1-\xi\big). \]
\textbf{Our desired solution has the 0th- and 1st-degree Taylor coefficients in a particular ratio. It should be given by a constant multiple of}
\[ F\big(\tfrac{2}{3}, \tfrac{4}{3}; \tfrac{3}{2}; \xi\big) - F\big(\tfrac{2}{3}, \tfrac{4}{3}; \tfrac{3}{2}; 1-\xi\big). \]
The quadratic transformation identity DLMF~15.8.27 then shows that
\[ \hat{K} \propto F\big(\tfrac{1}{3}, \tfrac{2}{3}; \tfrac{1}{2}; \zeta^2\big). \]

Using the DLMF identities 15.10.11 and 15.10.14, we can see that the functions
\begin{align*}
& (1-\xi)^{-1/2} F\big(\tfrac{1}{6}, \tfrac{5}{6}; \tfrac{1}{2}; \xi\big) \\
& (1-\xi)^{-1/2} F\big(\tfrac{1}{6}, \tfrac{5}{6}; \tfrac{1}{2}; 1-\xi\big)
\end{align*}
also satisfy the same hypergeometric equation as $\hat{K}$.
\subsection{Correspondence with Mari\~{n}o's series}
Let $f_1(z)$ be the holomorphic function corresponding to Mari\~{n}o's formal power series $\varphi_1(z^{-1})$. The formal power series corresponding to $f$ will be written in the variable $z$.

\begin{align*}
\Ai(x) & = \tfrac{1}{2\sqrt{\pi}} x^{-1/4} e^{-z} \varphi_1\big(\tfrac{2}{3} z^{-1}\big) \\
& = \tfrac{1}{2\sqrt{\pi}} x^{-1/4} e^{-z} f_1\big(\tfrac{3}{2} z\big) \\
\Ai(x) & = \frac{1}{\pi\sqrt{3}} x^{1/2} K(\tfrac{2}{3} x^{3/2})
\end{align*}
Putting together,
\begin{align*}
\tfrac{1}{2\sqrt{\pi}} x^{-1/4} e^{-z} f_1\big(\tfrac{3}{2} z\big) & = \frac{1}{\pi\sqrt{3}} x^{1/2} K(\tfrac{2}{3} x^{3/2}) \\
\tfrac{\sqrt{3\pi}}{2}\;x^{-3/4} e^{-z} f_1\big(\tfrac{3}{2} z\big) & = K(\tfrac{2}{3} x^{3/2}) \\
\tfrac{\sqrt{3\pi}}{2}\;\big(\tfrac{3}{2} z)^{-1/2} e^{-z} f_1\big(\tfrac{3}{2} z\big) & = K(z) \\
\sqrt{\tfrac{\pi}{2}}\;z^{-1/2} e^{-z} f_1\big(\tfrac{3}{2} z\big) & = K(z) \\
\sqrt{\tfrac{\pi}{2}}\;\big[\laplace^{-1} z^{-1/2}\big] * \big[\laplace^{-1} f_1\big(\tfrac{3}{2} z\big)\big](\zeta - 1) & = \hat{K}(\zeta) \\
\sqrt{\tfrac{\pi}{2}}\;\left[\Gamma\big({-\tfrac{1}{2}}\big)^{-1} \zeta^{-1/2}\right] * \tfrac{2}{3} \hat{f}_1\big[\tfrac{2}{3}(\zeta - 1)\big] & = \hat{K}(\zeta) \\
-\tfrac{1}{3\sqrt{2}}\;\left[\zeta^{-1/2}\right] * \hat{f}_1\big[\tfrac{2}{3}(\zeta - 1)\big] & = \hat{K}(\zeta) \\
\end{align*}
\end{document}